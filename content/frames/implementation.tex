\subsection{Overview}

\begin{frame}
    \frametitle{Overview of the Implementation}

    \begin{multicols}{2}

        % Properties.
        \begin{center}
            {\color{\accentcolor} \Large \textbf{Components}}
            \vspace*{0.5cm}

            \begin{minipage}{0.4\textwidth}
                \begin{itemize}
                    \item Prismatic Meshes
                    \item Polynomial Basis
                    \item Quadrature
                    \item Problem Assembly
                \end{itemize}
            \end{minipage}
        \end{center}

        \vfill\null
        \columnbreak

        % Generation.
        \begin{center}
            {\color{\accentcolor} \Large \textbf{Details}}
            \vspace*{0.125cm}

            \begin{minipage}{0.4\textwidth}
                \begin{description}
                    \item[Language] Written in \lstinline{C++23} 
                \end{description}
            \end{minipage}
        \end{center}
    \end{multicols}
    
\end{frame}

\subsection{Prismatic Meshes}

\begin{frame}
    \frametitle{Prismatic Meshes}

    % [!]
    
\end{frame}

\begin{frame} % Meshes.
    \frametitle{Building a Polygonal Mesh}

    \setbeamercovered{transparent}

    \vspace*{\fill}
    \begin{multicols}{2}
        
        \vspace*{\fill}
        \begin{center}
            {\color{\accentcolor} \Large \textbf{Mesh Building Steps}}
            \vspace*{0.5cm}

            \begin{minipage}{0.4\textwidth}
                \begin{enumerate}[<+>]
                    \item<2> Generation of $\nSpaceIndices$ random points
                    \item<3> Evaluation of the first Voronoi diagram
                    \item<4,5> Relaxation through Lloyd's algorithm
                    \item<6> Postprocessing
                \end{enumerate}
            \end{minipage}
        \end{center}
        \vspace*{\fill}

        \vfill\null
        \columnbreak

        \vspace*{\fill}
        \only<1>{
            \begin{figure}[!ht]
                \centering
                \begin{tikzpicture}[scale=4.0, line join=round]

	% Domain, filled.
    \draw[thick, color=\documentcolor, fill=white]
        (0,0) -- (1,0) -- (1,1) -- (0,1) -- cycle;

\end{tikzpicture}
            \end{figure}
        }

        \only<2>{
            \begin{figure}[!ht]
                \centering
                \begin{tikzpicture}[scale=4.0, line join=round]

	% Domain, filled.
    \draw[thick, color=\documentcolor, fill=white]
        (0,0) -- (1,0) -- (1,1) -- (0,1) -- cycle;
    
    % Points.
    \foreach \point in {
		(0.72127,0.389382),
		(0.33949,0.815996),
		(0.437955,0.7113),
		(0.822316,0.661117),
		(0.39474,0.389648),
		(0.817144,0.741499),
		(0.377311,0.459377),
		(0.745253,0.466636),
		(0.748747,0.186922),
		(0.594037,0.979896),
		(0.119892,0.0318298),
		(0.962932,0.000799264),
		(0.433235,0.380333),
		(0.264253,0.296984),
		(0.415665,0.0801278),
		(0.708583,0.152086),
		(0.106053,0.429443),
		(0.650318,0.896932),
		(0.736825,0.812514),
		(0.924622,0.119742),
		(0.500876,0.2207),
		(0.306865,0.484373),
		(0.848961,0.490905),
		(0.632636,0.712606),
		(0.774037,0.24235),
		(0.182251,0.0843225),
		(0.208095,0.444639),
		(0.0456983,0.0507024),
		(0.155496,0.423579),
		(0.0992755,0.523369),
		(0.266392,0.25296),
		(0.505792,0.851804),
		(0.272517,0.18911),
		(0.370514,0.228657),
		(0.042337,0.558294),
		(0.251592,0.508437),
		(0.295924,0.591539),
		(0.00221013,0.145576),
		(0.702387,0.0250446),
		(0.925336,0.124344),
		(0.841824,0.533834),
		(0.150118,0.035753),
		(0.900258,0.628199),
		(0.14044,0.373441),
		(0.427562,0.0399414),
		(0.295209,0.575902),
		(0.192981,0.428938),
		(0.162448,0.271859),
		(0.141587,0.65414),
		(0.134769,0.0650669),
		(0.57961,0.500233),
		(0.410215,0.475481),
		(0.408878,0.00823666),
		(0.433539,0.498153),
		(0.454675,0.729274),
		(0.909932,0.232016),
		(0.499707,0.567521),
		(0.321707,0.935148),
		(0.0288318,0.575444),
		(0.490847,0.671644),
		(0.319138,0.75213),
		(0.0513016,0.225936),
		(0.304792,0.646714),
		(0.324919,0.919322),
		(0.0447027,0.318903),
		(0.80667,0.695392),
		(0.44849,0.765687),
		(0.895921,0.744071),
		(0.599217,0.0481762),
		(0.697607,0.685196),
		(0.0940792,0.189337),
		(0.191543,0.265273),
		(0.450766,0.0189526),
		(0.536078,0.857182),
		(0.653675,0.317932),
		(0.48995,0.584697),
		(0.00416989,0.0832975),
		(0.981442,0.0930058),
		(0.148593,0.410725),
		(0.0581593,0.482716),
		(0.00634733,0.679611),
		(0.23032,0.980933),
		(0.549021,0.39758),
		(0.125327,0.378339),
		(0.751771,0.0143064),
		(0.448447,0.0498322),
		(0.529053,0.792066),
		(0.247601,0.427272),
		(0.157711,0.655285),
		(0.375021,0.984985),
		(0.63643,0.480752),
		(0.993462,0.11164),
		(0.328844,0.881809),
		(0.566091,0.297219),
		(0.359952,0.70537),
		(0.145231,0.893672),
		(0.94176,0.166414),
		(0.91373,0.063751),
		(0.46366,0.730053),
		(0.000622743,0.466438),
		(0.430218,0.665531),
		(0.580925,0.609608),
		(0.688652,0.176602),
		(0.15181,0.462417),
		(0.850529,0.83329),
		(0.0968576,0.886316),
		(0.320572,0.848394),
		(0.959553,0.205808),
		(0.0153658,0.252224),
		(0.129501,0.518222),
		(0.751421,0.133792),
		(0.637494,0.367684),
		(0.662129,0.407357),
		(0.445851,0.419898),
		(0.217669,0.356239),
		(0.304413,0.262017),
		(0.718117,0.386713),
		(0.477631,0.535993),
		(0.435177,0.0267962),
		(0.364522,0.51288),
		(0.974263,0.433769),
		(0.358799,0.335575),
		(0.0118437,0.0575584),
		(0.384258,0.219853),
		(0.0651194,0.461297)
    } {
        \draw[color=\accentcolor, fill=\accentcolor] \point circle (0.005);
    }

	% Domain.
    \draw[thick, color=\documentcolor]
        (0,0) -- (1,0) -- (1,1) -- (0,1) -- cycle;

\end{tikzpicture}
            \end{figure}
        }

        \only<3>{
            \begin{figure}[!ht]
                \centering
                \begin{tikzpicture}[scale=4.0, line join=round]

	% Domain, filled.
    \draw[thick, color=\documentcolor, fill=white]
        (0,0) -- (1,0) -- (1,1) -- (0,1) -- cycle;

	% Points.
    \foreach \point in {
		(0.72127,0.389382),
		(0.33949,0.815996),
		(0.437955,0.7113),
		(0.822316,0.661117),
		(0.39474,0.389648),
		(0.817144,0.741499),
		(0.377311,0.459377),
		(0.745253,0.466636),
		(0.748747,0.186922),
		(0.594037,0.979896),
		(0.119892,0.0318298),
		(0.962932,0.000799264),
		(0.433235,0.380333),
		(0.264253,0.296984),
		(0.415665,0.0801278),
		(0.708583,0.152086),
		(0.106053,0.429443),
		(0.650318,0.896932),
		(0.736825,0.812514),
		(0.924622,0.119742),
		(0.500876,0.2207),
		(0.306865,0.484373),
		(0.848961,0.490905),
		(0.632636,0.712606),
		(0.774037,0.24235),
		(0.182251,0.0843225),
		(0.208095,0.444639),
		(0.0456983,0.0507024),
		(0.155496,0.423579),
		(0.0992755,0.523369),
		(0.266392,0.25296),
		(0.505792,0.851804),
		(0.272517,0.18911),
		(0.370514,0.228657),
		(0.042337,0.558294),
		(0.251592,0.508437),
		(0.295924,0.591539),
		(0.00221013,0.145576),
		(0.702387,0.0250446),
		(0.925336,0.124344),
		(0.841824,0.533834),
		(0.150118,0.035753),
		(0.900258,0.628199),
		(0.14044,0.373441),
		(0.427562,0.0399414),
		(0.295209,0.575902),
		(0.192981,0.428938),
		(0.162448,0.271859),
		(0.141587,0.65414),
		(0.134769,0.0650669),
		(0.57961,0.500233),
		(0.410215,0.475481),
		(0.408878,0.00823666),
		(0.433539,0.498153),
		(0.454675,0.729274),
		(0.909932,0.232016),
		(0.499707,0.567521),
		(0.321707,0.935148),
		(0.0288318,0.575444),
		(0.490847,0.671644),
		(0.319138,0.75213),
		(0.0513016,0.225936),
		(0.304792,0.646714),
		(0.324919,0.919322),
		(0.0447027,0.318903),
		(0.80667,0.695392),
		(0.44849,0.765687),
		(0.895921,0.744071),
		(0.599217,0.0481762),
		(0.697607,0.685196),
		(0.0940792,0.189337),
		(0.191543,0.265273),
		(0.450766,0.0189526),
		(0.536078,0.857182),
		(0.653675,0.317932),
		(0.48995,0.584697),
		(0.00416989,0.0832975),
		(0.981442,0.0930058),
		(0.148593,0.410725),
		(0.0581593,0.482716),
		(0.00634733,0.679611),
		(0.23032,0.980933),
		(0.549021,0.39758),
		(0.125327,0.378339),
		(0.751771,0.0143064),
		(0.448447,0.0498322),
		(0.529053,0.792066),
		(0.247601,0.427272),
		(0.157711,0.655285),
		(0.375021,0.984985),
		(0.63643,0.480752),
		(0.993462,0.11164),
		(0.328844,0.881809),
		(0.566091,0.297219),
		(0.359952,0.70537),
		(0.145231,0.893672),
		(0.94176,0.166414),
		(0.91373,0.063751),
		(0.46366,0.730053),
		(0.000622743,0.466438),
		(0.430218,0.665531),
		(0.580925,0.609608),
		(0.688652,0.176602),
		(0.15181,0.462417),
		(0.850529,0.83329),
		(0.0968576,0.886316),
		(0.320572,0.848394),
		(0.959553,0.205808),
		(0.0153658,0.252224),
		(0.129501,0.518222),
		(0.751421,0.133792),
		(0.637494,0.367684),
		(0.662129,0.407357),
		(0.445851,0.419898),
		(0.217669,0.356239),
		(0.304413,0.262017),
		(0.718117,0.386713),
		(0.477631,0.535993),
		(0.435177,0.0267962),
		(0.364522,0.51288),
		(0.974263,0.433769),
		(0.358799,0.335575),
		(0.0118437,0.0575584),
		(0.384258,0.219853),
		(0.0651194,0.461297)
    } {
        \draw[color=\documentcolor, fill=\documentcolor, opacity=0.25] \point circle (0.0025);
    }
    
    % Cells.
    \foreach \polygon in {
		{(0.81491768079246,0.40265956226011)--(0.70351498956895,0.43724333517807)--(0.69678772190095,0.41510946443084)--(0.77306818472213,0.32498684923871)--(0.85078247275959,0.35287708055722)--(0.85268479062871,0.35515772932871)},
		{(0.2764530376866,0.80090852189881)--(0.38296955216628,0.76696413610026)--(0.41984917863072,0.84686746152159)--(0.41654175438734,0.86222777776676)--(0.36802123052607,0.85437879519475)},
		{(0.45895723675566,0.68421102720186)--(0.46831006710122,0.69668552928355)--(0.46117917728194,0.7064599270067)--(0.42241044350594,0.74252466591406)--(0.39596435397984,0.74764722511405)--(0.40003004557943,0.69417309111533)},
		{(0.86207132279396,0.69997381902801)--(0.75576986043882,0.65144772028038)--(0.74270409743836,0.58377890553351)--(0.84200246120739,0.59899793378281)--(0.87849436619317,0.68540158994539)},
		{(0.40570204902276,0.42943083535428)--(0.32112698747207,0.40829080504919)--(0.31922655012417,0.40085855998477)--(0.40416601277261,0.34440199183953)--(0.41921551498506,0.40659750714898)},
		{(0.79776743801987,0.80051273443099)--(0.74462848957745,0.74041138271774)--(0.74815279193124,0.73292892754748)--(0.85766648697157,0.70804992032895)--(0.85533151005542,0.779576506393)},
		{(0.34494256511528,0.47991983192898)--(0.32046944036361,0.41094741532368)--(0.32112698747207,0.40829080504919)--(0.40570204902276,0.42943083535428)--(0.40930151102783,0.43567991711595)--(0.38317611490883,0.48905895833332)},
		{(0.70351498956895,0.43724333517807)--(0.81491768079246,0.40265956226011)--(0.79134723887928,0.50338394330796)--(0.73831490564384,0.57959692011435)--(0.7035969044932,0.57202846998949)--(0.68879147678988,0.45788928555936)},
		{(0.7188640385544,0.18080352400507)--(0.737162565826,0.15970634748151)--(0.84032518849665,0.16489902681664)--(0.84115428013916,0.16723863882903)--(0.83762513083964,0.17985316987508)--(0.70894486827078,0.23856606830321)},
		{(0.71296155912231,1.0)--(0.484937150244,1.0)--(0.48403655852647,0.96123470828532)--(0.50448256867031,0.94714915240117)--(0.58145962342767,0.91079214595781)},
		{(0.072298550323714,1.6371245993997e-18)--(0.13939129777301,-8.7204753747234e-19)--(0.13345887328034,0.045705446797322)--(0.08898776827222,0.065610427782656)},
		{(0.89703652004504,-8.496744737258e-18)--(1.0,0.0)--(1.0,0.041319311713219)--(0.96014035979164,0.049320796215747)},
		{(0.41921551498506,0.40659750714898)--(0.40416601277261,0.34440199183953)--(0.43576762292633,0.29184645561424)--(0.44206071785289,0.28992586066053)--(0.4788564283864,0.3055171813609)--(0.49880644287549,0.33740678480698)--(0.49195536692368,0.38340230861383)},
		{(0.29119050723099,0.3661008405474)--(0.21646168418738,0.30735106496358)--(0.23130398767021,0.27331924495965)--(0.28105561393038,0.27573692899187)--(0.3130676881052,0.3125024910612)},
		{(0.4920735723557,0.12992343904819)--(0.44300721060946,0.15966608633375)--(0.34667451980727,0.1380125597604)--(0.29883092082001,0.075169912265488)--(0.29846712756044,0.054926375855413)--(0.37875600533919,0.047346392987191)--(0.42969744871829,0.062427885172013)},
		{(0.737162565826,0.15970634748151)--(0.7188640385544,0.18080352400507)--(0.63896570096093,0.11584961088955)--(0.66291659040003,0.090641176042336)--(0.70675525986754,0.088503281622324)},
		{(0.043813712627158,0.39169134740803)--(0.059808102573127,0.38281439801307)--(0.12109529190628,0.40592995387427)--(0.131004491351,0.4284508334431)--(0.13249153377149,0.44098975235087)--(0.10830029870213,0.47455859102972)},
		{(0.79330168747307,1.0)--(0.71296155912231,1.0)--(0.58145962342767,0.91079214595781)--(0.61451847493351,0.81578255166941)--(0.62246536258095,0.80659294065356)--(0.64453713193874,0.8044756189346)--(0.77299202284732,0.93610866178105)},
		{(0.77299202284732,0.93610866178105)--(0.64453713193874,0.8044756189346)--(0.68976355041114,0.75731130982598)--(0.74462848957745,0.74041138271774)--(0.79776743801987,0.80051273443099)},
		{(0.95799344194723,0.1169182909266)--(0.8387242872409,0.13543084779993)--(0.8364854829402,0.10783187604364)--(0.94388723277738,0.08693946775931)},
		{(0.4788564283864,0.3055171813609)--(0.44206071785289,0.28992586066053)--(0.44300721060946,0.15966608633375)--(0.4920735723557,0.12992343904819)--(0.52461137137799,0.11993960238694)--(0.58426459890015,0.15394297481185)--(0.59634516718861,0.20538395263939)},
		{(0.29349685398467,0.52917725146405)--(0.26643818854523,0.46702608858776)--(0.32046944036361,0.41094741532368)--(0.34494256511528,0.47991983192898)--(0.3189830015933,0.53242291831198)},
		{(0.94172921464231,0.52838594492475)--(0.79134723887928,0.50338394330796)--(0.81491768079246,0.40265956226011)--(0.85268479062871,0.35515772932871)--(0.86394986511225,0.35781078202526)},
		{(0.64176682088534,0.64354227744501)--(0.68976355041114,0.75731130982598)--(0.64453713193874,0.8044756189346)--(0.62246536258095,0.80659294065356)--(0.55433173261818,0.71777408333191)--(0.564582010025,0.68229342807613)},
		{(0.85078247275959,0.35287708055722)--(0.77306818472213,0.32498684923871)--(0.73203624977127,0.3090926198256)--(0.69726977440267,0.25372811877331)--(0.70894486827078,0.23856606830321)--(0.83762513083964,0.17985316987508)},
		{(0.2569340786675,-1.0136976840466e-17)--(0.28002663007953,-7.0148966784867e-18)--(0.29846712756044,0.054926375855413)--(0.29883092082001,0.075169912265488)--(0.18327949068838,0.17470850445308)--(0.13454432134724,0.133790045355)--(0.16381928133599,0.061602389221726)},
		{(0.18069712966087,0.4558871258114)--(0.21998466792267,0.41806869067827)--(0.242033983457,0.46822615943871)--(0.19477277146027,0.50044901219203)},
		{(0.017808859329408,-3.2242362240228e-19)--(0.072298550323714,-3.6027122931711e-19)--(0.08898776827222,0.065610427782656)--(0.080828852918195,0.11620195032545)--(0.070910661890026,0.11966321193268)--(0.063660301032814,0.11633980642789)--(0.033617789882254,0.078063609930191)},
		{(0.13249153377149,0.44098975235087)--(0.131004491351,0.4284508334431)--(0.17749395744838,0.40348625379436)--(0.17160187849493,0.44470222688338)},
		{(0.10355139017352,0.59421629322679)--(0.059530171783853,0.52244826813876)--(0.097789187353091,0.48375357888545)--(0.10719789273958,0.47857629032918)--(0.12567318042764,0.58705868273281)},
		{(0.28105561393038,0.27573692899187)--(0.23130398767021,0.27331924495965)--(0.22195346894011,0.21647883352707)--(0.2935353921711,0.22334503921447)},
		{(0.41654175438734,0.86222777776676)--(0.41984917863072,0.84686746152159)--(0.47950651139632,0.80717140096562)--(0.5261151507378,0.82531954984477)--(0.50448256867031,0.94714915240117)--(0.48403655852647,0.96123470828532)--(0.42320109421102,0.90149989587471)},
		{(0.18327949068838,0.17470850445308)--(0.29883092082001,0.075169912265488)--(0.34667451980727,0.1380125597604)--(0.33930259068455,0.16480732282195)--(0.32042710811992,0.2115802793121)--(0.2935353921711,0.22334503921447)--(0.22195346894011,0.21647883352707)--(0.18328032494326,0.17536328360184)},
		{(0.41821200434575,0.28798425564766)--(0.35551949547956,0.28111506769104)--(0.32042710811992,0.2115802793121)--(0.33930259068455,0.16480732282195)},
		{(-9.394099314051e-18,0.53884721269974)--(-9.5303926761132e-18,0.5221209362159)--(0.018290389757216,0.51381476588416)--(0.059530171783853,0.52244826813876)--(0.10355139017352,0.59421629322679)--(0.089192814691241,0.60908483270704)},
		{(0.19029265883876,0.51016314984628)--(0.19477277146027,0.50044901219203)--(0.242033983457,0.46822615943871)--(0.26643818854523,0.46702608858776)--(0.29349685398467,0.52917725146405)--(0.20834980476389,0.58422504544386)--(0.19557369936106,0.57605722853877)},
		{(0.21030919549961,0.58761888249994)--(0.36282951466794,0.58064565412964)--(0.38081463379964,0.59633021488474)--(0.37485167082317,0.60715289018846)--(0.23005311769621,0.63042739153665)},
		{(1.7367258949333e-18,0.19998426193243)--(1.7604470024043e-18,0.11433657409414)--(0.063660301032814,0.11633980642789)--(0.070910661890026,0.11966321193268)--(0.044628961597033,0.17483750654607)--(0.0043421685759054,0.19944862936459)},
		{(0.64259407254568,4.7111732237183e-18)--(0.72280082790447,4.0875614737244e-18)--(0.73889433232175,0.074011891053365)--(0.70675525986754,0.088503281622324)--(0.66291659040002,0.090641176042336)},
		{(0.86779285979695,0.17104959334438)--(0.84115428013916,0.16723863882903)--(0.84032518849665,0.16489902681664)--(0.8387242872409,0.13543084779993)--(0.95799344194723,0.1169182909266)--(0.95980288793588,0.120157844361)--(0.96240515495177,0.13411279303362)},
		{(0.84200246120739,0.59899793378281)--(0.74270409743836,0.58377890553351)--(0.73831490564384,0.57959692011435)--(0.79134723887928,0.50338394330796)--(0.94172921464231,0.52838594492475)--(0.94628875425442,0.53442062878311)},
		{(0.13939129777301,-8.7204753747234e-19)--(0.2569340786675,-2.4798149692361e-18)--(0.16381928133599,0.061602389221726)--(0.13345887328034,0.045705446797322)},
		{(1.0,0.55486457332503)--(1.0,0.68994896844859)--(0.87849436619317,0.68540158994539)--(0.84200246120739,0.59899793378281)--(0.94628875425442,0.53442062878311)},
		{(0.11292256040628,0.31430430903914)--(0.17058009573212,0.32679631765481)--(0.18323648816915,0.38361563383571)--(0.13855465034859,0.39338706492243)},
		{(0.42969744871829,0.062427885172013)--(0.37875600533919,0.047346392987191)--(0.41679787120389,0.024927156110052)--(0.44085670963409,0.038864627170034)},
		{(0.20834980476389,0.58422504544386)--(0.29349685398467,0.52917725146405)--(0.3189830015933,0.53242291831198)--(0.36282951466794,0.58064565412964)--(0.21030919549961,0.58761888249994)},
		{(0.2193526821086,0.39735212683294)--(0.21998466792267,0.41806869067827)--(0.18069712966087,0.4558871258114)--(0.17160187849493,0.44470222688338)--(0.17749395744838,0.40348625379436)--(0.1848223085064,0.38562605531908)},
		{(0.17058009573212,0.32679631765481)--(0.11292256040628,0.31430430903914)--(0.1108469391314,0.31358070639375)--(0.095755799344478,0.27580914374914)--(0.10723865242546,0.24801765073202)--(0.16206250398896,0.20259631064577)--(0.18764849573858,0.31562629180098)},
		{(0.092461778959338,0.76507283097678)--(0.068631366637277,0.63854518974231)--(0.089192814691241,0.60908483270704)--(0.10355139017352,0.59421629322679)--(0.12567318042763,0.58705868273281)--(0.15463530529741,0.58448326217391)--(0.14118446745633,0.7739398452686)--(0.13868496668193,0.77397786702595)},
		{(0.080828852918195,0.11620195032545)--(0.08898776827222,0.065610427782656)--(0.13345887328034,0.045705446797322)--(0.16381928133599,0.061602389221726)--(0.13454432134724,0.133790045355)},
		{(0.55760904079878,0.55519305264472)--(0.53021629682798,0.52266476711548)--(0.51002476855339,0.46508386857987)--(0.59103258681144,0.44094505268204)--(0.62990460023665,0.55432353818116)},
		{(0.38317611490883,0.48905895833332)--(0.40930151102783,0.43567991711595)--(0.44767090932061,0.46027995257834)--(0.39987125310726,0.50945658451621)},
		{(0.28002663007953,-4.3683120751993e-19)--(0.4332995415707,-7.646196514654e-19)--(0.43268008140293,0.0024214350371052)--(0.41679787120389,0.024927156110052)--(0.37875600533919,0.047346392987191)--(0.29846712756044,0.054926375855413)},
		{(0.39987125310726,0.50945658451621)--(0.44767090932061,0.46027995257834)--(0.49758667295564,0.46813302913076)--(0.41226668106765,0.56754699278944)},
		{(0.42241044350594,0.74252466591406)--(0.46117917728194,0.7064599270067)--(0.45753532695744,0.74849164711832)},
		{(0.86394986511225,0.35781078202526)--(0.85268479062871,0.35515772932871)--(0.85078247275959,0.35287708055722)--(0.83762513083964,0.17985316987508)--(0.84115428013916,0.16723863882903)--(0.86779285979695,0.17104959334438)--(0.92381970374935,0.19823176928504)--(0.98732709074017,0.31847099556908)},
		{(0.53495786093482,0.59890419023145)--(0.47222604353831,0.56327008741812)--(0.53021629682798,0.52266476711548)--(0.55760904079878,0.55519305264472)},
		{(0.31103514203778,1.0)--(0.29703528655886,1.0)--(0.25347854970964,0.91306121080099)--(0.37016545353716,0.93674400032224)},
		{(2.1443228462075e-17,0.62373115952369)--(1.2411893402441e-17,0.53884721269974)--(0.089192814691241,0.60908483270704)--(0.068631366637277,0.63854518974231)},
		{(0.46831006710122,0.69668552928355)--(0.45895723675566,0.68421102720186)--(0.46458047899867,0.62843705505097)--(0.52704763526984,0.62779212997707)--(0.564582010025,0.68229342807613)--(0.55433173261818,0.71777408333191)--(0.53011937586391,0.72545580183932)},
		{(0.38299954472878,0.76667795757367)--(0.38296955216628,0.76696413610026)--(0.2764530376866,0.80090852189881)--(0.21502946869708,0.80182333182705)--(0.19461191790395,0.77673701025447)--(0.2346864680375,0.70993863218517)--(0.30675585720203,0.70013106808951)},
		{(0.0043421685759053,0.19944862936459)--(0.044628961597033,0.17483750654607)--(0.10723865242546,0.24801765073202)--(0.095755799344478,0.27580914374914)--(0.058254947967382,0.27314731773162)},
		{(0.30675585720203,0.70013106808951)--(0.2346864680375,0.70993863218517)--(0.23005311769621,0.63042739153665)--(0.37485167082317,0.60715289018846)--(0.36979654915227,0.64084803401827)},
		{(0.40664343870089,0.90891046552274)--(0.37016545353716,0.93674400032224)--(0.25347854970964,0.91306121080099)--(0.23937785203937,0.89141065837622)},
		{(1.5536812770338e-18,0.38589973132066)--(-8.1135090646447e-19,0.29877786909705)--(0.058254947967382,0.27314731773162)--(0.095755799344478,0.27580914374914)--(0.1108469391314,0.31358070639375)--(0.059808102573128,0.38281439801307)--(0.043813712627158,0.39169134740803)--(0.027306298622061,0.39405820305984)},
		{(0.75576986043882,0.65144772028038)--(0.86207132279396,0.69997381902801)--(0.85766648697157,0.70804992032895)--(0.74815279193124,0.73292892754748)},
		{(0.41984917863072,0.84686746152159)--(0.38296955216628,0.76696413610026)--(0.38299954472878,0.76667795757367)--(0.39596435397984,0.74764722511405)--(0.42241044350594,0.74252466591406)--(0.45753532695744,0.74849164711832)--(0.49368151695606,0.7638799636651)--(0.47950651139632,0.80717140096562)},
		{(1.0,0.68994896844859)--(1.0,0.85318059068519)--(0.85533151005542,0.779576506393)--(0.85766648697157,0.70804992032895)--(0.86207132279396,0.69997381902801)--(0.87849436619317,0.68540158994539)},
		{(0.53159893867848,1.3774596629645e-18)--(0.64259407254568,5.4568986089327e-19)--(0.66291659040002,0.090641176042336)--(0.63896570096093,0.11584961088955)--(0.58426459890015,0.15394297481185)--(0.52461137137799,0.11993960238694)--(0.52373291894678,0.039958237609066)},
		{(0.7035969044932,0.57202846998949)--(0.73831490564384,0.57959692011435)--(0.74270409743836,0.58377890553351)--(0.75576986043882,0.65144772028038)--(0.74815279193124,0.73292892754748)--(0.74462848957745,0.74041138271774)--(0.68976355041114,0.75731130982598)--(0.64176682088534,0.64354227744501)--(0.68436642667714,0.57778291811654)},
		{(0.044628961597033,0.17483750654607)--(0.070910661890026,0.11966321193268)--(0.080828852918195,0.11620195032545)--(0.13454432134724,0.133790045355)--(0.18327949068838,0.17470850445308)--(0.18328032494326,0.17536328360184)--(0.16206250398896,0.20259631064577)--(0.10723865242546,0.24801765073202)},
		{(0.18328032494326,0.17536328360184)--(0.22195346894011,0.21647883352707)--(0.23130398767021,0.27331924495965)--(0.21646168418738,0.30735106496358)--(0.18764849573858,0.31562629180098)--(0.16206250398896,0.20259631064577)},
		{(0.4332995415707,9.1598126109942e-19)--(0.53159893867848,-2.4204540540085e-18)--(0.52373291894678,0.039958237609067)--(0.4487341692433,0.034326879481829)--(0.43268008140293,0.0024214350371053)},
		{(0.61451847493351,0.81578255166941)--(0.58145962342767,0.91079214595781)--(0.50448256867031,0.94714915240117)--(0.5261151507378,0.82531954984477)},
		{(0.68304668799349,0.3549919540442)--(0.60469588336363,0.3295097372055)--(0.62674582044354,0.23627455413941)--(0.69726977440267,0.25372811877331)--(0.73203624977127,0.3090926198256)},
		{(0.52704763526984,0.62779212997707)--(0.46458047899867,0.62843705505097)--(0.40629265276847,0.58536531422387)--(0.40992026943217,0.57902969949995)--(0.47222604353831,0.56327008741812)--(0.53495786093482,0.59890419023145)},
		{(-2.7636133040335e-17,0.11433657409414)--(-2.7709542413129e-17,0.068040822537518)--(0.033617789882254,0.078063609930191)--(0.063660301032814,0.11633980642789)},
		{(1.0,0.041319311713219)--(1.0,0.094228477320069)--(0.95980288793588,0.120157844361)--(0.95799344194723,0.1169182909266)--(0.94388723277738,0.08693946775931)--(0.96014035979164,0.049320796215747)},
		{(0.131004491351,0.4284508334431)--(0.12109529190628,0.40592995387427)--(0.13855465034859,0.39338706492243)--(0.18323648816915,0.38361563383571)--(0.1848223085064,0.38562605531908)--(0.17749395744838,0.40348625379436)},
		{(0.097789187353091,0.48375357888545)--(0.059530171783853,0.52244826813876)--(0.018290389757216,0.51381476588416)--(0.032772071952025,0.46262600750549)},
		{(-1.0199133931515e-18,0.80555925127614)--(1.0951678121777e-18,0.62373115952369)--(0.068631366637277,0.63854518974231)--(0.092461778959338,0.76507283097678)},
		{(0.29703528655886,1.0)--(0.12347670997767,1.0)--(0.23713019387941,0.88917660240315)--(0.23937785203937,0.89141065837622)--(0.25347854970964,0.91306121080099)},
		{(0.59103258681144,0.44094505268204)--(0.51002476855339,0.46508386857987)--(0.50957439155503,0.46485157361713)--(0.49195536692368,0.38340230861383)--(0.49880644287549,0.33740678480698)--(0.58280339373661,0.35169371245741)--(0.60443154695339,0.41569923038643)--(0.6033701017334,0.42797899736636)},
		{(0.12109529190628,0.40592995387427)--(0.059808102573128,0.38281439801307)--(0.1108469391314,0.31358070639375)--(0.11292256040628,0.31430430903914)--(0.13855465034859,0.39338706492243)},
		{(0.72280082790447,-4.0831101409703e-18)--(0.84466567680621,-4.5407790522216e-18)--(0.82199627575078,0.074255235773584)--(0.73889433232175,0.074011891053365)},
		{(0.52373291894678,0.039958237609067)--(0.52461137137799,0.11993960238694)--(0.4920735723557,0.12992343904819)--(0.42969744871829,0.062427885172013)--(0.44085670963409,0.038864627170034)--(0.4487341692433,0.034326879481829)},
		{(0.55433173261818,0.71777408333191)--(0.62246536258095,0.80659294065356)--(0.61451847493351,0.81578255166941)--(0.5261151507378,0.82531954984477)--(0.47950651139632,0.80717140096562)--(0.49368151695606,0.7638799636651)--(0.53011937586391,0.72545580183932)},
		{(0.31922655012417,0.40085855998477)--(0.32112698747207,0.40829080504919)--(0.32046944036361,0.41094741532368)--(0.26643818854523,0.46702608858776)--(0.242033983457,0.46822615943871)--(0.21998466792267,0.41806869067827)--(0.2193526821086,0.39735212683294)--(0.29132564916446,0.36702384870379)},
		{(0.14118446745633,0.7739398452686)--(0.15463530529741,0.58448326217391)--(0.19557369936106,0.57605722853877)--(0.20834980476389,0.58422504544386)--(0.21030919549961,0.58761888249994)--(0.23005311769621,0.63042739153665)--(0.2346864680375,0.70993863218517)--(0.19461191790395,0.77673701025447)},
		{(0.484937150244,1.0)--(0.31103514203778,1.0)--(0.37016545353716,0.93674400032224)--(0.40664343870089,0.90891046552274)--(0.42320109421102,0.90149989587471)--(0.48403655852647,0.96123470828532)},
		{(0.68879147678988,0.45788928555936)--(0.7035969044932,0.57202846998949)--(0.68436642667714,0.57778291811654)--(0.62990460023665,0.55432353818116)--(0.59103258681144,0.44094505268204)--(0.6033701017334,0.42797899736636)},
		{(1.0,0.094228477320069)--(1.0,0.16718329044542)--(0.99586260213138,0.16569346467065)--(0.96240515495177,0.13411279303362)--(0.95980288793588,0.120157844361)},
		{(0.36802123052607,0.85437879519475)--(0.41654175438734,0.86222777776676)--(0.42320109421102,0.90149989587471)--(0.40664343870089,0.90891046552274)--(0.23937785203937,0.89141065837622)--(0.23713019387941,0.88917660240315)--(0.23697796446859,0.88682035623598)},
		{(0.49880644287549,0.33740678480698)--(0.4788564283864,0.3055171813609)--(0.59634516718861,0.20538395263939)--(0.62674582044354,0.23627455413941)--(0.60469588336363,0.3295097372055)--(0.58280339373661,0.35169371245741)},
		{(0.40003004557943,0.69417309111533)--(0.39596435397984,0.74764722511405)--(0.38299954472878,0.76667795757367)--(0.30675585720203,0.70013106808951)--(0.36979654915227,0.64084803401827)},
		{(0.12347670997767,1.0)--(0.1043173504961,1.0)--(0.13868496668193,0.77397786702595)--(0.14118446745633,0.7739398452686)--(0.19461191790395,0.77673701025447)--(0.21502946869708,0.80182333182705)--(0.23697796446859,0.88682035623598)--(0.23713019387941,0.88917660240315)},
		{(0.92381970374935,0.19823176928504)--(0.86779285979695,0.17104959334438)--(0.96240515495177,0.13411279303362)--(0.99586260213138,0.16569346467065)},
		{(0.84466567680621,-2.4450320766493e-17)--(0.89703652004504,-3.4134829785379e-17)--(0.96014035979164,0.049320796215747)--(0.94388723277738,0.08693946775931)--(0.8364854829402,0.10783187604364)--(0.82199627575078,0.074255235773584)},
		{(0.46117917728194,0.7064599270067)--(0.46831006710122,0.69668552928355)--(0.53011937586391,0.72545580183932)--(0.49368151695606,0.7638799636651)--(0.45753532695744,0.74849164711832)},
		{(-9.5303926761132e-18,0.5221209362159)--(-1.1743299100151e-17,0.38589973132066)--(0.027306298622061,0.39405820305984)--(0.032772071952025,0.46262600750549)--(0.018290389757216,0.51381476588416)},
		{(0.45895723675566,0.68421102720186)--(0.40003004557943,0.69417309111533)--(0.36979654915227,0.64084803401827)--(0.37485167082317,0.60715289018846)--(0.38081463379964,0.59633021488474)--(0.40629265276847,0.58536531422387)--(0.46458047899867,0.62843705505097)},
		{(0.53495786093482,0.59890419023145)--(0.55760904079878,0.55519305264472)--(0.62990460023665,0.55432353818116)--(0.68436642667714,0.57778291811654)--(0.64176682088534,0.64354227744501)--(0.564582010025,0.68229342807613)--(0.52704763526984,0.62779212997707)},
		{(0.7188640385544,0.18080352400507)--(0.70894486827078,0.23856606830321)--(0.69726977440267,0.25372811877331)--(0.62674582044354,0.23627455413941)--(0.59634516718861,0.20538395263939)--(0.58426459890015,0.15394297481185)--(0.63896570096093,0.11584961088955)},
		{(0.10830029870213,0.47455859102972)--(0.13249153377149,0.44098975235087)--(0.17160187849493,0.44470222688338)--(0.18069712966087,0.4558871258114)--(0.19477277146027,0.50044901219203)--(0.19029265883876,0.51016314984628)--(0.10826395864066,0.4773705751988)},
		{(1.0,0.85318059068519)--(1.0,1.0)--(0.79330168747307,1.0)--(0.77299202284732,0.93610866178105)--(0.79776743801987,0.80051273443099)--(0.85533151005542,0.779576506393)},
		{(0.1043173504961,1.0)--(0.0,1.0)--(-1.0199133931515e-18,0.80555925127614)--(0.092461778959338,0.76507283097678)--(0.13868496668193,0.77397786702595)},
		{(0.21502946869708,0.80182333182705)--(0.2764530376866,0.80090852189881)--(0.36802123052607,0.85437879519475)--(0.23697796446859,0.88682035623598)},
		{(1.0,0.16718329044542)--(1.0,0.31765324128184)--(0.98732709074017,0.31847099556908)--(0.92381970374935,0.19823176928504)--(0.99586260213138,0.16569346467065)},
		{(-1.9865368616423e-17,0.29877786909705)--(-1.9906412481647e-17,0.19998426193243)--(0.0043421685759053,0.19944862936459)--(0.058254947967382,0.27314731773162)},
		{(0.12567318042764,0.58705868273281)--(0.10719789273958,0.47857629032918)--(0.10826395864066,0.4773705751988)--(0.19029265883876,0.51016314984628)--(0.19557369936106,0.57605722853877)--(0.15463530529741,0.58448326217391)},
		{(0.8364854829402,0.10783187604364)--(0.8387242872409,0.13543084779993)--(0.84032518849665,0.16489902681664)--(0.737162565826,0.15970634748151)--(0.70675525986754,0.088503281622324)--(0.73889433232175,0.074011891053365)--(0.82199627575078,0.074255235773584)},
		{(0.67975813505523,0.3689251492507)--(0.60443154695339,0.41569923038643)--(0.58280339373661,0.35169371245741)--(0.60469588336363,0.3295097372055)--(0.68304668799349,0.3549919540442)},
		{(0.69678772190095,0.41510946443084)--(0.70351498956895,0.43724333517807)--(0.68879147678988,0.45788928555936)--(0.6033701017334,0.42797899736636)--(0.60443154695339,0.41569923038643)--(0.67975813505523,0.3689251492507)},
		{(0.40930151102783,0.43567991711595)--(0.40570204902276,0.42943083535428)--(0.41921551498506,0.40659750714898)--(0.49195536692368,0.38340230861383)--(0.50957439155503,0.46485157361713)--(0.49758667295564,0.46813302913076)--(0.44767090932061,0.46027995257834)},
		{(0.2193526821086,0.39735212683294)--(0.1848223085064,0.38562605531908)--(0.18323648816915,0.38361563383571)--(0.17058009573212,0.32679631765481)--(0.18764849573858,0.31562629180098)--(0.21646168418738,0.30735106496358)--(0.29119050723099,0.3661008405474)--(0.29132564916446,0.36702384870379)},
		{(0.3130676881052,0.3125024910612)--(0.28105561393038,0.27573692899187)--(0.2935353921711,0.22334503921447)--(0.32042710811992,0.2115802793121)--(0.35551949547956,0.28111506769104)},
		{(0.77306818472213,0.32498684923871)--(0.69678772190095,0.41510946443084)--(0.67975813505523,0.3689251492507)--(0.68304668799349,0.3549919540442)--(0.73203624977127,0.3090926198256)},
		{(0.41226668106765,0.56754699278944)--(0.49758667295564,0.46813302913076)--(0.50957439155503,0.46485157361713)--(0.51002476855339,0.46508386857987)--(0.53021629682798,0.52266476711548)--(0.47222604353831,0.56327008741812)--(0.40992026943217,0.57902969949995)},
		{(0.44085670963409,0.038864627170034)--(0.41679787120389,0.024927156110052)--(0.43268008140293,0.0024214350371052)--(0.4487341692433,0.034326879481829)},
		{(0.38081463379964,0.59633021488474)--(0.36282951466794,0.58064565412964)--(0.3189830015933,0.53242291831198)--(0.34494256511528,0.47991983192898)--(0.38317611490883,0.48905895833332)--(0.39987125310726,0.50945658451621)--(0.41226668106765,0.56754699278944)--(0.40992026943217,0.57902969949995)--(0.40629265276847,0.58536531422387)},
		{(1.0,0.31765324128184)--(1.0,0.55486457332503)--(0.94628875425442,0.53442062878311)--(0.94172921464231,0.52838594492475)--(0.86394986511225,0.35781078202526)--(0.98732709074017,0.31847099556908)},
		{(0.43576762292633,0.29184645561424)--(0.40416601277261,0.34440199183953)--(0.31922655012417,0.40085855998477)--(0.29132564916446,0.36702384870379)--(0.29119050723099,0.3661008405474)--(0.3130676881052,0.3125024910612)--(0.35551949547956,0.28111506769104)--(0.41821200434575,0.28798425564766)},
		{(0.0,0.0)--(0.017808859329408,3.4673591993011e-19)--(0.033617789882254,0.078063609930191)--(4.0474899696953e-17,0.068040822537518)},
		{(0.44206071785289,0.28992586066053)--(0.43576762292633,0.29184645561424)--(0.41821200434575,0.28798425564766)--(0.33930259068455,0.16480732282195)--(0.34667451980727,0.1380125597604)--(0.44300721060946,0.15966608633375)},
		{(0.027306298622061,0.39405820305984)--(0.043813712627158,0.39169134740803)--(0.10830029870213,0.47455859102972)--(0.10826395864066,0.4773705751988)--(0.10719789273958,0.47857629032918)--(0.097789187353091,0.48375357888545)--(0.032772071952025,0.46262600750549)}
    } {
        \draw[thick, color=\accentcolor] \polygon -- cycle;
    }

	% Domain.
    \draw[thick, color=\documentcolor]
        (0,0) -- (1,0) -- (1,1) -- (0,1) -- cycle;

\end{tikzpicture}
            \end{figure}
        }

        \only<4>{
            \begin{figure}[!ht]
                \centering
                \begin{tikzpicture}[scale=4.0, line join=round]

	% Domain, filled.
    \draw[thick, color=\documentcolor, fill=white]
        (0,0) -- (1,0) -- (1,1) -- (0,1) -- cycle;

	% Cells.
    \foreach \polygon in {
		{(0.81491768079246,0.40265956226011)--(0.70351498956895,0.43724333517807)--(0.69678772190095,0.41510946443084)--(0.77306818472213,0.32498684923871)--(0.85078247275959,0.35287708055722)--(0.85268479062871,0.35515772932871)},
		{(0.2764530376866,0.80090852189881)--(0.38296955216628,0.76696413610026)--(0.41984917863072,0.84686746152159)--(0.41654175438734,0.86222777776676)--(0.36802123052607,0.85437879519475)},
		{(0.45895723675566,0.68421102720186)--(0.46831006710122,0.69668552928355)--(0.46117917728194,0.7064599270067)--(0.42241044350594,0.74252466591406)--(0.39596435397984,0.74764722511405)--(0.40003004557943,0.69417309111533)},
		{(0.86207132279396,0.69997381902801)--(0.75576986043882,0.65144772028038)--(0.74270409743836,0.58377890553351)--(0.84200246120739,0.59899793378281)--(0.87849436619317,0.68540158994539)},
		{(0.40570204902276,0.42943083535428)--(0.32112698747207,0.40829080504919)--(0.31922655012417,0.40085855998477)--(0.40416601277261,0.34440199183953)--(0.41921551498506,0.40659750714898)},
		{(0.79776743801987,0.80051273443099)--(0.74462848957745,0.74041138271774)--(0.74815279193124,0.73292892754748)--(0.85766648697157,0.70804992032895)--(0.85533151005542,0.779576506393)},
		{(0.34494256511528,0.47991983192898)--(0.32046944036361,0.41094741532368)--(0.32112698747207,0.40829080504919)--(0.40570204902276,0.42943083535428)--(0.40930151102783,0.43567991711595)--(0.38317611490883,0.48905895833332)},
		{(0.70351498956895,0.43724333517807)--(0.81491768079246,0.40265956226011)--(0.79134723887928,0.50338394330796)--(0.73831490564384,0.57959692011435)--(0.7035969044932,0.57202846998949)--(0.68879147678988,0.45788928555936)},
		{(0.7188640385544,0.18080352400507)--(0.737162565826,0.15970634748151)--(0.84032518849665,0.16489902681664)--(0.84115428013916,0.16723863882903)--(0.83762513083964,0.17985316987508)--(0.70894486827078,0.23856606830321)},
		{(0.71296155912231,1.0)--(0.484937150244,1.0)--(0.48403655852647,0.96123470828532)--(0.50448256867031,0.94714915240117)--(0.58145962342767,0.91079214595781)},
		{(0.072298550323714,1.6371245993997e-18)--(0.13939129777301,-8.7204753747234e-19)--(0.13345887328034,0.045705446797322)--(0.08898776827222,0.065610427782656)},
		{(0.89703652004504,-8.496744737258e-18)--(1.0,0.0)--(1.0,0.041319311713219)--(0.96014035979164,0.049320796215747)},
		{(0.41921551498506,0.40659750714898)--(0.40416601277261,0.34440199183953)--(0.43576762292633,0.29184645561424)--(0.44206071785289,0.28992586066053)--(0.4788564283864,0.3055171813609)--(0.49880644287549,0.33740678480698)--(0.49195536692368,0.38340230861383)},
		{(0.29119050723099,0.3661008405474)--(0.21646168418738,0.30735106496358)--(0.23130398767021,0.27331924495965)--(0.28105561393038,0.27573692899187)--(0.3130676881052,0.3125024910612)},
		{(0.4920735723557,0.12992343904819)--(0.44300721060946,0.15966608633375)--(0.34667451980727,0.1380125597604)--(0.29883092082001,0.075169912265488)--(0.29846712756044,0.054926375855413)--(0.37875600533919,0.047346392987191)--(0.42969744871829,0.062427885172013)},
		{(0.737162565826,0.15970634748151)--(0.7188640385544,0.18080352400507)--(0.63896570096093,0.11584961088955)--(0.66291659040003,0.090641176042336)--(0.70675525986754,0.088503281622324)},
		{(0.043813712627158,0.39169134740803)--(0.059808102573127,0.38281439801307)--(0.12109529190628,0.40592995387427)--(0.131004491351,0.4284508334431)--(0.13249153377149,0.44098975235087)--(0.10830029870213,0.47455859102972)},
		{(0.79330168747307,1.0)--(0.71296155912231,1.0)--(0.58145962342767,0.91079214595781)--(0.61451847493351,0.81578255166941)--(0.62246536258095,0.80659294065356)--(0.64453713193874,0.8044756189346)--(0.77299202284732,0.93610866178105)},
		{(0.77299202284732,0.93610866178105)--(0.64453713193874,0.8044756189346)--(0.68976355041114,0.75731130982598)--(0.74462848957745,0.74041138271774)--(0.79776743801987,0.80051273443099)},
		{(0.95799344194723,0.1169182909266)--(0.8387242872409,0.13543084779993)--(0.8364854829402,0.10783187604364)--(0.94388723277738,0.08693946775931)},
		{(0.4788564283864,0.3055171813609)--(0.44206071785289,0.28992586066053)--(0.44300721060946,0.15966608633375)--(0.4920735723557,0.12992343904819)--(0.52461137137799,0.11993960238694)--(0.58426459890015,0.15394297481185)--(0.59634516718861,0.20538395263939)},
		{(0.29349685398467,0.52917725146405)--(0.26643818854523,0.46702608858776)--(0.32046944036361,0.41094741532368)--(0.34494256511528,0.47991983192898)--(0.3189830015933,0.53242291831198)},
		{(0.94172921464231,0.52838594492475)--(0.79134723887928,0.50338394330796)--(0.81491768079246,0.40265956226011)--(0.85268479062871,0.35515772932871)--(0.86394986511225,0.35781078202526)},
		{(0.64176682088534,0.64354227744501)--(0.68976355041114,0.75731130982598)--(0.64453713193874,0.8044756189346)--(0.62246536258095,0.80659294065356)--(0.55433173261818,0.71777408333191)--(0.564582010025,0.68229342807613)},
		{(0.85078247275959,0.35287708055722)--(0.77306818472213,0.32498684923871)--(0.73203624977127,0.3090926198256)--(0.69726977440267,0.25372811877331)--(0.70894486827078,0.23856606830321)--(0.83762513083964,0.17985316987508)},
		{(0.2569340786675,-1.0136976840466e-17)--(0.28002663007953,-7.0148966784867e-18)--(0.29846712756044,0.054926375855413)--(0.29883092082001,0.075169912265488)--(0.18327949068838,0.17470850445308)--(0.13454432134724,0.133790045355)--(0.16381928133599,0.061602389221726)},
		{(0.18069712966087,0.4558871258114)--(0.21998466792267,0.41806869067827)--(0.242033983457,0.46822615943871)--(0.19477277146027,0.50044901219203)},
		{(0.017808859329408,-3.2242362240228e-19)--(0.072298550323714,-3.6027122931711e-19)--(0.08898776827222,0.065610427782656)--(0.080828852918195,0.11620195032545)--(0.070910661890026,0.11966321193268)--(0.063660301032814,0.11633980642789)--(0.033617789882254,0.078063609930191)},
		{(0.13249153377149,0.44098975235087)--(0.131004491351,0.4284508334431)--(0.17749395744838,0.40348625379436)--(0.17160187849493,0.44470222688338)},
		{(0.10355139017352,0.59421629322679)--(0.059530171783853,0.52244826813876)--(0.097789187353091,0.48375357888545)--(0.10719789273958,0.47857629032918)--(0.12567318042764,0.58705868273281)},
		{(0.28105561393038,0.27573692899187)--(0.23130398767021,0.27331924495965)--(0.22195346894011,0.21647883352707)--(0.2935353921711,0.22334503921447)},
		{(0.41654175438734,0.86222777776676)--(0.41984917863072,0.84686746152159)--(0.47950651139632,0.80717140096562)--(0.5261151507378,0.82531954984477)--(0.50448256867031,0.94714915240117)--(0.48403655852647,0.96123470828532)--(0.42320109421102,0.90149989587471)},
		{(0.18327949068838,0.17470850445308)--(0.29883092082001,0.075169912265488)--(0.34667451980727,0.1380125597604)--(0.33930259068455,0.16480732282195)--(0.32042710811992,0.2115802793121)--(0.2935353921711,0.22334503921447)--(0.22195346894011,0.21647883352707)--(0.18328032494326,0.17536328360184)},
		{(0.41821200434575,0.28798425564766)--(0.35551949547956,0.28111506769104)--(0.32042710811992,0.2115802793121)--(0.33930259068455,0.16480732282195)},
		{(-9.394099314051e-18,0.53884721269974)--(-9.5303926761132e-18,0.5221209362159)--(0.018290389757216,0.51381476588416)--(0.059530171783853,0.52244826813876)--(0.10355139017352,0.59421629322679)--(0.089192814691241,0.60908483270704)},
		{(0.19029265883876,0.51016314984628)--(0.19477277146027,0.50044901219203)--(0.242033983457,0.46822615943871)--(0.26643818854523,0.46702608858776)--(0.29349685398467,0.52917725146405)--(0.20834980476389,0.58422504544386)--(0.19557369936106,0.57605722853877)},
		{(0.21030919549961,0.58761888249994)--(0.36282951466794,0.58064565412964)--(0.38081463379964,0.59633021488474)--(0.37485167082317,0.60715289018846)--(0.23005311769621,0.63042739153665)},
		{(1.7367258949333e-18,0.19998426193243)--(1.7604470024043e-18,0.11433657409414)--(0.063660301032814,0.11633980642789)--(0.070910661890026,0.11966321193268)--(0.044628961597033,0.17483750654607)--(0.0043421685759054,0.19944862936459)},
		{(0.64259407254568,4.7111732237183e-18)--(0.72280082790447,4.0875614737244e-18)--(0.73889433232175,0.074011891053365)--(0.70675525986754,0.088503281622324)--(0.66291659040002,0.090641176042336)},
		{(0.86779285979695,0.17104959334438)--(0.84115428013916,0.16723863882903)--(0.84032518849665,0.16489902681664)--(0.8387242872409,0.13543084779993)--(0.95799344194723,0.1169182909266)--(0.95980288793588,0.120157844361)--(0.96240515495177,0.13411279303362)},
		{(0.84200246120739,0.59899793378281)--(0.74270409743836,0.58377890553351)--(0.73831490564384,0.57959692011435)--(0.79134723887928,0.50338394330796)--(0.94172921464231,0.52838594492475)--(0.94628875425442,0.53442062878311)},
		{(0.13939129777301,-8.7204753747234e-19)--(0.2569340786675,-2.4798149692361e-18)--(0.16381928133599,0.061602389221726)--(0.13345887328034,0.045705446797322)},
		{(1.0,0.55486457332503)--(1.0,0.68994896844859)--(0.87849436619317,0.68540158994539)--(0.84200246120739,0.59899793378281)--(0.94628875425442,0.53442062878311)},
		{(0.11292256040628,0.31430430903914)--(0.17058009573212,0.32679631765481)--(0.18323648816915,0.38361563383571)--(0.13855465034859,0.39338706492243)},
		{(0.42969744871829,0.062427885172013)--(0.37875600533919,0.047346392987191)--(0.41679787120389,0.024927156110052)--(0.44085670963409,0.038864627170034)},
		{(0.20834980476389,0.58422504544386)--(0.29349685398467,0.52917725146405)--(0.3189830015933,0.53242291831198)--(0.36282951466794,0.58064565412964)--(0.21030919549961,0.58761888249994)},
		{(0.2193526821086,0.39735212683294)--(0.21998466792267,0.41806869067827)--(0.18069712966087,0.4558871258114)--(0.17160187849493,0.44470222688338)--(0.17749395744838,0.40348625379436)--(0.1848223085064,0.38562605531908)},
		{(0.17058009573212,0.32679631765481)--(0.11292256040628,0.31430430903914)--(0.1108469391314,0.31358070639375)--(0.095755799344478,0.27580914374914)--(0.10723865242546,0.24801765073202)--(0.16206250398896,0.20259631064577)--(0.18764849573858,0.31562629180098)},
		{(0.092461778959338,0.76507283097678)--(0.068631366637277,0.63854518974231)--(0.089192814691241,0.60908483270704)--(0.10355139017352,0.59421629322679)--(0.12567318042763,0.58705868273281)--(0.15463530529741,0.58448326217391)--(0.14118446745633,0.7739398452686)--(0.13868496668193,0.77397786702595)},
		{(0.080828852918195,0.11620195032545)--(0.08898776827222,0.065610427782656)--(0.13345887328034,0.045705446797322)--(0.16381928133599,0.061602389221726)--(0.13454432134724,0.133790045355)},
		{(0.55760904079878,0.55519305264472)--(0.53021629682798,0.52266476711548)--(0.51002476855339,0.46508386857987)--(0.59103258681144,0.44094505268204)--(0.62990460023665,0.55432353818116)},
		{(0.38317611490883,0.48905895833332)--(0.40930151102783,0.43567991711595)--(0.44767090932061,0.46027995257834)--(0.39987125310726,0.50945658451621)},
		{(0.28002663007953,-4.3683120751993e-19)--(0.4332995415707,-7.646196514654e-19)--(0.43268008140293,0.0024214350371052)--(0.41679787120389,0.024927156110052)--(0.37875600533919,0.047346392987191)--(0.29846712756044,0.054926375855413)},
		{(0.39987125310726,0.50945658451621)--(0.44767090932061,0.46027995257834)--(0.49758667295564,0.46813302913076)--(0.41226668106765,0.56754699278944)},
		{(0.42241044350594,0.74252466591406)--(0.46117917728194,0.7064599270067)--(0.45753532695744,0.74849164711832)},
		{(0.86394986511225,0.35781078202526)--(0.85268479062871,0.35515772932871)--(0.85078247275959,0.35287708055722)--(0.83762513083964,0.17985316987508)--(0.84115428013916,0.16723863882903)--(0.86779285979695,0.17104959334438)--(0.92381970374935,0.19823176928504)--(0.98732709074017,0.31847099556908)},
		{(0.53495786093482,0.59890419023145)--(0.47222604353831,0.56327008741812)--(0.53021629682798,0.52266476711548)--(0.55760904079878,0.55519305264472)},
		{(0.31103514203778,1.0)--(0.29703528655886,1.0)--(0.25347854970964,0.91306121080099)--(0.37016545353716,0.93674400032224)},
		{(2.1443228462075e-17,0.62373115952369)--(1.2411893402441e-17,0.53884721269974)--(0.089192814691241,0.60908483270704)--(0.068631366637277,0.63854518974231)},
		{(0.46831006710122,0.69668552928355)--(0.45895723675566,0.68421102720186)--(0.46458047899867,0.62843705505097)--(0.52704763526984,0.62779212997707)--(0.564582010025,0.68229342807613)--(0.55433173261818,0.71777408333191)--(0.53011937586391,0.72545580183932)},
		{(0.38299954472878,0.76667795757367)--(0.38296955216628,0.76696413610026)--(0.2764530376866,0.80090852189881)--(0.21502946869708,0.80182333182705)--(0.19461191790395,0.77673701025447)--(0.2346864680375,0.70993863218517)--(0.30675585720203,0.70013106808951)},
		{(0.0043421685759053,0.19944862936459)--(0.044628961597033,0.17483750654607)--(0.10723865242546,0.24801765073202)--(0.095755799344478,0.27580914374914)--(0.058254947967382,0.27314731773162)},
		{(0.30675585720203,0.70013106808951)--(0.2346864680375,0.70993863218517)--(0.23005311769621,0.63042739153665)--(0.37485167082317,0.60715289018846)--(0.36979654915227,0.64084803401827)},
		{(0.40664343870089,0.90891046552274)--(0.37016545353716,0.93674400032224)--(0.25347854970964,0.91306121080099)--(0.23937785203937,0.89141065837622)},
		{(1.5536812770338e-18,0.38589973132066)--(-8.1135090646447e-19,0.29877786909705)--(0.058254947967382,0.27314731773162)--(0.095755799344478,0.27580914374914)--(0.1108469391314,0.31358070639375)--(0.059808102573128,0.38281439801307)--(0.043813712627158,0.39169134740803)--(0.027306298622061,0.39405820305984)},
		{(0.75576986043882,0.65144772028038)--(0.86207132279396,0.69997381902801)--(0.85766648697157,0.70804992032895)--(0.74815279193124,0.73292892754748)},
		{(0.41984917863072,0.84686746152159)--(0.38296955216628,0.76696413610026)--(0.38299954472878,0.76667795757367)--(0.39596435397984,0.74764722511405)--(0.42241044350594,0.74252466591406)--(0.45753532695744,0.74849164711832)--(0.49368151695606,0.7638799636651)--(0.47950651139632,0.80717140096562)},
		{(1.0,0.68994896844859)--(1.0,0.85318059068519)--(0.85533151005542,0.779576506393)--(0.85766648697157,0.70804992032895)--(0.86207132279396,0.69997381902801)--(0.87849436619317,0.68540158994539)},
		{(0.53159893867848,1.3774596629645e-18)--(0.64259407254568,5.4568986089327e-19)--(0.66291659040002,0.090641176042336)--(0.63896570096093,0.11584961088955)--(0.58426459890015,0.15394297481185)--(0.52461137137799,0.11993960238694)--(0.52373291894678,0.039958237609066)},
		{(0.7035969044932,0.57202846998949)--(0.73831490564384,0.57959692011435)--(0.74270409743836,0.58377890553351)--(0.75576986043882,0.65144772028038)--(0.74815279193124,0.73292892754748)--(0.74462848957745,0.74041138271774)--(0.68976355041114,0.75731130982598)--(0.64176682088534,0.64354227744501)--(0.68436642667714,0.57778291811654)},
		{(0.044628961597033,0.17483750654607)--(0.070910661890026,0.11966321193268)--(0.080828852918195,0.11620195032545)--(0.13454432134724,0.133790045355)--(0.18327949068838,0.17470850445308)--(0.18328032494326,0.17536328360184)--(0.16206250398896,0.20259631064577)--(0.10723865242546,0.24801765073202)},
		{(0.18328032494326,0.17536328360184)--(0.22195346894011,0.21647883352707)--(0.23130398767021,0.27331924495965)--(0.21646168418738,0.30735106496358)--(0.18764849573858,0.31562629180098)--(0.16206250398896,0.20259631064577)},
		{(0.4332995415707,9.1598126109942e-19)--(0.53159893867848,-2.4204540540085e-18)--(0.52373291894678,0.039958237609067)--(0.4487341692433,0.034326879481829)--(0.43268008140293,0.0024214350371053)},
		{(0.61451847493351,0.81578255166941)--(0.58145962342767,0.91079214595781)--(0.50448256867031,0.94714915240117)--(0.5261151507378,0.82531954984477)},
		{(0.68304668799349,0.3549919540442)--(0.60469588336363,0.3295097372055)--(0.62674582044354,0.23627455413941)--(0.69726977440267,0.25372811877331)--(0.73203624977127,0.3090926198256)},
		{(0.52704763526984,0.62779212997707)--(0.46458047899867,0.62843705505097)--(0.40629265276847,0.58536531422387)--(0.40992026943217,0.57902969949995)--(0.47222604353831,0.56327008741812)--(0.53495786093482,0.59890419023145)},
		{(-2.7636133040335e-17,0.11433657409414)--(-2.7709542413129e-17,0.068040822537518)--(0.033617789882254,0.078063609930191)--(0.063660301032814,0.11633980642789)},
		{(1.0,0.041319311713219)--(1.0,0.094228477320069)--(0.95980288793588,0.120157844361)--(0.95799344194723,0.1169182909266)--(0.94388723277738,0.08693946775931)--(0.96014035979164,0.049320796215747)},
		{(0.131004491351,0.4284508334431)--(0.12109529190628,0.40592995387427)--(0.13855465034859,0.39338706492243)--(0.18323648816915,0.38361563383571)--(0.1848223085064,0.38562605531908)--(0.17749395744838,0.40348625379436)},
		{(0.097789187353091,0.48375357888545)--(0.059530171783853,0.52244826813876)--(0.018290389757216,0.51381476588416)--(0.032772071952025,0.46262600750549)},
		{(-1.0199133931515e-18,0.80555925127614)--(1.0951678121777e-18,0.62373115952369)--(0.068631366637277,0.63854518974231)--(0.092461778959338,0.76507283097678)},
		{(0.29703528655886,1.0)--(0.12347670997767,1.0)--(0.23713019387941,0.88917660240315)--(0.23937785203937,0.89141065837622)--(0.25347854970964,0.91306121080099)},
		{(0.59103258681144,0.44094505268204)--(0.51002476855339,0.46508386857987)--(0.50957439155503,0.46485157361713)--(0.49195536692368,0.38340230861383)--(0.49880644287549,0.33740678480698)--(0.58280339373661,0.35169371245741)--(0.60443154695339,0.41569923038643)--(0.6033701017334,0.42797899736636)},
		{(0.12109529190628,0.40592995387427)--(0.059808102573128,0.38281439801307)--(0.1108469391314,0.31358070639375)--(0.11292256040628,0.31430430903914)--(0.13855465034859,0.39338706492243)},
		{(0.72280082790447,-4.0831101409703e-18)--(0.84466567680621,-4.5407790522216e-18)--(0.82199627575078,0.074255235773584)--(0.73889433232175,0.074011891053365)},
		{(0.52373291894678,0.039958237609067)--(0.52461137137799,0.11993960238694)--(0.4920735723557,0.12992343904819)--(0.42969744871829,0.062427885172013)--(0.44085670963409,0.038864627170034)--(0.4487341692433,0.034326879481829)},
		{(0.55433173261818,0.71777408333191)--(0.62246536258095,0.80659294065356)--(0.61451847493351,0.81578255166941)--(0.5261151507378,0.82531954984477)--(0.47950651139632,0.80717140096562)--(0.49368151695606,0.7638799636651)--(0.53011937586391,0.72545580183932)},
		{(0.31922655012417,0.40085855998477)--(0.32112698747207,0.40829080504919)--(0.32046944036361,0.41094741532368)--(0.26643818854523,0.46702608858776)--(0.242033983457,0.46822615943871)--(0.21998466792267,0.41806869067827)--(0.2193526821086,0.39735212683294)--(0.29132564916446,0.36702384870379)},
		{(0.14118446745633,0.7739398452686)--(0.15463530529741,0.58448326217391)--(0.19557369936106,0.57605722853877)--(0.20834980476389,0.58422504544386)--(0.21030919549961,0.58761888249994)--(0.23005311769621,0.63042739153665)--(0.2346864680375,0.70993863218517)--(0.19461191790395,0.77673701025447)},
		{(0.484937150244,1.0)--(0.31103514203778,1.0)--(0.37016545353716,0.93674400032224)--(0.40664343870089,0.90891046552274)--(0.42320109421102,0.90149989587471)--(0.48403655852647,0.96123470828532)},
		{(0.68879147678988,0.45788928555936)--(0.7035969044932,0.57202846998949)--(0.68436642667714,0.57778291811654)--(0.62990460023665,0.55432353818116)--(0.59103258681144,0.44094505268204)--(0.6033701017334,0.42797899736636)},
		{(1.0,0.094228477320069)--(1.0,0.16718329044542)--(0.99586260213138,0.16569346467065)--(0.96240515495177,0.13411279303362)--(0.95980288793588,0.120157844361)},
		{(0.36802123052607,0.85437879519475)--(0.41654175438734,0.86222777776676)--(0.42320109421102,0.90149989587471)--(0.40664343870089,0.90891046552274)--(0.23937785203937,0.89141065837622)--(0.23713019387941,0.88917660240315)--(0.23697796446859,0.88682035623598)},
		{(0.49880644287549,0.33740678480698)--(0.4788564283864,0.3055171813609)--(0.59634516718861,0.20538395263939)--(0.62674582044354,0.23627455413941)--(0.60469588336363,0.3295097372055)--(0.58280339373661,0.35169371245741)},
		{(0.40003004557943,0.69417309111533)--(0.39596435397984,0.74764722511405)--(0.38299954472878,0.76667795757367)--(0.30675585720203,0.70013106808951)--(0.36979654915227,0.64084803401827)},
		{(0.12347670997767,1.0)--(0.1043173504961,1.0)--(0.13868496668193,0.77397786702595)--(0.14118446745633,0.7739398452686)--(0.19461191790395,0.77673701025447)--(0.21502946869708,0.80182333182705)--(0.23697796446859,0.88682035623598)--(0.23713019387941,0.88917660240315)},
		{(0.92381970374935,0.19823176928504)--(0.86779285979695,0.17104959334438)--(0.96240515495177,0.13411279303362)--(0.99586260213138,0.16569346467065)},
		{(0.84466567680621,-2.4450320766493e-17)--(0.89703652004504,-3.4134829785379e-17)--(0.96014035979164,0.049320796215747)--(0.94388723277738,0.08693946775931)--(0.8364854829402,0.10783187604364)--(0.82199627575078,0.074255235773584)},
		{(0.46117917728194,0.7064599270067)--(0.46831006710122,0.69668552928355)--(0.53011937586391,0.72545580183932)--(0.49368151695606,0.7638799636651)--(0.45753532695744,0.74849164711832)},
		{(-9.5303926761132e-18,0.5221209362159)--(-1.1743299100151e-17,0.38589973132066)--(0.027306298622061,0.39405820305984)--(0.032772071952025,0.46262600750549)--(0.018290389757216,0.51381476588416)},
		{(0.45895723675566,0.68421102720186)--(0.40003004557943,0.69417309111533)--(0.36979654915227,0.64084803401827)--(0.37485167082317,0.60715289018846)--(0.38081463379964,0.59633021488474)--(0.40629265276847,0.58536531422387)--(0.46458047899867,0.62843705505097)},
		{(0.53495786093482,0.59890419023145)--(0.55760904079878,0.55519305264472)--(0.62990460023665,0.55432353818116)--(0.68436642667714,0.57778291811654)--(0.64176682088534,0.64354227744501)--(0.564582010025,0.68229342807613)--(0.52704763526984,0.62779212997707)},
		{(0.7188640385544,0.18080352400507)--(0.70894486827078,0.23856606830321)--(0.69726977440267,0.25372811877331)--(0.62674582044354,0.23627455413941)--(0.59634516718861,0.20538395263939)--(0.58426459890015,0.15394297481185)--(0.63896570096093,0.11584961088955)},
		{(0.10830029870213,0.47455859102972)--(0.13249153377149,0.44098975235087)--(0.17160187849493,0.44470222688338)--(0.18069712966087,0.4558871258114)--(0.19477277146027,0.50044901219203)--(0.19029265883876,0.51016314984628)--(0.10826395864066,0.4773705751988)},
		{(1.0,0.85318059068519)--(1.0,1.0)--(0.79330168747307,1.0)--(0.77299202284732,0.93610866178105)--(0.79776743801987,0.80051273443099)--(0.85533151005542,0.779576506393)},
		{(0.1043173504961,1.0)--(0.0,1.0)--(-1.0199133931515e-18,0.80555925127614)--(0.092461778959338,0.76507283097678)--(0.13868496668193,0.77397786702595)},
		{(0.21502946869708,0.80182333182705)--(0.2764530376866,0.80090852189881)--(0.36802123052607,0.85437879519475)--(0.23697796446859,0.88682035623598)},
		{(1.0,0.16718329044542)--(1.0,0.31765324128184)--(0.98732709074017,0.31847099556908)--(0.92381970374935,0.19823176928504)--(0.99586260213138,0.16569346467065)},
		{(-1.9865368616423e-17,0.29877786909705)--(-1.9906412481647e-17,0.19998426193243)--(0.0043421685759053,0.19944862936459)--(0.058254947967382,0.27314731773162)},
		{(0.12567318042764,0.58705868273281)--(0.10719789273958,0.47857629032918)--(0.10826395864066,0.4773705751988)--(0.19029265883876,0.51016314984628)--(0.19557369936106,0.57605722853877)--(0.15463530529741,0.58448326217391)},
		{(0.8364854829402,0.10783187604364)--(0.8387242872409,0.13543084779993)--(0.84032518849665,0.16489902681664)--(0.737162565826,0.15970634748151)--(0.70675525986754,0.088503281622324)--(0.73889433232175,0.074011891053365)--(0.82199627575078,0.074255235773584)},
		{(0.67975813505523,0.3689251492507)--(0.60443154695339,0.41569923038643)--(0.58280339373661,0.35169371245741)--(0.60469588336363,0.3295097372055)--(0.68304668799349,0.3549919540442)},
		{(0.69678772190095,0.41510946443084)--(0.70351498956895,0.43724333517807)--(0.68879147678988,0.45788928555936)--(0.6033701017334,0.42797899736636)--(0.60443154695339,0.41569923038643)--(0.67975813505523,0.3689251492507)},
		{(0.40930151102783,0.43567991711595)--(0.40570204902276,0.42943083535428)--(0.41921551498506,0.40659750714898)--(0.49195536692368,0.38340230861383)--(0.50957439155503,0.46485157361713)--(0.49758667295564,0.46813302913076)--(0.44767090932061,0.46027995257834)},
		{(0.2193526821086,0.39735212683294)--(0.1848223085064,0.38562605531908)--(0.18323648816915,0.38361563383571)--(0.17058009573212,0.32679631765481)--(0.18764849573858,0.31562629180098)--(0.21646168418738,0.30735106496358)--(0.29119050723099,0.3661008405474)--(0.29132564916446,0.36702384870379)},
		{(0.3130676881052,0.3125024910612)--(0.28105561393038,0.27573692899187)--(0.2935353921711,0.22334503921447)--(0.32042710811992,0.2115802793121)--(0.35551949547956,0.28111506769104)},
		{(0.77306818472213,0.32498684923871)--(0.69678772190095,0.41510946443084)--(0.67975813505523,0.3689251492507)--(0.68304668799349,0.3549919540442)--(0.73203624977127,0.3090926198256)},
		{(0.41226668106765,0.56754699278944)--(0.49758667295564,0.46813302913076)--(0.50957439155503,0.46485157361713)--(0.51002476855339,0.46508386857987)--(0.53021629682798,0.52266476711548)--(0.47222604353831,0.56327008741812)--(0.40992026943217,0.57902969949995)},
		{(0.44085670963409,0.038864627170034)--(0.41679787120389,0.024927156110052)--(0.43268008140293,0.0024214350371052)--(0.4487341692433,0.034326879481829)},
		{(0.38081463379964,0.59633021488474)--(0.36282951466794,0.58064565412964)--(0.3189830015933,0.53242291831198)--(0.34494256511528,0.47991983192898)--(0.38317611490883,0.48905895833332)--(0.39987125310726,0.50945658451621)--(0.41226668106765,0.56754699278944)--(0.40992026943217,0.57902969949995)--(0.40629265276847,0.58536531422387)},
		{(1.0,0.31765324128184)--(1.0,0.55486457332503)--(0.94628875425442,0.53442062878311)--(0.94172921464231,0.52838594492475)--(0.86394986511225,0.35781078202526)--(0.98732709074017,0.31847099556908)},
		{(0.43576762292633,0.29184645561424)--(0.40416601277261,0.34440199183953)--(0.31922655012417,0.40085855998477)--(0.29132564916446,0.36702384870379)--(0.29119050723099,0.3661008405474)--(0.3130676881052,0.3125024910612)--(0.35551949547956,0.28111506769104)--(0.41821200434575,0.28798425564766)},
		{(0.0,0.0)--(0.017808859329408,3.4673591993011e-19)--(0.033617789882254,0.078063609930191)--(4.0474899696953e-17,0.068040822537518)},
		{(0.44206071785289,0.28992586066053)--(0.43576762292633,0.29184645561424)--(0.41821200434575,0.28798425564766)--(0.33930259068455,0.16480732282195)--(0.34667451980727,0.1380125597604)--(0.44300721060946,0.15966608633375)},
		{(0.027306298622061,0.39405820305984)--(0.043813712627158,0.39169134740803)--(0.10830029870213,0.47455859102972)--(0.10826395864066,0.4773705751988)--(0.10719789273958,0.47857629032918)--(0.097789187353091,0.48375357888545)--(0.032772071952025,0.46262600750549)}
    } {
        \draw[ultra thin, color=\documentcolor, opacity=0.25] \polygon -- cycle;
    }
    
    % Cells.
    \foreach \polygon in {
		{(0.81377427336204,0.46644534298129)--(0.75923295959791,0.45791497191854)--(0.73757929240127,0.40647879170533)--(0.78065173980259,0.35052466099245)--(0.81430805704037,0.3492625046274)--(0.8529169955575,0.39200429940237)},
		{(0.36741249518689,0.77626288981837)--(0.40173646438035,0.83622596474369)--(0.40164375289576,0.83669224245549)--(0.33340493932519,0.87655499835545)--(0.277181252921,0.83569973176466)--(0.30639749172945,0.76731167055377)},
		{(0.43437329544108,0.62740173590702)--(0.5083806772443,0.67409511180354)--(0.50858499820828,0.69293340123319)--(0.46204277488079,0.7375764324665)--(0.4049860583089,0.71881236943179)--(0.39169770109416,0.68722681252046)},
		{(0.87010808480848,0.76375127032153)--(0.86486066743379,0.7633854702304)--(0.80441905817567,0.67917067226413)--(0.83032969363452,0.62359767026028)--(0.89331717649906,0.62303886472077)--(0.94354430216006,0.70124383163287)},
		{(0.40355601843553,0.42872965074704)--(0.3588209602385,0.4015672576613)--(0.35008775853055,0.35613482788822)--(0.43068864847863,0.33178421506826)--(0.45336697589585,0.35334587285169)--(0.43463087640918,0.41856066646997)},
		{(0.85715374101961,0.92297987260062)--(0.80004476488986,0.88237356857976)--(0.80342102854991,0.80640238532619)--(0.86486066743379,0.7633854702304)--(0.87010808480848,0.76375127032153)--(0.93684678804366,0.85558688155856)},
		{(0.34753238882745,0.49856889378558)--(0.30906560949117,0.44768482568236)--(0.3588209602385,0.4015672576613)--(0.40355601843553,0.42872965074704)--(0.388482560027,0.49227367804315)},
		{(0.75923295959791,0.45791497191854)--(0.81377427336204,0.46644534298129)--(0.84082490850973,0.50471392257231)--(0.80286587331314,0.57968601144072)--(0.7382976811111,0.5699697035951)--(0.71813669466341,0.50408202800119)},
		{(0.68949293175191,0.18372328197331)--(0.7409210086195,0.15466243293744)--(0.79933896048372,0.19006791029744)--(0.79443195417494,0.2353533946542)--(0.73454000357343,0.26962355692421)--(0.6855195201363,0.21835061501379)},
		{(0.71113639015299,1.0)--(0.55852156786264,1.0)--(0.60149122246882,0.90673278977818)--(0.70601161988357,0.94316683857761)},
		{(0.12227589551397,-6.5238950862231e-19)--(0.19986164251846,-1.1237323234243e-18)--(0.19087676357043,0.077958222400555)--(0.12826813184992,0.061154270436431)},
		{(0.89617395870701,5.4180189498655e-18)--(1.0,0.0)--(1.0,0.058752397294144)--(0.90540525465644,0.069583827622574)},
		{(0.45336697589585,0.35334587285169)--(0.43068864847863,0.33178421506826)--(0.42713017585419,0.26933930250147)--(0.4361607070855,0.25969230276856)--(0.5066165818898,0.24904686688435)--(0.51747615642121,0.25601153533809)--(0.53014756501118,0.32835653292115)--(0.50708420792836,0.35035636952691)},
		{(0.3471379812099,0.35374393205416)--(0.27319827639826,0.36291999297592)--(0.25470476380115,0.34283439110708)--(0.27798534189451,0.27152908778946)--(0.33650779393129,0.28850173339121)},
		{(0.41875055490304,0.17713421701335)--(0.36191909399227,0.17375752435787)--(0.3449033247313,0.097867452252301)--(0.34891842643741,0.093030281653553)--(0.42162657447946,0.079522330919597)--(0.44315769209849,0.096030757337953)--(0.44818331753831,0.15244903606967)},
		{(0.7409210086195,0.15466243293744)--(0.68949293175191,0.18372328197331)--(0.64175878731343,0.1425851464857)--(0.64727813653168,0.095138966823452)--(0.70680226884436,0.069917833972296)--(0.74387319795567,0.098055537160779)},
		{(0.11494878815847,0.41712994289509)--(0.1357631592805,0.43052789908513)--(0.13784570335467,0.48933922664861)--(0.094081638532771,0.51067505521418)--(0.056211952034956,0.47257967565105)--(0.070062125720471,0.43266444823048)},
		{(0.86476362211263,1.0)--(0.71113639015299,1.0)--(0.70601161988357,0.94316683857761)--(0.72647040305996,0.90703260488626)--(0.80004476488986,0.88237356857976)--(0.85715374101961,0.92297987260062)},
		{(0.80342102854991,0.80640238532619)--(0.80004476488986,0.88237356857976)--(0.72647040305996,0.90703260488626)--(0.68089400269524,0.81533560654476)--(0.73057105824356,0.77504881108474)},
		{(0.90088496095405,0.078372760582186)--(0.92607611600945,0.13634057968)--(0.89769469044754,0.18641868288881)--(0.83266992292899,0.1679093939665)--(0.83412505049535,0.096075256820406)},
		{(0.51747615642121,0.25601153533809)--(0.5066165818898,0.24904686688435)--(0.49921303152656,0.17157531809297)--(0.54643146255917,0.14321173137306)--(0.58650600383281,0.16757388619041)--(0.58822586097716,0.23184518129324)},
		{(0.27018293298545,0.54114656345302)--(0.25262695560923,0.51441347177955)--(0.28771661402906,0.44552749126863)--(0.30906560949117,0.44768482568236)--(0.34753238882745,0.49856889378558)--(0.3248640250729,0.53781612410112)},
		{(0.89764058000683,0.5074560818892)--(0.84082490850973,0.50471392257231)--(0.81377427336204,0.46644534298129)--(0.8529169955575,0.39200429940237)--(0.9012650040962,0.39179917057072)--(0.93589772115998,0.42699763567665)},
		{(0.72704884336493,0.69832391192792)--(0.73057105824356,0.77504881108474)--(0.68089400269524,0.81533560654476)--(0.65203360452762,0.81302854821066)--(0.60168563045276,0.74507135264944)--(0.62847917748123,0.70174833694654)--(0.71419851369649,0.68849848453681)},
		{(0.81430805704037,0.3492625046274)--(0.78065173980259,0.35052466099245)--(0.73070856698794,0.29301468150288)--(0.73454000357343,0.26962355692421)--(0.79443195417494,0.2353533946542)--(0.8483200717027,0.28110781218028)},
		{(0.26227070461306,0.1671848011338)--(0.22656771464951,0.17919876762108)--(0.18832616281389,0.16340518200672)--(0.17763208949406,0.135999459889)--(0.19457272517073,0.083498197174869)--(0.26193615148853,0.086523888431298)--(0.27756497004741,0.10263598416091)},
		{(0.28771661402906,0.44552749126863)--(0.25262695560923,0.51441347177955)--(0.2013224108042,0.50151646237705)--(0.21005394498235,0.42931809165649)--(0.26471063334888,0.42245576174845)},
		{(0.061662486707602,3.4075156807917e-18)--(0.12227589551397,-6.5238950862231e-19)--(0.12826813184992,0.061154270436431)--(0.093716962981599,0.090157628363625)--(0.059127538435393,0.069293444671796)},
		{(0.13784570335467,0.48933922664861)--(0.1357631592805,0.43052789908513)--(0.18932285452885,0.41326685130257)--(0.21005394498235,0.42931809165649)--(0.2013224108042,0.50151646237705)--(0.18355624333685,0.51039392719948)},
		{(0.05502355466207,0.6439219076586)--(0.075705740334593,0.5626589437364)--(0.0846824115434,0.55765118315718)--(0.14320805808351,0.58732992451543)--(0.11610429352201,0.67207753949949)--(0.096409201823942,0.67404340318591)},
		{(0.27699152395929,0.26994362912223)--(0.22210332898952,0.25169595854152)--(0.22656771464951,0.17919876762108)--(0.26227070461306,0.1671848011338)--(0.30145693717806,0.19669714013291)},
		{(0.55852156786264,1.0)--(0.5223377782225,1.0)--(0.47826990299094,0.90645345131304)--(0.53335982731871,0.85501455453458)--(0.59266654390433,0.879667353541)--(0.60149122246882,0.90673278977818)},
		{(0.3449033247313,0.097867452252301)--(0.36191909399227,0.17375752435787)--(0.34559181978407,0.19135712502723)--(0.30145693717806,0.19669714013291)--(0.26227070461306,0.1671848011338)--(0.27756497004741,0.10263598416091)},
		{(0.4361607070855,0.25969230276856)--(0.42713017585419,0.26933930250147)--(0.36554422157249,0.26256516276973)--(0.34559181978407,0.19135712502723)--(0.36191909399227,0.17375752435787)--(0.41875055490304,0.17713421701335)},
		{(1.5376327988901e-17,0.64772798558207)--(1.6034554889537e-17,0.55058297186392)--(0.075705740334593,0.5626589437364)--(0.05502355466207,0.6439219076586)},
		{(0.17134169331082,0.57938374097505)--(0.18355624333685,0.51039392719948)--(0.2013224108042,0.50151646237705)--(0.25262695560923,0.51441347177955)--(0.27018293298545,0.54114656345302)--(0.25502554267202,0.58279516484096)--(0.20301198576883,0.59939207586982)},
		{(0.20301198576883,0.59939207586982)--(0.25502554267202,0.58279516484096)--(0.29959034399522,0.6369959559015)--(0.23001348019483,0.68383060276134)--(0.20636245832899,0.66983963838826)},
		{(2.7170782714942e-18,0.22137505583613)--(-3.2495733020338e-18,0.13512764986591)--(0.08220092118446,0.15466338357064)--(0.082398760918983,0.15605657582294)--(0.046633614663434,0.21938783290738)},
		{(0.60783424510423,-1.2794324155343e-17)--(0.70868263841806,-1.7121612108467e-17)--(0.70680226884436,0.069917833972296)--(0.64727813653168,0.095138966823452)--(0.60686045679871,0.059896034265045)},
		{(0.88651321148548,0.27353618403783)--(0.8483200717027,0.28110781218028)--(0.79443195417494,0.2353533946542)--(0.79933896048372,0.19006791029744)--(0.83266992292899,0.1679093939665)--(0.89769469044754,0.18641868288881)--(0.91004711516282,0.21507335330943)},
		{(0.89331717649906,0.62303886472077)--(0.83032969363452,0.62359767026028)--(0.80286587331314,0.57968601144072)--(0.84082490850973,0.50471392257231)--(0.89764058000683,0.5074560818892)--(0.939423153296,0.55741955645906)},
		{(0.19986164251846,-1.1237323234243e-18)--(0.2746330686814,-1.7745093106029e-20)--(0.26193615148853,0.086523888431298)--(0.19457272517073,0.083498197174869)--(0.19087676357043,0.077958222400555)},
		{(1.0,0.55764643983381)--(1.0,0.70413255616401)--(0.94354430216006,0.70124383163286)--(0.89331717649906,0.62303886472077)--(0.939423153296,0.55741955645906)},
		{(0.13133932211366,0.30081906788868)--(0.18073157431698,0.26585067423534)--(0.18945151817242,0.26764872284988)--(0.21047334568522,0.34170913727606)--(0.18913323621749,0.35978679702952)--(0.1343623311339,0.34355393370436)},
		{(0.33601345693271,-2.6880139561383e-18)--(0.42198060132998,1.2434732069417e-19)--(0.42162657447946,0.079522330919597)--(0.34891842643741,0.093030281653553)},
		{(0.31587490906051,0.63990510205951)--(0.29959034399522,0.6369959559015)--(0.25502554267202,0.58279516484096)--(0.27018293298545,0.54114656345302)--(0.3248640250729,0.53781612410112)--(0.35434648910764,0.5868771168903)},
		{(0.26471063334888,0.42245576174845)--(0.21005394498235,0.42931809165649)--(0.18932285452885,0.41326685130257)--(0.18913323621749,0.35978679702952)--(0.21047334568522,0.34170913727606)--(0.25470476380115,0.34283439110708)--(0.27319827639826,0.36291999297592)},
		{(0.083046851941488,0.25643782319375)--(0.13702408055966,0.21168455741157)--(0.14211272985908,0.21201871064309)--(0.18073157431698,0.26585067423534)--(0.13133932211366,0.30081906788868)--(0.083479811319291,0.27406640426661)},
		{(0.052111004327275,0.75459090538481)--(0.096409201823942,0.67404340318591)--(0.11610429352201,0.67207753949949)--(0.14424787658226,0.6876940756699)--(0.15790469123836,0.78029851948679)--(0.095071274608359,0.79123815887059)},
		{(0.093716962981599,0.090157628363625)--(0.12826813184992,0.061154270436431)--(0.19087676357043,0.077958222400555)--(0.19457272517073,0.083498197174869)--(0.17763208949406,0.135999459889)--(0.097995603708305,0.12822258324854)},
		{(0.65984709131333,0.49171876411262)--(0.61842882474402,0.53009581684405)--(0.57116713378872,0.51656430017137)--(0.55170197411666,0.4431006298003)--(0.60287487120645,0.41309615236067)--(0.6429799752324,0.42440434156506)},
		{(0.388482560027,0.49227367804315)--(0.40355601843553,0.42872965074704)--(0.43463087640918,0.41856066646997)--(0.48344882740163,0.45437050808152)--(0.46888078913854,0.50567323874642)--(0.41646568801911,0.51582103688304)},
		{(0.2746330686814,-1.7745093106029e-20)--(0.33601345693271,2.6316515385181e-18)--(0.34891842643741,0.093030281653553)--(0.3449033247313,0.097867452252301)--(0.27756497004741,0.10263598416091)--(0.26193615148853,0.086523888431298)},
		{(0.41646568801911,0.51582103688304)--(0.46888078913854,0.50567323874642)--(0.51173107082738,0.55559599825558)--(0.4319737972028,0.61146700444916)--(0.40419477576661,0.58568832829771)},
		{(0.40173646438035,0.83622596474369)--(0.36741249518689,0.77626288981837)--(0.4049860583089,0.71881236943179)--(0.46204277488079,0.7375764324665)--(0.47229351202466,0.79005207789208)},
		{(0.9012650040962,0.39179917057072)--(0.8529169955575,0.39200429940237)--(0.81430805704037,0.3492625046274)--(0.8483200717027,0.28110781218028)--(0.88651321148548,0.27353618403783)--(0.92847309663915,0.31554888410497)},
		{(0.59685918847794,0.63026005594793)--(0.55056341443673,0.627321238202)--(0.51490714181101,0.55628429897008)--(0.57116713378872,0.51656430017137)--(0.61842882474402,0.53009581684405)--(0.63344655969018,0.58950859602035)},
		{(0.38213822664684,1.0)--(0.23719520344954,1.0)--(0.24199831168402,0.95559295318398)--(0.33180829244124,0.9174469012483)--(0.38085488971948,0.97235962211152)},
		{(-3.6877698504793e-18,0.7553149622307)--(1.5316222932689e-18,0.64772798558207)--(0.05502355466207,0.6439219076586)--(0.096409201823942,0.67404340318591)--(0.052111004327275,0.75459090538481)},
		{(0.50858499820828,0.69293340123319)--(0.5083806772443,0.67409511180354)--(0.55056341443673,0.627321238202)--(0.59685918847794,0.63026005594793)--(0.62847917748123,0.70174833694654)--(0.60168563045276,0.74507135264944)--(0.57838182567746,0.74779793288446)},
		{(0.30639749172945,0.76731167055377)--(0.277181252921,0.83569973176466)--(0.26660883903922,0.83862561325022)--(0.18183945569358,0.79795414606115)--(0.17832003444239,0.7908790200153)--(0.24419439537107,0.72706041884685)--(0.29619454381117,0.74821429496436)},
		{(0.046633614663434,0.21938783290738)--(0.082398760918983,0.15605657582294)--(0.13702408055966,0.21168455741157)--(0.083046851941488,0.25643782319375)},
		{(0.29619454381117,0.74821429496436)--(0.24419439537107,0.72706041884685)--(0.23001348019483,0.68383060276134)--(0.29959034399522,0.6369959559015)--(0.31587490906051,0.63990510205951)--(0.33682886116887,0.67271577105837)},
		{(0.26660883903922,0.83862561325022)--(0.277181252921,0.83569973176466)--(0.33340493932519,0.87655499835545)--(0.33180829244124,0.9174469012483)--(0.24199831168402,0.95559295318398)--(0.2230856608485,0.91942162464549)},
		{(1.8156558448054e-17,0.3886600085509)--(1.7392932991998e-17,0.29801984598924)--(0.053378087503395,0.30362644622448)--(0.067568804973452,0.35136838230796)--(0.02932558649174,0.38863580481067)},
		{(0.80441905817567,0.67917067226413)--(0.86486066743379,0.7633854702304)--(0.80342102854991,0.80640238532619)--(0.73057105824356,0.77504881108474)--(0.72704884336493,0.69832391192792)},
		{(0.40164375289576,0.83669224245549)--(0.40173646438035,0.83622596474369)--(0.47229351202466,0.79005207789208)--(0.52181571356122,0.81467654481876)--(0.53335982731871,0.85501455453458)--(0.47826990299094,0.90645345131304)--(0.44563014749115,0.90163220700976)},
		{(1.0,0.70413255616401)--(1.0,0.86136896446465)--(0.93684678804366,0.85558688155856)--(0.87010808480848,0.76375127032153)--(0.94354430216006,0.70124383163286)},
		{(0.60686045679871,0.059896034265045)--(0.64727813653168,0.095138966823452)--(0.64175878731343,0.1425851464857)--(0.58650600383281,0.16757388619041)--(0.54643146255917,0.14321173137306)--(0.54301461872237,0.092797036270243)},
		{(0.7382976811111,0.5699697035951)--(0.80286587331314,0.57968601144072)--(0.83032969363452,0.62359767026028)--(0.80441905817567,0.67917067226413)--(0.72704884336493,0.69832391192792)--(0.71419851369649,0.68849848453681)--(0.7031996012139,0.60475064770785)},
		{(0.17763208949406,0.135999459889)--(0.18832616281389,0.16340518200672)--(0.14211272985908,0.21201871064309)--(0.13702408055966,0.21168455741157)--(0.082398760918983,0.15605657582294)--(0.08220092118446,0.15466338357064)--(0.097995603708305,0.12822258324854)},
		{(0.18832616281389,0.16340518200672)--(0.22656771464951,0.17919876762108)--(0.22210332898952,0.25169595854152)--(0.18945151817242,0.26764872284988)--(0.18073157431698,0.26585067423534)--(0.14211272985908,0.21201871064309)},
		{(0.50783652063976,5.7524658570437e-19)--(0.60783424510423,-7.6741652620722e-19)--(0.60686045679871,0.059896034265045)--(0.54301461872237,0.092797036270243)--(0.51066058749423,0.068531486885363)},
		{(0.65203360452762,0.81302854821066)--(0.68089400269524,0.81533560654476)--(0.72647040305996,0.90703260488626)--(0.70601161988357,0.94316683857761)--(0.60149122246882,0.90673278977818)--(0.59266654390433,0.879667353541)},
		{(0.6855195201363,0.21835061501379)--(0.73454000357343,0.26962355692421)--(0.73070856698794,0.29301468150288)--(0.67417167828656,0.3284451063664)--(0.62594643337379,0.30737678237245)--(0.62023001744523,0.25384088092476)},
		{(0.5083806772443,0.67409511180354)--(0.43437329544108,0.62740173590702)--(0.4319737972028,0.61146700444916)--(0.51173107082738,0.55559599825558)--(0.51490714181101,0.55628429897008)--(0.55056341443673,0.627321238202)},
		{(0.08220092118446,0.15466338357064)--(-1.1578803197835e-18,0.13512764986591)--(1.7277520457165e-18,0.085345674513439)--(0.059127538435393,0.069293444671796)--(0.0937169629816,0.090157628363625)--(0.097995603708305,0.12822258324854)},
		{(1.0,0.058752397294144)--(1.0,0.13765144661038)--(0.92607611600945,0.13634057968)--(0.90088496095405,0.078372760582186)--(0.90540525465644,0.069583827622574)},
		{(0.1357631592805,0.43052789908513)--(0.11494878815847,0.41712994289509)--(0.11118580357978,0.36293561472925)--(0.1343623311339,0.34355393370436)--(0.18913323621749,0.35978679702952)--(0.18932285452885,0.41326685130257)},
		{(-1.1374511936236e-17,0.55058297186392)--(-1.071091355383e-17,0.4839874495359)--(0.056211952034956,0.47257967565105)--(0.094081638532771,0.51067505521418)--(0.0846824115434,0.55765118315718)--(0.075705740334594,0.5626589437364)},
		{(-2.1659226667175e-18,0.88316080301788)--(-1.152287776871e-18,0.7553149622307)--(0.052111004327275,0.75459090538481)--(0.095071274608359,0.79123815887059)--(0.071117184385222,0.8747996608725)},
		{(0.23719520344954,1.0)--(0.10768490850862,1.0)--(0.11826183840611,0.90964515917231)--(0.15891801509917,0.8938576092032)--(0.2230856608485,0.91942162464549)--(0.24199831168402,0.95559295318398)},
		{(0.60287487120645,0.41309615236067)--(0.55170197411666,0.4431006298003)--(0.5291990062865,0.4359771852093)--(0.50708420792836,0.35035636952691)--(0.53014756501118,0.32835653292115)--(0.58641460002033,0.33863084952938)},
		{(0.083479811319291,0.27406640426661)--(0.13133932211366,0.30081906788868)--(0.1343623311339,0.34355393370436)--(0.11118580357978,0.36293561472925)--(0.067568804973452,0.35136838230796)--(0.053378087503395,0.30362644622448)},
		{(0.70868263841806,-1.4819045118148e-17)--(0.80532088406662,-1.3986236596368e-17)--(0.80466912304714,0.074630587799916)--(0.74387319795567,0.098055537160779)--(0.70680226884436,0.069917833972296)},
		{(0.54301461872237,0.092797036270243)--(0.54643146255917,0.14321173137306)--(0.49921303152656,0.17157531809297)--(0.44818331753831,0.15244903606967)--(0.44315769209849,0.096030757337953)--(0.51066058749423,0.068531486885363)},
		{(0.60168563045276,0.74507135264944)--(0.65203360452762,0.81302854821066)--(0.59266654390433,0.879667353541)--(0.53335982731871,0.85501455453458)--(0.52181571356122,0.81467654481876)--(0.57838182567746,0.74779793288446)},
		{(0.35008775853055,0.35613482788822)--(0.3588209602385,0.4015672576613)--(0.30906560949117,0.44768482568236)--(0.28771661402906,0.44552749126863)--(0.26471063334888,0.42245576174845)--(0.27319827639826,0.36291999297592)--(0.3471379812099,0.35374393205416)},
		{(0.15790469123836,0.78029851948679)--(0.14424787658226,0.6876940756699)--(0.20636245832899,0.66983963838826)--(0.23001348019483,0.68383060276134)--(0.24419439537107,0.72706041884685)--(0.17832003444239,0.7908790200153)},
		{(0.5223377782225,1.0)--(0.38213822664684,1.0)--(0.38085488971948,0.97235962211152)--(0.44563014749115,0.90163220700976)--(0.47826990299094,0.90645345131304)},
		{(0.71813669466341,0.50408202800119)--(0.7382976811111,0.5699697035951)--(0.7031996012139,0.60475064770785)--(0.63344655969018,0.58950859602035)--(0.61842882474402,0.53009581684405)--(0.65984709131333,0.49171876411262)},
		{(1.0,0.13765144661038)--(1.0,0.22274154013077)--(0.91004711516282,0.21507335330943)--(0.89769469044754,0.18641868288881)--(0.92607611600945,0.13634057968)},
		{(0.38085488971948,0.97235962211152)--(0.33180829244124,0.9174469012483)--(0.33340493932519,0.87655499835545)--(0.40164375289576,0.83669224245549)--(0.44563014749115,0.90163220700976)},
		{(0.53014756501118,0.32835653292115)--(0.51747615642121,0.25601153533809)--(0.58822586097716,0.23184518129324)--(0.62023001744523,0.25384088092476)--(0.62594643337379,0.30737678237245)--(0.58641460002033,0.33863084952938)},
		{(0.39169770109416,0.68722681252046)--(0.4049860583089,0.71881236943179)--(0.36741249518689,0.77626288981837)--(0.30639749172945,0.76731167055377)--(0.29619454381117,0.74821429496436)--(0.33682886116887,0.67271577105837)},
		{(0.071117184385222,0.8747996608725)--(0.095071274608359,0.79123815887059)--(0.15790469123836,0.78029851948679)--(0.17832003444239,0.7908790200153)--(0.18183945569358,0.79795414606115)--(0.15891801509917,0.8938576092032)--(0.11826183840611,0.90964515917231)},
		{(1.0,0.22274154013077)--(1.0,0.30839580497562)--(0.92847309663915,0.31554888410497)--(0.88651321148548,0.27353618403783)--(0.91004711516282,0.21507335330943)},
		{(0.80532088406662,2.7645539268707e-17)--(0.89617395870701,4.0256052199517e-17)--(0.90540525465644,0.069583827622574)--(0.90088496095405,0.078372760582186)--(0.83412505049535,0.096075256820406)--(0.80466912304714,0.074630587799916)},
		{(0.46204277488079,0.7375764324665)--(0.50858499820828,0.69293340123319)--(0.57838182567746,0.74779793288446)--(0.52181571356122,0.81467654481876)--(0.47229351202466,0.79005207789208)},
		{(-6.5897971399554e-18,0.4839874495359)--(-6.1295619438285e-18,0.3886600085509)--(0.02932558649174,0.38863580481067)--(0.070062125720471,0.43266444823048)--(0.056211952034956,0.47257967565105)},
		{(0.43437329544108,0.62740173590702)--(0.39169770109416,0.68722681252046)--(0.33682886116887,0.67271577105837)--(0.31587490906051,0.63990510205951)--(0.35434648910764,0.5868771168903)--(0.40419477576661,0.58568832829771)--(0.4319737972028,0.61146700444916)},
		{(0.59685918847794,0.63026005594793)--(0.63344655969018,0.58950859602035)--(0.7031996012139,0.60475064770785)--(0.71419851369649,0.68849848453681)--(0.62847917748123,0.70174833694654)},
		{(0.68949293175191,0.18372328197331)--(0.6855195201363,0.21835061501379)--(0.62023001744523,0.25384088092476)--(0.58822586097716,0.23184518129324)--(0.58650600383281,0.16757388619041)--(0.64175878731343,0.1425851464857)},
		{(0.094081638532771,0.51067505521418)--(0.13784570335467,0.48933922664861)--(0.18355624333685,0.51039392719948)--(0.17134169331082,0.57938374097505)--(0.14320805808351,0.58732992451543)--(0.0846824115434,0.55765118315718)},
		{(1.0,0.86136896446465)--(1.0,1.0)--(0.86476362211263,1.0)--(0.85715374101961,0.92297987260062)--(0.93684678804366,0.85558688155856)},
		{(0.10768490850862,1.0)--(0.0,1.0)--(-2.1659226667175e-18,0.88316080301788)--(0.071117184385222,0.8747996608725)--(0.11826183840611,0.90964515917231)},
		{(0.18183945569358,0.79795414606115)--(0.26660883903922,0.83862561325022)--(0.2230856608485,0.91942162464549)--(0.15891801509917,0.8938576092032)},
		{(1.0,0.30839580497562)--(1.0,0.42725080647529)--(0.93589772115998,0.42699763567665)--(0.9012650040962,0.39179917057072)--(0.92847309663915,0.31554888410497)},
		{(3.5910707340084e-18,0.29801984598924)--(6.5685858886698e-18,0.22137505583613)--(0.046633614663434,0.21938783290738)--(0.083046851941488,0.25643782319375)--(0.083479811319291,0.27406640426661)--(0.053378087503395,0.30362644622448)},
		{(0.11610429352201,0.67207753949949)--(0.14320805808351,0.58732992451543)--(0.17134169331082,0.57938374097505)--(0.20301198576883,0.59939207586982)--(0.20636245832899,0.66983963838826)--(0.14424787658226,0.6876940756699)},
		{(0.83412505049535,0.096075256820406)--(0.83266992292899,0.1679093939665)--(0.79933896048372,0.19006791029744)--(0.7409210086195,0.15466243293744)--(0.74387319795567,0.098055537160779)--(0.80466912304714,0.074630587799916)},
		{(0.68400242448273,0.39264458059972)--(0.6429799752324,0.42440434156506)--(0.60287487120645,0.41309615236067)--(0.58641460002033,0.33863084952938)--(0.62594643337379,0.30737678237245)--(0.67417167828656,0.3284451063664)},
		{(0.73757929240127,0.40647879170533)--(0.75923295959791,0.45791497191854)--(0.71813669466341,0.50408202800119)--(0.65984709131333,0.49171876411262)--(0.6429799752324,0.42440434156506)--(0.68400242448273,0.39264458059972)},
		{(0.43463087640918,0.41856066646997)--(0.45336697589585,0.35334587285169)--(0.50708420792836,0.35035636952691)--(0.5291990062865,0.4359771852093)--(0.48344882740163,0.45437050808152)},
		{(0.22210332898952,0.25169595854152)--(0.27699152395929,0.26994362912223)--(0.27798534189451,0.27152908778946)--(0.25470476380115,0.34283439110708)--(0.21047334568522,0.34170913727606)--(0.18945151817242,0.26764872284988)},
		{(0.33650779393129,0.28850173339121)--(0.27798534189451,0.27152908778946)--(0.27699152395929,0.26994362912223)--(0.30145693717806,0.19669714013291)--(0.34559181978407,0.19135712502723)--(0.36554422157249,0.26256516276973)},
		{(0.78065173980259,0.35052466099245)--(0.73757929240127,0.40647879170533)--(0.68400242448273,0.39264458059972)--(0.67417167828656,0.3284451063664)--(0.73070856698794,0.29301468150288)},
		{(0.51173107082738,0.55559599825558)--(0.46888078913854,0.50567323874642)--(0.48344882740163,0.45437050808152)--(0.5291990062865,0.4359771852093)--(0.55170197411666,0.4431006298003)--(0.57116713378872,0.51656430017137)--(0.51490714181101,0.55628429897008)},
		{(0.42198060132998,9.2966782523004e-18)--(0.50783652063976,4.0921541492543e-18)--(0.51066058749423,0.068531486885363)--(0.44315769209849,0.096030757337953)--(0.42162657447946,0.079522330919597)},
		{(0.35434648910764,0.5868771168903)--(0.3248640250729,0.53781612410112)--(0.34753238882745,0.49856889378558)--(0.388482560027,0.49227367804315)--(0.41646568801911,0.51582103688304)--(0.40419477576661,0.58568832829771)},
		{(1.0,0.42725080647529)--(1.0,0.55764643983381)--(0.939423153296,0.55741955645906)--(0.89764058000683,0.5074560818892)--(0.93589772115998,0.42699763567665)},
		{(0.43068864847863,0.33178421506826)--(0.35008775853055,0.35613482788822)--(0.3471379812099,0.35374393205416)--(0.33650779393129,0.28850173339121)--(0.36554422157249,0.26256516276973)--(0.42713017585419,0.26933930250147)},
		{(0.0,0.0)--(0.061662486707602,3.4075156807917e-18)--(0.059127538435393,0.069293444671796)--(1.7277520457165e-18,0.085345674513439)},
		{(0.5066165818898,0.24904686688435)--(0.4361607070855,0.25969230276856)--(0.41875055490304,0.17713421701335)--(0.44818331753831,0.15244903606967)--(0.49921303152656,0.17157531809297)},
		{(0.02932558649174,0.38863580481067)--(0.067568804973452,0.35136838230796)--(0.11118580357978,0.36293561472925)--(0.11494878815847,0.41712994289509)--(0.070062125720471,0.43266444823048)}
    } {
        \draw[thick, color=\accentcolor] \polygon -- cycle;
    }

	% Domain.
    \draw[thick, color=\documentcolor]
        (0,0) -- (1,0) -- (1,1) -- (0,1) -- cycle;

\end{tikzpicture}
            \end{figure}
        }

        \only<5>{
            \begin{figure}[!ht]
                \centering
                \begin{tikzpicture}[scale=4.0, line join=round]

	% Domain, filled.
    \draw[thick, color=\documentcolor, fill=white]
        (0,0) -- (1,0) -- (1,1) -- (0,1) -- cycle;

	% Cells.
    \foreach \polygon in {
		{(0.81377427336204,0.46644534298129)--(0.75923295959791,0.45791497191854)--(0.73757929240127,0.40647879170533)--(0.78065173980259,0.35052466099245)--(0.81430805704037,0.3492625046274)--(0.8529169955575,0.39200429940237)},
		{(0.36741249518689,0.77626288981837)--(0.40173646438035,0.83622596474369)--(0.40164375289576,0.83669224245549)--(0.33340493932519,0.87655499835545)--(0.277181252921,0.83569973176466)--(0.30639749172945,0.76731167055377)},
		{(0.43437329544108,0.62740173590702)--(0.5083806772443,0.67409511180354)--(0.50858499820828,0.69293340123319)--(0.46204277488079,0.7375764324665)--(0.4049860583089,0.71881236943179)--(0.39169770109416,0.68722681252046)},
		{(0.87010808480848,0.76375127032153)--(0.86486066743379,0.7633854702304)--(0.80441905817567,0.67917067226413)--(0.83032969363452,0.62359767026028)--(0.89331717649906,0.62303886472077)--(0.94354430216006,0.70124383163287)},
		{(0.40355601843553,0.42872965074704)--(0.3588209602385,0.4015672576613)--(0.35008775853055,0.35613482788822)--(0.43068864847863,0.33178421506826)--(0.45336697589585,0.35334587285169)--(0.43463087640918,0.41856066646997)},
		{(0.85715374101961,0.92297987260062)--(0.80004476488986,0.88237356857976)--(0.80342102854991,0.80640238532619)--(0.86486066743379,0.7633854702304)--(0.87010808480848,0.76375127032153)--(0.93684678804366,0.85558688155856)},
		{(0.34753238882745,0.49856889378558)--(0.30906560949117,0.44768482568236)--(0.3588209602385,0.4015672576613)--(0.40355601843553,0.42872965074704)--(0.388482560027,0.49227367804315)},
		{(0.75923295959791,0.45791497191854)--(0.81377427336204,0.46644534298129)--(0.84082490850973,0.50471392257231)--(0.80286587331314,0.57968601144072)--(0.7382976811111,0.5699697035951)--(0.71813669466341,0.50408202800119)},
		{(0.68949293175191,0.18372328197331)--(0.7409210086195,0.15466243293744)--(0.79933896048372,0.19006791029744)--(0.79443195417494,0.2353533946542)--(0.73454000357343,0.26962355692421)--(0.6855195201363,0.21835061501379)},
		{(0.71113639015299,1.0)--(0.55852156786264,1.0)--(0.60149122246882,0.90673278977818)--(0.70601161988357,0.94316683857761)},
		{(0.12227589551397,-6.5238950862231e-19)--(0.19986164251846,-1.1237323234243e-18)--(0.19087676357043,0.077958222400555)--(0.12826813184992,0.061154270436431)},
		{(0.89617395870701,5.4180189498655e-18)--(1.0,0.0)--(1.0,0.058752397294144)--(0.90540525465644,0.069583827622574)},
		{(0.45336697589585,0.35334587285169)--(0.43068864847863,0.33178421506826)--(0.42713017585419,0.26933930250147)--(0.4361607070855,0.25969230276856)--(0.5066165818898,0.24904686688435)--(0.51747615642121,0.25601153533809)--(0.53014756501118,0.32835653292115)--(0.50708420792836,0.35035636952691)},
		{(0.3471379812099,0.35374393205416)--(0.27319827639826,0.36291999297592)--(0.25470476380115,0.34283439110708)--(0.27798534189451,0.27152908778946)--(0.33650779393129,0.28850173339121)},
		{(0.41875055490304,0.17713421701335)--(0.36191909399227,0.17375752435787)--(0.3449033247313,0.097867452252301)--(0.34891842643741,0.093030281653553)--(0.42162657447946,0.079522330919597)--(0.44315769209849,0.096030757337953)--(0.44818331753831,0.15244903606967)},
		{(0.7409210086195,0.15466243293744)--(0.68949293175191,0.18372328197331)--(0.64175878731343,0.1425851464857)--(0.64727813653168,0.095138966823452)--(0.70680226884436,0.069917833972296)--(0.74387319795567,0.098055537160779)},
		{(0.11494878815847,0.41712994289509)--(0.1357631592805,0.43052789908513)--(0.13784570335467,0.48933922664861)--(0.094081638532771,0.51067505521418)--(0.056211952034956,0.47257967565105)--(0.070062125720471,0.43266444823048)},
		{(0.86476362211263,1.0)--(0.71113639015299,1.0)--(0.70601161988357,0.94316683857761)--(0.72647040305996,0.90703260488626)--(0.80004476488986,0.88237356857976)--(0.85715374101961,0.92297987260062)},
		{(0.80342102854991,0.80640238532619)--(0.80004476488986,0.88237356857976)--(0.72647040305996,0.90703260488626)--(0.68089400269524,0.81533560654476)--(0.73057105824356,0.77504881108474)},
		{(0.90088496095405,0.078372760582186)--(0.92607611600945,0.13634057968)--(0.89769469044754,0.18641868288881)--(0.83266992292899,0.1679093939665)--(0.83412505049535,0.096075256820406)},
		{(0.51747615642121,0.25601153533809)--(0.5066165818898,0.24904686688435)--(0.49921303152656,0.17157531809297)--(0.54643146255917,0.14321173137306)--(0.58650600383281,0.16757388619041)--(0.58822586097716,0.23184518129324)},
		{(0.27018293298545,0.54114656345302)--(0.25262695560923,0.51441347177955)--(0.28771661402906,0.44552749126863)--(0.30906560949117,0.44768482568236)--(0.34753238882745,0.49856889378558)--(0.3248640250729,0.53781612410112)},
		{(0.89764058000683,0.5074560818892)--(0.84082490850973,0.50471392257231)--(0.81377427336204,0.46644534298129)--(0.8529169955575,0.39200429940237)--(0.9012650040962,0.39179917057072)--(0.93589772115998,0.42699763567665)},
		{(0.72704884336493,0.69832391192792)--(0.73057105824356,0.77504881108474)--(0.68089400269524,0.81533560654476)--(0.65203360452762,0.81302854821066)--(0.60168563045276,0.74507135264944)--(0.62847917748123,0.70174833694654)--(0.71419851369649,0.68849848453681)},
		{(0.81430805704037,0.3492625046274)--(0.78065173980259,0.35052466099245)--(0.73070856698794,0.29301468150288)--(0.73454000357343,0.26962355692421)--(0.79443195417494,0.2353533946542)--(0.8483200717027,0.28110781218028)},
		{(0.26227070461306,0.1671848011338)--(0.22656771464951,0.17919876762108)--(0.18832616281389,0.16340518200672)--(0.17763208949406,0.135999459889)--(0.19457272517073,0.083498197174869)--(0.26193615148853,0.086523888431298)--(0.27756497004741,0.10263598416091)},
		{(0.28771661402906,0.44552749126863)--(0.25262695560923,0.51441347177955)--(0.2013224108042,0.50151646237705)--(0.21005394498235,0.42931809165649)--(0.26471063334888,0.42245576174845)},
		{(0.061662486707602,3.4075156807917e-18)--(0.12227589551397,-6.5238950862231e-19)--(0.12826813184992,0.061154270436431)--(0.093716962981599,0.090157628363625)--(0.059127538435393,0.069293444671796)},
		{(0.13784570335467,0.48933922664861)--(0.1357631592805,0.43052789908513)--(0.18932285452885,0.41326685130257)--(0.21005394498235,0.42931809165649)--(0.2013224108042,0.50151646237705)--(0.18355624333685,0.51039392719948)},
		{(0.05502355466207,0.6439219076586)--(0.075705740334593,0.5626589437364)--(0.0846824115434,0.55765118315718)--(0.14320805808351,0.58732992451543)--(0.11610429352201,0.67207753949949)--(0.096409201823942,0.67404340318591)},
		{(0.27699152395929,0.26994362912223)--(0.22210332898952,0.25169595854152)--(0.22656771464951,0.17919876762108)--(0.26227070461306,0.1671848011338)--(0.30145693717806,0.19669714013291)},
		{(0.55852156786264,1.0)--(0.5223377782225,1.0)--(0.47826990299094,0.90645345131304)--(0.53335982731871,0.85501455453458)--(0.59266654390433,0.879667353541)--(0.60149122246882,0.90673278977818)},
		{(0.3449033247313,0.097867452252301)--(0.36191909399227,0.17375752435787)--(0.34559181978407,0.19135712502723)--(0.30145693717806,0.19669714013291)--(0.26227070461306,0.1671848011338)--(0.27756497004741,0.10263598416091)},
		{(0.4361607070855,0.25969230276856)--(0.42713017585419,0.26933930250147)--(0.36554422157249,0.26256516276973)--(0.34559181978407,0.19135712502723)--(0.36191909399227,0.17375752435787)--(0.41875055490304,0.17713421701335)},
		{(1.5376327988901e-17,0.64772798558207)--(1.6034554889537e-17,0.55058297186392)--(0.075705740334593,0.5626589437364)--(0.05502355466207,0.6439219076586)},
		{(0.17134169331082,0.57938374097505)--(0.18355624333685,0.51039392719948)--(0.2013224108042,0.50151646237705)--(0.25262695560923,0.51441347177955)--(0.27018293298545,0.54114656345302)--(0.25502554267202,0.58279516484096)--(0.20301198576883,0.59939207586982)},
		{(0.20301198576883,0.59939207586982)--(0.25502554267202,0.58279516484096)--(0.29959034399522,0.6369959559015)--(0.23001348019483,0.68383060276134)--(0.20636245832899,0.66983963838826)},
		{(2.7170782714942e-18,0.22137505583613)--(-3.2495733020338e-18,0.13512764986591)--(0.08220092118446,0.15466338357064)--(0.082398760918983,0.15605657582294)--(0.046633614663434,0.21938783290738)},
		{(0.60783424510423,-1.2794324155343e-17)--(0.70868263841806,-1.7121612108467e-17)--(0.70680226884436,0.069917833972296)--(0.64727813653168,0.095138966823452)--(0.60686045679871,0.059896034265045)},
		{(0.88651321148548,0.27353618403783)--(0.8483200717027,0.28110781218028)--(0.79443195417494,0.2353533946542)--(0.79933896048372,0.19006791029744)--(0.83266992292899,0.1679093939665)--(0.89769469044754,0.18641868288881)--(0.91004711516282,0.21507335330943)},
		{(0.89331717649906,0.62303886472077)--(0.83032969363452,0.62359767026028)--(0.80286587331314,0.57968601144072)--(0.84082490850973,0.50471392257231)--(0.89764058000683,0.5074560818892)--(0.939423153296,0.55741955645906)},
		{(0.19986164251846,-1.1237323234243e-18)--(0.2746330686814,-1.7745093106029e-20)--(0.26193615148853,0.086523888431298)--(0.19457272517073,0.083498197174869)--(0.19087676357043,0.077958222400555)},
		{(1.0,0.55764643983381)--(1.0,0.70413255616401)--(0.94354430216006,0.70124383163286)--(0.89331717649906,0.62303886472077)--(0.939423153296,0.55741955645906)},
		{(0.13133932211366,0.30081906788868)--(0.18073157431698,0.26585067423534)--(0.18945151817242,0.26764872284988)--(0.21047334568522,0.34170913727606)--(0.18913323621749,0.35978679702952)--(0.1343623311339,0.34355393370436)},
		{(0.33601345693271,-2.6880139561383e-18)--(0.42198060132998,1.2434732069417e-19)--(0.42162657447946,0.079522330919597)--(0.34891842643741,0.093030281653553)},
		{(0.31587490906051,0.63990510205951)--(0.29959034399522,0.6369959559015)--(0.25502554267202,0.58279516484096)--(0.27018293298545,0.54114656345302)--(0.3248640250729,0.53781612410112)--(0.35434648910764,0.5868771168903)},
		{(0.26471063334888,0.42245576174845)--(0.21005394498235,0.42931809165649)--(0.18932285452885,0.41326685130257)--(0.18913323621749,0.35978679702952)--(0.21047334568522,0.34170913727606)--(0.25470476380115,0.34283439110708)--(0.27319827639826,0.36291999297592)},
		{(0.083046851941488,0.25643782319375)--(0.13702408055966,0.21168455741157)--(0.14211272985908,0.21201871064309)--(0.18073157431698,0.26585067423534)--(0.13133932211366,0.30081906788868)--(0.083479811319291,0.27406640426661)},
		{(0.052111004327275,0.75459090538481)--(0.096409201823942,0.67404340318591)--(0.11610429352201,0.67207753949949)--(0.14424787658226,0.6876940756699)--(0.15790469123836,0.78029851948679)--(0.095071274608359,0.79123815887059)},
		{(0.093716962981599,0.090157628363625)--(0.12826813184992,0.061154270436431)--(0.19087676357043,0.077958222400555)--(0.19457272517073,0.083498197174869)--(0.17763208949406,0.135999459889)--(0.097995603708305,0.12822258324854)},
		{(0.65984709131333,0.49171876411262)--(0.61842882474402,0.53009581684405)--(0.57116713378872,0.51656430017137)--(0.55170197411666,0.4431006298003)--(0.60287487120645,0.41309615236067)--(0.6429799752324,0.42440434156506)},
		{(0.388482560027,0.49227367804315)--(0.40355601843553,0.42872965074704)--(0.43463087640918,0.41856066646997)--(0.48344882740163,0.45437050808152)--(0.46888078913854,0.50567323874642)--(0.41646568801911,0.51582103688304)},
		{(0.2746330686814,-1.7745093106029e-20)--(0.33601345693271,2.6316515385181e-18)--(0.34891842643741,0.093030281653553)--(0.3449033247313,0.097867452252301)--(0.27756497004741,0.10263598416091)--(0.26193615148853,0.086523888431298)},
		{(0.41646568801911,0.51582103688304)--(0.46888078913854,0.50567323874642)--(0.51173107082738,0.55559599825558)--(0.4319737972028,0.61146700444916)--(0.40419477576661,0.58568832829771)},
		{(0.40173646438035,0.83622596474369)--(0.36741249518689,0.77626288981837)--(0.4049860583089,0.71881236943179)--(0.46204277488079,0.7375764324665)--(0.47229351202466,0.79005207789208)},
		{(0.9012650040962,0.39179917057072)--(0.8529169955575,0.39200429940237)--(0.81430805704037,0.3492625046274)--(0.8483200717027,0.28110781218028)--(0.88651321148548,0.27353618403783)--(0.92847309663915,0.31554888410497)},
		{(0.59685918847794,0.63026005594793)--(0.55056341443673,0.627321238202)--(0.51490714181101,0.55628429897008)--(0.57116713378872,0.51656430017137)--(0.61842882474402,0.53009581684405)--(0.63344655969018,0.58950859602035)},
		{(0.38213822664684,1.0)--(0.23719520344954,1.0)--(0.24199831168402,0.95559295318398)--(0.33180829244124,0.9174469012483)--(0.38085488971948,0.97235962211152)},
		{(-3.6877698504793e-18,0.7553149622307)--(1.5316222932689e-18,0.64772798558207)--(0.05502355466207,0.6439219076586)--(0.096409201823942,0.67404340318591)--(0.052111004327275,0.75459090538481)},
		{(0.50858499820828,0.69293340123319)--(0.5083806772443,0.67409511180354)--(0.55056341443673,0.627321238202)--(0.59685918847794,0.63026005594793)--(0.62847917748123,0.70174833694654)--(0.60168563045276,0.74507135264944)--(0.57838182567746,0.74779793288446)},
		{(0.30639749172945,0.76731167055377)--(0.277181252921,0.83569973176466)--(0.26660883903922,0.83862561325022)--(0.18183945569358,0.79795414606115)--(0.17832003444239,0.7908790200153)--(0.24419439537107,0.72706041884685)--(0.29619454381117,0.74821429496436)},
		{(0.046633614663434,0.21938783290738)--(0.082398760918983,0.15605657582294)--(0.13702408055966,0.21168455741157)--(0.083046851941488,0.25643782319375)},
		{(0.29619454381117,0.74821429496436)--(0.24419439537107,0.72706041884685)--(0.23001348019483,0.68383060276134)--(0.29959034399522,0.6369959559015)--(0.31587490906051,0.63990510205951)--(0.33682886116887,0.67271577105837)},
		{(0.26660883903922,0.83862561325022)--(0.277181252921,0.83569973176466)--(0.33340493932519,0.87655499835545)--(0.33180829244124,0.9174469012483)--(0.24199831168402,0.95559295318398)--(0.2230856608485,0.91942162464549)},
		{(1.8156558448054e-17,0.3886600085509)--(1.7392932991998e-17,0.29801984598924)--(0.053378087503395,0.30362644622448)--(0.067568804973452,0.35136838230796)--(0.02932558649174,0.38863580481067)},
		{(0.80441905817567,0.67917067226413)--(0.86486066743379,0.7633854702304)--(0.80342102854991,0.80640238532619)--(0.73057105824356,0.77504881108474)--(0.72704884336493,0.69832391192792)},
		{(0.40164375289576,0.83669224245549)--(0.40173646438035,0.83622596474369)--(0.47229351202466,0.79005207789208)--(0.52181571356122,0.81467654481876)--(0.53335982731871,0.85501455453458)--(0.47826990299094,0.90645345131304)--(0.44563014749115,0.90163220700976)},
		{(1.0,0.70413255616401)--(1.0,0.86136896446465)--(0.93684678804366,0.85558688155856)--(0.87010808480848,0.76375127032153)--(0.94354430216006,0.70124383163286)},
		{(0.60686045679871,0.059896034265045)--(0.64727813653168,0.095138966823452)--(0.64175878731343,0.1425851464857)--(0.58650600383281,0.16757388619041)--(0.54643146255917,0.14321173137306)--(0.54301461872237,0.092797036270243)},
		{(0.7382976811111,0.5699697035951)--(0.80286587331314,0.57968601144072)--(0.83032969363452,0.62359767026028)--(0.80441905817567,0.67917067226413)--(0.72704884336493,0.69832391192792)--(0.71419851369649,0.68849848453681)--(0.7031996012139,0.60475064770785)},
		{(0.17763208949406,0.135999459889)--(0.18832616281389,0.16340518200672)--(0.14211272985908,0.21201871064309)--(0.13702408055966,0.21168455741157)--(0.082398760918983,0.15605657582294)--(0.08220092118446,0.15466338357064)--(0.097995603708305,0.12822258324854)},
		{(0.18832616281389,0.16340518200672)--(0.22656771464951,0.17919876762108)--(0.22210332898952,0.25169595854152)--(0.18945151817242,0.26764872284988)--(0.18073157431698,0.26585067423534)--(0.14211272985908,0.21201871064309)},
		{(0.50783652063976,5.7524658570437e-19)--(0.60783424510423,-7.6741652620722e-19)--(0.60686045679871,0.059896034265045)--(0.54301461872237,0.092797036270243)--(0.51066058749423,0.068531486885363)},
		{(0.65203360452762,0.81302854821066)--(0.68089400269524,0.81533560654476)--(0.72647040305996,0.90703260488626)--(0.70601161988357,0.94316683857761)--(0.60149122246882,0.90673278977818)--(0.59266654390433,0.879667353541)},
		{(0.6855195201363,0.21835061501379)--(0.73454000357343,0.26962355692421)--(0.73070856698794,0.29301468150288)--(0.67417167828656,0.3284451063664)--(0.62594643337379,0.30737678237245)--(0.62023001744523,0.25384088092476)},
		{(0.5083806772443,0.67409511180354)--(0.43437329544108,0.62740173590702)--(0.4319737972028,0.61146700444916)--(0.51173107082738,0.55559599825558)--(0.51490714181101,0.55628429897008)--(0.55056341443673,0.627321238202)},
		{(0.08220092118446,0.15466338357064)--(-1.1578803197835e-18,0.13512764986591)--(1.7277520457165e-18,0.085345674513439)--(0.059127538435393,0.069293444671796)--(0.0937169629816,0.090157628363625)--(0.097995603708305,0.12822258324854)},
		{(1.0,0.058752397294144)--(1.0,0.13765144661038)--(0.92607611600945,0.13634057968)--(0.90088496095405,0.078372760582186)--(0.90540525465644,0.069583827622574)},
		{(0.1357631592805,0.43052789908513)--(0.11494878815847,0.41712994289509)--(0.11118580357978,0.36293561472925)--(0.1343623311339,0.34355393370436)--(0.18913323621749,0.35978679702952)--(0.18932285452885,0.41326685130257)},
		{(-1.1374511936236e-17,0.55058297186392)--(-1.071091355383e-17,0.4839874495359)--(0.056211952034956,0.47257967565105)--(0.094081638532771,0.51067505521418)--(0.0846824115434,0.55765118315718)--(0.075705740334594,0.5626589437364)},
		{(-2.1659226667175e-18,0.88316080301788)--(-1.152287776871e-18,0.7553149622307)--(0.052111004327275,0.75459090538481)--(0.095071274608359,0.79123815887059)--(0.071117184385222,0.8747996608725)},
		{(0.23719520344954,1.0)--(0.10768490850862,1.0)--(0.11826183840611,0.90964515917231)--(0.15891801509917,0.8938576092032)--(0.2230856608485,0.91942162464549)--(0.24199831168402,0.95559295318398)},
		{(0.60287487120645,0.41309615236067)--(0.55170197411666,0.4431006298003)--(0.5291990062865,0.4359771852093)--(0.50708420792836,0.35035636952691)--(0.53014756501118,0.32835653292115)--(0.58641460002033,0.33863084952938)},
		{(0.083479811319291,0.27406640426661)--(0.13133932211366,0.30081906788868)--(0.1343623311339,0.34355393370436)--(0.11118580357978,0.36293561472925)--(0.067568804973452,0.35136838230796)--(0.053378087503395,0.30362644622448)},
		{(0.70868263841806,-1.4819045118148e-17)--(0.80532088406662,-1.3986236596368e-17)--(0.80466912304714,0.074630587799916)--(0.74387319795567,0.098055537160779)--(0.70680226884436,0.069917833972296)},
		{(0.54301461872237,0.092797036270243)--(0.54643146255917,0.14321173137306)--(0.49921303152656,0.17157531809297)--(0.44818331753831,0.15244903606967)--(0.44315769209849,0.096030757337953)--(0.51066058749423,0.068531486885363)},
		{(0.60168563045276,0.74507135264944)--(0.65203360452762,0.81302854821066)--(0.59266654390433,0.879667353541)--(0.53335982731871,0.85501455453458)--(0.52181571356122,0.81467654481876)--(0.57838182567746,0.74779793288446)},
		{(0.35008775853055,0.35613482788822)--(0.3588209602385,0.4015672576613)--(0.30906560949117,0.44768482568236)--(0.28771661402906,0.44552749126863)--(0.26471063334888,0.42245576174845)--(0.27319827639826,0.36291999297592)--(0.3471379812099,0.35374393205416)},
		{(0.15790469123836,0.78029851948679)--(0.14424787658226,0.6876940756699)--(0.20636245832899,0.66983963838826)--(0.23001348019483,0.68383060276134)--(0.24419439537107,0.72706041884685)--(0.17832003444239,0.7908790200153)},
		{(0.5223377782225,1.0)--(0.38213822664684,1.0)--(0.38085488971948,0.97235962211152)--(0.44563014749115,0.90163220700976)--(0.47826990299094,0.90645345131304)},
		{(0.71813669466341,0.50408202800119)--(0.7382976811111,0.5699697035951)--(0.7031996012139,0.60475064770785)--(0.63344655969018,0.58950859602035)--(0.61842882474402,0.53009581684405)--(0.65984709131333,0.49171876411262)},
		{(1.0,0.13765144661038)--(1.0,0.22274154013077)--(0.91004711516282,0.21507335330943)--(0.89769469044754,0.18641868288881)--(0.92607611600945,0.13634057968)},
		{(0.38085488971948,0.97235962211152)--(0.33180829244124,0.9174469012483)--(0.33340493932519,0.87655499835545)--(0.40164375289576,0.83669224245549)--(0.44563014749115,0.90163220700976)},
		{(0.53014756501118,0.32835653292115)--(0.51747615642121,0.25601153533809)--(0.58822586097716,0.23184518129324)--(0.62023001744523,0.25384088092476)--(0.62594643337379,0.30737678237245)--(0.58641460002033,0.33863084952938)},
		{(0.39169770109416,0.68722681252046)--(0.4049860583089,0.71881236943179)--(0.36741249518689,0.77626288981837)--(0.30639749172945,0.76731167055377)--(0.29619454381117,0.74821429496436)--(0.33682886116887,0.67271577105837)},
		{(0.071117184385222,0.8747996608725)--(0.095071274608359,0.79123815887059)--(0.15790469123836,0.78029851948679)--(0.17832003444239,0.7908790200153)--(0.18183945569358,0.79795414606115)--(0.15891801509917,0.8938576092032)--(0.11826183840611,0.90964515917231)},
		{(1.0,0.22274154013077)--(1.0,0.30839580497562)--(0.92847309663915,0.31554888410497)--(0.88651321148548,0.27353618403783)--(0.91004711516282,0.21507335330943)},
		{(0.80532088406662,2.7645539268707e-17)--(0.89617395870701,4.0256052199517e-17)--(0.90540525465644,0.069583827622574)--(0.90088496095405,0.078372760582186)--(0.83412505049535,0.096075256820406)--(0.80466912304714,0.074630587799916)},
		{(0.46204277488079,0.7375764324665)--(0.50858499820828,0.69293340123319)--(0.57838182567746,0.74779793288446)--(0.52181571356122,0.81467654481876)--(0.47229351202466,0.79005207789208)},
		{(-6.5897971399554e-18,0.4839874495359)--(-6.1295619438285e-18,0.3886600085509)--(0.02932558649174,0.38863580481067)--(0.070062125720471,0.43266444823048)--(0.056211952034956,0.47257967565105)},
		{(0.43437329544108,0.62740173590702)--(0.39169770109416,0.68722681252046)--(0.33682886116887,0.67271577105837)--(0.31587490906051,0.63990510205951)--(0.35434648910764,0.5868771168903)--(0.40419477576661,0.58568832829771)--(0.4319737972028,0.61146700444916)},
		{(0.59685918847794,0.63026005594793)--(0.63344655969018,0.58950859602035)--(0.7031996012139,0.60475064770785)--(0.71419851369649,0.68849848453681)--(0.62847917748123,0.70174833694654)},
		{(0.68949293175191,0.18372328197331)--(0.6855195201363,0.21835061501379)--(0.62023001744523,0.25384088092476)--(0.58822586097716,0.23184518129324)--(0.58650600383281,0.16757388619041)--(0.64175878731343,0.1425851464857)},
		{(0.094081638532771,0.51067505521418)--(0.13784570335467,0.48933922664861)--(0.18355624333685,0.51039392719948)--(0.17134169331082,0.57938374097505)--(0.14320805808351,0.58732992451543)--(0.0846824115434,0.55765118315718)},
		{(1.0,0.86136896446465)--(1.0,1.0)--(0.86476362211263,1.0)--(0.85715374101961,0.92297987260062)--(0.93684678804366,0.85558688155856)},
		{(0.10768490850862,1.0)--(0.0,1.0)--(-2.1659226667175e-18,0.88316080301788)--(0.071117184385222,0.8747996608725)--(0.11826183840611,0.90964515917231)},
		{(0.18183945569358,0.79795414606115)--(0.26660883903922,0.83862561325022)--(0.2230856608485,0.91942162464549)--(0.15891801509917,0.8938576092032)},
		{(1.0,0.30839580497562)--(1.0,0.42725080647529)--(0.93589772115998,0.42699763567665)--(0.9012650040962,0.39179917057072)--(0.92847309663915,0.31554888410497)},
		{(3.5910707340084e-18,0.29801984598924)--(6.5685858886698e-18,0.22137505583613)--(0.046633614663434,0.21938783290738)--(0.083046851941488,0.25643782319375)--(0.083479811319291,0.27406640426661)--(0.053378087503395,0.30362644622448)},
		{(0.11610429352201,0.67207753949949)--(0.14320805808351,0.58732992451543)--(0.17134169331082,0.57938374097505)--(0.20301198576883,0.59939207586982)--(0.20636245832899,0.66983963838826)--(0.14424787658226,0.6876940756699)},
		{(0.83412505049535,0.096075256820406)--(0.83266992292899,0.1679093939665)--(0.79933896048372,0.19006791029744)--(0.7409210086195,0.15466243293744)--(0.74387319795567,0.098055537160779)--(0.80466912304714,0.074630587799916)},
		{(0.68400242448273,0.39264458059972)--(0.6429799752324,0.42440434156506)--(0.60287487120645,0.41309615236067)--(0.58641460002033,0.33863084952938)--(0.62594643337379,0.30737678237245)--(0.67417167828656,0.3284451063664)},
		{(0.73757929240127,0.40647879170533)--(0.75923295959791,0.45791497191854)--(0.71813669466341,0.50408202800119)--(0.65984709131333,0.49171876411262)--(0.6429799752324,0.42440434156506)--(0.68400242448273,0.39264458059972)},
		{(0.43463087640918,0.41856066646997)--(0.45336697589585,0.35334587285169)--(0.50708420792836,0.35035636952691)--(0.5291990062865,0.4359771852093)--(0.48344882740163,0.45437050808152)},
		{(0.22210332898952,0.25169595854152)--(0.27699152395929,0.26994362912223)--(0.27798534189451,0.27152908778946)--(0.25470476380115,0.34283439110708)--(0.21047334568522,0.34170913727606)--(0.18945151817242,0.26764872284988)},
		{(0.33650779393129,0.28850173339121)--(0.27798534189451,0.27152908778946)--(0.27699152395929,0.26994362912223)--(0.30145693717806,0.19669714013291)--(0.34559181978407,0.19135712502723)--(0.36554422157249,0.26256516276973)},
		{(0.78065173980259,0.35052466099245)--(0.73757929240127,0.40647879170533)--(0.68400242448273,0.39264458059972)--(0.67417167828656,0.3284451063664)--(0.73070856698794,0.29301468150288)},
		{(0.51173107082738,0.55559599825558)--(0.46888078913854,0.50567323874642)--(0.48344882740163,0.45437050808152)--(0.5291990062865,0.4359771852093)--(0.55170197411666,0.4431006298003)--(0.57116713378872,0.51656430017137)--(0.51490714181101,0.55628429897008)},
		{(0.42198060132998,9.2966782523004e-18)--(0.50783652063976,4.0921541492543e-18)--(0.51066058749423,0.068531486885363)--(0.44315769209849,0.096030757337953)--(0.42162657447946,0.079522330919597)},
		{(0.35434648910764,0.5868771168903)--(0.3248640250729,0.53781612410112)--(0.34753238882745,0.49856889378558)--(0.388482560027,0.49227367804315)--(0.41646568801911,0.51582103688304)--(0.40419477576661,0.58568832829771)},
		{(1.0,0.42725080647529)--(1.0,0.55764643983381)--(0.939423153296,0.55741955645906)--(0.89764058000683,0.5074560818892)--(0.93589772115998,0.42699763567665)},
		{(0.43068864847863,0.33178421506826)--(0.35008775853055,0.35613482788822)--(0.3471379812099,0.35374393205416)--(0.33650779393129,0.28850173339121)--(0.36554422157249,0.26256516276973)--(0.42713017585419,0.26933930250147)},
		{(0.0,0.0)--(0.061662486707602,3.4075156807917e-18)--(0.059127538435393,0.069293444671796)--(1.7277520457165e-18,0.085345674513439)},
		{(0.5066165818898,0.24904686688435)--(0.4361607070855,0.25969230276856)--(0.41875055490304,0.17713421701335)--(0.44818331753831,0.15244903606967)--(0.49921303152656,0.17157531809297)},
		{(0.02932558649174,0.38863580481067)--(0.067568804973452,0.35136838230796)--(0.11118580357978,0.36293561472925)--(0.11494878815847,0.41712994289509)--(0.070062125720471,0.43266444823048)}
    } {
        \draw[ultra thin, color=\documentcolor, opacity=0.25] \polygon -- cycle;
    }
    
    % Cells.
    \foreach \polygon in {
		{(0.90562086673874,0.49745731734894)--(0.84964648501335,0.53928911983049)--(0.81370165301096,0.52736499787452)--(0.8007778398891,0.4599745910587)--(0.83830034032296,0.43123635442729)--(0.90332534096838,0.45661227040353)},
		{(0.43025505707094,0.85761111287619)--(0.439061613054,0.90102324862771)--(0.38639873239114,0.94248964143183)--(0.3362779132294,0.9097346449815)--(0.34039514150195,0.8459774418655)--(0.34850454506219,0.83910337452696)},
		{(0.41960644813702,0.76076625934297)--(0.43497718019503,0.71938447619238)--(0.49650365527531,0.70994989273511)--(0.52534847863401,0.74460186576363)--(0.49862928696564,0.81151654105998)--(0.45835143401538,0.81604376787072)},
		{(0.90671027792458,0.81411931008239)--(0.85765138373302,0.80955051545943)--(0.83790156000265,0.73211635896392)--(0.8604940273552,0.70719355410405)--(0.90814401638341,0.70795658852666)--(0.94673361824338,0.77371752222486)},
		{(0.4751613409377,0.44151674314935)--(0.4129547037095,0.42947908675186)--(0.39954754712646,0.40726327112996)--(0.42341532865985,0.34383551279366)--(0.49200407224827,0.34636662618389)--(0.51141822893407,0.37800387101546)},
		{(0.89872511778609,0.92046603344255)--(0.83534136701563,0.90145174413154)--(0.82885329691237,0.83443746367247)--(0.85765138373302,0.80955051545943)--(0.90671027792458,0.81411931008239)--(0.93546381809738,0.88744169879805)},
		{(0.41307582812368,0.53250224411983)--(0.38416850586663,0.49547567047401)--(0.4129547037095,0.42947908675186)--(0.4751613409377,0.44151674314935)--(0.49034178123502,0.47091119134154)--(0.45787096486917,0.52996812060041)},
		{(0.81370165301096,0.52736499787452)--(0.84964648501335,0.53928911983049)--(0.87264270846591,0.60125240420503)--(0.83676682446274,0.63778614798418)--(0.77960877800131,0.63045738124194)--(0.76810727094859,0.55768934658571)},
		{(0.73771722997482,0.19370941566284)--(0.79947883339628,0.16792494306375)--(0.82938320652562,0.18770003404512)--(0.83429833271736,0.2568333450831)--(0.77387850186301,0.28075926513421)--(0.74280602817792,0.2633168184927)},
		{(0.80019557685243,1.0)--(0.69409490600805,1.0)--(0.69380893033619,0.93886438567463)--(0.74061201671033,0.90481746293136)--(0.79830652878874,0.9305431229772)},
		{(0.17162073339073,1.6394271461714e-18)--(0.23365308807361,1.7367166623212e-18)--(0.25041478300468,0.081611205832619)--(0.23543050583935,0.10190903177243)--(0.18754064066743,0.10705758438686)--(0.15742045948643,0.072781138213485)},
		{(0.92118477967377,8.7100309126292e-18)--(1.0,0.0)--(1.0,0.1017721264846)--(0.91877817870729,0.093562841047918)},
		{(0.51141822893407,0.37800387101546)--(0.49200407224827,0.34636662618389)--(0.51369814765298,0.28579605366575)--(0.55562384132193,0.2731671852179)--(0.59337533904548,0.29548058688286)--(0.59669976992875,0.3664850638665)--(0.56758692480709,0.38788786901398)},
		{(0.42341532865985,0.34383551279366)--(0.39954754712646,0.40726327112996)--(0.33571803465189,0.3993410856253)--(0.31391253481761,0.35504452404102)--(0.34010653205604,0.3145727941542)--(0.40391634620773,0.31430714048907)},
		{(0.47634554951811,0.18883945644299)--(0.43008332996357,0.1546868218746)--(0.43044293484973,0.10112503702838)--(0.48908546549103,0.079415129582681)--(0.52692325543634,0.10290397147954)--(0.52062018995623,0.17130903915791)},
		{(0.79947883339628,0.16792494306375)--(0.73771722997482,0.19370941566284)--(0.70512093060736,0.17459887224965)--(0.70191732877822,0.10494698038419)--(0.74807260579378,0.081494092512622)--(0.79671105343376,0.10386391156366)},
		{(0.15759072575741,0.51702466680117)--(0.15593199973037,0.59378548549754)--(0.091974921981432,0.60200545404424)--(0.062479596074471,0.55461108764192)--(0.090334998477803,0.5105201576241)},
		{(0.90680120300739,1.0)--(0.80019557685243,1.0)--(0.79830652878874,0.9305431229772)--(0.83534136701563,0.90145174413154)--(0.89872511778609,0.92046603344255)},
		{(0.82885329691237,0.83443746367247)--(0.83534136701563,0.90145174413154)--(0.79830652878874,0.9305431229772)--(0.74061201671033,0.90481746293136)--(0.7349356515337,0.84345835452141)--(0.76668198957367,0.82014503182108)},
		{(0.91271273664315,0.25689411904908)--(0.86024354144915,0.27228365156758)--(0.83429833271736,0.2568333450831)--(0.82938320652562,0.18770003404512)--(0.89585777469058,0.1634898855985)--(0.93239287598605,0.20022992740407)},
		{(0.65166539015414,0.27080032503853)--(0.59337533904548,0.29548058688286)--(0.55562384132193,0.2731671852179)--(0.56413378794781,0.19754467506834)--(0.61287670363839,0.17984726382333)--(0.64665276281395,0.19867656194114)},
		{(0.33273072785904,0.59201745391288)--(0.30263941099775,0.55177266151939)--(0.33118275533446,0.4950870597933)--(0.38416850586663,0.49547567047401)--(0.41307582812368,0.53250224411983)--(0.38830912710478,0.58809988488896)},
		{(0.91244065180808,0.6062954385183)--(0.87264270846591,0.60125240420503)--(0.84964648501335,0.53928911983049)--(0.90562086673874,0.49745731734894)--(0.9639873004165,0.54000827924375)},
		{(0.76119275347245,0.73699949460978)--(0.76668198957367,0.82014503182108)--(0.7349356515337,0.84345835452141)--(0.68802573327946,0.82818753870301)--(0.67031785177334,0.76315890060833)--(0.68780188208613,0.74023887820549)},
		{(0.77387850186301,0.28075926513421)--(0.83429833271736,0.2568333450831)--(0.86024354144915,0.27228365156758)--(0.86470240133282,0.34852266459074)--(0.83603929287926,0.36893428567055)--(0.77941790003513,0.35019630476697)},
		{(0.34160589697019,0.094513672527824)--(0.33155858885746,0.15563799507262)--(0.26897529043921,0.16733835457815)--(0.23543050583935,0.10190903177243)--(0.25041478300468,0.081611205832619)--(0.31983036119436,0.073052449092787)},
		{(0.33118275533446,0.4950870597933)--(0.30263941099775,0.55177266151939)--(0.25380798875983,0.5526194384777)--(0.2195515088868,0.50389409457052)--(0.25806478562061,0.45046005555177)--(0.30317607746907,0.4529211417038)},
		{(0.075010034820169,7.3506279061383e-19)--(0.17162073339073,-1.6816816415712e-18)--(0.15742045948643,0.072781138213485)--(0.086485529574877,0.081635364593292)},
		{(0.15593199973037,0.59378548549754)--(0.15759072575741,0.51702466680117)--(0.17481075348646,0.50192012981204)--(0.2195515088868,0.50389409457052)--(0.25380798875983,0.5526194384777)--(0.22097129842883,0.61071460758486)--(0.17929801850874,0.61385198935047)},
		{(0.075297295409805,0.73877396112494)--(0.099723556961436,0.69058794606744)--(0.15562110912384,0.68802351791194)--(0.18641782187622,0.72150151437522)--(0.15931631518952,0.78772111258074)--(0.10100386127141,0.78593004898778)},
		{(0.27310724622345,0.25053361869666)--(0.25045836513368,0.20227624362407)--(0.26897529043921,0.16733835457815)--(0.33155858885746,0.15563799507262)--(0.3661000967184,0.19021492301758)--(0.36572639063023,0.21281149275499)--(0.31655376524573,0.25920567249135)},
		{(0.69409490600805,1.0)--(0.58760378421051,1.0)--(0.58883658367539,0.93864567278327)--(0.63779877032203,0.90304341614145)--(0.69380893033619,0.93886438567463)},
		{(0.43044293484973,0.10112503702838)--(0.43008332996357,0.1546868218746)--(0.3661000967184,0.19021492301758)--(0.33155858885746,0.15563799507262)--(0.34160589697019,0.094513672527824)--(0.40448175620263,0.081719073239906)},
		{(0.42172692966925,0.26084124339668)--(0.36572639063023,0.21281149275499)--(0.3661000967184,0.19021492301758)--(0.43008332996357,0.1546868218746)--(0.47634554951811,0.18883945644299)--(0.46399205474278,0.24528359574481)},
		{(-2.4250333963585e-18,0.738887607386)--(4.0334572221406e-18,0.64955618583817)--(0.071762165041819,0.64637933734506)--(0.099723556961436,0.69058794606744)--(0.075297295409805,0.73877396112494)},
		{(0.22097129842883,0.61071460758486)--(0.25380798875983,0.5526194384777)--(0.30263941099775,0.55177266151939)--(0.33273072785904,0.59201745391288)--(0.3095388704372,0.64957990647192)--(0.2606647579083,0.6547465077279)},
		{(0.32650250881848,0.74398270321248)--(0.26880077293031,0.75467651375032)--(0.23382313120445,0.71763074899324)--(0.2606647579083,0.6547465077279)--(0.3095388704372,0.64957990647192)--(0.34147746269111,0.68661019501426)},
		{(-2.2359986723035e-17,0.27877773252863)--(-2.3837580172329e-17,0.18670401126797)--(0.07534211873675,0.18740617822472)--(0.096392616615221,0.24522214340638)--(0.074353229052474,0.27885841601452)},
		{(0.66040691263801,4.1890025364447e-18)--(0.74707185567762,4.2548661771588e-18)--(0.74807260579378,0.081494092512622)--(0.70191732877822,0.10494698038419)--(0.66094052300372,0.084859226540502)},
		{(0.91415706365105,0.35924220050188)--(0.86470240133282,0.34852266459074)--(0.86024354144915,0.27228365156758)--(0.91271273664315,0.25689411904908)--(0.96417188630523,0.30864267927584)},
		{(0.90814401638341,0.70795658852666)--(0.8604940273552,0.70719355410405)--(0.83676682446274,0.63778614798418)--(0.87264270846591,0.60125240420503)--(0.91244065180808,0.6062954385183)--(0.9491529504042,0.65873122068471)},
		{(0.23365308807361,-3.3477505575059e-18)--(0.32628143694767,2.5803824807885e-19)--(0.31983036119436,0.073052449092787)--(0.25041478300468,0.081611205832619)},
		{(1.0,0.66016748325509)--(1.0,0.77458628824672)--(0.94673361824338,0.77371752222486)--(0.90814401638341,0.70795658852666)--(0.9491529504042,0.65873122068471)},
		{(0.15425357771257,0.33629539947812)--(0.18476486038891,0.29544336901431)--(0.23396036591514,0.30034838204361)--(0.26040160220422,0.35173388212644)--(0.2268937956406,0.39890898912141)--(0.17955310904827,0.39738992577799)},
		{(0.40587680220629,-2.9667951719802e-18)--(0.48992125349184,-4.0389342343263e-18)--(0.48908546549103,0.079415129582681)--(0.43044293484973,0.10112503702838)--(0.40448175620263,0.081719073239906)},
		{(0.40382770580073,0.67612206011063)--(0.34147746269111,0.68661019501426)--(0.3095388704372,0.64957990647192)--(0.33273072785904,0.59201745391288)--(0.38830912710478,0.58809988488896)--(0.41926742897499,0.63048085523994)},
		{(0.30317607746907,0.4529211417038)--(0.25806478562061,0.45046005555177)--(0.2268937956406,0.39890898912141)--(0.26040160220422,0.35173388212644)--(0.31391253481761,0.35504452404102)--(0.33571803465189,0.3993410856253)},
		{(0.074353229052474,0.27885841601452)--(0.096392616615221,0.24522214340638)--(0.16174055514051,0.24576440244375)--(0.18476486038891,0.29544336901431)--(0.15425357771257,0.33629539947812)--(0.096914643160764,0.33401537300282)},
		{(0.10544294737487,0.89435209215953)--(0.073916802589346,0.82932116499525)--(0.10100386127141,0.78593004898778)--(0.15931631518952,0.78772111258074)--(0.18080173347841,0.81699036861983)--(0.15190615583306,0.89009117210229)},
		{(0.15742045948643,0.072781138213485)--(0.18754064066743,0.10705758438686)--(0.16226733481266,0.15984078446317)--(0.095969171266849,0.15820612677018)--(0.077785334870339,0.093268733958092)--(0.086485529574877,0.081635364593292)},
		{(0.72463388956437,0.54269559300798)--(0.67446146919368,0.57507159324725)--(0.63797036149734,0.56086228377793)--(0.63074047496544,0.47875490164281)--(0.6609860319228,0.45893561983003)--(0.7129412682258,0.47288353123979)},
		{(0.45787096486917,0.52996812060041)--(0.49034178123502,0.47091119134154)--(0.55216201591839,0.4848763144971)--(0.55139757442056,0.5556585621165)--(0.49617510278083,0.57455891874653)},
		{(0.32628143694767,2.5803824807885e-19)--(0.40587680220629,-2.9667951719802e-18)--(0.40448175620263,0.081719073239906)--(0.34160589697019,0.094513672527824)--(0.31983036119436,0.073052449092787)},
		{(0.49617510278083,0.57455891874653)--(0.55139757442056,0.5556585621165)--(0.59114493366791,0.58104088182392)--(0.58301801629956,0.64682676411061)--(0.51236359176674,0.65572323824777)--(0.48396747686858,0.61838879428297)},
		{(0.439061613054,0.90102324862771)--(0.43025505707094,0.85761111287619)--(0.45835143401538,0.81604376787072)--(0.49862928696564,0.81151654105998)--(0.53886520818497,0.84806503605021)--(0.53058029364196,0.90099660511096)--(0.48760913777198,0.92532574237882)},
		{(0.90332534096838,0.45661227040353)--(0.83830034032296,0.43123635442729)--(0.83603929287926,0.36893428567055)--(0.86470240133282,0.34852266459074)--(0.91415706365105,0.35924220050188)--(0.9392022151889,0.42197668711201)},
		{(0.66377499068603,0.66963966040744)--(0.6033778721243,0.66931898593838)--(0.58301801629956,0.64682676411061)--(0.59114493366791,0.58104088182392)--(0.63797036149734,0.56086228377793)--(0.67446146919368,0.57507159324725)--(0.68882881153311,0.64088601503834)},
		{(0.388733913137,1.0)--(0.28284840015166,1.0)--(0.28488988344102,0.93715032027721)--(0.3362779132294,0.9097346449815)--(0.38639873239114,0.94248964143183)},
		{(-9.7620980360604e-18,0.83111983168827)--(-2.4250333963585e-18,0.738887607386)--(0.075297295409805,0.73877396112494)--(0.10100386127141,0.78593004898778)--(0.073916802589346,0.82932116499525)},
		{(0.6033778721243,0.66931898593838)--(0.66377499068603,0.66963966040744)--(0.68780188208613,0.74023887820549)--(0.67031785177334,0.76315890060833)--(0.60209322151793,0.7656981620798)--(0.58225899722008,0.7433741875493)},
		{(0.34850454506219,0.83910337452696)--(0.34039514150195,0.8459774418655)--(0.25627073354994,0.83167390407978)--(0.25037144272769,0.82338973935646)--(0.26880077293031,0.75467651375032)--(0.32650250881848,0.74398270321248)--(0.3579089973552,0.77651570635267)},
		{(0.16174055514051,0.24576440244375)--(0.096392616615221,0.24522214340638)--(0.07534211873675,0.18740617822472)--(0.095969171266849,0.15820612677018)--(0.16226733481266,0.15984078446317)--(0.18498554460672,0.2030287379494)},
		{(0.43497718019503,0.71938447619237)--(0.41960644813702,0.76076625934297)--(0.3579089973552,0.77651570635267)--(0.32650250881848,0.74398270321248)--(0.34147746269111,0.68661019501426)--(0.40382770580073,0.67612206011063)},
		{(0.34039514150195,0.8459774418655)--(0.3362779132294,0.9097346449815)--(0.28488988344102,0.93715032027721)--(0.2440991387246,0.90612103394854)--(0.25627073354994,0.83167390407978)},
		{(-4.0196166610004e-18,0.46062128402398)--(-4.04488462454e-18,0.36975068838999)--(0.071761178403807,0.37062088708733)--(0.094779729161039,0.42552694771552)--(0.069655043672125,0.46252170906469)},
		{(0.83790156000265,0.73211635896392)--(0.85765138373302,0.80955051545943)--(0.82885329691237,0.83443746367247)--(0.76668198957367,0.82014503182108)--(0.76119275347245,0.73699949460978)--(0.76501151428016,0.73318890229155)},
		{(0.58931938009406,0.83016295112284)--(0.63751433633477,0.86498201132943)--(0.63779877032203,0.90304341614145)--(0.58883658367539,0.93864567278327)--(0.53058029364196,0.90099660511096)--(0.53886520818497,0.84806503605021)},
		{(1.0,0.77458628824672)--(1.0,0.89221791742536)--(0.93546381809738,0.88744169879805)--(0.90671027792458,0.81411931008239)--(0.94673361824338,0.77371752222486)},
		{(0.70191732877822,0.10494698038419)--(0.70512093060736,0.17459887224965)--(0.64665276281395,0.19867656194114)--(0.61287670363839,0.17984726382333)--(0.61334150870758,0.10502616284931)--(0.66094052300372,0.084859226540502)},
		{(0.77960877800131,0.63045738124194)--(0.83676682446274,0.63778614798418)--(0.8604940273552,0.70719355410405)--(0.83790156000265,0.73211635896392)--(0.76501151428016,0.73318890229155)--(0.75843704503209,0.64936436375305)},
		{(0.18754064066743,0.10705758438686)--(0.23543050583935,0.10190903177243)--(0.26897529043921,0.16733835457815)--(0.25045836513368,0.20227624362407)--(0.18498554460672,0.2030287379494)--(0.16226733481266,0.15984078446317)},
		{(0.25045836513368,0.20227624362407)--(0.27310724622345,0.25053361869666)--(0.23396036591514,0.30034838204361)--(0.18476486038891,0.29544336901431)--(0.16174055514051,0.24576440244375)--(0.18498554460672,0.2030287379494)},
		{(0.57579246010271,-2.5769654542781e-17)--(0.66040691263801,-2.5225589931724e-17)--(0.66094052300372,0.084859226540502)--(0.61334150870758,0.10502616284931)--(0.57444644094745,0.084653717689775)},
		{(0.68802573327946,0.82818753870301)--(0.7349356515337,0.84345835452141)--(0.74061201671033,0.90481746293136)--(0.69380893033619,0.93886438567463)--(0.63779877032203,0.90304341614145)--(0.63751433633477,0.86498201132943)},
		{(0.69037814035163,0.36010268815439)--(0.68418875779986,0.28843130921516)--(0.74280602817792,0.2633168184927)--(0.77387850186301,0.28075926513421)--(0.77941790003513,0.35019630476697)--(0.74153850688122,0.3763468622677)},
		{(0.52534847863401,0.74460186576363)--(0.49650365527531,0.70994989273511)--(0.51236359176674,0.65572323824777)--(0.58301801629956,0.64682676411061)--(0.6033778721243,0.66931898593838)--(0.58225899722008,0.7433741875493)},
		{(0.07534211873675,0.18740617822472)--(8.7360618592155e-18,0.18670401126797)--(7.9754597912455e-18,0.096242174388719)--(0.077785334870339,0.093268733958092)--(0.095969171266849,0.15820612677018)},
		{(1.0,0.1017721264846)--(1.0,0.19140579317574)--(0.93239287598605,0.20022992740407)--(0.89585777469058,0.1634898855985)--(0.90267708939022,0.10524041471726)--(0.91877817870729,0.093562841047918)},
		{(0.1539172541449,0.42682959832593)--(0.17955310904827,0.39738992577799)--(0.2268937956406,0.39890898912141)--(0.25806478562061,0.45046005555177)--(0.2195515088868,0.50389409457052)--(0.17481075348646,0.50192012981204)},
		{(4.0334572221406e-18,0.64955618583817)--(2.6410555968065e-19,0.55501911596182)--(0.062479596074471,0.55461108764192)--(0.091974921981432,0.60200545404424)--(0.071762165041819,0.64637933734506)},
		{(1.445338404923e-18,0.91774728313323)--(-2.8252575743211e-18,0.83111983168827)--(0.073916802589346,0.82932116499525)--(0.10544294737487,0.89435209215953)--(0.087187661788198,0.91762708112664)},
		{(0.28284840015166,1.0)--(0.17980756855261,1.0)--(0.18915376767155,0.92224702668953)--(0.2440991387246,0.90612103394854)--(0.28488988344102,0.93715032027721)},
		{(0.56758692480709,0.38788786901398)--(0.59669976992875,0.3664850638665)--(0.65555621324255,0.38349657492187)--(0.6609860319228,0.45893561983003)--(0.63074047496544,0.47875490164281)--(0.5770556404014,0.46733679559338)},
		{(0.096914643160764,0.33401537300282)--(0.15425357771257,0.33629539947812)--(0.17955310904827,0.39738992577799)--(0.1539172541449,0.42682959832593)--(0.094779729161039,0.42552694771552)--(0.071761178403807,0.37062088708733)},
		{(0.74707185567762,-1.6560246142566e-17)--(0.83588912585082,-1.4264266249735e-17)--(0.83515530993165,0.079572391411158)--(0.79671105343376,0.10386391156366)--(0.74807260579378,0.081494092512622)},
		{(0.61334150870758,0.10502616284931)--(0.61287670363839,0.17984726382333)--(0.56413378794781,0.19754467506834)--(0.52062018995623,0.17130903915791)--(0.52692325543634,0.10290397147954)--(0.57444644094745,0.084653717689775)},
		{(0.60209322151793,0.7656981620798)--(0.67031785177334,0.76315890060833)--(0.68802573327946,0.82818753870301)--(0.63751433633477,0.86498201132943)--(0.58931938009406,0.83016295112284)},
		{(0.39954754712646,0.40726327112996)--(0.4129547037095,0.42947908675186)--(0.38416850586663,0.49547567047401)--(0.33118275533446,0.4950870597933)--(0.30317607746907,0.4529211417038)--(0.33571803465189,0.3993410856253)},
		{(0.18080173347841,0.81699036861983)--(0.15931631518952,0.78772111258074)--(0.18641782187622,0.72150151437522)--(0.23382313120445,0.71763074899324)--(0.26880077293031,0.75467651375032)--(0.25037144272769,0.82338973935646)},
		{(0.58760378421051,1.0)--(0.48771105281588,1.0)--(0.48760913777198,0.92532574237882)--(0.53058029364196,0.90099660511096)--(0.58883658367539,0.93864567278327)},
		{(0.76810727094859,0.55768934658571)--(0.77960877800131,0.63045738124194)--(0.75843704503209,0.64936436375305)--(0.68882881153311,0.64088601503834)--(0.67446146919368,0.57507159324725)--(0.72463388956437,0.54269559300798)},
		{(1.0,0.19140579317574)--(1.0,0.30836423392629)--(0.96417188630523,0.30864267927584)--(0.91271273664315,0.25689411904908)--(0.93239287598605,0.20022992740407)},
		{(0.48771105281588,1.0)--(0.388733913137,1.0)--(0.38639873239114,0.94248964143183)--(0.439061613054,0.90102324862771)--(0.48760913777198,0.92532574237882)},
		{(0.59669976992875,0.3664850638665)--(0.59337533904548,0.29548058688286)--(0.65166539015414,0.27080032503853)--(0.68418875779986,0.28843130921516)--(0.69037814035163,0.36010268815439)--(0.65555621324255,0.38349657492187)},
		{(0.43025505707094,0.85761111287619)--(0.34850454506219,0.83910337452696)--(0.3579089973552,0.77651570635267)--(0.41960644813702,0.76076625934297)--(0.45835143401538,0.81604376787072)},
		{(0.17980756855261,1.0)--(0.098282855212861,1.0)--(0.087187661788198,0.91762708112664)--(0.10544294737487,0.89435209215953)--(0.15190615583306,0.89009117210229)--(0.18915376767155,0.92224702668953)},
		{(1.0,0.30836423392629)--(1.0,0.4260285521325)--(0.9392022151889,0.42197668711201)--(0.91415706365105,0.35924220050188)--(0.96417188630523,0.30864267927584)},
		{(0.83588912585082,-3.8706852340109e-19)--(0.92118477967377,8.7100309126292e-18)--(0.91877817870729,0.093562841047918)--(0.90267708939022,0.10524041471726)--(0.83515530993165,0.079572391411158)},
		{(0.49862928696564,0.81151654105998)--(0.52534847863401,0.74460186576363)--(0.58225899722008,0.7433741875493)--(0.60209322151793,0.7656981620798)--(0.58931938009406,0.83016295112284)--(0.53886520818497,0.84806503605021)},
		{(2.6410555968065e-19,0.55501911596182)--(1.4808354815746e-18,0.46062128402398)--(0.069655043672125,0.46252170906469)--(0.090334998477803,0.5105201576241)--(0.062479596074471,0.55461108764192)},
		{(0.49650365527531,0.70994989273511)--(0.43497718019503,0.71938447619238)--(0.40382770580073,0.67612206011063)--(0.41926742897499,0.63048085523994)--(0.48396747686858,0.61838879428297)--(0.51236359176674,0.65572323824777)},
		{(0.66377499068603,0.66963966040744)--(0.68882881153311,0.64088601503834)--(0.75843704503209,0.64936436375305)--(0.76501151428016,0.73318890229155)--(0.76119275347245,0.73699949460978)--(0.68780188208613,0.74023887820549)},
		{(0.73771722997482,0.19370941566284)--(0.74280602817792,0.2633168184927)--(0.68418875779986,0.28843130921516)--(0.65166539015414,0.27080032503853)--(0.64665276281395,0.19867656194114)--(0.70512093060736,0.17459887224965)},
		{(0.091974921981432,0.60200545404424)--(0.15593199973037,0.59378548549754)--(0.17929801850874,0.61385198935047)--(0.15562110912384,0.68802351791194)--(0.099723556961436,0.69058794606744)--(0.071762165041819,0.64637933734506)},
		{(1.0,0.89221791742536)--(1.0,1.0)--(0.90680120300739,1.0)--(0.89872511778609,0.92046603344255)--(0.93546381809738,0.88744169879805)},
		{(0.098282855212861,1.0)--(0.0,1.0)--(3.2382657123805e-18,0.91774728313323)--(0.087187661788198,0.91762708112664)},
		{(0.15190615583306,0.89009117210229)--(0.18080173347841,0.81699036861983)--(0.25037144272769,0.82338973935646)--(0.25627073354994,0.83167390407978)--(0.2440991387246,0.90612103394854)--(0.18915376767155,0.92224702668953)},
		{(1.0,0.4260285521325)--(1.0,0.53758135732378)--(0.9639873004165,0.54000827924375)--(0.90562086673874,0.49745731734894)--(0.90332534096838,0.45661227040353)--(0.9392022151889,0.42197668711201)},
		{(-8.0568679736084e-19,0.36975068838999)--(3.2226216693698e-18,0.27877773252863)--(0.074353229052474,0.27885841601452)--(0.096914643160764,0.33401537300282)--(0.071761178403807,0.37062088708733)},
		{(0.18641782187622,0.72150151437522)--(0.15562110912384,0.68802351791194)--(0.17929801850874,0.61385198935047)--(0.22097129842883,0.61071460758486)--(0.2606647579083,0.6547465077279)--(0.23382313120445,0.71763074899324)},
		{(0.90267708939022,0.10524041471726)--(0.89585777469058,0.1634898855985)--(0.82938320652562,0.18770003404512)--(0.79947883339628,0.16792494306375)--(0.79671105343376,0.10386391156366)--(0.83515530993165,0.079572391411158)},
		{(0.7510082818126,0.44540964367599)--(0.7129412682258,0.47288353123979)--(0.6609860319228,0.45893561983003)--(0.65555621324255,0.38349657492187)--(0.69037814035163,0.36010268815439)--(0.74153850688122,0.3763468622677)},
		{(0.8007778398891,0.4599745910587)--(0.81370165301096,0.52736499787452)--(0.76810727094859,0.55768934658571)--(0.72463388956437,0.54269559300798)--(0.7129412682258,0.47288353123979)--(0.7510082818126,0.44540964367599)},
		{(0.49034178123502,0.47091119134154)--(0.4751613409377,0.44151674314935)--(0.51141822893407,0.37800387101546)--(0.56758692480709,0.38788786901398)--(0.5770556404014,0.46733679559338)--(0.55216201591839,0.4848763144971)},
		{(0.34010653205604,0.3145727941542)--(0.31391253481761,0.35504452404102)--(0.26040160220422,0.35173388212644)--(0.23396036591514,0.30034838204361)--(0.27310724622345,0.25053361869666)--(0.31655376524573,0.25920567249135)},
		{(0.40391634620773,0.31430714048907)--(0.34010653205604,0.3145727941542)--(0.31655376524573,0.25920567249135)--(0.36572639063023,0.21281149275499)--(0.42172692966924,0.26084124339668)},
		{(0.83830034032296,0.43123635442729)--(0.8007778398891,0.4599745910587)--(0.7510082818126,0.44540964367599)--(0.74153850688122,0.3763468622677)--(0.77941790003513,0.35019630476697)--(0.83603929287926,0.36893428567055)},
		{(0.55139757442056,0.5556585621165)--(0.55216201591839,0.4848763144971)--(0.5770556404014,0.46733679559338)--(0.63074047496544,0.47875490164281)--(0.63797036149734,0.56086228377793)--(0.59114493366791,0.58104088182392)},
		{(0.48992125349184,-6.3178208718301e-18)--(0.57579246010271,-4.9556037627779e-18)--(0.57444644094745,0.084653717689775)--(0.52692325543634,0.10290397147954)--(0.48908546549103,0.079415129582681)},
		{(0.41926742897499,0.63048085523994)--(0.38830912710478,0.58809988488896)--(0.41307582812368,0.53250224411983)--(0.45787096486917,0.52996812060041)--(0.49617510278083,0.57455891874653)--(0.48396747686858,0.61838879428297)},
		{(1.0,0.53758135732378)--(1.0,0.66016748325509)--(0.9491529504042,0.65873122068471)--(0.91244065180808,0.6062954385183)--(0.9639873004165,0.54000827924375)},
		{(0.51369814765298,0.28579605366575)--(0.49200407224827,0.34636662618389)--(0.42341532865985,0.34383551279366)--(0.40391634620773,0.31430714048907)--(0.42172692966924,0.26084124339668)--(0.46399205474278,0.24528359574481)},
		{(0.0,0.0)--(0.075010034820169,7.3506279061383e-19)--(0.086485529574877,0.081635364593292)--(0.077785334870339,0.093268733958092)--(7.9754597912455e-18,0.096242174388719)},
		{(0.55562384132193,0.2731671852179)--(0.51369814765298,0.28579605366575)--(0.46399205474278,0.24528359574481)--(0.47634554951811,0.18883945644299)--(0.52062018995623,0.17130903915791)--(0.56413378794781,0.19754467506834)},
		{(0.15759072575741,0.51702466680117)--(0.090334998477803,0.5105201576241)--(0.069655043672125,0.46252170906469)--(0.094779729161039,0.42552694771552)--(0.1539172541449,0.42682959832593)--(0.17481075348646,0.50192012981204)}
    } {
        \draw[thick, color=\accentcolor] \polygon -- cycle;
    }

	% Domain.
    \draw[thick, color=\documentcolor]
        (0,0) -- (1,0) -- (1,1) -- (0,1) -- cycle;

\end{tikzpicture}
            \end{figure}
        }

        \only<6>{
            \begin{figure}[!ht]
                \centering
                \begin{tikzpicture}[scale=4.0, line join=round]

	% Cells.
    \foreach \polygon in {
		{(0.90562086673874,0.49745731734894)--(0.84964648501335,0.53928911983049)--(0.81370165301096,0.52736499787452)--(0.8007778398891,0.4599745910587)--(0.83830034032296,0.43123635442729)--(0.90332534096838,0.45661227040353)},
		{(0.43025505707094,0.85761111287619)--(0.439061613054,0.90102324862771)--(0.38639873239114,0.94248964143183)--(0.3362779132294,0.9097346449815)--(0.34039514150195,0.8459774418655)--(0.34850454506219,0.83910337452696)},
		{(0.41960644813702,0.76076625934297)--(0.43497718019503,0.71938447619238)--(0.49650365527531,0.70994989273511)--(0.52534847863401,0.74460186576363)--(0.49862928696564,0.81151654105998)--(0.45835143401538,0.81604376787072)},
		{(0.90671027792458,0.81411931008239)--(0.85765138373302,0.80955051545943)--(0.83790156000265,0.73211635896392)--(0.8604940273552,0.70719355410405)--(0.90814401638341,0.70795658852666)--(0.94673361824338,0.77371752222486)},
		{(0.4751613409377,0.44151674314935)--(0.4129547037095,0.42947908675186)--(0.39954754712646,0.40726327112996)--(0.42341532865985,0.34383551279366)--(0.49200407224827,0.34636662618389)--(0.51141822893407,0.37800387101546)},
		{(0.89872511778609,0.92046603344255)--(0.83534136701563,0.90145174413154)--(0.82885329691237,0.83443746367247)--(0.85765138373302,0.80955051545943)--(0.90671027792458,0.81411931008239)--(0.93546381809738,0.88744169879805)},
		{(0.41307582812368,0.53250224411983)--(0.38416850586663,0.49547567047401)--(0.4129547037095,0.42947908675186)--(0.4751613409377,0.44151674314935)--(0.49034178123502,0.47091119134154)--(0.45787096486917,0.52996812060041)},
		{(0.81370165301096,0.52736499787452)--(0.84964648501335,0.53928911983049)--(0.87264270846591,0.60125240420503)--(0.83676682446274,0.63778614798418)--(0.77960877800131,0.63045738124194)--(0.76810727094859,0.55768934658571)},
		{(0.73771722997482,0.19370941566284)--(0.79947883339628,0.16792494306375)--(0.82938320652562,0.18770003404512)--(0.83429833271736,0.2568333450831)--(0.77387850186301,0.28075926513421)--(0.74280602817792,0.2633168184927)},
		{(0.80019557685243,1.0)--(0.69409490600805,1.0)--(0.69380893033619,0.93886438567463)--(0.74061201671033,0.90481746293136)--(0.79830652878874,0.9305431229772)},
		{(0.17162073339073,1.6394271461714e-18)--(0.23365308807361,1.7367166623212e-18)--(0.25041478300468,0.081611205832619)--(0.23543050583935,0.10190903177243)--(0.18754064066743,0.10705758438686)--(0.15742045948643,0.072781138213485)},
		{(0.92118477967377,8.7100309126292e-18)--(1.0,0.0)--(1.0,0.1017721264846)--(0.91877817870729,0.093562841047918)},
		{(0.51141822893407,0.37800387101546)--(0.49200407224827,0.34636662618389)--(0.51369814765298,0.28579605366575)--(0.55562384132193,0.2731671852179)--(0.59337533904548,0.29548058688286)--(0.59669976992875,0.3664850638665)--(0.56758692480709,0.38788786901398)},
		{(0.42341532865985,0.34383551279366)--(0.39954754712646,0.40726327112996)--(0.33571803465189,0.3993410856253)--(0.31391253481761,0.35504452404102)--(0.34010653205604,0.3145727941542)--(0.40391634620773,0.31430714048907)},
		{(0.47634554951811,0.18883945644299)--(0.43008332996357,0.1546868218746)--(0.43044293484973,0.10112503702838)--(0.48908546549103,0.079415129582681)--(0.52692325543634,0.10290397147954)--(0.52062018995623,0.17130903915791)},
		{(0.79947883339628,0.16792494306375)--(0.73771722997482,0.19370941566284)--(0.70512093060736,0.17459887224965)--(0.70191732877822,0.10494698038419)--(0.74807260579378,0.081494092512622)--(0.79671105343376,0.10386391156366)},
		{(0.15759072575741,0.51702466680117)--(0.15593199973037,0.59378548549754)--(0.091974921981432,0.60200545404424)--(0.062479596074471,0.55461108764192)--(0.090334998477803,0.5105201576241)},
		{(0.90680120300739,1.0)--(0.80019557685243,1.0)--(0.79830652878874,0.9305431229772)--(0.83534136701563,0.90145174413154)--(0.89872511778609,0.92046603344255)},
		{(0.82885329691237,0.83443746367247)--(0.83534136701563,0.90145174413154)--(0.79830652878874,0.9305431229772)--(0.74061201671033,0.90481746293136)--(0.7349356515337,0.84345835452141)--(0.76668198957367,0.82014503182108)},
		{(0.91271273664315,0.25689411904908)--(0.86024354144915,0.27228365156758)--(0.83429833271736,0.2568333450831)--(0.82938320652562,0.18770003404512)--(0.89585777469058,0.1634898855985)--(0.93239287598605,0.20022992740407)},
		{(0.65166539015414,0.27080032503853)--(0.59337533904548,0.29548058688286)--(0.55562384132193,0.2731671852179)--(0.56413378794781,0.19754467506834)--(0.61287670363839,0.17984726382333)--(0.64665276281395,0.19867656194114)},
		{(0.33273072785904,0.59201745391288)--(0.30263941099775,0.55177266151939)--(0.33118275533446,0.4950870597933)--(0.38416850586663,0.49547567047401)--(0.41307582812368,0.53250224411983)--(0.38830912710478,0.58809988488896)},
		{(0.91244065180808,0.6062954385183)--(0.87264270846591,0.60125240420503)--(0.84964648501335,0.53928911983049)--(0.90562086673874,0.49745731734894)--(0.9639873004165,0.54000827924375)},
		{(0.76119275347245,0.73699949460978)--(0.76668198957367,0.82014503182108)--(0.7349356515337,0.84345835452141)--(0.68802573327946,0.82818753870301)--(0.67031785177334,0.76315890060833)--(0.68780188208613,0.74023887820549)},
		{(0.77387850186301,0.28075926513421)--(0.83429833271736,0.2568333450831)--(0.86024354144915,0.27228365156758)--(0.86470240133282,0.34852266459074)--(0.83603929287926,0.36893428567055)--(0.77941790003513,0.35019630476697)},
		{(0.34160589697019,0.094513672527824)--(0.33155858885746,0.15563799507262)--(0.26897529043921,0.16733835457815)--(0.23543050583935,0.10190903177243)--(0.25041478300468,0.081611205832619)--(0.31983036119436,0.073052449092787)},
		{(0.33118275533446,0.4950870597933)--(0.30263941099775,0.55177266151939)--(0.25380798875983,0.5526194384777)--(0.2195515088868,0.50389409457052)--(0.25806478562061,0.45046005555177)--(0.30317607746907,0.4529211417038)},
		{(0.075010034820169,7.3506279061383e-19)--(0.17162073339073,-1.6816816415712e-18)--(0.15742045948643,0.072781138213485)--(0.086485529574877,0.081635364593292)},
		{(0.15593199973037,0.59378548549754)--(0.15759072575741,0.51702466680117)--(0.17481075348646,0.50192012981204)--(0.2195515088868,0.50389409457052)--(0.25380798875983,0.5526194384777)--(0.22097129842883,0.61071460758486)--(0.17929801850874,0.61385198935047)},
		{(0.075297295409805,0.73877396112494)--(0.099723556961436,0.69058794606744)--(0.15562110912384,0.68802351791194)--(0.18641782187622,0.72150151437522)--(0.15931631518952,0.78772111258074)--(0.10100386127141,0.78593004898778)},
		{(0.27310724622345,0.25053361869666)--(0.25045836513368,0.20227624362407)--(0.26897529043921,0.16733835457815)--(0.33155858885746,0.15563799507262)--(0.3661000967184,0.19021492301758)--(0.36572639063023,0.21281149275499)--(0.31655376524573,0.25920567249135)},
		{(0.69409490600805,1.0)--(0.58760378421051,1.0)--(0.58883658367539,0.93864567278327)--(0.63779877032203,0.90304341614145)--(0.69380893033619,0.93886438567463)},
		{(0.43044293484973,0.10112503702838)--(0.43008332996357,0.1546868218746)--(0.3661000967184,0.19021492301758)--(0.33155858885746,0.15563799507262)--(0.34160589697019,0.094513672527824)--(0.40448175620263,0.081719073239906)},
		{(0.42172692966925,0.26084124339668)--(0.36572639063023,0.21281149275499)--(0.3661000967184,0.19021492301758)--(0.43008332996357,0.1546868218746)--(0.47634554951811,0.18883945644299)--(0.46399205474278,0.24528359574481)},
		{(-2.4250333963585e-18,0.738887607386)--(4.0334572221406e-18,0.64955618583817)--(0.071762165041819,0.64637933734506)--(0.099723556961436,0.69058794606744)--(0.075297295409805,0.73877396112494)},
		{(0.22097129842883,0.61071460758486)--(0.25380798875983,0.5526194384777)--(0.30263941099775,0.55177266151939)--(0.33273072785904,0.59201745391288)--(0.3095388704372,0.64957990647192)--(0.2606647579083,0.6547465077279)},
		{(0.32650250881848,0.74398270321248)--(0.26880077293031,0.75467651375032)--(0.23382313120445,0.71763074899324)--(0.2606647579083,0.6547465077279)--(0.3095388704372,0.64957990647192)--(0.34147746269111,0.68661019501426)},
		{(-2.2359986723035e-17,0.27877773252863)--(-2.3837580172329e-17,0.18670401126797)--(0.07534211873675,0.18740617822472)--(0.096392616615221,0.24522214340638)--(0.074353229052474,0.27885841601452)},
		{(0.66040691263801,4.1890025364447e-18)--(0.74707185567762,4.2548661771588e-18)--(0.74807260579378,0.081494092512622)--(0.70191732877822,0.10494698038419)--(0.66094052300372,0.084859226540502)},
		{(0.91415706365105,0.35924220050188)--(0.86470240133282,0.34852266459074)--(0.86024354144915,0.27228365156758)--(0.91271273664315,0.25689411904908)--(0.96417188630523,0.30864267927584)},
		{(0.90814401638341,0.70795658852666)--(0.8604940273552,0.70719355410405)--(0.83676682446274,0.63778614798418)--(0.87264270846591,0.60125240420503)--(0.91244065180808,0.6062954385183)--(0.9491529504042,0.65873122068471)},
		{(0.23365308807361,-3.3477505575059e-18)--(0.32628143694767,2.5803824807885e-19)--(0.31983036119436,0.073052449092787)--(0.25041478300468,0.081611205832619)},
		{(1.0,0.66016748325509)--(1.0,0.77458628824672)--(0.94673361824338,0.77371752222486)--(0.90814401638341,0.70795658852666)--(0.9491529504042,0.65873122068471)},
		{(0.15425357771257,0.33629539947812)--(0.18476486038891,0.29544336901431)--(0.23396036591514,0.30034838204361)--(0.26040160220422,0.35173388212644)--(0.2268937956406,0.39890898912141)--(0.17955310904827,0.39738992577799)},
		{(0.40587680220629,-2.9667951719802e-18)--(0.48992125349184,-4.0389342343263e-18)--(0.48908546549103,0.079415129582681)--(0.43044293484973,0.10112503702838)--(0.40448175620263,0.081719073239906)},
		{(0.40382770580073,0.67612206011063)--(0.34147746269111,0.68661019501426)--(0.3095388704372,0.64957990647192)--(0.33273072785904,0.59201745391288)--(0.38830912710478,0.58809988488896)--(0.41926742897499,0.63048085523994)},
		{(0.30317607746907,0.4529211417038)--(0.25806478562061,0.45046005555177)--(0.2268937956406,0.39890898912141)--(0.26040160220422,0.35173388212644)--(0.31391253481761,0.35504452404102)--(0.33571803465189,0.3993410856253)},
		{(0.074353229052474,0.27885841601452)--(0.096392616615221,0.24522214340638)--(0.16174055514051,0.24576440244375)--(0.18476486038891,0.29544336901431)--(0.15425357771257,0.33629539947812)--(0.096914643160764,0.33401537300282)},
		{(0.10544294737487,0.89435209215953)--(0.073916802589346,0.82932116499525)--(0.10100386127141,0.78593004898778)--(0.15931631518952,0.78772111258074)--(0.18080173347841,0.81699036861983)--(0.15190615583306,0.89009117210229)},
		{(0.15742045948643,0.072781138213485)--(0.18754064066743,0.10705758438686)--(0.16226733481266,0.15984078446317)--(0.095969171266849,0.15820612677018)--(0.077785334870339,0.093268733958092)--(0.086485529574877,0.081635364593292)},
		{(0.72463388956437,0.54269559300798)--(0.67446146919368,0.57507159324725)--(0.63797036149734,0.56086228377793)--(0.63074047496544,0.47875490164281)--(0.6609860319228,0.45893561983003)--(0.7129412682258,0.47288353123979)},
		{(0.45787096486917,0.52996812060041)--(0.49034178123502,0.47091119134154)--(0.55216201591839,0.4848763144971)--(0.55139757442056,0.5556585621165)--(0.49617510278083,0.57455891874653)},
		{(0.32628143694767,2.5803824807885e-19)--(0.40587680220629,-2.9667951719802e-18)--(0.40448175620263,0.081719073239906)--(0.34160589697019,0.094513672527824)--(0.31983036119436,0.073052449092787)},
		{(0.49617510278083,0.57455891874653)--(0.55139757442056,0.5556585621165)--(0.59114493366791,0.58104088182392)--(0.58301801629956,0.64682676411061)--(0.51236359176674,0.65572323824777)--(0.48396747686858,0.61838879428297)},
		{(0.439061613054,0.90102324862771)--(0.43025505707094,0.85761111287619)--(0.45835143401538,0.81604376787072)--(0.49862928696564,0.81151654105998)--(0.53886520818497,0.84806503605021)--(0.53058029364196,0.90099660511096)--(0.48760913777198,0.92532574237882)},
		{(0.90332534096838,0.45661227040353)--(0.83830034032296,0.43123635442729)--(0.83603929287926,0.36893428567055)--(0.86470240133282,0.34852266459074)--(0.91415706365105,0.35924220050188)--(0.9392022151889,0.42197668711201)},
		{(0.66377499068603,0.66963966040744)--(0.6033778721243,0.66931898593838)--(0.58301801629956,0.64682676411061)--(0.59114493366791,0.58104088182392)--(0.63797036149734,0.56086228377793)--(0.67446146919368,0.57507159324725)--(0.68882881153311,0.64088601503834)},
		{(0.388733913137,1.0)--(0.28284840015166,1.0)--(0.28488988344102,0.93715032027721)--(0.3362779132294,0.9097346449815)--(0.38639873239114,0.94248964143183)},
		{(-9.7620980360604e-18,0.83111983168827)--(-2.4250333963585e-18,0.738887607386)--(0.075297295409805,0.73877396112494)--(0.10100386127141,0.78593004898778)--(0.073916802589346,0.82932116499525)},
		{(0.6033778721243,0.66931898593838)--(0.66377499068603,0.66963966040744)--(0.68780188208613,0.74023887820549)--(0.67031785177334,0.76315890060833)--(0.60209322151793,0.7656981620798)--(0.58225899722008,0.7433741875493)},
		{(0.34850454506219,0.83910337452696)--(0.34039514150195,0.8459774418655)--(0.25627073354994,0.83167390407978)--(0.25037144272769,0.82338973935646)--(0.26880077293031,0.75467651375032)--(0.32650250881848,0.74398270321248)--(0.3579089973552,0.77651570635267)},
		{(0.16174055514051,0.24576440244375)--(0.096392616615221,0.24522214340638)--(0.07534211873675,0.18740617822472)--(0.095969171266849,0.15820612677018)--(0.16226733481266,0.15984078446317)--(0.18498554460672,0.2030287379494)},
		{(0.43497718019503,0.71938447619237)--(0.41960644813702,0.76076625934297)--(0.3579089973552,0.77651570635267)--(0.32650250881848,0.74398270321248)--(0.34147746269111,0.68661019501426)--(0.40382770580073,0.67612206011063)},
		{(0.34039514150195,0.8459774418655)--(0.3362779132294,0.9097346449815)--(0.28488988344102,0.93715032027721)--(0.2440991387246,0.90612103394854)--(0.25627073354994,0.83167390407978)},
		{(-4.0196166610004e-18,0.46062128402398)--(-4.04488462454e-18,0.36975068838999)--(0.071761178403807,0.37062088708733)--(0.094779729161039,0.42552694771552)--(0.069655043672125,0.46252170906469)},
		{(0.83790156000265,0.73211635896392)--(0.85765138373302,0.80955051545943)--(0.82885329691237,0.83443746367247)--(0.76668198957367,0.82014503182108)--(0.76119275347245,0.73699949460978)--(0.76501151428016,0.73318890229155)},
		{(0.58931938009406,0.83016295112284)--(0.63751433633477,0.86498201132943)--(0.63779877032203,0.90304341614145)--(0.58883658367539,0.93864567278327)--(0.53058029364196,0.90099660511096)--(0.53886520818497,0.84806503605021)},
		{(1.0,0.77458628824672)--(1.0,0.89221791742536)--(0.93546381809738,0.88744169879805)--(0.90671027792458,0.81411931008239)--(0.94673361824338,0.77371752222486)},
		{(0.70191732877822,0.10494698038419)--(0.70512093060736,0.17459887224965)--(0.64665276281395,0.19867656194114)--(0.61287670363839,0.17984726382333)--(0.61334150870758,0.10502616284931)--(0.66094052300372,0.084859226540502)},
		{(0.77960877800131,0.63045738124194)--(0.83676682446274,0.63778614798418)--(0.8604940273552,0.70719355410405)--(0.83790156000265,0.73211635896392)--(0.76501151428016,0.73318890229155)--(0.75843704503209,0.64936436375305)},
		{(0.18754064066743,0.10705758438686)--(0.23543050583935,0.10190903177243)--(0.26897529043921,0.16733835457815)--(0.25045836513368,0.20227624362407)--(0.18498554460672,0.2030287379494)--(0.16226733481266,0.15984078446317)},
		{(0.25045836513368,0.20227624362407)--(0.27310724622345,0.25053361869666)--(0.23396036591514,0.30034838204361)--(0.18476486038891,0.29544336901431)--(0.16174055514051,0.24576440244375)--(0.18498554460672,0.2030287379494)},
		{(0.57579246010271,-2.5769654542781e-17)--(0.66040691263801,-2.5225589931724e-17)--(0.66094052300372,0.084859226540502)--(0.61334150870758,0.10502616284931)--(0.57444644094745,0.084653717689775)},
		{(0.68802573327946,0.82818753870301)--(0.7349356515337,0.84345835452141)--(0.74061201671033,0.90481746293136)--(0.69380893033619,0.93886438567463)--(0.63779877032203,0.90304341614145)--(0.63751433633477,0.86498201132943)},
		{(0.69037814035163,0.36010268815439)--(0.68418875779986,0.28843130921516)--(0.74280602817792,0.2633168184927)--(0.77387850186301,0.28075926513421)--(0.77941790003513,0.35019630476697)--(0.74153850688122,0.3763468622677)},
		{(0.52534847863401,0.74460186576363)--(0.49650365527531,0.70994989273511)--(0.51236359176674,0.65572323824777)--(0.58301801629956,0.64682676411061)--(0.6033778721243,0.66931898593838)--(0.58225899722008,0.7433741875493)},
		{(0.07534211873675,0.18740617822472)--(8.7360618592155e-18,0.18670401126797)--(7.9754597912455e-18,0.096242174388719)--(0.077785334870339,0.093268733958092)--(0.095969171266849,0.15820612677018)},
		{(1.0,0.1017721264846)--(1.0,0.19140579317574)--(0.93239287598605,0.20022992740407)--(0.89585777469058,0.1634898855985)--(0.90267708939022,0.10524041471726)--(0.91877817870729,0.093562841047918)},
		{(0.1539172541449,0.42682959832593)--(0.17955310904827,0.39738992577799)--(0.2268937956406,0.39890898912141)--(0.25806478562061,0.45046005555177)--(0.2195515088868,0.50389409457052)--(0.17481075348646,0.50192012981204)},
		{(4.0334572221406e-18,0.64955618583817)--(2.6410555968065e-19,0.55501911596182)--(0.062479596074471,0.55461108764192)--(0.091974921981432,0.60200545404424)--(0.071762165041819,0.64637933734506)},
		{(1.445338404923e-18,0.91774728313323)--(-2.8252575743211e-18,0.83111983168827)--(0.073916802589346,0.82932116499525)--(0.10544294737487,0.89435209215953)--(0.087187661788198,0.91762708112664)},
		{(0.28284840015166,1.0)--(0.17980756855261,1.0)--(0.18915376767155,0.92224702668953)--(0.2440991387246,0.90612103394854)--(0.28488988344102,0.93715032027721)},
		{(0.56758692480709,0.38788786901398)--(0.59669976992875,0.3664850638665)--(0.65555621324255,0.38349657492187)--(0.6609860319228,0.45893561983003)--(0.63074047496544,0.47875490164281)--(0.5770556404014,0.46733679559338)},
		{(0.096914643160764,0.33401537300282)--(0.15425357771257,0.33629539947812)--(0.17955310904827,0.39738992577799)--(0.1539172541449,0.42682959832593)--(0.094779729161039,0.42552694771552)--(0.071761178403807,0.37062088708733)},
		{(0.74707185567762,-1.6560246142566e-17)--(0.83588912585082,-1.4264266249735e-17)--(0.83515530993165,0.079572391411158)--(0.79671105343376,0.10386391156366)--(0.74807260579378,0.081494092512622)},
		{(0.61334150870758,0.10502616284931)--(0.61287670363839,0.17984726382333)--(0.56413378794781,0.19754467506834)--(0.52062018995623,0.17130903915791)--(0.52692325543634,0.10290397147954)--(0.57444644094745,0.084653717689775)},
		{(0.60209322151793,0.7656981620798)--(0.67031785177334,0.76315890060833)--(0.68802573327946,0.82818753870301)--(0.63751433633477,0.86498201132943)--(0.58931938009406,0.83016295112284)},
		{(0.39954754712646,0.40726327112996)--(0.4129547037095,0.42947908675186)--(0.38416850586663,0.49547567047401)--(0.33118275533446,0.4950870597933)--(0.30317607746907,0.4529211417038)--(0.33571803465189,0.3993410856253)},
		{(0.18080173347841,0.81699036861983)--(0.15931631518952,0.78772111258074)--(0.18641782187622,0.72150151437522)--(0.23382313120445,0.71763074899324)--(0.26880077293031,0.75467651375032)--(0.25037144272769,0.82338973935646)},
		{(0.58760378421051,1.0)--(0.48771105281588,1.0)--(0.48760913777198,0.92532574237882)--(0.53058029364196,0.90099660511096)--(0.58883658367539,0.93864567278327)},
		{(0.76810727094859,0.55768934658571)--(0.77960877800131,0.63045738124194)--(0.75843704503209,0.64936436375305)--(0.68882881153311,0.64088601503834)--(0.67446146919368,0.57507159324725)--(0.72463388956437,0.54269559300798)},
		{(1.0,0.19140579317574)--(1.0,0.30836423392629)--(0.96417188630523,0.30864267927584)--(0.91271273664315,0.25689411904908)--(0.93239287598605,0.20022992740407)},
		{(0.48771105281588,1.0)--(0.388733913137,1.0)--(0.38639873239114,0.94248964143183)--(0.439061613054,0.90102324862771)--(0.48760913777198,0.92532574237882)},
		{(0.59669976992875,0.3664850638665)--(0.59337533904548,0.29548058688286)--(0.65166539015414,0.27080032503853)--(0.68418875779986,0.28843130921516)--(0.69037814035163,0.36010268815439)--(0.65555621324255,0.38349657492187)},
		{(0.43025505707094,0.85761111287619)--(0.34850454506219,0.83910337452696)--(0.3579089973552,0.77651570635267)--(0.41960644813702,0.76076625934297)--(0.45835143401538,0.81604376787072)},
		{(0.17980756855261,1.0)--(0.098282855212861,1.0)--(0.087187661788198,0.91762708112664)--(0.10544294737487,0.89435209215953)--(0.15190615583306,0.89009117210229)--(0.18915376767155,0.92224702668953)},
		{(1.0,0.30836423392629)--(1.0,0.4260285521325)--(0.9392022151889,0.42197668711201)--(0.91415706365105,0.35924220050188)--(0.96417188630523,0.30864267927584)},
		{(0.83588912585082,-3.8706852340109e-19)--(0.92118477967377,8.7100309126292e-18)--(0.91877817870729,0.093562841047918)--(0.90267708939022,0.10524041471726)--(0.83515530993165,0.079572391411158)},
		{(0.49862928696564,0.81151654105998)--(0.52534847863401,0.74460186576363)--(0.58225899722008,0.7433741875493)--(0.60209322151793,0.7656981620798)--(0.58931938009406,0.83016295112284)--(0.53886520818497,0.84806503605021)},
		{(2.6410555968065e-19,0.55501911596182)--(1.4808354815746e-18,0.46062128402398)--(0.069655043672125,0.46252170906469)--(0.090334998477803,0.5105201576241)--(0.062479596074471,0.55461108764192)},
		{(0.49650365527531,0.70994989273511)--(0.43497718019503,0.71938447619238)--(0.40382770580073,0.67612206011063)--(0.41926742897499,0.63048085523994)--(0.48396747686858,0.61838879428297)--(0.51236359176674,0.65572323824777)},
		{(0.66377499068603,0.66963966040744)--(0.68882881153311,0.64088601503834)--(0.75843704503209,0.64936436375305)--(0.76501151428016,0.73318890229155)--(0.76119275347245,0.73699949460978)--(0.68780188208613,0.74023887820549)},
		{(0.73771722997482,0.19370941566284)--(0.74280602817792,0.2633168184927)--(0.68418875779986,0.28843130921516)--(0.65166539015414,0.27080032503853)--(0.64665276281395,0.19867656194114)--(0.70512093060736,0.17459887224965)},
		{(0.091974921981432,0.60200545404424)--(0.15593199973037,0.59378548549754)--(0.17929801850874,0.61385198935047)--(0.15562110912384,0.68802351791194)--(0.099723556961436,0.69058794606744)--(0.071762165041819,0.64637933734506)},
		{(1.0,0.89221791742536)--(1.0,1.0)--(0.90680120300739,1.0)--(0.89872511778609,0.92046603344255)--(0.93546381809738,0.88744169879805)},
		{(0.098282855212861,1.0)--(0.0,1.0)--(3.2382657123805e-18,0.91774728313323)--(0.087187661788198,0.91762708112664)},
		{(0.15190615583306,0.89009117210229)--(0.18080173347841,0.81699036861983)--(0.25037144272769,0.82338973935646)--(0.25627073354994,0.83167390407978)--(0.2440991387246,0.90612103394854)--(0.18915376767155,0.92224702668953)},
		{(1.0,0.4260285521325)--(1.0,0.53758135732378)--(0.9639873004165,0.54000827924375)--(0.90562086673874,0.49745731734894)--(0.90332534096838,0.45661227040353)--(0.9392022151889,0.42197668711201)},
		{(-8.0568679736084e-19,0.36975068838999)--(3.2226216693698e-18,0.27877773252863)--(0.074353229052474,0.27885841601452)--(0.096914643160764,0.33401537300282)--(0.071761178403807,0.37062088708733)},
		{(0.18641782187622,0.72150151437522)--(0.15562110912384,0.68802351791194)--(0.17929801850874,0.61385198935047)--(0.22097129842883,0.61071460758486)--(0.2606647579083,0.6547465077279)--(0.23382313120445,0.71763074899324)},
		{(0.90267708939022,0.10524041471726)--(0.89585777469058,0.1634898855985)--(0.82938320652562,0.18770003404512)--(0.79947883339628,0.16792494306375)--(0.79671105343376,0.10386391156366)--(0.83515530993165,0.079572391411158)},
		{(0.7510082818126,0.44540964367599)--(0.7129412682258,0.47288353123979)--(0.6609860319228,0.45893561983003)--(0.65555621324255,0.38349657492187)--(0.69037814035163,0.36010268815439)--(0.74153850688122,0.3763468622677)},
		{(0.8007778398891,0.4599745910587)--(0.81370165301096,0.52736499787452)--(0.76810727094859,0.55768934658571)--(0.72463388956437,0.54269559300798)--(0.7129412682258,0.47288353123979)--(0.7510082818126,0.44540964367599)},
		{(0.49034178123502,0.47091119134154)--(0.4751613409377,0.44151674314935)--(0.51141822893407,0.37800387101546)--(0.56758692480709,0.38788786901398)--(0.5770556404014,0.46733679559338)--(0.55216201591839,0.4848763144971)},
		{(0.34010653205604,0.3145727941542)--(0.31391253481761,0.35504452404102)--(0.26040160220422,0.35173388212644)--(0.23396036591514,0.30034838204361)--(0.27310724622345,0.25053361869666)--(0.31655376524573,0.25920567249135)},
		{(0.40391634620773,0.31430714048907)--(0.34010653205604,0.3145727941542)--(0.31655376524573,0.25920567249135)--(0.36572639063023,0.21281149275499)--(0.42172692966924,0.26084124339668)},
		{(0.83830034032296,0.43123635442729)--(0.8007778398891,0.4599745910587)--(0.7510082818126,0.44540964367599)--(0.74153850688122,0.3763468622677)--(0.77941790003513,0.35019630476697)--(0.83603929287926,0.36893428567055)},
		{(0.55139757442056,0.5556585621165)--(0.55216201591839,0.4848763144971)--(0.5770556404014,0.46733679559338)--(0.63074047496544,0.47875490164281)--(0.63797036149734,0.56086228377793)--(0.59114493366791,0.58104088182392)},
		{(0.48992125349184,-6.3178208718301e-18)--(0.57579246010271,-4.9556037627779e-18)--(0.57444644094745,0.084653717689775)--(0.52692325543634,0.10290397147954)--(0.48908546549103,0.079415129582681)},
		{(0.41926742897499,0.63048085523994)--(0.38830912710478,0.58809988488896)--(0.41307582812368,0.53250224411983)--(0.45787096486917,0.52996812060041)--(0.49617510278083,0.57455891874653)--(0.48396747686858,0.61838879428297)},
		{(1.0,0.53758135732378)--(1.0,0.66016748325509)--(0.9491529504042,0.65873122068471)--(0.91244065180808,0.6062954385183)--(0.9639873004165,0.54000827924375)},
		{(0.51369814765298,0.28579605366575)--(0.49200407224827,0.34636662618389)--(0.42341532865985,0.34383551279366)--(0.40391634620773,0.31430714048907)--(0.42172692966924,0.26084124339668)--(0.46399205474278,0.24528359574481)},
		{(0.0,0.0)--(0.075010034820169,7.3506279061383e-19)--(0.086485529574877,0.081635364593292)--(0.077785334870339,0.093268733958092)--(7.9754597912455e-18,0.096242174388719)},
		{(0.55562384132193,0.2731671852179)--(0.51369814765298,0.28579605366575)--(0.46399205474278,0.24528359574481)--(0.47634554951811,0.18883945644299)--(0.52062018995623,0.17130903915791)--(0.56413378794781,0.19754467506834)},
		{(0.15759072575741,0.51702466680117)--(0.090334998477803,0.5105201576241)--(0.069655043672125,0.46252170906469)--(0.094779729161039,0.42552694771552)--(0.1539172541449,0.42682959832593)--(0.17481075348646,0.50192012981204)}
    } {
        \draw[ultra thin, color=\documentcolor, opacity=0.25] \polygon -- cycle;
    }
    
    % Cells.
    \foreach \polygon in {
		{(0.90562086673874,0.49745731734894)--(0.84964648501335,0.53928911983049)--(0.81370165301096,0.52736499787452)--(0.8007778398891,0.4599745910587)--(0.83830034032296,0.43123635442729)--(0.90332534096838,0.45661227040353)},
		{(0.43025505707094,0.85761111287619)--(0.439061613054,0.90102324862771)--(0.38639873239114,0.94248964143183)--(0.3362779132294,0.9097346449815)--(0.34039514150195,0.8459774418655)--(0.34850454506219,0.83910337452696)},
		{(0.41960644813702,0.76076625934297)--(0.43497718019503,0.71938447619238)--(0.49650365527531,0.70994989273511)--(0.52534847863401,0.74460186576363)--(0.49862928696564,0.81151654105998)--(0.45835143401538,0.81604376787072)},
		{(0.90671027792458,0.81411931008239)--(0.85765138373302,0.80955051545943)--(0.83790156000265,0.73211635896392)--(0.8604940273552,0.70719355410405)--(0.90814401638341,0.70795658852666)--(0.94673361824338,0.77371752222486)},
		{(0.4751613409377,0.44151674314935)--(0.4129547037095,0.42947908675186)--(0.39954754712646,0.40726327112996)--(0.42341532865985,0.34383551279366)--(0.49200407224827,0.34636662618389)--(0.51141822893407,0.37800387101546)},
		{(0.89872511778609,0.92046603344255)--(0.83534136701563,0.90145174413154)--(0.82885329691237,0.83443746367247)--(0.85765138373302,0.80955051545943)--(0.90671027792458,0.81411931008239)--(0.93546381809738,0.88744169879805)},
		{(0.41307582812368,0.53250224411983)--(0.38416850586663,0.49547567047401)--(0.4129547037095,0.42947908675186)--(0.4751613409377,0.44151674314935)--(0.49034178123502,0.47091119134154)--(0.45787096486917,0.52996812060041)},
		{(0.81370165301096,0.52736499787452)--(0.84964648501335,0.53928911983049)--(0.87264270846591,0.60125240420503)--(0.83676682446274,0.63778614798418)--(0.77960877800131,0.63045738124194)--(0.76810727094859,0.55768934658571)},
		{(0.73771722997482,0.19370941566284)--(0.79947883339628,0.16792494306375)--(0.82938320652562,0.18770003404512)--(0.83429833271736,0.2568333450831)--(0.77387850186301,0.28075926513421)--(0.74280602817792,0.2633168184927)},
		{(0.80019557685243,1.0)--(0.69409490600805,1.0)--(0.69380893033619,0.93886438567463)--(0.74061201671033,0.90481746293136)--(0.79830652878874,0.9305431229772)},
		{(0.17162073339073,1.6394271461714e-18)--(0.23365308807361,1.7367166623212e-18)--(0.25041478300468,0.081611205832619)--(0.23543050583935,0.10190903177243)--(0.18754064066743,0.10705758438686)--(0.15742045948643,0.072781138213485)},
		{(0.92118477967377,8.7100309126292e-18)--(1.0,0.0)--(1.0,0.1017721264846)--(0.91877817870729,0.093562841047918)},
		{(0.51141822893407,0.37800387101546)--(0.49200407224827,0.34636662618389)--(0.51369814765298,0.28579605366575)--(0.55562384132193,0.2731671852179)--(0.59337533904548,0.29548058688286)--(0.59669976992875,0.3664850638665)--(0.56758692480709,0.38788786901398)},
		{(0.42341532865985,0.34383551279366)--(0.39954754712646,0.40726327112996)--(0.33571803465189,0.3993410856253)--(0.31391253481761,0.35504452404102)--(0.34010653205604,0.3145727941542)--(0.40391634620773,0.31430714048907)},
		{(0.47634554951811,0.18883945644299)--(0.43008332996357,0.1546868218746)--(0.43044293484973,0.10112503702838)--(0.48908546549103,0.079415129582681)--(0.52692325543634,0.10290397147954)--(0.52062018995623,0.17130903915791)},
		{(0.79947883339628,0.16792494306375)--(0.73771722997482,0.19370941566284)--(0.70512093060736,0.17459887224965)--(0.70191732877822,0.10494698038419)--(0.74807260579378,0.081494092512622)--(0.79671105343376,0.10386391156366)},
		{(0.15759072575741,0.51702466680117)--(0.15593199973037,0.59378548549754)--(0.091974921981432,0.60200545404424)--(0.062479596074471,0.55461108764192)--(0.090334998477803,0.5105201576241)},
		{(0.90680120300739,1.0)--(0.80019557685243,1.0)--(0.79830652878874,0.9305431229772)--(0.83534136701563,0.90145174413154)--(0.89872511778609,0.92046603344255)},
		{(0.82885329691237,0.83443746367247)--(0.83534136701563,0.90145174413154)--(0.79830652878874,0.9305431229772)--(0.74061201671033,0.90481746293136)--(0.7349356515337,0.84345835452141)--(0.76668198957367,0.82014503182108)},
		{(0.91271273664315,0.25689411904908)--(0.86024354144915,0.27228365156758)--(0.83429833271736,0.2568333450831)--(0.82938320652562,0.18770003404512)--(0.89585777469058,0.1634898855985)--(0.93239287598605,0.20022992740407)},
		{(0.65166539015414,0.27080032503853)--(0.59337533904548,0.29548058688286)--(0.55562384132193,0.2731671852179)--(0.56413378794781,0.19754467506834)--(0.61287670363839,0.17984726382333)--(0.64665276281395,0.19867656194114)},
		{(0.33273072785904,0.59201745391288)--(0.30263941099775,0.55177266151939)--(0.33118275533446,0.4950870597933)--(0.38416850586663,0.49547567047401)--(0.41307582812368,0.53250224411983)--(0.38830912710478,0.58809988488896)},
		{(0.91244065180808,0.6062954385183)--(0.87264270846591,0.60125240420503)--(0.84964648501335,0.53928911983049)--(0.90562086673874,0.49745731734894)--(0.9639873004165,0.54000827924375)},
		{(0.7631021338763,0.73509419845066)--(0.76668198957367,0.82014503182108)--(0.7349356515337,0.84345835452141)--(0.68802573327946,0.82818753870301)--(0.67031785177334,0.76315890060833)--(0.68780188208613,0.74023887820549)},
		{(0.77387850186301,0.28075926513421)--(0.83429833271736,0.2568333450831)--(0.86024354144915,0.27228365156758)--(0.86470240133282,0.34852266459074)--(0.83603929287926,0.36893428567055)--(0.77941790003513,0.35019630476697)},
		{(0.34160589697019,0.094513672527824)--(0.33155858885746,0.15563799507262)--(0.26897529043921,0.16733835457815)--(0.23543050583935,0.10190903177243)--(0.25041478300468,0.081611205832619)--(0.31983036119436,0.073052449092787)},
		{(0.33118275533446,0.4950870597933)--(0.30263941099775,0.55177266151939)--(0.25380798875983,0.5526194384777)--(0.2195515088868,0.50389409457052)--(0.25806478562061,0.45046005555177)--(0.30317607746907,0.4529211417038)},
		{(0.075010034820169,7.3506279061383e-19)--(0.17162073339073,-1.6816816415712e-18)--(0.15742045948643,0.072781138213485)--(0.086485529574877,0.081635364593292)},
		{(0.15593199973037,0.59378548549754)--(0.15759072575741,0.51702466680117)--(0.17481075348646,0.50192012981204)--(0.2195515088868,0.50389409457052)--(0.25380798875983,0.5526194384777)--(0.22097129842883,0.61071460758486)--(0.17929801850874,0.61385198935047)},
		{(0.075297295409805,0.73877396112494)--(0.099723556961436,0.69058794606744)--(0.15562110912384,0.68802351791194)--(0.18641782187622,0.72150151437522)--(0.15931631518952,0.78772111258074)--(0.10100386127141,0.78593004898778)},
		{(0.27310724622345,0.25053361869666)--(0.25045836513368,0.20227624362407)--(0.26897529043921,0.16733835457815)--(0.33155858885746,0.15563799507262)--(0.3661000967184,0.19021492301758)--(0.36572639063023,0.21281149275499)--(0.31655376524573,0.25920567249135)},
		{(0.69409490600805,1.0)--(0.58760378421051,1.0)--(0.58883658367539,0.93864567278327)--(0.63779877032203,0.90304341614145)--(0.69380893033619,0.93886438567463)},
		{(0.43044293484973,0.10112503702838)--(0.43008332996357,0.1546868218746)--(0.3661000967184,0.19021492301758)--(0.33155858885746,0.15563799507262)--(0.34160589697019,0.094513672527824)--(0.40448175620263,0.081719073239906)},
		{(0.42172692966925,0.26084124339668)--(0.36572639063023,0.21281149275499)--(0.3661000967184,0.19021492301758)--(0.43008332996357,0.1546868218746)--(0.47634554951811,0.18883945644299)--(0.46399205474278,0.24528359574481)},
		{(-2.4250333963585e-18,0.738887607386)--(4.0334572221406e-18,0.64955618583817)--(0.071762165041819,0.64637933734506)--(0.099723556961436,0.69058794606744)--(0.075297295409805,0.73877396112494)},
		{(0.22097129842883,0.61071460758486)--(0.25380798875983,0.5526194384777)--(0.30263941099775,0.55177266151939)--(0.33273072785904,0.59201745391288)--(0.3095388704372,0.64957990647192)--(0.2606647579083,0.6547465077279)},
		{(0.32650250881848,0.74398270321248)--(0.26880077293031,0.75467651375032)--(0.23382313120445,0.71763074899324)--(0.2606647579083,0.6547465077279)--(0.3095388704372,0.64957990647192)--(0.34147746269111,0.68661019501426)},
		{(-2.2359986723035e-17,0.27877773252863)--(-2.3837580172329e-17,0.18670401126797)--(0.07534211873675,0.18740617822472)--(0.096392616615221,0.24522214340638)--(0.074353229052474,0.27885841601452)},
		{(0.66040691263801,4.1890025364447e-18)--(0.74707185567762,4.2548661771588e-18)--(0.74807260579378,0.081494092512622)--(0.70191732877822,0.10494698038419)--(0.66094052300372,0.084859226540502)},
		{(0.91415706365105,0.35924220050188)--(0.86470240133282,0.34852266459074)--(0.86024354144915,0.27228365156758)--(0.91271273664315,0.25689411904908)--(0.96417188630523,0.30864267927584)},
		{(0.90814401638341,0.70795658852666)--(0.8604940273552,0.70719355410405)--(0.83676682446274,0.63778614798418)--(0.87264270846591,0.60125240420503)--(0.91244065180808,0.6062954385183)--(0.9491529504042,0.65873122068471)},
		{(0.23365308807361,-3.3477505575059e-18)--(0.32628143694767,2.5803824807885e-19)--(0.31983036119436,0.073052449092787)--(0.25041478300468,0.081611205832619)},
		{(1.0,0.66016748325509)--(1.0,0.77458628824672)--(0.94673361824338,0.77371752222486)--(0.90814401638341,0.70795658852666)--(0.9491529504042,0.65873122068471)},
		{(0.15425357771257,0.33629539947812)--(0.18476486038891,0.29544336901431)--(0.23396036591514,0.30034838204361)--(0.26040160220422,0.35173388212644)--(0.2268937956406,0.39890898912141)--(0.17955310904827,0.39738992577799)},
		{(0.40587680220629,-2.9667951719802e-18)--(0.48992125349184,-4.0389342343263e-18)--(0.48908546549103,0.079415129582681)--(0.43044293484973,0.10112503702838)--(0.40448175620263,0.081719073239906)},
		{(0.40382770580073,0.67612206011063)--(0.34147746269111,0.68661019501426)--(0.3095388704372,0.64957990647192)--(0.33273072785904,0.59201745391288)--(0.38830912710478,0.58809988488896)--(0.41926742897499,0.63048085523994)},
		{(0.30317607746907,0.4529211417038)--(0.25806478562061,0.45046005555177)--(0.2268937956406,0.39890898912141)--(0.26040160220422,0.35173388212644)--(0.31391253481761,0.35504452404102)--(0.33571803465189,0.3993410856253)},
		{(0.074353229052474,0.27885841601452)--(0.096392616615221,0.24522214340638)--(0.16174055514051,0.24576440244375)--(0.18476486038891,0.29544336901431)--(0.15425357771257,0.33629539947812)--(0.096914643160764,0.33401537300282)},
		{(0.10544294737487,0.89435209215953)--(0.073916802589346,0.82932116499525)--(0.10100386127141,0.78593004898778)--(0.15931631518952,0.78772111258074)--(0.18080173347841,0.81699036861983)--(0.15190615583306,0.89009117210229)},
		{(0.15742045948643,0.072781138213485)--(0.18754064066743,0.10705758438686)--(0.16226733481266,0.15984078446317)--(0.095969171266849,0.15820612677018)--(0.077785334870339,0.093268733958092)--(0.086485529574877,0.081635364593292)},
		{(0.72463388956437,0.54269559300798)--(0.67446146919368,0.57507159324725)--(0.63797036149734,0.56086228377793)--(0.63074047496544,0.47875490164281)--(0.6609860319228,0.45893561983003)--(0.7129412682258,0.47288353123979)},
		{(0.45787096486917,0.52996812060041)--(0.49034178123502,0.47091119134154)--(0.55216201591839,0.4848763144971)--(0.55139757442056,0.5556585621165)--(0.49617510278083,0.57455891874653)},
		{(0.32628143694767,2.5803824807885e-19)--(0.40587680220629,-2.9667951719802e-18)--(0.40448175620263,0.081719073239906)--(0.34160589697019,0.094513672527824)--(0.31983036119436,0.073052449092787)},
		{(0.49617510278083,0.57455891874653)--(0.55139757442056,0.5556585621165)--(0.59114493366791,0.58104088182392)--(0.58301801629956,0.64682676411061)--(0.51236359176674,0.65572323824777)--(0.48396747686858,0.61838879428297)},
		{(0.439061613054,0.90102324862771)--(0.43025505707094,0.85761111287619)--(0.45835143401538,0.81604376787072)--(0.49862928696564,0.81151654105998)--(0.53886520818497,0.84806503605021)--(0.53058029364196,0.90099660511096)--(0.48760913777198,0.92532574237882)},
		{(0.90332534096838,0.45661227040353)--(0.83830034032296,0.43123635442729)--(0.83603929287926,0.36893428567055)--(0.86470240133282,0.34852266459074)--(0.91415706365105,0.35924220050188)--(0.9392022151889,0.42197668711201)},
		{(0.66377499068603,0.66963966040744)--(0.6033778721243,0.66931898593838)--(0.58301801629956,0.64682676411061)--(0.59114493366791,0.58104088182392)--(0.63797036149734,0.56086228377793)--(0.67446146919368,0.57507159324725)--(0.68882881153311,0.64088601503834)},
		{(0.388733913137,1.0)--(0.28284840015166,1.0)--(0.28488988344102,0.93715032027721)--(0.3362779132294,0.9097346449815)--(0.38639873239114,0.94248964143183)},
		{(-9.7620980360604e-18,0.83111983168827)--(-2.4250333963585e-18,0.738887607386)--(0.075297295409805,0.73877396112494)--(0.10100386127141,0.78593004898778)--(0.073916802589346,0.82932116499525)},
		{(0.6033778721243,0.66931898593838)--(0.66377499068603,0.66963966040744)--(0.68780188208613,0.74023887820549)--(0.67031785177334,0.76315890060833)--(0.60209322151793,0.7656981620798)--(0.58225899722008,0.7433741875493)},
		{(0.34850454506219,0.83910337452696)--(0.34039514150195,0.8459774418655)--(0.25627073354994,0.83167390407978)--(0.25037144272769,0.82338973935646)--(0.26880077293031,0.75467651375032)--(0.32650250881848,0.74398270321248)--(0.3579089973552,0.77651570635267)},
		{(0.16174055514051,0.24576440244375)--(0.096392616615221,0.24522214340638)--(0.07534211873675,0.18740617822472)--(0.095969171266849,0.15820612677018)--(0.16226733481266,0.15984078446317)--(0.18498554460672,0.2030287379494)},
		{(0.43497718019503,0.71938447619237)--(0.41960644813702,0.76076625934297)--(0.3579089973552,0.77651570635267)--(0.32650250881848,0.74398270321248)--(0.34147746269111,0.68661019501426)--(0.40382770580073,0.67612206011063)},
		{(0.34039514150195,0.8459774418655)--(0.3362779132294,0.9097346449815)--(0.28488988344102,0.93715032027721)--(0.2440991387246,0.90612103394854)--(0.25627073354994,0.83167390407978)},
		{(-4.0196166610004e-18,0.46062128402398)--(-4.04488462454e-18,0.36975068838999)--(0.071761178403807,0.37062088708733)--(0.094779729161039,0.42552694771552)--(0.069655043672125,0.46252170906469)},
		{(0.83790156000265,0.73211635896392)--(0.85765138373302,0.80955051545943)--(0.82885329691237,0.83443746367247)--(0.76668198957367,0.82014503182108)--(0.7631021338763,0.73509419845066)},
		{(0.58931938009406,0.83016295112284)--(0.63751433633477,0.86498201132943)--(0.63779877032203,0.90304341614145)--(0.58883658367539,0.93864567278327)--(0.53058029364196,0.90099660511096)--(0.53886520818497,0.84806503605021)},
		{(1.0,0.77458628824672)--(1.0,0.89221791742536)--(0.93546381809738,0.88744169879805)--(0.90671027792458,0.81411931008239)--(0.94673361824338,0.77371752222486)},
		{(0.70191732877822,0.10494698038419)--(0.70512093060736,0.17459887224965)--(0.64665276281395,0.19867656194114)--(0.61287670363839,0.17984726382333)--(0.61334150870758,0.10502616284931)--(0.66094052300372,0.084859226540502)},
		{(0.77960877800131,0.63045738124194)--(0.83676682446274,0.63778614798418)--(0.8604940273552,0.70719355410405)--(0.83790156000265,0.73211635896392)--(0.7631021338763,0.73509419845066)--(0.75843704503209,0.64936436375305)},
		{(0.18754064066743,0.10705758438686)--(0.23543050583935,0.10190903177243)--(0.26897529043921,0.16733835457815)--(0.25045836513368,0.20227624362407)--(0.18498554460672,0.2030287379494)--(0.16226733481266,0.15984078446317)},
		{(0.25045836513368,0.20227624362407)--(0.27310724622345,0.25053361869666)--(0.23396036591514,0.30034838204361)--(0.18476486038891,0.29544336901431)--(0.16174055514051,0.24576440244375)--(0.18498554460672,0.2030287379494)},
		{(0.57579246010271,-2.5769654542781e-17)--(0.66040691263801,-2.5225589931724e-17)--(0.66094052300372,0.084859226540502)--(0.61334150870758,0.10502616284931)--(0.57444644094745,0.084653717689775)},
		{(0.68802573327946,0.82818753870301)--(0.7349356515337,0.84345835452141)--(0.74061201671033,0.90481746293136)--(0.69380893033619,0.93886438567463)--(0.63779877032203,0.90304341614145)--(0.63751433633477,0.86498201132943)},
		{(0.69037814035163,0.36010268815439)--(0.68418875779986,0.28843130921516)--(0.74280602817792,0.2633168184927)--(0.77387850186301,0.28075926513421)--(0.77941790003513,0.35019630476697)--(0.74153850688122,0.3763468622677)},
		{(0.52534847863401,0.74460186576363)--(0.49650365527531,0.70994989273511)--(0.51236359176674,0.65572323824777)--(0.58301801629956,0.64682676411061)--(0.6033778721243,0.66931898593838)--(0.58225899722008,0.7433741875493)},
		{(0.07534211873675,0.18740617822472)--(8.7360618592155e-18,0.18670401126797)--(7.9754597912455e-18,0.096242174388719)--(0.077785334870339,0.093268733958092)--(0.095969171266849,0.15820612677018)},
		{(1.0,0.1017721264846)--(1.0,0.19140579317574)--(0.93239287598605,0.20022992740407)--(0.89585777469058,0.1634898855985)--(0.90267708939022,0.10524041471726)--(0.91877817870729,0.093562841047918)},
		{(0.1539172541449,0.42682959832593)--(0.17955310904827,0.39738992577799)--(0.2268937956406,0.39890898912141)--(0.25806478562061,0.45046005555177)--(0.2195515088868,0.50389409457052)--(0.17481075348646,0.50192012981204)},
		{(4.0334572221406e-18,0.64955618583817)--(2.6410555968065e-19,0.55501911596182)--(0.062479596074471,0.55461108764192)--(0.091974921981432,0.60200545404424)--(0.071762165041819,0.64637933734506)},
		{(1.445338404923e-18,0.91774728313323)--(-2.8252575743211e-18,0.83111983168827)--(0.073916802589346,0.82932116499525)--(0.10544294737487,0.89435209215953)--(0.087187661788198,0.91762708112664)},
		{(0.28284840015166,1.0)--(0.17980756855261,1.0)--(0.18915376767155,0.92224702668953)--(0.2440991387246,0.90612103394854)--(0.28488988344102,0.93715032027721)},
		{(0.56758692480709,0.38788786901398)--(0.59669976992875,0.3664850638665)--(0.65555621324255,0.38349657492187)--(0.6609860319228,0.45893561983003)--(0.63074047496544,0.47875490164281)--(0.5770556404014,0.46733679559338)},
		{(0.096914643160764,0.33401537300282)--(0.15425357771257,0.33629539947812)--(0.17955310904827,0.39738992577799)--(0.1539172541449,0.42682959832593)--(0.094779729161039,0.42552694771552)--(0.071761178403807,0.37062088708733)},
		{(0.74707185567762,-1.6560246142566e-17)--(0.83588912585082,-1.4264266249735e-17)--(0.83515530993165,0.079572391411158)--(0.79671105343376,0.10386391156366)--(0.74807260579378,0.081494092512622)},
		{(0.61334150870758,0.10502616284931)--(0.61287670363839,0.17984726382333)--(0.56413378794781,0.19754467506834)--(0.52062018995623,0.17130903915791)--(0.52692325543634,0.10290397147954)--(0.57444644094745,0.084653717689775)},
		{(0.60209322151793,0.7656981620798)--(0.67031785177334,0.76315890060833)--(0.68802573327946,0.82818753870301)--(0.63751433633477,0.86498201132943)--(0.58931938009406,0.83016295112284)},
		{(0.39954754712646,0.40726327112996)--(0.4129547037095,0.42947908675186)--(0.38416850586663,0.49547567047401)--(0.33118275533446,0.4950870597933)--(0.30317607746907,0.4529211417038)--(0.33571803465189,0.3993410856253)},
		{(0.18080173347841,0.81699036861983)--(0.15931631518952,0.78772111258074)--(0.18641782187622,0.72150151437522)--(0.23382313120445,0.71763074899324)--(0.26880077293031,0.75467651375032)--(0.25037144272769,0.82338973935646)},
		{(0.58760378421051,1.0)--(0.48771105281588,1.0)--(0.48760913777198,0.92532574237882)--(0.53058029364196,0.90099660511096)--(0.58883658367539,0.93864567278327)},
		{(0.76810727094859,0.55768934658571)--(0.77960877800131,0.63045738124194)--(0.75843704503209,0.64936436375305)--(0.68882881153311,0.64088601503834)--(0.67446146919368,0.57507159324725)--(0.72463388956437,0.54269559300798)},
		{(1.0,0.19140579317574)--(1.0,0.30836423392629)--(0.96417188630523,0.30864267927584)--(0.91271273664315,0.25689411904908)--(0.93239287598605,0.20022992740407)},
		{(0.48771105281588,1.0)--(0.388733913137,1.0)--(0.38639873239114,0.94248964143183)--(0.439061613054,0.90102324862771)--(0.48760913777198,0.92532574237882)},
		{(0.59669976992875,0.3664850638665)--(0.59337533904548,0.29548058688286)--(0.65166539015414,0.27080032503853)--(0.68418875779986,0.28843130921516)--(0.69037814035163,0.36010268815439)--(0.65555621324255,0.38349657492187)},
		{(0.43025505707094,0.85761111287619)--(0.34850454506219,0.83910337452696)--(0.3579089973552,0.77651570635267)--(0.41960644813702,0.76076625934297)--(0.45835143401538,0.81604376787072)},
		{(0.17980756855261,1.0)--(0.098282855212861,1.0)--(0.087187661788198,0.91762708112664)--(0.10544294737487,0.89435209215953)--(0.15190615583306,0.89009117210229)--(0.18915376767155,0.92224702668953)},
		{(1.0,0.30836423392629)--(1.0,0.4260285521325)--(0.9392022151889,0.42197668711201)--(0.91415706365105,0.35924220050188)--(0.96417188630523,0.30864267927584)},
		{(0.83588912585082,-3.8706852340109e-19)--(0.92118477967377,8.7100309126292e-18)--(0.91877817870729,0.093562841047918)--(0.90267708939022,0.10524041471726)--(0.83515530993165,0.079572391411158)},
		{(0.49862928696564,0.81151654105998)--(0.52534847863401,0.74460186576363)--(0.58225899722008,0.7433741875493)--(0.60209322151793,0.7656981620798)--(0.58931938009406,0.83016295112284)--(0.53886520818497,0.84806503605021)},
		{(2.6410555968065e-19,0.55501911596182)--(1.4808354815746e-18,0.46062128402398)--(0.069655043672125,0.46252170906469)--(0.090334998477803,0.5105201576241)--(0.062479596074471,0.55461108764192)},
		{(0.49650365527531,0.70994989273511)--(0.43497718019503,0.71938447619238)--(0.40382770580073,0.67612206011063)--(0.41926742897499,0.63048085523994)--(0.48396747686858,0.61838879428297)--(0.51236359176674,0.65572323824777)},
		{(0.66377499068603,0.66963966040744)--(0.68882881153311,0.64088601503834)--(0.75843704503209,0.64936436375305)--(0.7631021338763,0.73509419845066)--(0.68780188208613,0.74023887820549)},
		{(0.73771722997482,0.19370941566284)--(0.74280602817792,0.2633168184927)--(0.68418875779986,0.28843130921516)--(0.65166539015414,0.27080032503853)--(0.64665276281395,0.19867656194114)--(0.70512093060736,0.17459887224965)},
		{(0.091974921981432,0.60200545404424)--(0.15593199973037,0.59378548549754)--(0.17929801850874,0.61385198935047)--(0.15562110912384,0.68802351791194)--(0.099723556961436,0.69058794606744)--(0.071762165041819,0.64637933734506)},
		{(1.0,0.89221791742536)--(1.0,1.0)--(0.90680120300739,1.0)--(0.89872511778609,0.92046603344255)--(0.93546381809738,0.88744169879805)},
		{(0.098282855212861,1.0)--(0.0,1.0)--(3.2382657123805e-18,0.91774728313323)--(0.087187661788198,0.91762708112664)},
		{(0.15190615583306,0.89009117210229)--(0.18080173347841,0.81699036861983)--(0.25037144272769,0.82338973935646)--(0.25627073354994,0.83167390407978)--(0.2440991387246,0.90612103394854)--(0.18915376767155,0.92224702668953)},
		{(1.0,0.4260285521325)--(1.0,0.53758135732378)--(0.9639873004165,0.54000827924375)--(0.90562086673874,0.49745731734894)--(0.90332534096838,0.45661227040353)--(0.9392022151889,0.42197668711201)},
		{(-8.0568679736084e-19,0.36975068838999)--(3.2226216693698e-18,0.27877773252863)--(0.074353229052474,0.27885841601452)--(0.096914643160764,0.33401537300282)--(0.071761178403807,0.37062088708733)},
		{(0.18641782187622,0.72150151437522)--(0.15562110912384,0.68802351791194)--(0.17929801850874,0.61385198935047)--(0.22097129842883,0.61071460758486)--(0.2606647579083,0.6547465077279)--(0.23382313120445,0.71763074899324)},
		{(0.90267708939022,0.10524041471726)--(0.89585777469058,0.1634898855985)--(0.82938320652562,0.18770003404512)--(0.79947883339628,0.16792494306375)--(0.79671105343376,0.10386391156366)--(0.83515530993165,0.079572391411158)},
		{(0.7510082818126,0.44540964367599)--(0.7129412682258,0.47288353123979)--(0.6609860319228,0.45893561983003)--(0.65555621324255,0.38349657492187)--(0.69037814035163,0.36010268815439)--(0.74153850688122,0.3763468622677)},
		{(0.8007778398891,0.4599745910587)--(0.81370165301096,0.52736499787452)--(0.76810727094859,0.55768934658571)--(0.72463388956437,0.54269559300798)--(0.7129412682258,0.47288353123979)--(0.7510082818126,0.44540964367599)},
		{(0.49034178123502,0.47091119134154)--(0.4751613409377,0.44151674314935)--(0.51141822893407,0.37800387101546)--(0.56758692480709,0.38788786901398)--(0.5770556404014,0.46733679559338)--(0.55216201591839,0.4848763144971)},
		{(0.34010653205604,0.3145727941542)--(0.31391253481761,0.35504452404102)--(0.26040160220422,0.35173388212644)--(0.23396036591514,0.30034838204361)--(0.27310724622345,0.25053361869666)--(0.31655376524573,0.25920567249135)},
		{(0.40391634620773,0.31430714048907)--(0.34010653205604,0.3145727941542)--(0.31655376524573,0.25920567249135)--(0.36572639063023,0.21281149275499)--(0.42172692966924,0.26084124339668)},
		{(0.83830034032296,0.43123635442729)--(0.8007778398891,0.4599745910587)--(0.7510082818126,0.44540964367599)--(0.74153850688122,0.3763468622677)--(0.77941790003513,0.35019630476697)--(0.83603929287926,0.36893428567055)},
		{(0.55139757442056,0.5556585621165)--(0.55216201591839,0.4848763144971)--(0.5770556404014,0.46733679559338)--(0.63074047496544,0.47875490164281)--(0.63797036149734,0.56086228377793)--(0.59114493366791,0.58104088182392)},
		{(0.48992125349184,-6.3178208718301e-18)--(0.57579246010271,-4.9556037627779e-18)--(0.57444644094745,0.084653717689775)--(0.52692325543634,0.10290397147954)--(0.48908546549103,0.079415129582681)},
		{(0.41926742897499,0.63048085523994)--(0.38830912710478,0.58809988488896)--(0.41307582812368,0.53250224411983)--(0.45787096486917,0.52996812060041)--(0.49617510278083,0.57455891874653)--(0.48396747686858,0.61838879428297)},
		{(1.0,0.53758135732378)--(1.0,0.66016748325509)--(0.9491529504042,0.65873122068471)--(0.91244065180808,0.6062954385183)--(0.9639873004165,0.54000827924375)},
		{(0.51369814765298,0.28579605366575)--(0.49200407224827,0.34636662618389)--(0.42341532865985,0.34383551279366)--(0.40391634620773,0.31430714048907)--(0.42172692966924,0.26084124339668)--(0.46399205474278,0.24528359574481)},
		{(0.0,0.0)--(0.075010034820169,7.3506279061383e-19)--(0.086485529574877,0.081635364593292)--(0.077785334870339,0.093268733958092)--(7.9754597912455e-18,0.096242174388719)},
		{(0.55562384132193,0.2731671852179)--(0.51369814765298,0.28579605366575)--(0.46399205474278,0.24528359574481)--(0.47634554951811,0.18883945644299)--(0.52062018995623,0.17130903915791)--(0.56413378794781,0.19754467506834)},
		{(0.15759072575741,0.51702466680117)--(0.090334998477803,0.5105201576241)--(0.069655043672125,0.46252170906469)--(0.094779729161039,0.42552694771552)--(0.1539172541449,0.42682959832593)--(0.17481075348646,0.50192012981204)}
    } {
        \draw[thick, color=\accentcolor] \polygon -- cycle;
    }

	% Domain.
    \draw[thick, color=\documentcolor]
        (0,0) -- (1,0) -- (1,1) -- (0,1) -- cycle;

\end{tikzpicture}
            \end{figure}
        }
        \vspace*{\fill}

    \end{multicols}
    \vspace*{\fill}
    
\end{frame}

\subsection{Polynomial Basis and Quadratures}

\begin{frame}
    \frametitle{Polynomials}

    \vspace*{\fill}
    \begin{center}
        {\color{\accentcolor} \Large \textbf{Legendre Basis}}
    \end{center}

    \begin{multicols}{2}

        % Properties.
        \begin{center}
            {\color{\accentcolor} \Large \textbf{Properties}}
            \vspace*{0.5cm}

            \begin{minipage}{0.4\textwidth}
                \begin{itemize}
                    \item Easy to generate through an iterative formula
                    \item Estimates of regularity of solutions from coefficients decay
                \end{itemize}
            \end{minipage}
        \end{center}

        \vfill\null
        \columnbreak

        % Generation.
        \begin{center}
            {\color{\accentcolor} \Large \textbf{Generation}}
            \vspace*{0.125cm}

            \begin{minipage}{0.4\textwidth}
                \begin{align*}
                    \Legendre_n(\xi) &= \sum_{j = 0}^{n} \binom{n}{j} \binom{n + j}{j} \left( \frac{\xi - 1}{2} \right)^{j} \\
                    \Legendre_n^{\prime}(\xi) &= \frac12 \sum_{j = 1}^{n} \binom{n}{j} \binom{n + j}{j} \left( \frac{\xi - 1}{2} \right)^{j - 1}
                \end{align*}
            \end{minipage}
        \end{center}
    \end{multicols}

    \vspace*{\fill}

    \begin{center}
        {\color{\accentcolor} \Large \textbf{Gauss-Legendre Quadrature}}
    \end{center}

    \begin{center}
        \begin{minipage}{0.75\textwidth}
            \begin{description}
                \item[Sub-triangulation] Easier handling of reference maps
                \item[Arbitrary quadrature order] Sufficiently high quadrature order to avoid introducing quadrature errors
            \end{description}
        \end{minipage}
    \end{center}
    \vspace*{\fill}
    
\end{frame}

\subsection{\texorpdfstring{\acrshort{cdr}}{} Problems}

\begin{frame} % Matrices.
    \frametitle{Assembly of a \acrshort{cdr} Problem}

    \setbeamercovered{transparent}

    \vspace*{\fill}
    \begin{multicols}{2}
        
        \vspace*{\fill}
        \begin{center}
            {\color{\accentcolor} \Large \textbf{Matrix Building Steps}}
            \vspace*{0.5cm}

            \begin{minipage}{0.4\textwidth}
                \begin{enumerate}[<+>]
                    \item<2> Time Volume Integrals
                    \item<3> Space Volume Integrals
                    \item<4> Space Interface Integrals
                    \item<5> Time Interface Integrals
                \end{enumerate}
            \end{minipage}
        \end{center}
        \vspace*{\fill}

        \vfill\null
        \columnbreak

        \vspace*{\fill}
        \only<1>{
            \begin{figure}[!ht]
                \centering
                \begin{tikzpicture}[scale=4.0, line join=round]

	% Domain, filled.
    \draw[thick, color=\documentcolor, fill=white]
        (-0.0081967213,0) -- (1,0) -- (1,1.0081967213) -- (-0.0081967213,1.0081967213) -- cycle;

	% Points.
    \foreach \point in {
		(0.0081967213, 0.9918032787),
		(0.0081967213, 0.9836065574),
		(0.0081967213, 0.9754098361),
		(0.0081967213, 0.9672131148),
		(0.0081967213, 0.9590163934),
		(0.0081967213, 0.9508196721),
		(0.0081967213, 0.9426229508),
		(0.0081967213, 0.9344262295),
		(0.0081967213, 0.9262295082),
		(0.0081967213, 0.9180327869),
		(0.0081967213, 0.9098360656),
		(0.0081967213, 0.9016393443),
		(0.0081967213, 0.7950819672),
		(0.0081967213, 0.7868852459),
		(0.0081967213, 0.7786885246),
		(0.0081967213, 0.7704918033),
		(0.0081967213, 0.7622950820),
		(0.0081967213, 0.7540983607),
		(0.0081967213, 0.7459016393),
		(0.0081967213, 0.7377049180),
		(0.0081967213, 0.7295081967),
		(0.0081967213, 0.7213114754),
		(0.0081967213, 0.7131147541),
		(0.0081967213, 0.7049180328),
		(0.0081967213, 0.6967213115),
		(0.0081967213, 0.6885245902),
		(0.0081967213, 0.6803278689),
		(0.0081967213, 0.6721311475),
		(0.0081967213, 0.6639344262),
		(0.0081967213, 0.6557377049),
		(0.0081967213, 0.5491803279),
		(0.0081967213, 0.5409836066),
		(0.0081967213, 0.5327868852),
		(0.0081967213, 0.5245901639),
		(0.0081967213, 0.5163934426),
		(0.0081967213, 0.5081967213),
		(0.0163934426, 0.9918032787),
		(0.0163934426, 0.9836065574),
		(0.0163934426, 0.9754098361),
		(0.0163934426, 0.9672131148),
		(0.0163934426, 0.9590163934),
		(0.0163934426, 0.9508196721),
		(0.0163934426, 0.9426229508),
		(0.0163934426, 0.9344262295),
		(0.0163934426, 0.9262295082),
		(0.0163934426, 0.9180327869),
		(0.0163934426, 0.9098360656),
		(0.0163934426, 0.9016393443),
		(0.0163934426, 0.7950819672),
		(0.0163934426, 0.7868852459),
		(0.0163934426, 0.7786885246),
		(0.0163934426, 0.7704918033),
		(0.0163934426, 0.7622950820),
		(0.0163934426, 0.7540983607),
		(0.0163934426, 0.7459016393),
		(0.0163934426, 0.7377049180),
		(0.0163934426, 0.7295081967),
		(0.0163934426, 0.7213114754),
		(0.0163934426, 0.7131147541),
		(0.0163934426, 0.7049180328),
		(0.0163934426, 0.6967213115),
		(0.0163934426, 0.6885245902),
		(0.0163934426, 0.6803278689),
		(0.0163934426, 0.6721311475),
		(0.0163934426, 0.6639344262),
		(0.0163934426, 0.6557377049),
		(0.0163934426, 0.5491803279),
		(0.0163934426, 0.5409836066),
		(0.0163934426, 0.5327868852),
		(0.0163934426, 0.5245901639),
		(0.0163934426, 0.5163934426),
		(0.0163934426, 0.5081967213),
		(0.0245901639, 0.9918032787),
		(0.0245901639, 0.9836065574),
		(0.0245901639, 0.9754098361),
		(0.0245901639, 0.9672131148),
		(0.0245901639, 0.9590163934),
		(0.0245901639, 0.9508196721),
		(0.0245901639, 0.9426229508),
		(0.0245901639, 0.9344262295),
		(0.0245901639, 0.9262295082),
		(0.0245901639, 0.9180327869),
		(0.0245901639, 0.9098360656),
		(0.0245901639, 0.9016393443),
		(0.0245901639, 0.7950819672),
		(0.0245901639, 0.7868852459),
		(0.0245901639, 0.7786885246),
		(0.0245901639, 0.7704918033),
		(0.0245901639, 0.7622950820),
		(0.0245901639, 0.7540983607),
		(0.0245901639, 0.7459016393),
		(0.0245901639, 0.7377049180),
		(0.0245901639, 0.7295081967),
		(0.0245901639, 0.7213114754),
		(0.0245901639, 0.7131147541),
		(0.0245901639, 0.7049180328),
		(0.0245901639, 0.6967213115),
		(0.0245901639, 0.6885245902),
		(0.0245901639, 0.6803278689),
		(0.0245901639, 0.6721311475),
		(0.0245901639, 0.6639344262),
		(0.0245901639, 0.6557377049),
		(0.0245901639, 0.5491803279),
		(0.0245901639, 0.5409836066),
		(0.0245901639, 0.5327868852),
		(0.0245901639, 0.5245901639),
		(0.0245901639, 0.5163934426),
		(0.0245901639, 0.5081967213),
		(0.0327868852, 0.9918032787),
		(0.0327868852, 0.9836065574),
		(0.0327868852, 0.9754098361),
		(0.0327868852, 0.9672131148),
		(0.0327868852, 0.9590163934),
		(0.0327868852, 0.9508196721),
		(0.0327868852, 0.9426229508),
		(0.0327868852, 0.9344262295),
		(0.0327868852, 0.9262295082),
		(0.0327868852, 0.9180327869),
		(0.0327868852, 0.9098360656),
		(0.0327868852, 0.9016393443),
		(0.0327868852, 0.7950819672),
		(0.0327868852, 0.7868852459),
		(0.0327868852, 0.7786885246),
		(0.0327868852, 0.7704918033),
		(0.0327868852, 0.7622950820),
		(0.0327868852, 0.7540983607),
		(0.0327868852, 0.7459016393),
		(0.0327868852, 0.7377049180),
		(0.0327868852, 0.7295081967),
		(0.0327868852, 0.7213114754),
		(0.0327868852, 0.7131147541),
		(0.0327868852, 0.7049180328),
		(0.0327868852, 0.6967213115),
		(0.0327868852, 0.6885245902),
		(0.0327868852, 0.6803278689),
		(0.0327868852, 0.6721311475),
		(0.0327868852, 0.6639344262),
		(0.0327868852, 0.6557377049),
		(0.0327868852, 0.5491803279),
		(0.0327868852, 0.5409836066),
		(0.0327868852, 0.5327868852),
		(0.0327868852, 0.5245901639),
		(0.0327868852, 0.5163934426),
		(0.0327868852, 0.5081967213),
		(0.0409836066, 0.9918032787),
		(0.0409836066, 0.9836065574),
		(0.0409836066, 0.9754098361),
		(0.0409836066, 0.9672131148),
		(0.0409836066, 0.9590163934),
		(0.0409836066, 0.9508196721),
		(0.0409836066, 0.9426229508),
		(0.0409836066, 0.9344262295),
		(0.0409836066, 0.9262295082),
		(0.0409836066, 0.9180327869),
		(0.0409836066, 0.9098360656),
		(0.0409836066, 0.9016393443),
		(0.0409836066, 0.7950819672),
		(0.0409836066, 0.7868852459),
		(0.0409836066, 0.7786885246),
		(0.0409836066, 0.7704918033),
		(0.0409836066, 0.7622950820),
		(0.0409836066, 0.7540983607),
		(0.0409836066, 0.7459016393),
		(0.0409836066, 0.7377049180),
		(0.0409836066, 0.7295081967),
		(0.0409836066, 0.7213114754),
		(0.0409836066, 0.7131147541),
		(0.0409836066, 0.7049180328),
		(0.0409836066, 0.6967213115),
		(0.0409836066, 0.6885245902),
		(0.0409836066, 0.6803278689),
		(0.0409836066, 0.6721311475),
		(0.0409836066, 0.6639344262),
		(0.0409836066, 0.6557377049),
		(0.0409836066, 0.5491803279),
		(0.0409836066, 0.5409836066),
		(0.0409836066, 0.5327868852),
		(0.0409836066, 0.5245901639),
		(0.0409836066, 0.5163934426),
		(0.0409836066, 0.5081967213),
		(0.0491803279, 0.9918032787),
		(0.0491803279, 0.9836065574),
		(0.0491803279, 0.9754098361),
		(0.0491803279, 0.9672131148),
		(0.0491803279, 0.9590163934),
		(0.0491803279, 0.9508196721),
		(0.0491803279, 0.9426229508),
		(0.0491803279, 0.9344262295),
		(0.0491803279, 0.9262295082),
		(0.0491803279, 0.9180327869),
		(0.0491803279, 0.9098360656),
		(0.0491803279, 0.9016393443),
		(0.0491803279, 0.7950819672),
		(0.0491803279, 0.7868852459),
		(0.0491803279, 0.7786885246),
		(0.0491803279, 0.7704918033),
		(0.0491803279, 0.7622950820),
		(0.0491803279, 0.7540983607),
		(0.0491803279, 0.7459016393),
		(0.0491803279, 0.7377049180),
		(0.0491803279, 0.7295081967),
		(0.0491803279, 0.7213114754),
		(0.0491803279, 0.7131147541),
		(0.0491803279, 0.7049180328),
		(0.0491803279, 0.6967213115),
		(0.0491803279, 0.6885245902),
		(0.0491803279, 0.6803278689),
		(0.0491803279, 0.6721311475),
		(0.0491803279, 0.6639344262),
		(0.0491803279, 0.6557377049),
		(0.0491803279, 0.5491803279),
		(0.0491803279, 0.5409836066),
		(0.0491803279, 0.5327868852),
		(0.0491803279, 0.5245901639),
		(0.0491803279, 0.5163934426),
		(0.0491803279, 0.5081967213),
		(0.0573770492, 0.9918032787),
		(0.0573770492, 0.9836065574),
		(0.0573770492, 0.9754098361),
		(0.0573770492, 0.9672131148),
		(0.0573770492, 0.9590163934),
		(0.0573770492, 0.9508196721),
		(0.0573770492, 0.9426229508),
		(0.0573770492, 0.9344262295),
		(0.0573770492, 0.9262295082),
		(0.0573770492, 0.9180327869),
		(0.0573770492, 0.9098360656),
		(0.0573770492, 0.9016393443),
		(0.0573770492, 0.8442622951),
		(0.0573770492, 0.8360655738),
		(0.0573770492, 0.8278688525),
		(0.0573770492, 0.8196721311),
		(0.0573770492, 0.8114754098),
		(0.0573770492, 0.8032786885),
		(0.0573770492, 0.7459016393),
		(0.0573770492, 0.7377049180),
		(0.0573770492, 0.7295081967),
		(0.0573770492, 0.7213114754),
		(0.0573770492, 0.7131147541),
		(0.0573770492, 0.7049180328),
		(0.0573770492, 0.6967213115),
		(0.0573770492, 0.6885245902),
		(0.0573770492, 0.6803278689),
		(0.0573770492, 0.6721311475),
		(0.0573770492, 0.6639344262),
		(0.0573770492, 0.6557377049),
		(0.0655737705, 0.9918032787),
		(0.0655737705, 0.9836065574),
		(0.0655737705, 0.9754098361),
		(0.0655737705, 0.9672131148),
		(0.0655737705, 0.9590163934),
		(0.0655737705, 0.9508196721),
		(0.0655737705, 0.9426229508),
		(0.0655737705, 0.9344262295),
		(0.0655737705, 0.9262295082),
		(0.0655737705, 0.9180327869),
		(0.0655737705, 0.9098360656),
		(0.0655737705, 0.9016393443),
		(0.0655737705, 0.8442622951),
		(0.0655737705, 0.8360655738),
		(0.0655737705, 0.8278688525),
		(0.0655737705, 0.8196721311),
		(0.0655737705, 0.8114754098),
		(0.0655737705, 0.8032786885),
		(0.0655737705, 0.7459016393),
		(0.0655737705, 0.7377049180),
		(0.0655737705, 0.7295081967),
		(0.0655737705, 0.7213114754),
		(0.0655737705, 0.7131147541),
		(0.0655737705, 0.7049180328),
		(0.0655737705, 0.6967213115),
		(0.0655737705, 0.6885245902),
		(0.0655737705, 0.6803278689),
		(0.0655737705, 0.6721311475),
		(0.0655737705, 0.6639344262),
		(0.0655737705, 0.6557377049),
		(0.0737704918, 0.9918032787),
		(0.0737704918, 0.9836065574),
		(0.0737704918, 0.9754098361),
		(0.0737704918, 0.9672131148),
		(0.0737704918, 0.9590163934),
		(0.0737704918, 0.9508196721),
		(0.0737704918, 0.9426229508),
		(0.0737704918, 0.9344262295),
		(0.0737704918, 0.9262295082),
		(0.0737704918, 0.9180327869),
		(0.0737704918, 0.9098360656),
		(0.0737704918, 0.9016393443),
		(0.0737704918, 0.8442622951),
		(0.0737704918, 0.8360655738),
		(0.0737704918, 0.8278688525),
		(0.0737704918, 0.8196721311),
		(0.0737704918, 0.8114754098),
		(0.0737704918, 0.8032786885),
		(0.0737704918, 0.7459016393),
		(0.0737704918, 0.7377049180),
		(0.0737704918, 0.7295081967),
		(0.0737704918, 0.7213114754),
		(0.0737704918, 0.7131147541),
		(0.0737704918, 0.7049180328),
		(0.0737704918, 0.6967213115),
		(0.0737704918, 0.6885245902),
		(0.0737704918, 0.6803278689),
		(0.0737704918, 0.6721311475),
		(0.0737704918, 0.6639344262),
		(0.0737704918, 0.6557377049),
		(0.0819672131, 0.9918032787),
		(0.0819672131, 0.9836065574),
		(0.0819672131, 0.9754098361),
		(0.0819672131, 0.9672131148),
		(0.0819672131, 0.9590163934),
		(0.0819672131, 0.9508196721),
		(0.0819672131, 0.9426229508),
		(0.0819672131, 0.9344262295),
		(0.0819672131, 0.9262295082),
		(0.0819672131, 0.9180327869),
		(0.0819672131, 0.9098360656),
		(0.0819672131, 0.9016393443),
		(0.0819672131, 0.8442622951),
		(0.0819672131, 0.8360655738),
		(0.0819672131, 0.8278688525),
		(0.0819672131, 0.8196721311),
		(0.0819672131, 0.8114754098),
		(0.0819672131, 0.8032786885),
		(0.0819672131, 0.7459016393),
		(0.0819672131, 0.7377049180),
		(0.0819672131, 0.7295081967),
		(0.0819672131, 0.7213114754),
		(0.0819672131, 0.7131147541),
		(0.0819672131, 0.7049180328),
		(0.0819672131, 0.6967213115),
		(0.0819672131, 0.6885245902),
		(0.0819672131, 0.6803278689),
		(0.0819672131, 0.6721311475),
		(0.0819672131, 0.6639344262),
		(0.0819672131, 0.6557377049),
		(0.0901639344, 0.9918032787),
		(0.0901639344, 0.9836065574),
		(0.0901639344, 0.9754098361),
		(0.0901639344, 0.9672131148),
		(0.0901639344, 0.9590163934),
		(0.0901639344, 0.9508196721),
		(0.0901639344, 0.9426229508),
		(0.0901639344, 0.9344262295),
		(0.0901639344, 0.9262295082),
		(0.0901639344, 0.9180327869),
		(0.0901639344, 0.9098360656),
		(0.0901639344, 0.9016393443),
		(0.0901639344, 0.8442622951),
		(0.0901639344, 0.8360655738),
		(0.0901639344, 0.8278688525),
		(0.0901639344, 0.8196721311),
		(0.0901639344, 0.8114754098),
		(0.0901639344, 0.8032786885),
		(0.0901639344, 0.7459016393),
		(0.0901639344, 0.7377049180),
		(0.0901639344, 0.7295081967),
		(0.0901639344, 0.7213114754),
		(0.0901639344, 0.7131147541),
		(0.0901639344, 0.7049180328),
		(0.0901639344, 0.6967213115),
		(0.0901639344, 0.6885245902),
		(0.0901639344, 0.6803278689),
		(0.0901639344, 0.6721311475),
		(0.0901639344, 0.6639344262),
		(0.0901639344, 0.6557377049),
		(0.0983606557, 0.9918032787),
		(0.0983606557, 0.9836065574),
		(0.0983606557, 0.9754098361),
		(0.0983606557, 0.9672131148),
		(0.0983606557, 0.9590163934),
		(0.0983606557, 0.9508196721),
		(0.0983606557, 0.9426229508),
		(0.0983606557, 0.9344262295),
		(0.0983606557, 0.9262295082),
		(0.0983606557, 0.9180327869),
		(0.0983606557, 0.9098360656),
		(0.0983606557, 0.9016393443),
		(0.0983606557, 0.8442622951),
		(0.0983606557, 0.8360655738),
		(0.0983606557, 0.8278688525),
		(0.0983606557, 0.8196721311),
		(0.0983606557, 0.8114754098),
		(0.0983606557, 0.8032786885),
		(0.0983606557, 0.7459016393),
		(0.0983606557, 0.7377049180),
		(0.0983606557, 0.7295081967),
		(0.0983606557, 0.7213114754),
		(0.0983606557, 0.7131147541),
		(0.0983606557, 0.7049180328),
		(0.0983606557, 0.6967213115),
		(0.0983606557, 0.6885245902),
		(0.0983606557, 0.6803278689),
		(0.0983606557, 0.6721311475),
		(0.0983606557, 0.6639344262),
		(0.0983606557, 0.6557377049),
		(0.1065573770, 0.8934426230),
		(0.1065573770, 0.8852459016),
		(0.1065573770, 0.8770491803),
		(0.1065573770, 0.8688524590),
		(0.1065573770, 0.8606557377),
		(0.1065573770, 0.8524590164),
		(0.1065573770, 0.7950819672),
		(0.1065573770, 0.7868852459),
		(0.1065573770, 0.7786885246),
		(0.1065573770, 0.7704918033),
		(0.1065573770, 0.7622950820),
		(0.1065573770, 0.7540983607),
		(0.1065573770, 0.5491803279),
		(0.1065573770, 0.5409836066),
		(0.1065573770, 0.5327868852),
		(0.1065573770, 0.5245901639),
		(0.1065573770, 0.5163934426),
		(0.1065573770, 0.5081967213),
		(0.1147540984, 0.8934426230),
		(0.1147540984, 0.8852459016),
		(0.1147540984, 0.8770491803),
		(0.1147540984, 0.8688524590),
		(0.1147540984, 0.8606557377),
		(0.1147540984, 0.8524590164),
		(0.1147540984, 0.7950819672),
		(0.1147540984, 0.7868852459),
		(0.1147540984, 0.7786885246),
		(0.1147540984, 0.7704918033),
		(0.1147540984, 0.7622950820),
		(0.1147540984, 0.7540983607),
		(0.1147540984, 0.5491803279),
		(0.1147540984, 0.5409836066),
		(0.1147540984, 0.5327868852),
		(0.1147540984, 0.5245901639),
		(0.1147540984, 0.5163934426),
		(0.1147540984, 0.5081967213),
		(0.1229508197, 0.8934426230),
		(0.1229508197, 0.8852459016),
		(0.1229508197, 0.8770491803),
		(0.1229508197, 0.8688524590),
		(0.1229508197, 0.8606557377),
		(0.1229508197, 0.8524590164),
		(0.1229508197, 0.7950819672),
		(0.1229508197, 0.7868852459),
		(0.1229508197, 0.7786885246),
		(0.1229508197, 0.7704918033),
		(0.1229508197, 0.7622950820),
		(0.1229508197, 0.7540983607),
		(0.1229508197, 0.5491803279),
		(0.1229508197, 0.5409836066),
		(0.1229508197, 0.5327868852),
		(0.1229508197, 0.5245901639),
		(0.1229508197, 0.5163934426),
		(0.1229508197, 0.5081967213),
		(0.1311475410, 0.8934426230),
		(0.1311475410, 0.8852459016),
		(0.1311475410, 0.8770491803),
		(0.1311475410, 0.8688524590),
		(0.1311475410, 0.8606557377),
		(0.1311475410, 0.8524590164),
		(0.1311475410, 0.7950819672),
		(0.1311475410, 0.7868852459),
		(0.1311475410, 0.7786885246),
		(0.1311475410, 0.7704918033),
		(0.1311475410, 0.7622950820),
		(0.1311475410, 0.7540983607),
		(0.1311475410, 0.5491803279),
		(0.1311475410, 0.5409836066),
		(0.1311475410, 0.5327868852),
		(0.1311475410, 0.5245901639),
		(0.1311475410, 0.5163934426),
		(0.1311475410, 0.5081967213),
		(0.1393442623, 0.8934426230),
		(0.1393442623, 0.8852459016),
		(0.1393442623, 0.8770491803),
		(0.1393442623, 0.8688524590),
		(0.1393442623, 0.8606557377),
		(0.1393442623, 0.8524590164),
		(0.1393442623, 0.7950819672),
		(0.1393442623, 0.7868852459),
		(0.1393442623, 0.7786885246),
		(0.1393442623, 0.7704918033),
		(0.1393442623, 0.7622950820),
		(0.1393442623, 0.7540983607),
		(0.1393442623, 0.5491803279),
		(0.1393442623, 0.5409836066),
		(0.1393442623, 0.5327868852),
		(0.1393442623, 0.5245901639),
		(0.1393442623, 0.5163934426),
		(0.1393442623, 0.5081967213),
		(0.1475409836, 0.8934426230),
		(0.1475409836, 0.8852459016),
		(0.1475409836, 0.8770491803),
		(0.1475409836, 0.8688524590),
		(0.1475409836, 0.8606557377),
		(0.1475409836, 0.8524590164),
		(0.1475409836, 0.7950819672),
		(0.1475409836, 0.7868852459),
		(0.1475409836, 0.7786885246),
		(0.1475409836, 0.7704918033),
		(0.1475409836, 0.7622950820),
		(0.1475409836, 0.7540983607),
		(0.1475409836, 0.5491803279),
		(0.1475409836, 0.5409836066),
		(0.1475409836, 0.5327868852),
		(0.1475409836, 0.5245901639),
		(0.1475409836, 0.5163934426),
		(0.1475409836, 0.5081967213),
		(0.1557377049, 0.9426229508),
		(0.1557377049, 0.9344262295),
		(0.1557377049, 0.9262295082),
		(0.1557377049, 0.9180327869),
		(0.1557377049, 0.9098360656),
		(0.1557377049, 0.9016393443),
		(0.1557377049, 0.8442622951),
		(0.1557377049, 0.8360655738),
		(0.1557377049, 0.8278688525),
		(0.1557377049, 0.8196721311),
		(0.1557377049, 0.8114754098),
		(0.1557377049, 0.8032786885),
		(0.1557377049, 0.6967213115),
		(0.1557377049, 0.6885245902),
		(0.1557377049, 0.6803278689),
		(0.1557377049, 0.6721311475),
		(0.1557377049, 0.6639344262),
		(0.1557377049, 0.6557377049),
		(0.1639344262, 0.9426229508),
		(0.1639344262, 0.9344262295),
		(0.1639344262, 0.9262295082),
		(0.1639344262, 0.9180327869),
		(0.1639344262, 0.9098360656),
		(0.1639344262, 0.9016393443),
		(0.1639344262, 0.8442622951),
		(0.1639344262, 0.8360655738),
		(0.1639344262, 0.8278688525),
		(0.1639344262, 0.8196721311),
		(0.1639344262, 0.8114754098),
		(0.1639344262, 0.8032786885),
		(0.1639344262, 0.6967213115),
		(0.1639344262, 0.6885245902),
		(0.1639344262, 0.6803278689),
		(0.1639344262, 0.6721311475),
		(0.1639344262, 0.6639344262),
		(0.1639344262, 0.6557377049),
		(0.1721311475, 0.9426229508),
		(0.1721311475, 0.9344262295),
		(0.1721311475, 0.9262295082),
		(0.1721311475, 0.9180327869),
		(0.1721311475, 0.9098360656),
		(0.1721311475, 0.9016393443),
		(0.1721311475, 0.8442622951),
		(0.1721311475, 0.8360655738),
		(0.1721311475, 0.8278688525),
		(0.1721311475, 0.8196721311),
		(0.1721311475, 0.8114754098),
		(0.1721311475, 0.8032786885),
		(0.1721311475, 0.6967213115),
		(0.1721311475, 0.6885245902),
		(0.1721311475, 0.6803278689),
		(0.1721311475, 0.6721311475),
		(0.1721311475, 0.6639344262),
		(0.1721311475, 0.6557377049),
		(0.1803278689, 0.9426229508),
		(0.1803278689, 0.9344262295),
		(0.1803278689, 0.9262295082),
		(0.1803278689, 0.9180327869),
		(0.1803278689, 0.9098360656),
		(0.1803278689, 0.9016393443),
		(0.1803278689, 0.8442622951),
		(0.1803278689, 0.8360655738),
		(0.1803278689, 0.8278688525),
		(0.1803278689, 0.8196721311),
		(0.1803278689, 0.8114754098),
		(0.1803278689, 0.8032786885),
		(0.1803278689, 0.6967213115),
		(0.1803278689, 0.6885245902),
		(0.1803278689, 0.6803278689),
		(0.1803278689, 0.6721311475),
		(0.1803278689, 0.6639344262),
		(0.1803278689, 0.6557377049),
		(0.1885245902, 0.9426229508),
		(0.1885245902, 0.9344262295),
		(0.1885245902, 0.9262295082),
		(0.1885245902, 0.9180327869),
		(0.1885245902, 0.9098360656),
		(0.1885245902, 0.9016393443),
		(0.1885245902, 0.8442622951),
		(0.1885245902, 0.8360655738),
		(0.1885245902, 0.8278688525),
		(0.1885245902, 0.8196721311),
		(0.1885245902, 0.8114754098),
		(0.1885245902, 0.8032786885),
		(0.1885245902, 0.6967213115),
		(0.1885245902, 0.6885245902),
		(0.1885245902, 0.6803278689),
		(0.1885245902, 0.6721311475),
		(0.1885245902, 0.6639344262),
		(0.1885245902, 0.6557377049),
		(0.1967213115, 0.9426229508),
		(0.1967213115, 0.9344262295),
		(0.1967213115, 0.9262295082),
		(0.1967213115, 0.9180327869),
		(0.1967213115, 0.9098360656),
		(0.1967213115, 0.9016393443),
		(0.1967213115, 0.8442622951),
		(0.1967213115, 0.8360655738),
		(0.1967213115, 0.8278688525),
		(0.1967213115, 0.8196721311),
		(0.1967213115, 0.8114754098),
		(0.1967213115, 0.8032786885),
		(0.1967213115, 0.6967213115),
		(0.1967213115, 0.6885245902),
		(0.1967213115, 0.6803278689),
		(0.1967213115, 0.6721311475),
		(0.1967213115, 0.6639344262),
		(0.1967213115, 0.6557377049),
		(0.2049180328, 0.9918032787),
		(0.2049180328, 0.9836065574),
		(0.2049180328, 0.9754098361),
		(0.2049180328, 0.9672131148),
		(0.2049180328, 0.9590163934),
		(0.2049180328, 0.9508196721),
		(0.2049180328, 0.8934426230),
		(0.2049180328, 0.8852459016),
		(0.2049180328, 0.8770491803),
		(0.2049180328, 0.8688524590),
		(0.2049180328, 0.8606557377),
		(0.2049180328, 0.8524590164),
		(0.2049180328, 0.7950819672),
		(0.2049180328, 0.7868852459),
		(0.2049180328, 0.7786885246),
		(0.2049180328, 0.7704918033),
		(0.2049180328, 0.7622950820),
		(0.2049180328, 0.7540983607),
		(0.2049180328, 0.7459016393),
		(0.2049180328, 0.7377049180),
		(0.2049180328, 0.7295081967),
		(0.2049180328, 0.7213114754),
		(0.2049180328, 0.7131147541),
		(0.2049180328, 0.7049180328),
		(0.2049180328, 0.6475409836),
		(0.2049180328, 0.6393442623),
		(0.2049180328, 0.6311475410),
		(0.2049180328, 0.6229508197),
		(0.2049180328, 0.6147540984),
		(0.2049180328, 0.6065573770),
		(0.2131147541, 0.9918032787),
		(0.2131147541, 0.9836065574),
		(0.2131147541, 0.9754098361),
		(0.2131147541, 0.9672131148),
		(0.2131147541, 0.9590163934),
		(0.2131147541, 0.9508196721),
		(0.2131147541, 0.8934426230),
		(0.2131147541, 0.8852459016),
		(0.2131147541, 0.8770491803),
		(0.2131147541, 0.8688524590),
		(0.2131147541, 0.8606557377),
		(0.2131147541, 0.8524590164),
		(0.2131147541, 0.7950819672),
		(0.2131147541, 0.7868852459),
		(0.2131147541, 0.7786885246),
		(0.2131147541, 0.7704918033),
		(0.2131147541, 0.7622950820),
		(0.2131147541, 0.7540983607),
		(0.2131147541, 0.7459016393),
		(0.2131147541, 0.7377049180),
		(0.2131147541, 0.7295081967),
		(0.2131147541, 0.7213114754),
		(0.2131147541, 0.7131147541),
		(0.2131147541, 0.7049180328),
		(0.2131147541, 0.6475409836),
		(0.2131147541, 0.6393442623),
		(0.2131147541, 0.6311475410),
		(0.2131147541, 0.6229508197),
		(0.2131147541, 0.6147540984),
		(0.2131147541, 0.6065573770),
		(0.2213114754, 0.9918032787),
		(0.2213114754, 0.9836065574),
		(0.2213114754, 0.9754098361),
		(0.2213114754, 0.9672131148),
		(0.2213114754, 0.9590163934),
		(0.2213114754, 0.9508196721),
		(0.2213114754, 0.8934426230),
		(0.2213114754, 0.8852459016),
		(0.2213114754, 0.8770491803),
		(0.2213114754, 0.8688524590),
		(0.2213114754, 0.8606557377),
		(0.2213114754, 0.8524590164),
		(0.2213114754, 0.7950819672),
		(0.2213114754, 0.7868852459),
		(0.2213114754, 0.7786885246),
		(0.2213114754, 0.7704918033),
		(0.2213114754, 0.7622950820),
		(0.2213114754, 0.7540983607),
		(0.2213114754, 0.7459016393),
		(0.2213114754, 0.7377049180),
		(0.2213114754, 0.7295081967),
		(0.2213114754, 0.7213114754),
		(0.2213114754, 0.7131147541),
		(0.2213114754, 0.7049180328),
		(0.2213114754, 0.6475409836),
		(0.2213114754, 0.6393442623),
		(0.2213114754, 0.6311475410),
		(0.2213114754, 0.6229508197),
		(0.2213114754, 0.6147540984),
		(0.2213114754, 0.6065573770),
		(0.2295081967, 0.9918032787),
		(0.2295081967, 0.9836065574),
		(0.2295081967, 0.9754098361),
		(0.2295081967, 0.9672131148),
		(0.2295081967, 0.9590163934),
		(0.2295081967, 0.9508196721),
		(0.2295081967, 0.8934426230),
		(0.2295081967, 0.8852459016),
		(0.2295081967, 0.8770491803),
		(0.2295081967, 0.8688524590),
		(0.2295081967, 0.8606557377),
		(0.2295081967, 0.8524590164),
		(0.2295081967, 0.7950819672),
		(0.2295081967, 0.7868852459),
		(0.2295081967, 0.7786885246),
		(0.2295081967, 0.7704918033),
		(0.2295081967, 0.7622950820),
		(0.2295081967, 0.7540983607),
		(0.2295081967, 0.7459016393),
		(0.2295081967, 0.7377049180),
		(0.2295081967, 0.7295081967),
		(0.2295081967, 0.7213114754),
		(0.2295081967, 0.7131147541),
		(0.2295081967, 0.7049180328),
		(0.2295081967, 0.6475409836),
		(0.2295081967, 0.6393442623),
		(0.2295081967, 0.6311475410),
		(0.2295081967, 0.6229508197),
		(0.2295081967, 0.6147540984),
		(0.2295081967, 0.6065573770),
		(0.2377049180, 0.9918032787),
		(0.2377049180, 0.9836065574),
		(0.2377049180, 0.9754098361),
		(0.2377049180, 0.9672131148),
		(0.2377049180, 0.9590163934),
		(0.2377049180, 0.9508196721),
		(0.2377049180, 0.8934426230),
		(0.2377049180, 0.8852459016),
		(0.2377049180, 0.8770491803),
		(0.2377049180, 0.8688524590),
		(0.2377049180, 0.8606557377),
		(0.2377049180, 0.8524590164),
		(0.2377049180, 0.7950819672),
		(0.2377049180, 0.7868852459),
		(0.2377049180, 0.7786885246),
		(0.2377049180, 0.7704918033),
		(0.2377049180, 0.7622950820),
		(0.2377049180, 0.7540983607),
		(0.2377049180, 0.7459016393),
		(0.2377049180, 0.7377049180),
		(0.2377049180, 0.7295081967),
		(0.2377049180, 0.7213114754),
		(0.2377049180, 0.7131147541),
		(0.2377049180, 0.7049180328),
		(0.2377049180, 0.6475409836),
		(0.2377049180, 0.6393442623),
		(0.2377049180, 0.6311475410),
		(0.2377049180, 0.6229508197),
		(0.2377049180, 0.6147540984),
		(0.2377049180, 0.6065573770),
		(0.2459016393, 0.9918032787),
		(0.2459016393, 0.9836065574),
		(0.2459016393, 0.9754098361),
		(0.2459016393, 0.9672131148),
		(0.2459016393, 0.9590163934),
		(0.2459016393, 0.9508196721),
		(0.2459016393, 0.8934426230),
		(0.2459016393, 0.8852459016),
		(0.2459016393, 0.8770491803),
		(0.2459016393, 0.8688524590),
		(0.2459016393, 0.8606557377),
		(0.2459016393, 0.8524590164),
		(0.2459016393, 0.7950819672),
		(0.2459016393, 0.7868852459),
		(0.2459016393, 0.7786885246),
		(0.2459016393, 0.7704918033),
		(0.2459016393, 0.7622950820),
		(0.2459016393, 0.7540983607),
		(0.2459016393, 0.7459016393),
		(0.2459016393, 0.7377049180),
		(0.2459016393, 0.7295081967),
		(0.2459016393, 0.7213114754),
		(0.2459016393, 0.7131147541),
		(0.2459016393, 0.7049180328),
		(0.2459016393, 0.6475409836),
		(0.2459016393, 0.6393442623),
		(0.2459016393, 0.6311475410),
		(0.2459016393, 0.6229508197),
		(0.2459016393, 0.6147540984),
		(0.2459016393, 0.6065573770),
		(0.2540983607, 0.9918032787),
		(0.2540983607, 0.9836065574),
		(0.2540983607, 0.9754098361),
		(0.2540983607, 0.9672131148),
		(0.2540983607, 0.9590163934),
		(0.2540983607, 0.9508196721),
		(0.2540983607, 0.9426229508),
		(0.2540983607, 0.9344262295),
		(0.2540983607, 0.9262295082),
		(0.2540983607, 0.9180327869),
		(0.2540983607, 0.9098360656),
		(0.2540983607, 0.9016393443),
		(0.2540983607, 0.7950819672),
		(0.2540983607, 0.7868852459),
		(0.2540983607, 0.7786885246),
		(0.2540983607, 0.7704918033),
		(0.2540983607, 0.7622950820),
		(0.2540983607, 0.7540983607),
		(0.2540983607, 0.7459016393),
		(0.2540983607, 0.7377049180),
		(0.2540983607, 0.7295081967),
		(0.2540983607, 0.7213114754),
		(0.2540983607, 0.7131147541),
		(0.2540983607, 0.7049180328),
		(0.2540983607, 0.6475409836),
		(0.2540983607, 0.6393442623),
		(0.2540983607, 0.6311475410),
		(0.2540983607, 0.6229508197),
		(0.2540983607, 0.6147540984),
		(0.2540983607, 0.6065573770),
		(0.2622950820, 0.9918032787),
		(0.2622950820, 0.9836065574),
		(0.2622950820, 0.9754098361),
		(0.2622950820, 0.9672131148),
		(0.2622950820, 0.9590163934),
		(0.2622950820, 0.9508196721),
		(0.2622950820, 0.9426229508),
		(0.2622950820, 0.9344262295),
		(0.2622950820, 0.9262295082),
		(0.2622950820, 0.9180327869),
		(0.2622950820, 0.9098360656),
		(0.2622950820, 0.9016393443),
		(0.2622950820, 0.7950819672),
		(0.2622950820, 0.7868852459),
		(0.2622950820, 0.7786885246),
		(0.2622950820, 0.7704918033),
		(0.2622950820, 0.7622950820),
		(0.2622950820, 0.7540983607),
		(0.2622950820, 0.7459016393),
		(0.2622950820, 0.7377049180),
		(0.2622950820, 0.7295081967),
		(0.2622950820, 0.7213114754),
		(0.2622950820, 0.7131147541),
		(0.2622950820, 0.7049180328),
		(0.2622950820, 0.6475409836),
		(0.2622950820, 0.6393442623),
		(0.2622950820, 0.6311475410),
		(0.2622950820, 0.6229508197),
		(0.2622950820, 0.6147540984),
		(0.2622950820, 0.6065573770),
		(0.2704918033, 0.9918032787),
		(0.2704918033, 0.9836065574),
		(0.2704918033, 0.9754098361),
		(0.2704918033, 0.9672131148),
		(0.2704918033, 0.9590163934),
		(0.2704918033, 0.9508196721),
		(0.2704918033, 0.9426229508),
		(0.2704918033, 0.9344262295),
		(0.2704918033, 0.9262295082),
		(0.2704918033, 0.9180327869),
		(0.2704918033, 0.9098360656),
		(0.2704918033, 0.9016393443),
		(0.2704918033, 0.7950819672),
		(0.2704918033, 0.7868852459),
		(0.2704918033, 0.7786885246),
		(0.2704918033, 0.7704918033),
		(0.2704918033, 0.7622950820),
		(0.2704918033, 0.7540983607),
		(0.2704918033, 0.7459016393),
		(0.2704918033, 0.7377049180),
		(0.2704918033, 0.7295081967),
		(0.2704918033, 0.7213114754),
		(0.2704918033, 0.7131147541),
		(0.2704918033, 0.7049180328),
		(0.2704918033, 0.6475409836),
		(0.2704918033, 0.6393442623),
		(0.2704918033, 0.6311475410),
		(0.2704918033, 0.6229508197),
		(0.2704918033, 0.6147540984),
		(0.2704918033, 0.6065573770),
		(0.2786885246, 0.9918032787),
		(0.2786885246, 0.9836065574),
		(0.2786885246, 0.9754098361),
		(0.2786885246, 0.9672131148),
		(0.2786885246, 0.9590163934),
		(0.2786885246, 0.9508196721),
		(0.2786885246, 0.9426229508),
		(0.2786885246, 0.9344262295),
		(0.2786885246, 0.9262295082),
		(0.2786885246, 0.9180327869),
		(0.2786885246, 0.9098360656),
		(0.2786885246, 0.9016393443),
		(0.2786885246, 0.7950819672),
		(0.2786885246, 0.7868852459),
		(0.2786885246, 0.7786885246),
		(0.2786885246, 0.7704918033),
		(0.2786885246, 0.7622950820),
		(0.2786885246, 0.7540983607),
		(0.2786885246, 0.7459016393),
		(0.2786885246, 0.7377049180),
		(0.2786885246, 0.7295081967),
		(0.2786885246, 0.7213114754),
		(0.2786885246, 0.7131147541),
		(0.2786885246, 0.7049180328),
		(0.2786885246, 0.6475409836),
		(0.2786885246, 0.6393442623),
		(0.2786885246, 0.6311475410),
		(0.2786885246, 0.6229508197),
		(0.2786885246, 0.6147540984),
		(0.2786885246, 0.6065573770),
		(0.2868852459, 0.9918032787),
		(0.2868852459, 0.9836065574),
		(0.2868852459, 0.9754098361),
		(0.2868852459, 0.9672131148),
		(0.2868852459, 0.9590163934),
		(0.2868852459, 0.9508196721),
		(0.2868852459, 0.9426229508),
		(0.2868852459, 0.9344262295),
		(0.2868852459, 0.9262295082),
		(0.2868852459, 0.9180327869),
		(0.2868852459, 0.9098360656),
		(0.2868852459, 0.9016393443),
		(0.2868852459, 0.7950819672),
		(0.2868852459, 0.7868852459),
		(0.2868852459, 0.7786885246),
		(0.2868852459, 0.7704918033),
		(0.2868852459, 0.7622950820),
		(0.2868852459, 0.7540983607),
		(0.2868852459, 0.7459016393),
		(0.2868852459, 0.7377049180),
		(0.2868852459, 0.7295081967),
		(0.2868852459, 0.7213114754),
		(0.2868852459, 0.7131147541),
		(0.2868852459, 0.7049180328),
		(0.2868852459, 0.6475409836),
		(0.2868852459, 0.6393442623),
		(0.2868852459, 0.6311475410),
		(0.2868852459, 0.6229508197),
		(0.2868852459, 0.6147540984),
		(0.2868852459, 0.6065573770),
		(0.2950819672, 0.9918032787),
		(0.2950819672, 0.9836065574),
		(0.2950819672, 0.9754098361),
		(0.2950819672, 0.9672131148),
		(0.2950819672, 0.9590163934),
		(0.2950819672, 0.9508196721),
		(0.2950819672, 0.9426229508),
		(0.2950819672, 0.9344262295),
		(0.2950819672, 0.9262295082),
		(0.2950819672, 0.9180327869),
		(0.2950819672, 0.9098360656),
		(0.2950819672, 0.9016393443),
		(0.2950819672, 0.7950819672),
		(0.2950819672, 0.7868852459),
		(0.2950819672, 0.7786885246),
		(0.2950819672, 0.7704918033),
		(0.2950819672, 0.7622950820),
		(0.2950819672, 0.7540983607),
		(0.2950819672, 0.7459016393),
		(0.2950819672, 0.7377049180),
		(0.2950819672, 0.7295081967),
		(0.2950819672, 0.7213114754),
		(0.2950819672, 0.7131147541),
		(0.2950819672, 0.7049180328),
		(0.2950819672, 0.6475409836),
		(0.2950819672, 0.6393442623),
		(0.2950819672, 0.6311475410),
		(0.2950819672, 0.6229508197),
		(0.2950819672, 0.6147540984),
		(0.2950819672, 0.6065573770),
		(0.3032786885, 0.9918032787),
		(0.3032786885, 0.9836065574),
		(0.3032786885, 0.9754098361),
		(0.3032786885, 0.9672131148),
		(0.3032786885, 0.9590163934),
		(0.3032786885, 0.9508196721),
		(0.3032786885, 0.9426229508),
		(0.3032786885, 0.9344262295),
		(0.3032786885, 0.9262295082),
		(0.3032786885, 0.9180327869),
		(0.3032786885, 0.9098360656),
		(0.3032786885, 0.9016393443),
		(0.3032786885, 0.8442622951),
		(0.3032786885, 0.8360655738),
		(0.3032786885, 0.8278688525),
		(0.3032786885, 0.8196721311),
		(0.3032786885, 0.8114754098),
		(0.3032786885, 0.8032786885),
		(0.3032786885, 0.6967213115),
		(0.3032786885, 0.6885245902),
		(0.3032786885, 0.6803278689),
		(0.3032786885, 0.6721311475),
		(0.3032786885, 0.6639344262),
		(0.3032786885, 0.6557377049),
		(0.3032786885, 0.5983606557),
		(0.3032786885, 0.5901639344),
		(0.3032786885, 0.5819672131),
		(0.3032786885, 0.5737704918),
		(0.3032786885, 0.5655737705),
		(0.3032786885, 0.5573770492),
		(0.3114754098, 0.9918032787),
		(0.3114754098, 0.9836065574),
		(0.3114754098, 0.9754098361),
		(0.3114754098, 0.9672131148),
		(0.3114754098, 0.9590163934),
		(0.3114754098, 0.9508196721),
		(0.3114754098, 0.9426229508),
		(0.3114754098, 0.9344262295),
		(0.3114754098, 0.9262295082),
		(0.3114754098, 0.9180327869),
		(0.3114754098, 0.9098360656),
		(0.3114754098, 0.9016393443),
		(0.3114754098, 0.8442622951),
		(0.3114754098, 0.8360655738),
		(0.3114754098, 0.8278688525),
		(0.3114754098, 0.8196721311),
		(0.3114754098, 0.8114754098),
		(0.3114754098, 0.8032786885),
		(0.3114754098, 0.6967213115),
		(0.3114754098, 0.6885245902),
		(0.3114754098, 0.6803278689),
		(0.3114754098, 0.6721311475),
		(0.3114754098, 0.6639344262),
		(0.3114754098, 0.6557377049),
		(0.3114754098, 0.5983606557),
		(0.3114754098, 0.5901639344),
		(0.3114754098, 0.5819672131),
		(0.3114754098, 0.5737704918),
		(0.3114754098, 0.5655737705),
		(0.3114754098, 0.5573770492),
		(0.3196721311, 0.9918032787),
		(0.3196721311, 0.9836065574),
		(0.3196721311, 0.9754098361),
		(0.3196721311, 0.9672131148),
		(0.3196721311, 0.9590163934),
		(0.3196721311, 0.9508196721),
		(0.3196721311, 0.9426229508),
		(0.3196721311, 0.9344262295),
		(0.3196721311, 0.9262295082),
		(0.3196721311, 0.9180327869),
		(0.3196721311, 0.9098360656),
		(0.3196721311, 0.9016393443),
		(0.3196721311, 0.8442622951),
		(0.3196721311, 0.8360655738),
		(0.3196721311, 0.8278688525),
		(0.3196721311, 0.8196721311),
		(0.3196721311, 0.8114754098),
		(0.3196721311, 0.8032786885),
		(0.3196721311, 0.6967213115),
		(0.3196721311, 0.6885245902),
		(0.3196721311, 0.6803278689),
		(0.3196721311, 0.6721311475),
		(0.3196721311, 0.6639344262),
		(0.3196721311, 0.6557377049),
		(0.3196721311, 0.5983606557),
		(0.3196721311, 0.5901639344),
		(0.3196721311, 0.5819672131),
		(0.3196721311, 0.5737704918),
		(0.3196721311, 0.5655737705),
		(0.3196721311, 0.5573770492),
		(0.3278688525, 0.9918032787),
		(0.3278688525, 0.9836065574),
		(0.3278688525, 0.9754098361),
		(0.3278688525, 0.9672131148),
		(0.3278688525, 0.9590163934),
		(0.3278688525, 0.9508196721),
		(0.3278688525, 0.9426229508),
		(0.3278688525, 0.9344262295),
		(0.3278688525, 0.9262295082),
		(0.3278688525, 0.9180327869),
		(0.3278688525, 0.9098360656),
		(0.3278688525, 0.9016393443),
		(0.3278688525, 0.8442622951),
		(0.3278688525, 0.8360655738),
		(0.3278688525, 0.8278688525),
		(0.3278688525, 0.8196721311),
		(0.3278688525, 0.8114754098),
		(0.3278688525, 0.8032786885),
		(0.3278688525, 0.6967213115),
		(0.3278688525, 0.6885245902),
		(0.3278688525, 0.6803278689),
		(0.3278688525, 0.6721311475),
		(0.3278688525, 0.6639344262),
		(0.3278688525, 0.6557377049),
		(0.3278688525, 0.5983606557),
		(0.3278688525, 0.5901639344),
		(0.3278688525, 0.5819672131),
		(0.3278688525, 0.5737704918),
		(0.3278688525, 0.5655737705),
		(0.3278688525, 0.5573770492),
		(0.3360655738, 0.9918032787),
		(0.3360655738, 0.9836065574),
		(0.3360655738, 0.9754098361),
		(0.3360655738, 0.9672131148),
		(0.3360655738, 0.9590163934),
		(0.3360655738, 0.9508196721),
		(0.3360655738, 0.9426229508),
		(0.3360655738, 0.9344262295),
		(0.3360655738, 0.9262295082),
		(0.3360655738, 0.9180327869),
		(0.3360655738, 0.9098360656),
		(0.3360655738, 0.9016393443),
		(0.3360655738, 0.8442622951),
		(0.3360655738, 0.8360655738),
		(0.3360655738, 0.8278688525),
		(0.3360655738, 0.8196721311),
		(0.3360655738, 0.8114754098),
		(0.3360655738, 0.8032786885),
		(0.3360655738, 0.6967213115),
		(0.3360655738, 0.6885245902),
		(0.3360655738, 0.6803278689),
		(0.3360655738, 0.6721311475),
		(0.3360655738, 0.6639344262),
		(0.3360655738, 0.6557377049),
		(0.3360655738, 0.5983606557),
		(0.3360655738, 0.5901639344),
		(0.3360655738, 0.5819672131),
		(0.3360655738, 0.5737704918),
		(0.3360655738, 0.5655737705),
		(0.3360655738, 0.5573770492),
		(0.3442622951, 0.9918032787),
		(0.3442622951, 0.9836065574),
		(0.3442622951, 0.9754098361),
		(0.3442622951, 0.9672131148),
		(0.3442622951, 0.9590163934),
		(0.3442622951, 0.9508196721),
		(0.3442622951, 0.9426229508),
		(0.3442622951, 0.9344262295),
		(0.3442622951, 0.9262295082),
		(0.3442622951, 0.9180327869),
		(0.3442622951, 0.9098360656),
		(0.3442622951, 0.9016393443),
		(0.3442622951, 0.8442622951),
		(0.3442622951, 0.8360655738),
		(0.3442622951, 0.8278688525),
		(0.3442622951, 0.8196721311),
		(0.3442622951, 0.8114754098),
		(0.3442622951, 0.8032786885),
		(0.3442622951, 0.6967213115),
		(0.3442622951, 0.6885245902),
		(0.3442622951, 0.6803278689),
		(0.3442622951, 0.6721311475),
		(0.3442622951, 0.6639344262),
		(0.3442622951, 0.6557377049),
		(0.3442622951, 0.5983606557),
		(0.3442622951, 0.5901639344),
		(0.3442622951, 0.5819672131),
		(0.3442622951, 0.5737704918),
		(0.3442622951, 0.5655737705),
		(0.3442622951, 0.5573770492),
		(0.3524590164, 0.7950819672),
		(0.3524590164, 0.7868852459),
		(0.3524590164, 0.7786885246),
		(0.3524590164, 0.7704918033),
		(0.3524590164, 0.7622950820),
		(0.3524590164, 0.7540983607),
		(0.3524590164, 0.7459016393),
		(0.3524590164, 0.7377049180),
		(0.3524590164, 0.7295081967),
		(0.3524590164, 0.7213114754),
		(0.3524590164, 0.7131147541),
		(0.3524590164, 0.7049180328),
		(0.3524590164, 0.6475409836),
		(0.3524590164, 0.6393442623),
		(0.3524590164, 0.6311475410),
		(0.3524590164, 0.6229508197),
		(0.3524590164, 0.6147540984),
		(0.3524590164, 0.6065573770),
		(0.3606557377, 0.7950819672),
		(0.3606557377, 0.7868852459),
		(0.3606557377, 0.7786885246),
		(0.3606557377, 0.7704918033),
		(0.3606557377, 0.7622950820),
		(0.3606557377, 0.7540983607),
		(0.3606557377, 0.7459016393),
		(0.3606557377, 0.7377049180),
		(0.3606557377, 0.7295081967),
		(0.3606557377, 0.7213114754),
		(0.3606557377, 0.7131147541),
		(0.3606557377, 0.7049180328),
		(0.3606557377, 0.6475409836),
		(0.3606557377, 0.6393442623),
		(0.3606557377, 0.6311475410),
		(0.3606557377, 0.6229508197),
		(0.3606557377, 0.6147540984),
		(0.3606557377, 0.6065573770),
		(0.3688524590, 0.7950819672),
		(0.3688524590, 0.7868852459),
		(0.3688524590, 0.7786885246),
		(0.3688524590, 0.7704918033),
		(0.3688524590, 0.7622950820),
		(0.3688524590, 0.7540983607),
		(0.3688524590, 0.7459016393),
		(0.3688524590, 0.7377049180),
		(0.3688524590, 0.7295081967),
		(0.3688524590, 0.7213114754),
		(0.3688524590, 0.7131147541),
		(0.3688524590, 0.7049180328),
		(0.3688524590, 0.6475409836),
		(0.3688524590, 0.6393442623),
		(0.3688524590, 0.6311475410),
		(0.3688524590, 0.6229508197),
		(0.3688524590, 0.6147540984),
		(0.3688524590, 0.6065573770),
		(0.3770491803, 0.7950819672),
		(0.3770491803, 0.7868852459),
		(0.3770491803, 0.7786885246),
		(0.3770491803, 0.7704918033),
		(0.3770491803, 0.7622950820),
		(0.3770491803, 0.7540983607),
		(0.3770491803, 0.7459016393),
		(0.3770491803, 0.7377049180),
		(0.3770491803, 0.7295081967),
		(0.3770491803, 0.7213114754),
		(0.3770491803, 0.7131147541),
		(0.3770491803, 0.7049180328),
		(0.3770491803, 0.6475409836),
		(0.3770491803, 0.6393442623),
		(0.3770491803, 0.6311475410),
		(0.3770491803, 0.6229508197),
		(0.3770491803, 0.6147540984),
		(0.3770491803, 0.6065573770),
		(0.3852459016, 0.7950819672),
		(0.3852459016, 0.7868852459),
		(0.3852459016, 0.7786885246),
		(0.3852459016, 0.7704918033),
		(0.3852459016, 0.7622950820),
		(0.3852459016, 0.7540983607),
		(0.3852459016, 0.7459016393),
		(0.3852459016, 0.7377049180),
		(0.3852459016, 0.7295081967),
		(0.3852459016, 0.7213114754),
		(0.3852459016, 0.7131147541),
		(0.3852459016, 0.7049180328),
		(0.3852459016, 0.6475409836),
		(0.3852459016, 0.6393442623),
		(0.3852459016, 0.6311475410),
		(0.3852459016, 0.6229508197),
		(0.3852459016, 0.6147540984),
		(0.3852459016, 0.6065573770),
		(0.3934426230, 0.7950819672),
		(0.3934426230, 0.7868852459),
		(0.3934426230, 0.7786885246),
		(0.3934426230, 0.7704918033),
		(0.3934426230, 0.7622950820),
		(0.3934426230, 0.7540983607),
		(0.3934426230, 0.7459016393),
		(0.3934426230, 0.7377049180),
		(0.3934426230, 0.7295081967),
		(0.3934426230, 0.7213114754),
		(0.3934426230, 0.7131147541),
		(0.3934426230, 0.7049180328),
		(0.3934426230, 0.6475409836),
		(0.3934426230, 0.6393442623),
		(0.3934426230, 0.6311475410),
		(0.3934426230, 0.6229508197),
		(0.3934426230, 0.6147540984),
		(0.3934426230, 0.6065573770),
		(0.4016393443, 0.6967213115),
		(0.4016393443, 0.6885245902),
		(0.4016393443, 0.6803278689),
		(0.4016393443, 0.6721311475),
		(0.4016393443, 0.6639344262),
		(0.4016393443, 0.6557377049),
		(0.4016393443, 0.5983606557),
		(0.4016393443, 0.5901639344),
		(0.4016393443, 0.5819672131),
		(0.4016393443, 0.5737704918),
		(0.4016393443, 0.5655737705),
		(0.4016393443, 0.5573770492),
		(0.4016393443, 0.5491803279),
		(0.4016393443, 0.5409836066),
		(0.4016393443, 0.5327868852),
		(0.4016393443, 0.5245901639),
		(0.4016393443, 0.5163934426),
		(0.4016393443, 0.5081967213),
		(0.4098360656, 0.6967213115),
		(0.4098360656, 0.6885245902),
		(0.4098360656, 0.6803278689),
		(0.4098360656, 0.6721311475),
		(0.4098360656, 0.6639344262),
		(0.4098360656, 0.6557377049),
		(0.4098360656, 0.5983606557),
		(0.4098360656, 0.5901639344),
		(0.4098360656, 0.5819672131),
		(0.4098360656, 0.5737704918),
		(0.4098360656, 0.5655737705),
		(0.4098360656, 0.5573770492),
		(0.4098360656, 0.5491803279),
		(0.4098360656, 0.5409836066),
		(0.4098360656, 0.5327868852),
		(0.4098360656, 0.5245901639),
		(0.4098360656, 0.5163934426),
		(0.4098360656, 0.5081967213),
		(0.4180327869, 0.6967213115),
		(0.4180327869, 0.6885245902),
		(0.4180327869, 0.6803278689),
		(0.4180327869, 0.6721311475),
		(0.4180327869, 0.6639344262),
		(0.4180327869, 0.6557377049),
		(0.4180327869, 0.5983606557),
		(0.4180327869, 0.5901639344),
		(0.4180327869, 0.5819672131),
		(0.4180327869, 0.5737704918),
		(0.4180327869, 0.5655737705),
		(0.4180327869, 0.5573770492),
		(0.4180327869, 0.5491803279),
		(0.4180327869, 0.5409836066),
		(0.4180327869, 0.5327868852),
		(0.4180327869, 0.5245901639),
		(0.4180327869, 0.5163934426),
		(0.4180327869, 0.5081967213),
		(0.4262295082, 0.6967213115),
		(0.4262295082, 0.6885245902),
		(0.4262295082, 0.6803278689),
		(0.4262295082, 0.6721311475),
		(0.4262295082, 0.6639344262),
		(0.4262295082, 0.6557377049),
		(0.4262295082, 0.5983606557),
		(0.4262295082, 0.5901639344),
		(0.4262295082, 0.5819672131),
		(0.4262295082, 0.5737704918),
		(0.4262295082, 0.5655737705),
		(0.4262295082, 0.5573770492),
		(0.4262295082, 0.5491803279),
		(0.4262295082, 0.5409836066),
		(0.4262295082, 0.5327868852),
		(0.4262295082, 0.5245901639),
		(0.4262295082, 0.5163934426),
		(0.4262295082, 0.5081967213),
		(0.4344262295, 0.6967213115),
		(0.4344262295, 0.6885245902),
		(0.4344262295, 0.6803278689),
		(0.4344262295, 0.6721311475),
		(0.4344262295, 0.6639344262),
		(0.4344262295, 0.6557377049),
		(0.4344262295, 0.5983606557),
		(0.4344262295, 0.5901639344),
		(0.4344262295, 0.5819672131),
		(0.4344262295, 0.5737704918),
		(0.4344262295, 0.5655737705),
		(0.4344262295, 0.5573770492),
		(0.4344262295, 0.5491803279),
		(0.4344262295, 0.5409836066),
		(0.4344262295, 0.5327868852),
		(0.4344262295, 0.5245901639),
		(0.4344262295, 0.5163934426),
		(0.4344262295, 0.5081967213),
		(0.4426229508, 0.6967213115),
		(0.4426229508, 0.6885245902),
		(0.4426229508, 0.6803278689),
		(0.4426229508, 0.6721311475),
		(0.4426229508, 0.6639344262),
		(0.4426229508, 0.6557377049),
		(0.4426229508, 0.5983606557),
		(0.4426229508, 0.5901639344),
		(0.4426229508, 0.5819672131),
		(0.4426229508, 0.5737704918),
		(0.4426229508, 0.5655737705),
		(0.4426229508, 0.5573770492),
		(0.4426229508, 0.5491803279),
		(0.4426229508, 0.5409836066),
		(0.4426229508, 0.5327868852),
		(0.4426229508, 0.5245901639),
		(0.4426229508, 0.5163934426),
		(0.4426229508, 0.5081967213),
		(0.4508196721, 0.9918032787),
		(0.4508196721, 0.9836065574),
		(0.4508196721, 0.9754098361),
		(0.4508196721, 0.9672131148),
		(0.4508196721, 0.9590163934),
		(0.4508196721, 0.9508196721),
		(0.4508196721, 0.8934426230),
		(0.4508196721, 0.8852459016),
		(0.4508196721, 0.8770491803),
		(0.4508196721, 0.8688524590),
		(0.4508196721, 0.8606557377),
		(0.4508196721, 0.8524590164),
		(0.4508196721, 0.5983606557),
		(0.4508196721, 0.5901639344),
		(0.4508196721, 0.5819672131),
		(0.4508196721, 0.5737704918),
		(0.4508196721, 0.5655737705),
		(0.4508196721, 0.5573770492),
		(0.4508196721, 0.5491803279),
		(0.4508196721, 0.5409836066),
		(0.4508196721, 0.5327868852),
		(0.4508196721, 0.5245901639),
		(0.4508196721, 0.5163934426),
		(0.4508196721, 0.5081967213),
		(0.4590163934, 0.9918032787),
		(0.4590163934, 0.9836065574),
		(0.4590163934, 0.9754098361),
		(0.4590163934, 0.9672131148),
		(0.4590163934, 0.9590163934),
		(0.4590163934, 0.9508196721),
		(0.4590163934, 0.8934426230),
		(0.4590163934, 0.8852459016),
		(0.4590163934, 0.8770491803),
		(0.4590163934, 0.8688524590),
		(0.4590163934, 0.8606557377),
		(0.4590163934, 0.8524590164),
		(0.4590163934, 0.5983606557),
		(0.4590163934, 0.5901639344),
		(0.4590163934, 0.5819672131),
		(0.4590163934, 0.5737704918),
		(0.4590163934, 0.5655737705),
		(0.4590163934, 0.5573770492),
		(0.4590163934, 0.5491803279),
		(0.4590163934, 0.5409836066),
		(0.4590163934, 0.5327868852),
		(0.4590163934, 0.5245901639),
		(0.4590163934, 0.5163934426),
		(0.4590163934, 0.5081967213),
		(0.4672131148, 0.9918032787),
		(0.4672131148, 0.9836065574),
		(0.4672131148, 0.9754098361),
		(0.4672131148, 0.9672131148),
		(0.4672131148, 0.9590163934),
		(0.4672131148, 0.9508196721),
		(0.4672131148, 0.8934426230),
		(0.4672131148, 0.8852459016),
		(0.4672131148, 0.8770491803),
		(0.4672131148, 0.8688524590),
		(0.4672131148, 0.8606557377),
		(0.4672131148, 0.8524590164),
		(0.4672131148, 0.5983606557),
		(0.4672131148, 0.5901639344),
		(0.4672131148, 0.5819672131),
		(0.4672131148, 0.5737704918),
		(0.4672131148, 0.5655737705),
		(0.4672131148, 0.5573770492),
		(0.4672131148, 0.5491803279),
		(0.4672131148, 0.5409836066),
		(0.4672131148, 0.5327868852),
		(0.4672131148, 0.5245901639),
		(0.4672131148, 0.5163934426),
		(0.4672131148, 0.5081967213),
		(0.4754098361, 0.9918032787),
		(0.4754098361, 0.9836065574),
		(0.4754098361, 0.9754098361),
		(0.4754098361, 0.9672131148),
		(0.4754098361, 0.9590163934),
		(0.4754098361, 0.9508196721),
		(0.4754098361, 0.8934426230),
		(0.4754098361, 0.8852459016),
		(0.4754098361, 0.8770491803),
		(0.4754098361, 0.8688524590),
		(0.4754098361, 0.8606557377),
		(0.4754098361, 0.8524590164),
		(0.4754098361, 0.5983606557),
		(0.4754098361, 0.5901639344),
		(0.4754098361, 0.5819672131),
		(0.4754098361, 0.5737704918),
		(0.4754098361, 0.5655737705),
		(0.4754098361, 0.5573770492),
		(0.4754098361, 0.5491803279),
		(0.4754098361, 0.5409836066),
		(0.4754098361, 0.5327868852),
		(0.4754098361, 0.5245901639),
		(0.4754098361, 0.5163934426),
		(0.4754098361, 0.5081967213),
		(0.4836065574, 0.9918032787),
		(0.4836065574, 0.9836065574),
		(0.4836065574, 0.9754098361),
		(0.4836065574, 0.9672131148),
		(0.4836065574, 0.9590163934),
		(0.4836065574, 0.9508196721),
		(0.4836065574, 0.8934426230),
		(0.4836065574, 0.8852459016),
		(0.4836065574, 0.8770491803),
		(0.4836065574, 0.8688524590),
		(0.4836065574, 0.8606557377),
		(0.4836065574, 0.8524590164),
		(0.4836065574, 0.5983606557),
		(0.4836065574, 0.5901639344),
		(0.4836065574, 0.5819672131),
		(0.4836065574, 0.5737704918),
		(0.4836065574, 0.5655737705),
		(0.4836065574, 0.5573770492),
		(0.4836065574, 0.5491803279),
		(0.4836065574, 0.5409836066),
		(0.4836065574, 0.5327868852),
		(0.4836065574, 0.5245901639),
		(0.4836065574, 0.5163934426),
		(0.4836065574, 0.5081967213),
		(0.4918032787, 0.9918032787),
		(0.4918032787, 0.9836065574),
		(0.4918032787, 0.9754098361),
		(0.4918032787, 0.9672131148),
		(0.4918032787, 0.9590163934),
		(0.4918032787, 0.9508196721),
		(0.4918032787, 0.8934426230),
		(0.4918032787, 0.8852459016),
		(0.4918032787, 0.8770491803),
		(0.4918032787, 0.8688524590),
		(0.4918032787, 0.8606557377),
		(0.4918032787, 0.8524590164),
		(0.4918032787, 0.5983606557),
		(0.4918032787, 0.5901639344),
		(0.4918032787, 0.5819672131),
		(0.4918032787, 0.5737704918),
		(0.4918032787, 0.5655737705),
		(0.4918032787, 0.5573770492),
		(0.4918032787, 0.5491803279),
		(0.4918032787, 0.5409836066),
		(0.4918032787, 0.5327868852),
		(0.4918032787, 0.5245901639),
		(0.4918032787, 0.5163934426),
		(0.4918032787, 0.5081967213),
		(0.5000000000, 0.5000000000),
		(0.5000000000, 0.4918032787),
		(0.5000000000, 0.4836065574),
		(0.5000000000, 0.4754098361),
		(0.5000000000, 0.4672131148),
		(0.5000000000, 0.4590163934),
		(0.5000000000, 0.4508196721),
		(0.5000000000, 0.4426229508),
		(0.5000000000, 0.4344262295),
		(0.5000000000, 0.4262295082),
		(0.5000000000, 0.4180327869),
		(0.5000000000, 0.4098360656),
		(0.5000000000, 0.3032786885),
		(0.5000000000, 0.2950819672),
		(0.5000000000, 0.2868852459),
		(0.5000000000, 0.2786885246),
		(0.5000000000, 0.2704918033),
		(0.5000000000, 0.2622950820),
		(0.5000000000, 0.2540983607),
		(0.5000000000, 0.2459016393),
		(0.5000000000, 0.2377049180),
		(0.5000000000, 0.2295081967),
		(0.5000000000, 0.2213114754),
		(0.5000000000, 0.2131147541),
		(0.5000000000, 0.2049180328),
		(0.5000000000, 0.1967213115),
		(0.5000000000, 0.1885245902),
		(0.5000000000, 0.1803278689),
		(0.5000000000, 0.1721311475),
		(0.5000000000, 0.1639344262),
		(0.5000000000, 0.0573770492),
		(0.5000000000, 0.0491803279),
		(0.5000000000, 0.0409836066),
		(0.5000000000, 0.0327868852),
		(0.5000000000, 0.0245901639),
		(0.5000000000, 0.0163934426),
		(0.5081967213, 0.5000000000),
		(0.5081967213, 0.4918032787),
		(0.5081967213, 0.4836065574),
		(0.5081967213, 0.4754098361),
		(0.5081967213, 0.4672131148),
		(0.5081967213, 0.4590163934),
		(0.5081967213, 0.4508196721),
		(0.5081967213, 0.4426229508),
		(0.5081967213, 0.4344262295),
		(0.5081967213, 0.4262295082),
		(0.5081967213, 0.4180327869),
		(0.5081967213, 0.4098360656),
		(0.5081967213, 0.3032786885),
		(0.5081967213, 0.2950819672),
		(0.5081967213, 0.2868852459),
		(0.5081967213, 0.2786885246),
		(0.5081967213, 0.2704918033),
		(0.5081967213, 0.2622950820),
		(0.5081967213, 0.2540983607),
		(0.5081967213, 0.2459016393),
		(0.5081967213, 0.2377049180),
		(0.5081967213, 0.2295081967),
		(0.5081967213, 0.2213114754),
		(0.5081967213, 0.2131147541),
		(0.5081967213, 0.2049180328),
		(0.5081967213, 0.1967213115),
		(0.5081967213, 0.1885245902),
		(0.5081967213, 0.1803278689),
		(0.5081967213, 0.1721311475),
		(0.5081967213, 0.1639344262),
		(0.5081967213, 0.0573770492),
		(0.5081967213, 0.0491803279),
		(0.5081967213, 0.0409836066),
		(0.5081967213, 0.0327868852),
		(0.5081967213, 0.0245901639),
		(0.5081967213, 0.0163934426),
		(0.5163934426, 0.5000000000),
		(0.5163934426, 0.4918032787),
		(0.5163934426, 0.4836065574),
		(0.5163934426, 0.4754098361),
		(0.5163934426, 0.4672131148),
		(0.5163934426, 0.4590163934),
		(0.5163934426, 0.4508196721),
		(0.5163934426, 0.4426229508),
		(0.5163934426, 0.4344262295),
		(0.5163934426, 0.4262295082),
		(0.5163934426, 0.4180327869),
		(0.5163934426, 0.4098360656),
		(0.5163934426, 0.3032786885),
		(0.5163934426, 0.2950819672),
		(0.5163934426, 0.2868852459),
		(0.5163934426, 0.2786885246),
		(0.5163934426, 0.2704918033),
		(0.5163934426, 0.2622950820),
		(0.5163934426, 0.2540983607),
		(0.5163934426, 0.2459016393),
		(0.5163934426, 0.2377049180),
		(0.5163934426, 0.2295081967),
		(0.5163934426, 0.2213114754),
		(0.5163934426, 0.2131147541),
		(0.5163934426, 0.2049180328),
		(0.5163934426, 0.1967213115),
		(0.5163934426, 0.1885245902),
		(0.5163934426, 0.1803278689),
		(0.5163934426, 0.1721311475),
		(0.5163934426, 0.1639344262),
		(0.5163934426, 0.0573770492),
		(0.5163934426, 0.0491803279),
		(0.5163934426, 0.0409836066),
		(0.5163934426, 0.0327868852),
		(0.5163934426, 0.0245901639),
		(0.5163934426, 0.0163934426),
		(0.5245901639, 0.5000000000),
		(0.5245901639, 0.4918032787),
		(0.5245901639, 0.4836065574),
		(0.5245901639, 0.4754098361),
		(0.5245901639, 0.4672131148),
		(0.5245901639, 0.4590163934),
		(0.5245901639, 0.4508196721),
		(0.5245901639, 0.4426229508),
		(0.5245901639, 0.4344262295),
		(0.5245901639, 0.4262295082),
		(0.5245901639, 0.4180327869),
		(0.5245901639, 0.4098360656),
		(0.5245901639, 0.3032786885),
		(0.5245901639, 0.2950819672),
		(0.5245901639, 0.2868852459),
		(0.5245901639, 0.2786885246),
		(0.5245901639, 0.2704918033),
		(0.5245901639, 0.2622950820),
		(0.5245901639, 0.2540983607),
		(0.5245901639, 0.2459016393),
		(0.5245901639, 0.2377049180),
		(0.5245901639, 0.2295081967),
		(0.5245901639, 0.2213114754),
		(0.5245901639, 0.2131147541),
		(0.5245901639, 0.2049180328),
		(0.5245901639, 0.1967213115),
		(0.5245901639, 0.1885245902),
		(0.5245901639, 0.1803278689),
		(0.5245901639, 0.1721311475),
		(0.5245901639, 0.1639344262),
		(0.5245901639, 0.0573770492),
		(0.5245901639, 0.0491803279),
		(0.5245901639, 0.0409836066),
		(0.5245901639, 0.0327868852),
		(0.5245901639, 0.0245901639),
		(0.5245901639, 0.0163934426),
		(0.5327868852, 0.5000000000),
		(0.5327868852, 0.4918032787),
		(0.5327868852, 0.4836065574),
		(0.5327868852, 0.4754098361),
		(0.5327868852, 0.4672131148),
		(0.5327868852, 0.4590163934),
		(0.5327868852, 0.4508196721),
		(0.5327868852, 0.4426229508),
		(0.5327868852, 0.4344262295),
		(0.5327868852, 0.4262295082),
		(0.5327868852, 0.4180327869),
		(0.5327868852, 0.4098360656),
		(0.5327868852, 0.3032786885),
		(0.5327868852, 0.2950819672),
		(0.5327868852, 0.2868852459),
		(0.5327868852, 0.2786885246),
		(0.5327868852, 0.2704918033),
		(0.5327868852, 0.2622950820),
		(0.5327868852, 0.2540983607),
		(0.5327868852, 0.2459016393),
		(0.5327868852, 0.2377049180),
		(0.5327868852, 0.2295081967),
		(0.5327868852, 0.2213114754),
		(0.5327868852, 0.2131147541),
		(0.5327868852, 0.2049180328),
		(0.5327868852, 0.1967213115),
		(0.5327868852, 0.1885245902),
		(0.5327868852, 0.1803278689),
		(0.5327868852, 0.1721311475),
		(0.5327868852, 0.1639344262),
		(0.5327868852, 0.0573770492),
		(0.5327868852, 0.0491803279),
		(0.5327868852, 0.0409836066),
		(0.5327868852, 0.0327868852),
		(0.5327868852, 0.0245901639),
		(0.5327868852, 0.0163934426),
		(0.5409836066, 0.5000000000),
		(0.5409836066, 0.4918032787),
		(0.5409836066, 0.4836065574),
		(0.5409836066, 0.4754098361),
		(0.5409836066, 0.4672131148),
		(0.5409836066, 0.4590163934),
		(0.5409836066, 0.4508196721),
		(0.5409836066, 0.4426229508),
		(0.5409836066, 0.4344262295),
		(0.5409836066, 0.4262295082),
		(0.5409836066, 0.4180327869),
		(0.5409836066, 0.4098360656),
		(0.5409836066, 0.3032786885),
		(0.5409836066, 0.2950819672),
		(0.5409836066, 0.2868852459),
		(0.5409836066, 0.2786885246),
		(0.5409836066, 0.2704918033),
		(0.5409836066, 0.2622950820),
		(0.5409836066, 0.2540983607),
		(0.5409836066, 0.2459016393),
		(0.5409836066, 0.2377049180),
		(0.5409836066, 0.2295081967),
		(0.5409836066, 0.2213114754),
		(0.5409836066, 0.2131147541),
		(0.5409836066, 0.2049180328),
		(0.5409836066, 0.1967213115),
		(0.5409836066, 0.1885245902),
		(0.5409836066, 0.1803278689),
		(0.5409836066, 0.1721311475),
		(0.5409836066, 0.1639344262),
		(0.5409836066, 0.0573770492),
		(0.5409836066, 0.0491803279),
		(0.5409836066, 0.0409836066),
		(0.5409836066, 0.0327868852),
		(0.5409836066, 0.0245901639),
		(0.5409836066, 0.0163934426),
		(0.5491803279, 0.5000000000),
		(0.5491803279, 0.4918032787),
		(0.5491803279, 0.4836065574),
		(0.5491803279, 0.4754098361),
		(0.5491803279, 0.4672131148),
		(0.5491803279, 0.4590163934),
		(0.5491803279, 0.4508196721),
		(0.5491803279, 0.4426229508),
		(0.5491803279, 0.4344262295),
		(0.5491803279, 0.4262295082),
		(0.5491803279, 0.4180327869),
		(0.5491803279, 0.4098360656),
		(0.5491803279, 0.3524590164),
		(0.5491803279, 0.3442622951),
		(0.5491803279, 0.3360655738),
		(0.5491803279, 0.3278688525),
		(0.5491803279, 0.3196721311),
		(0.5491803279, 0.3114754098),
		(0.5491803279, 0.2540983607),
		(0.5491803279, 0.2459016393),
		(0.5491803279, 0.2377049180),
		(0.5491803279, 0.2295081967),
		(0.5491803279, 0.2213114754),
		(0.5491803279, 0.2131147541),
		(0.5491803279, 0.2049180328),
		(0.5491803279, 0.1967213115),
		(0.5491803279, 0.1885245902),
		(0.5491803279, 0.1803278689),
		(0.5491803279, 0.1721311475),
		(0.5491803279, 0.1639344262),
		(0.5573770492, 0.5000000000),
		(0.5573770492, 0.4918032787),
		(0.5573770492, 0.4836065574),
		(0.5573770492, 0.4754098361),
		(0.5573770492, 0.4672131148),
		(0.5573770492, 0.4590163934),
		(0.5573770492, 0.4508196721),
		(0.5573770492, 0.4426229508),
		(0.5573770492, 0.4344262295),
		(0.5573770492, 0.4262295082),
		(0.5573770492, 0.4180327869),
		(0.5573770492, 0.4098360656),
		(0.5573770492, 0.3524590164),
		(0.5573770492, 0.3442622951),
		(0.5573770492, 0.3360655738),
		(0.5573770492, 0.3278688525),
		(0.5573770492, 0.3196721311),
		(0.5573770492, 0.3114754098),
		(0.5573770492, 0.2540983607),
		(0.5573770492, 0.2459016393),
		(0.5573770492, 0.2377049180),
		(0.5573770492, 0.2295081967),
		(0.5573770492, 0.2213114754),
		(0.5573770492, 0.2131147541),
		(0.5573770492, 0.2049180328),
		(0.5573770492, 0.1967213115),
		(0.5573770492, 0.1885245902),
		(0.5573770492, 0.1803278689),
		(0.5573770492, 0.1721311475),
		(0.5573770492, 0.1639344262),
		(0.5655737705, 0.5000000000),
		(0.5655737705, 0.4918032787),
		(0.5655737705, 0.4836065574),
		(0.5655737705, 0.4754098361),
		(0.5655737705, 0.4672131148),
		(0.5655737705, 0.4590163934),
		(0.5655737705, 0.4508196721),
		(0.5655737705, 0.4426229508),
		(0.5655737705, 0.4344262295),
		(0.5655737705, 0.4262295082),
		(0.5655737705, 0.4180327869),
		(0.5655737705, 0.4098360656),
		(0.5655737705, 0.3524590164),
		(0.5655737705, 0.3442622951),
		(0.5655737705, 0.3360655738),
		(0.5655737705, 0.3278688525),
		(0.5655737705, 0.3196721311),
		(0.5655737705, 0.3114754098),
		(0.5655737705, 0.2540983607),
		(0.5655737705, 0.2459016393),
		(0.5655737705, 0.2377049180),
		(0.5655737705, 0.2295081967),
		(0.5655737705, 0.2213114754),
		(0.5655737705, 0.2131147541),
		(0.5655737705, 0.2049180328),
		(0.5655737705, 0.1967213115),
		(0.5655737705, 0.1885245902),
		(0.5655737705, 0.1803278689),
		(0.5655737705, 0.1721311475),
		(0.5655737705, 0.1639344262),
		(0.5737704918, 0.5000000000),
		(0.5737704918, 0.4918032787),
		(0.5737704918, 0.4836065574),
		(0.5737704918, 0.4754098361),
		(0.5737704918, 0.4672131148),
		(0.5737704918, 0.4590163934),
		(0.5737704918, 0.4508196721),
		(0.5737704918, 0.4426229508),
		(0.5737704918, 0.4344262295),
		(0.5737704918, 0.4262295082),
		(0.5737704918, 0.4180327869),
		(0.5737704918, 0.4098360656),
		(0.5737704918, 0.3524590164),
		(0.5737704918, 0.3442622951),
		(0.5737704918, 0.3360655738),
		(0.5737704918, 0.3278688525),
		(0.5737704918, 0.3196721311),
		(0.5737704918, 0.3114754098),
		(0.5737704918, 0.2540983607),
		(0.5737704918, 0.2459016393),
		(0.5737704918, 0.2377049180),
		(0.5737704918, 0.2295081967),
		(0.5737704918, 0.2213114754),
		(0.5737704918, 0.2131147541),
		(0.5737704918, 0.2049180328),
		(0.5737704918, 0.1967213115),
		(0.5737704918, 0.1885245902),
		(0.5737704918, 0.1803278689),
		(0.5737704918, 0.1721311475),
		(0.5737704918, 0.1639344262),
		(0.5819672131, 0.5000000000),
		(0.5819672131, 0.4918032787),
		(0.5819672131, 0.4836065574),
		(0.5819672131, 0.4754098361),
		(0.5819672131, 0.4672131148),
		(0.5819672131, 0.4590163934),
		(0.5819672131, 0.4508196721),
		(0.5819672131, 0.4426229508),
		(0.5819672131, 0.4344262295),
		(0.5819672131, 0.4262295082),
		(0.5819672131, 0.4180327869),
		(0.5819672131, 0.4098360656),
		(0.5819672131, 0.3524590164),
		(0.5819672131, 0.3442622951),
		(0.5819672131, 0.3360655738),
		(0.5819672131, 0.3278688525),
		(0.5819672131, 0.3196721311),
		(0.5819672131, 0.3114754098),
		(0.5819672131, 0.2540983607),
		(0.5819672131, 0.2459016393),
		(0.5819672131, 0.2377049180),
		(0.5819672131, 0.2295081967),
		(0.5819672131, 0.2213114754),
		(0.5819672131, 0.2131147541),
		(0.5819672131, 0.2049180328),
		(0.5819672131, 0.1967213115),
		(0.5819672131, 0.1885245902),
		(0.5819672131, 0.1803278689),
		(0.5819672131, 0.1721311475),
		(0.5819672131, 0.1639344262),
		(0.5901639344, 0.5000000000),
		(0.5901639344, 0.4918032787),
		(0.5901639344, 0.4836065574),
		(0.5901639344, 0.4754098361),
		(0.5901639344, 0.4672131148),
		(0.5901639344, 0.4590163934),
		(0.5901639344, 0.4508196721),
		(0.5901639344, 0.4426229508),
		(0.5901639344, 0.4344262295),
		(0.5901639344, 0.4262295082),
		(0.5901639344, 0.4180327869),
		(0.5901639344, 0.4098360656),
		(0.5901639344, 0.3524590164),
		(0.5901639344, 0.3442622951),
		(0.5901639344, 0.3360655738),
		(0.5901639344, 0.3278688525),
		(0.5901639344, 0.3196721311),
		(0.5901639344, 0.3114754098),
		(0.5901639344, 0.2540983607),
		(0.5901639344, 0.2459016393),
		(0.5901639344, 0.2377049180),
		(0.5901639344, 0.2295081967),
		(0.5901639344, 0.2213114754),
		(0.5901639344, 0.2131147541),
		(0.5901639344, 0.2049180328),
		(0.5901639344, 0.1967213115),
		(0.5901639344, 0.1885245902),
		(0.5901639344, 0.1803278689),
		(0.5901639344, 0.1721311475),
		(0.5901639344, 0.1639344262),
		(0.5983606557, 0.4016393443),
		(0.5983606557, 0.3934426230),
		(0.5983606557, 0.3852459016),
		(0.5983606557, 0.3770491803),
		(0.5983606557, 0.3688524590),
		(0.5983606557, 0.3606557377),
		(0.5983606557, 0.3032786885),
		(0.5983606557, 0.2950819672),
		(0.5983606557, 0.2868852459),
		(0.5983606557, 0.2786885246),
		(0.5983606557, 0.2704918033),
		(0.5983606557, 0.2622950820),
		(0.5983606557, 0.0573770492),
		(0.5983606557, 0.0491803279),
		(0.5983606557, 0.0409836066),
		(0.5983606557, 0.0327868852),
		(0.5983606557, 0.0245901639),
		(0.5983606557, 0.0163934426),
		(0.6065573770, 0.4016393443),
		(0.6065573770, 0.3934426230),
		(0.6065573770, 0.3852459016),
		(0.6065573770, 0.3770491803),
		(0.6065573770, 0.3688524590),
		(0.6065573770, 0.3606557377),
		(0.6065573770, 0.3032786885),
		(0.6065573770, 0.2950819672),
		(0.6065573770, 0.2868852459),
		(0.6065573770, 0.2786885246),
		(0.6065573770, 0.2704918033),
		(0.6065573770, 0.2622950820),
		(0.6065573770, 0.0573770492),
		(0.6065573770, 0.0491803279),
		(0.6065573770, 0.0409836066),
		(0.6065573770, 0.0327868852),
		(0.6065573770, 0.0245901639),
		(0.6065573770, 0.0163934426),
		(0.6147540984, 0.4016393443),
		(0.6147540984, 0.3934426230),
		(0.6147540984, 0.3852459016),
		(0.6147540984, 0.3770491803),
		(0.6147540984, 0.3688524590),
		(0.6147540984, 0.3606557377),
		(0.6147540984, 0.3032786885),
		(0.6147540984, 0.2950819672),
		(0.6147540984, 0.2868852459),
		(0.6147540984, 0.2786885246),
		(0.6147540984, 0.2704918033),
		(0.6147540984, 0.2622950820),
		(0.6147540984, 0.0573770492),
		(0.6147540984, 0.0491803279),
		(0.6147540984, 0.0409836066),
		(0.6147540984, 0.0327868852),
		(0.6147540984, 0.0245901639),
		(0.6147540984, 0.0163934426),
		(0.6229508197, 0.4016393443),
		(0.6229508197, 0.3934426230),
		(0.6229508197, 0.3852459016),
		(0.6229508197, 0.3770491803),
		(0.6229508197, 0.3688524590),
		(0.6229508197, 0.3606557377),
		(0.6229508197, 0.3032786885),
		(0.6229508197, 0.2950819672),
		(0.6229508197, 0.2868852459),
		(0.6229508197, 0.2786885246),
		(0.6229508197, 0.2704918033),
		(0.6229508197, 0.2622950820),
		(0.6229508197, 0.0573770492),
		(0.6229508197, 0.0491803279),
		(0.6229508197, 0.0409836066),
		(0.6229508197, 0.0327868852),
		(0.6229508197, 0.0245901639),
		(0.6229508197, 0.0163934426),
		(0.6311475410, 0.4016393443),
		(0.6311475410, 0.3934426230),
		(0.6311475410, 0.3852459016),
		(0.6311475410, 0.3770491803),
		(0.6311475410, 0.3688524590),
		(0.6311475410, 0.3606557377),
		(0.6311475410, 0.3032786885),
		(0.6311475410, 0.2950819672),
		(0.6311475410, 0.2868852459),
		(0.6311475410, 0.2786885246),
		(0.6311475410, 0.2704918033),
		(0.6311475410, 0.2622950820),
		(0.6311475410, 0.0573770492),
		(0.6311475410, 0.0491803279),
		(0.6311475410, 0.0409836066),
		(0.6311475410, 0.0327868852),
		(0.6311475410, 0.0245901639),
		(0.6311475410, 0.0163934426),
		(0.6393442623, 0.4016393443),
		(0.6393442623, 0.3934426230),
		(0.6393442623, 0.3852459016),
		(0.6393442623, 0.3770491803),
		(0.6393442623, 0.3688524590),
		(0.6393442623, 0.3606557377),
		(0.6393442623, 0.3032786885),
		(0.6393442623, 0.2950819672),
		(0.6393442623, 0.2868852459),
		(0.6393442623, 0.2786885246),
		(0.6393442623, 0.2704918033),
		(0.6393442623, 0.2622950820),
		(0.6393442623, 0.0573770492),
		(0.6393442623, 0.0491803279),
		(0.6393442623, 0.0409836066),
		(0.6393442623, 0.0327868852),
		(0.6393442623, 0.0245901639),
		(0.6393442623, 0.0163934426),
		(0.6475409836, 0.4508196721),
		(0.6475409836, 0.4426229508),
		(0.6475409836, 0.4344262295),
		(0.6475409836, 0.4262295082),
		(0.6475409836, 0.4180327869),
		(0.6475409836, 0.4098360656),
		(0.6475409836, 0.3524590164),
		(0.6475409836, 0.3442622951),
		(0.6475409836, 0.3360655738),
		(0.6475409836, 0.3278688525),
		(0.6475409836, 0.3196721311),
		(0.6475409836, 0.3114754098),
		(0.6475409836, 0.2049180328),
		(0.6475409836, 0.1967213115),
		(0.6475409836, 0.1885245902),
		(0.6475409836, 0.1803278689),
		(0.6475409836, 0.1721311475),
		(0.6475409836, 0.1639344262),
		(0.6557377049, 0.4508196721),
		(0.6557377049, 0.4426229508),
		(0.6557377049, 0.4344262295),
		(0.6557377049, 0.4262295082),
		(0.6557377049, 0.4180327869),
		(0.6557377049, 0.4098360656),
		(0.6557377049, 0.3524590164),
		(0.6557377049, 0.3442622951),
		(0.6557377049, 0.3360655738),
		(0.6557377049, 0.3278688525),
		(0.6557377049, 0.3196721311),
		(0.6557377049, 0.3114754098),
		(0.6557377049, 0.2049180328),
		(0.6557377049, 0.1967213115),
		(0.6557377049, 0.1885245902),
		(0.6557377049, 0.1803278689),
		(0.6557377049, 0.1721311475),
		(0.6557377049, 0.1639344262),
		(0.6639344262, 0.4508196721),
		(0.6639344262, 0.4426229508),
		(0.6639344262, 0.4344262295),
		(0.6639344262, 0.4262295082),
		(0.6639344262, 0.4180327869),
		(0.6639344262, 0.4098360656),
		(0.6639344262, 0.3524590164),
		(0.6639344262, 0.3442622951),
		(0.6639344262, 0.3360655738),
		(0.6639344262, 0.3278688525),
		(0.6639344262, 0.3196721311),
		(0.6639344262, 0.3114754098),
		(0.6639344262, 0.2049180328),
		(0.6639344262, 0.1967213115),
		(0.6639344262, 0.1885245902),
		(0.6639344262, 0.1803278689),
		(0.6639344262, 0.1721311475),
		(0.6639344262, 0.1639344262),
		(0.6721311475, 0.4508196721),
		(0.6721311475, 0.4426229508),
		(0.6721311475, 0.4344262295),
		(0.6721311475, 0.4262295082),
		(0.6721311475, 0.4180327869),
		(0.6721311475, 0.4098360656),
		(0.6721311475, 0.3524590164),
		(0.6721311475, 0.3442622951),
		(0.6721311475, 0.3360655738),
		(0.6721311475, 0.3278688525),
		(0.6721311475, 0.3196721311),
		(0.6721311475, 0.3114754098),
		(0.6721311475, 0.2049180328),
		(0.6721311475, 0.1967213115),
		(0.6721311475, 0.1885245902),
		(0.6721311475, 0.1803278689),
		(0.6721311475, 0.1721311475),
		(0.6721311475, 0.1639344262),
		(0.6803278689, 0.4508196721),
		(0.6803278689, 0.4426229508),
		(0.6803278689, 0.4344262295),
		(0.6803278689, 0.4262295082),
		(0.6803278689, 0.4180327869),
		(0.6803278689, 0.4098360656),
		(0.6803278689, 0.3524590164),
		(0.6803278689, 0.3442622951),
		(0.6803278689, 0.3360655738),
		(0.6803278689, 0.3278688525),
		(0.6803278689, 0.3196721311),
		(0.6803278689, 0.3114754098),
		(0.6803278689, 0.2049180328),
		(0.6803278689, 0.1967213115),
		(0.6803278689, 0.1885245902),
		(0.6803278689, 0.1803278689),
		(0.6803278689, 0.1721311475),
		(0.6803278689, 0.1639344262),
		(0.6885245902, 0.4508196721),
		(0.6885245902, 0.4426229508),
		(0.6885245902, 0.4344262295),
		(0.6885245902, 0.4262295082),
		(0.6885245902, 0.4180327869),
		(0.6885245902, 0.4098360656),
		(0.6885245902, 0.3524590164),
		(0.6885245902, 0.3442622951),
		(0.6885245902, 0.3360655738),
		(0.6885245902, 0.3278688525),
		(0.6885245902, 0.3196721311),
		(0.6885245902, 0.3114754098),
		(0.6885245902, 0.2049180328),
		(0.6885245902, 0.1967213115),
		(0.6885245902, 0.1885245902),
		(0.6885245902, 0.1803278689),
		(0.6885245902, 0.1721311475),
		(0.6885245902, 0.1639344262),
		(0.6967213115, 0.5000000000),
		(0.6967213115, 0.4918032787),
		(0.6967213115, 0.4836065574),
		(0.6967213115, 0.4754098361),
		(0.6967213115, 0.4672131148),
		(0.6967213115, 0.4590163934),
		(0.6967213115, 0.4016393443),
		(0.6967213115, 0.3934426230),
		(0.6967213115, 0.3852459016),
		(0.6967213115, 0.3770491803),
		(0.6967213115, 0.3688524590),
		(0.6967213115, 0.3606557377),
		(0.6967213115, 0.3032786885),
		(0.6967213115, 0.2950819672),
		(0.6967213115, 0.2868852459),
		(0.6967213115, 0.2786885246),
		(0.6967213115, 0.2704918033),
		(0.6967213115, 0.2622950820),
		(0.6967213115, 0.2540983607),
		(0.6967213115, 0.2459016393),
		(0.6967213115, 0.2377049180),
		(0.6967213115, 0.2295081967),
		(0.6967213115, 0.2213114754),
		(0.6967213115, 0.2131147541),
		(0.6967213115, 0.1557377049),
		(0.6967213115, 0.1475409836),
		(0.6967213115, 0.1393442623),
		(0.6967213115, 0.1311475410),
		(0.6967213115, 0.1229508197),
		(0.6967213115, 0.1147540984),
		(0.7049180328, 0.5000000000),
		(0.7049180328, 0.4918032787),
		(0.7049180328, 0.4836065574),
		(0.7049180328, 0.4754098361),
		(0.7049180328, 0.4672131148),
		(0.7049180328, 0.4590163934),
		(0.7049180328, 0.4016393443),
		(0.7049180328, 0.3934426230),
		(0.7049180328, 0.3852459016),
		(0.7049180328, 0.3770491803),
		(0.7049180328, 0.3688524590),
		(0.7049180328, 0.3606557377),
		(0.7049180328, 0.3032786885),
		(0.7049180328, 0.2950819672),
		(0.7049180328, 0.2868852459),
		(0.7049180328, 0.2786885246),
		(0.7049180328, 0.2704918033),
		(0.7049180328, 0.2622950820),
		(0.7049180328, 0.2540983607),
		(0.7049180328, 0.2459016393),
		(0.7049180328, 0.2377049180),
		(0.7049180328, 0.2295081967),
		(0.7049180328, 0.2213114754),
		(0.7049180328, 0.2131147541),
		(0.7049180328, 0.1557377049),
		(0.7049180328, 0.1475409836),
		(0.7049180328, 0.1393442623),
		(0.7049180328, 0.1311475410),
		(0.7049180328, 0.1229508197),
		(0.7049180328, 0.1147540984),
		(0.7131147541, 0.5000000000),
		(0.7131147541, 0.4918032787),
		(0.7131147541, 0.4836065574),
		(0.7131147541, 0.4754098361),
		(0.7131147541, 0.4672131148),
		(0.7131147541, 0.4590163934),
		(0.7131147541, 0.4016393443),
		(0.7131147541, 0.3934426230),
		(0.7131147541, 0.3852459016),
		(0.7131147541, 0.3770491803),
		(0.7131147541, 0.3688524590),
		(0.7131147541, 0.3606557377),
		(0.7131147541, 0.3032786885),
		(0.7131147541, 0.2950819672),
		(0.7131147541, 0.2868852459),
		(0.7131147541, 0.2786885246),
		(0.7131147541, 0.2704918033),
		(0.7131147541, 0.2622950820),
		(0.7131147541, 0.2540983607),
		(0.7131147541, 0.2459016393),
		(0.7131147541, 0.2377049180),
		(0.7131147541, 0.2295081967),
		(0.7131147541, 0.2213114754),
		(0.7131147541, 0.2131147541),
		(0.7131147541, 0.1557377049),
		(0.7131147541, 0.1475409836),
		(0.7131147541, 0.1393442623),
		(0.7131147541, 0.1311475410),
		(0.7131147541, 0.1229508197),
		(0.7131147541, 0.1147540984),
		(0.7213114754, 0.5000000000),
		(0.7213114754, 0.4918032787),
		(0.7213114754, 0.4836065574),
		(0.7213114754, 0.4754098361),
		(0.7213114754, 0.4672131148),
		(0.7213114754, 0.4590163934),
		(0.7213114754, 0.4016393443),
		(0.7213114754, 0.3934426230),
		(0.7213114754, 0.3852459016),
		(0.7213114754, 0.3770491803),
		(0.7213114754, 0.3688524590),
		(0.7213114754, 0.3606557377),
		(0.7213114754, 0.3032786885),
		(0.7213114754, 0.2950819672),
		(0.7213114754, 0.2868852459),
		(0.7213114754, 0.2786885246),
		(0.7213114754, 0.2704918033),
		(0.7213114754, 0.2622950820),
		(0.7213114754, 0.2540983607),
		(0.7213114754, 0.2459016393),
		(0.7213114754, 0.2377049180),
		(0.7213114754, 0.2295081967),
		(0.7213114754, 0.2213114754),
		(0.7213114754, 0.2131147541),
		(0.7213114754, 0.1557377049),
		(0.7213114754, 0.1475409836),
		(0.7213114754, 0.1393442623),
		(0.7213114754, 0.1311475410),
		(0.7213114754, 0.1229508197),
		(0.7213114754, 0.1147540984),
		(0.7295081967, 0.5000000000),
		(0.7295081967, 0.4918032787),
		(0.7295081967, 0.4836065574),
		(0.7295081967, 0.4754098361),
		(0.7295081967, 0.4672131148),
		(0.7295081967, 0.4590163934),
		(0.7295081967, 0.4016393443),
		(0.7295081967, 0.3934426230),
		(0.7295081967, 0.3852459016),
		(0.7295081967, 0.3770491803),
		(0.7295081967, 0.3688524590),
		(0.7295081967, 0.3606557377),
		(0.7295081967, 0.3032786885),
		(0.7295081967, 0.2950819672),
		(0.7295081967, 0.2868852459),
		(0.7295081967, 0.2786885246),
		(0.7295081967, 0.2704918033),
		(0.7295081967, 0.2622950820),
		(0.7295081967, 0.2540983607),
		(0.7295081967, 0.2459016393),
		(0.7295081967, 0.2377049180),
		(0.7295081967, 0.2295081967),
		(0.7295081967, 0.2213114754),
		(0.7295081967, 0.2131147541),
		(0.7295081967, 0.1557377049),
		(0.7295081967, 0.1475409836),
		(0.7295081967, 0.1393442623),
		(0.7295081967, 0.1311475410),
		(0.7295081967, 0.1229508197),
		(0.7295081967, 0.1147540984),
		(0.7377049180, 0.5000000000),
		(0.7377049180, 0.4918032787),
		(0.7377049180, 0.4836065574),
		(0.7377049180, 0.4754098361),
		(0.7377049180, 0.4672131148),
		(0.7377049180, 0.4590163934),
		(0.7377049180, 0.4016393443),
		(0.7377049180, 0.3934426230),
		(0.7377049180, 0.3852459016),
		(0.7377049180, 0.3770491803),
		(0.7377049180, 0.3688524590),
		(0.7377049180, 0.3606557377),
		(0.7377049180, 0.3032786885),
		(0.7377049180, 0.2950819672),
		(0.7377049180, 0.2868852459),
		(0.7377049180, 0.2786885246),
		(0.7377049180, 0.2704918033),
		(0.7377049180, 0.2622950820),
		(0.7377049180, 0.2540983607),
		(0.7377049180, 0.2459016393),
		(0.7377049180, 0.2377049180),
		(0.7377049180, 0.2295081967),
		(0.7377049180, 0.2213114754),
		(0.7377049180, 0.2131147541),
		(0.7377049180, 0.1557377049),
		(0.7377049180, 0.1475409836),
		(0.7377049180, 0.1393442623),
		(0.7377049180, 0.1311475410),
		(0.7377049180, 0.1229508197),
		(0.7377049180, 0.1147540984),
		(0.7459016393, 0.5000000000),
		(0.7459016393, 0.4918032787),
		(0.7459016393, 0.4836065574),
		(0.7459016393, 0.4754098361),
		(0.7459016393, 0.4672131148),
		(0.7459016393, 0.4590163934),
		(0.7459016393, 0.4508196721),
		(0.7459016393, 0.4426229508),
		(0.7459016393, 0.4344262295),
		(0.7459016393, 0.4262295082),
		(0.7459016393, 0.4180327869),
		(0.7459016393, 0.4098360656),
		(0.7459016393, 0.3032786885),
		(0.7459016393, 0.2950819672),
		(0.7459016393, 0.2868852459),
		(0.7459016393, 0.2786885246),
		(0.7459016393, 0.2704918033),
		(0.7459016393, 0.2622950820),
		(0.7459016393, 0.2540983607),
		(0.7459016393, 0.2459016393),
		(0.7459016393, 0.2377049180),
		(0.7459016393, 0.2295081967),
		(0.7459016393, 0.2213114754),
		(0.7459016393, 0.2131147541),
		(0.7459016393, 0.1557377049),
		(0.7459016393, 0.1475409836),
		(0.7459016393, 0.1393442623),
		(0.7459016393, 0.1311475410),
		(0.7459016393, 0.1229508197),
		(0.7459016393, 0.1147540984),
		(0.7540983607, 0.5000000000),
		(0.7540983607, 0.4918032787),
		(0.7540983607, 0.4836065574),
		(0.7540983607, 0.4754098361),
		(0.7540983607, 0.4672131148),
		(0.7540983607, 0.4590163934),
		(0.7540983607, 0.4508196721),
		(0.7540983607, 0.4426229508),
		(0.7540983607, 0.4344262295),
		(0.7540983607, 0.4262295082),
		(0.7540983607, 0.4180327869),
		(0.7540983607, 0.4098360656),
		(0.7540983607, 0.3032786885),
		(0.7540983607, 0.2950819672),
		(0.7540983607, 0.2868852459),
		(0.7540983607, 0.2786885246),
		(0.7540983607, 0.2704918033),
		(0.7540983607, 0.2622950820),
		(0.7540983607, 0.2540983607),
		(0.7540983607, 0.2459016393),
		(0.7540983607, 0.2377049180),
		(0.7540983607, 0.2295081967),
		(0.7540983607, 0.2213114754),
		(0.7540983607, 0.2131147541),
		(0.7540983607, 0.1557377049),
		(0.7540983607, 0.1475409836),
		(0.7540983607, 0.1393442623),
		(0.7540983607, 0.1311475410),
		(0.7540983607, 0.1229508197),
		(0.7540983607, 0.1147540984),
		(0.7622950820, 0.5000000000),
		(0.7622950820, 0.4918032787),
		(0.7622950820, 0.4836065574),
		(0.7622950820, 0.4754098361),
		(0.7622950820, 0.4672131148),
		(0.7622950820, 0.4590163934),
		(0.7622950820, 0.4508196721),
		(0.7622950820, 0.4426229508),
		(0.7622950820, 0.4344262295),
		(0.7622950820, 0.4262295082),
		(0.7622950820, 0.4180327869),
		(0.7622950820, 0.4098360656),
		(0.7622950820, 0.3032786885),
		(0.7622950820, 0.2950819672),
		(0.7622950820, 0.2868852459),
		(0.7622950820, 0.2786885246),
		(0.7622950820, 0.2704918033),
		(0.7622950820, 0.2622950820),
		(0.7622950820, 0.2540983607),
		(0.7622950820, 0.2459016393),
		(0.7622950820, 0.2377049180),
		(0.7622950820, 0.2295081967),
		(0.7622950820, 0.2213114754),
		(0.7622950820, 0.2131147541),
		(0.7622950820, 0.1557377049),
		(0.7622950820, 0.1475409836),
		(0.7622950820, 0.1393442623),
		(0.7622950820, 0.1311475410),
		(0.7622950820, 0.1229508197),
		(0.7622950820, 0.1147540984),
		(0.7704918033, 0.5000000000),
		(0.7704918033, 0.4918032787),
		(0.7704918033, 0.4836065574),
		(0.7704918033, 0.4754098361),
		(0.7704918033, 0.4672131148),
		(0.7704918033, 0.4590163934),
		(0.7704918033, 0.4508196721),
		(0.7704918033, 0.4426229508),
		(0.7704918033, 0.4344262295),
		(0.7704918033, 0.4262295082),
		(0.7704918033, 0.4180327869),
		(0.7704918033, 0.4098360656),
		(0.7704918033, 0.3032786885),
		(0.7704918033, 0.2950819672),
		(0.7704918033, 0.2868852459),
		(0.7704918033, 0.2786885246),
		(0.7704918033, 0.2704918033),
		(0.7704918033, 0.2622950820),
		(0.7704918033, 0.2540983607),
		(0.7704918033, 0.2459016393),
		(0.7704918033, 0.2377049180),
		(0.7704918033, 0.2295081967),
		(0.7704918033, 0.2213114754),
		(0.7704918033, 0.2131147541),
		(0.7704918033, 0.1557377049),
		(0.7704918033, 0.1475409836),
		(0.7704918033, 0.1393442623),
		(0.7704918033, 0.1311475410),
		(0.7704918033, 0.1229508197),
		(0.7704918033, 0.1147540984),
		(0.7786885246, 0.5000000000),
		(0.7786885246, 0.4918032787),
		(0.7786885246, 0.4836065574),
		(0.7786885246, 0.4754098361),
		(0.7786885246, 0.4672131148),
		(0.7786885246, 0.4590163934),
		(0.7786885246, 0.4508196721),
		(0.7786885246, 0.4426229508),
		(0.7786885246, 0.4344262295),
		(0.7786885246, 0.4262295082),
		(0.7786885246, 0.4180327869),
		(0.7786885246, 0.4098360656),
		(0.7786885246, 0.3032786885),
		(0.7786885246, 0.2950819672),
		(0.7786885246, 0.2868852459),
		(0.7786885246, 0.2786885246),
		(0.7786885246, 0.2704918033),
		(0.7786885246, 0.2622950820),
		(0.7786885246, 0.2540983607),
		(0.7786885246, 0.2459016393),
		(0.7786885246, 0.2377049180),
		(0.7786885246, 0.2295081967),
		(0.7786885246, 0.2213114754),
		(0.7786885246, 0.2131147541),
		(0.7786885246, 0.1557377049),
		(0.7786885246, 0.1475409836),
		(0.7786885246, 0.1393442623),
		(0.7786885246, 0.1311475410),
		(0.7786885246, 0.1229508197),
		(0.7786885246, 0.1147540984),
		(0.7868852459, 0.5000000000),
		(0.7868852459, 0.4918032787),
		(0.7868852459, 0.4836065574),
		(0.7868852459, 0.4754098361),
		(0.7868852459, 0.4672131148),
		(0.7868852459, 0.4590163934),
		(0.7868852459, 0.4508196721),
		(0.7868852459, 0.4426229508),
		(0.7868852459, 0.4344262295),
		(0.7868852459, 0.4262295082),
		(0.7868852459, 0.4180327869),
		(0.7868852459, 0.4098360656),
		(0.7868852459, 0.3032786885),
		(0.7868852459, 0.2950819672),
		(0.7868852459, 0.2868852459),
		(0.7868852459, 0.2786885246),
		(0.7868852459, 0.2704918033),
		(0.7868852459, 0.2622950820),
		(0.7868852459, 0.2540983607),
		(0.7868852459, 0.2459016393),
		(0.7868852459, 0.2377049180),
		(0.7868852459, 0.2295081967),
		(0.7868852459, 0.2213114754),
		(0.7868852459, 0.2131147541),
		(0.7868852459, 0.1557377049),
		(0.7868852459, 0.1475409836),
		(0.7868852459, 0.1393442623),
		(0.7868852459, 0.1311475410),
		(0.7868852459, 0.1229508197),
		(0.7868852459, 0.1147540984),
		(0.7950819672, 0.5000000000),
		(0.7950819672, 0.4918032787),
		(0.7950819672, 0.4836065574),
		(0.7950819672, 0.4754098361),
		(0.7950819672, 0.4672131148),
		(0.7950819672, 0.4590163934),
		(0.7950819672, 0.4508196721),
		(0.7950819672, 0.4426229508),
		(0.7950819672, 0.4344262295),
		(0.7950819672, 0.4262295082),
		(0.7950819672, 0.4180327869),
		(0.7950819672, 0.4098360656),
		(0.7950819672, 0.3524590164),
		(0.7950819672, 0.3442622951),
		(0.7950819672, 0.3360655738),
		(0.7950819672, 0.3278688525),
		(0.7950819672, 0.3196721311),
		(0.7950819672, 0.3114754098),
		(0.7950819672, 0.2049180328),
		(0.7950819672, 0.1967213115),
		(0.7950819672, 0.1885245902),
		(0.7950819672, 0.1803278689),
		(0.7950819672, 0.1721311475),
		(0.7950819672, 0.1639344262),
		(0.7950819672, 0.1065573770),
		(0.7950819672, 0.0983606557),
		(0.7950819672, 0.0901639344),
		(0.7950819672, 0.0819672131),
		(0.7950819672, 0.0737704918),
		(0.7950819672, 0.0655737705),
		(0.8032786885, 0.5000000000),
		(0.8032786885, 0.4918032787),
		(0.8032786885, 0.4836065574),
		(0.8032786885, 0.4754098361),
		(0.8032786885, 0.4672131148),
		(0.8032786885, 0.4590163934),
		(0.8032786885, 0.4508196721),
		(0.8032786885, 0.4426229508),
		(0.8032786885, 0.4344262295),
		(0.8032786885, 0.4262295082),
		(0.8032786885, 0.4180327869),
		(0.8032786885, 0.4098360656),
		(0.8032786885, 0.3524590164),
		(0.8032786885, 0.3442622951),
		(0.8032786885, 0.3360655738),
		(0.8032786885, 0.3278688525),
		(0.8032786885, 0.3196721311),
		(0.8032786885, 0.3114754098),
		(0.8032786885, 0.2049180328),
		(0.8032786885, 0.1967213115),
		(0.8032786885, 0.1885245902),
		(0.8032786885, 0.1803278689),
		(0.8032786885, 0.1721311475),
		(0.8032786885, 0.1639344262),
		(0.8032786885, 0.1065573770),
		(0.8032786885, 0.0983606557),
		(0.8032786885, 0.0901639344),
		(0.8032786885, 0.0819672131),
		(0.8032786885, 0.0737704918),
		(0.8032786885, 0.0655737705),
		(0.8114754098, 0.5000000000),
		(0.8114754098, 0.4918032787),
		(0.8114754098, 0.4836065574),
		(0.8114754098, 0.4754098361),
		(0.8114754098, 0.4672131148),
		(0.8114754098, 0.4590163934),
		(0.8114754098, 0.4508196721),
		(0.8114754098, 0.4426229508),
		(0.8114754098, 0.4344262295),
		(0.8114754098, 0.4262295082),
		(0.8114754098, 0.4180327869),
		(0.8114754098, 0.4098360656),
		(0.8114754098, 0.3524590164),
		(0.8114754098, 0.3442622951),
		(0.8114754098, 0.3360655738),
		(0.8114754098, 0.3278688525),
		(0.8114754098, 0.3196721311),
		(0.8114754098, 0.3114754098),
		(0.8114754098, 0.2049180328),
		(0.8114754098, 0.1967213115),
		(0.8114754098, 0.1885245902),
		(0.8114754098, 0.1803278689),
		(0.8114754098, 0.1721311475),
		(0.8114754098, 0.1639344262),
		(0.8114754098, 0.1065573770),
		(0.8114754098, 0.0983606557),
		(0.8114754098, 0.0901639344),
		(0.8114754098, 0.0819672131),
		(0.8114754098, 0.0737704918),
		(0.8114754098, 0.0655737705),
		(0.8196721311, 0.5000000000),
		(0.8196721311, 0.4918032787),
		(0.8196721311, 0.4836065574),
		(0.8196721311, 0.4754098361),
		(0.8196721311, 0.4672131148),
		(0.8196721311, 0.4590163934),
		(0.8196721311, 0.4508196721),
		(0.8196721311, 0.4426229508),
		(0.8196721311, 0.4344262295),
		(0.8196721311, 0.4262295082),
		(0.8196721311, 0.4180327869),
		(0.8196721311, 0.4098360656),
		(0.8196721311, 0.3524590164),
		(0.8196721311, 0.3442622951),
		(0.8196721311, 0.3360655738),
		(0.8196721311, 0.3278688525),
		(0.8196721311, 0.3196721311),
		(0.8196721311, 0.3114754098),
		(0.8196721311, 0.2049180328),
		(0.8196721311, 0.1967213115),
		(0.8196721311, 0.1885245902),
		(0.8196721311, 0.1803278689),
		(0.8196721311, 0.1721311475),
		(0.8196721311, 0.1639344262),
		(0.8196721311, 0.1065573770),
		(0.8196721311, 0.0983606557),
		(0.8196721311, 0.0901639344),
		(0.8196721311, 0.0819672131),
		(0.8196721311, 0.0737704918),
		(0.8196721311, 0.0655737705),
		(0.8278688525, 0.5000000000),
		(0.8278688525, 0.4918032787),
		(0.8278688525, 0.4836065574),
		(0.8278688525, 0.4754098361),
		(0.8278688525, 0.4672131148),
		(0.8278688525, 0.4590163934),
		(0.8278688525, 0.4508196721),
		(0.8278688525, 0.4426229508),
		(0.8278688525, 0.4344262295),
		(0.8278688525, 0.4262295082),
		(0.8278688525, 0.4180327869),
		(0.8278688525, 0.4098360656),
		(0.8278688525, 0.3524590164),
		(0.8278688525, 0.3442622951),
		(0.8278688525, 0.3360655738),
		(0.8278688525, 0.3278688525),
		(0.8278688525, 0.3196721311),
		(0.8278688525, 0.3114754098),
		(0.8278688525, 0.2049180328),
		(0.8278688525, 0.1967213115),
		(0.8278688525, 0.1885245902),
		(0.8278688525, 0.1803278689),
		(0.8278688525, 0.1721311475),
		(0.8278688525, 0.1639344262),
		(0.8278688525, 0.1065573770),
		(0.8278688525, 0.0983606557),
		(0.8278688525, 0.0901639344),
		(0.8278688525, 0.0819672131),
		(0.8278688525, 0.0737704918),
		(0.8278688525, 0.0655737705),
		(0.8360655738, 0.5000000000),
		(0.8360655738, 0.4918032787),
		(0.8360655738, 0.4836065574),
		(0.8360655738, 0.4754098361),
		(0.8360655738, 0.4672131148),
		(0.8360655738, 0.4590163934),
		(0.8360655738, 0.4508196721),
		(0.8360655738, 0.4426229508),
		(0.8360655738, 0.4344262295),
		(0.8360655738, 0.4262295082),
		(0.8360655738, 0.4180327869),
		(0.8360655738, 0.4098360656),
		(0.8360655738, 0.3524590164),
		(0.8360655738, 0.3442622951),
		(0.8360655738, 0.3360655738),
		(0.8360655738, 0.3278688525),
		(0.8360655738, 0.3196721311),
		(0.8360655738, 0.3114754098),
		(0.8360655738, 0.2049180328),
		(0.8360655738, 0.1967213115),
		(0.8360655738, 0.1885245902),
		(0.8360655738, 0.1803278689),
		(0.8360655738, 0.1721311475),
		(0.8360655738, 0.1639344262),
		(0.8360655738, 0.1065573770),
		(0.8360655738, 0.0983606557),
		(0.8360655738, 0.0901639344),
		(0.8360655738, 0.0819672131),
		(0.8360655738, 0.0737704918),
		(0.8360655738, 0.0655737705),
		(0.8442622951, 0.3032786885),
		(0.8442622951, 0.2950819672),
		(0.8442622951, 0.2868852459),
		(0.8442622951, 0.2786885246),
		(0.8442622951, 0.2704918033),
		(0.8442622951, 0.2622950820),
		(0.8442622951, 0.2540983607),
		(0.8442622951, 0.2459016393),
		(0.8442622951, 0.2377049180),
		(0.8442622951, 0.2295081967),
		(0.8442622951, 0.2213114754),
		(0.8442622951, 0.2131147541),
		(0.8442622951, 0.1557377049),
		(0.8442622951, 0.1475409836),
		(0.8442622951, 0.1393442623),
		(0.8442622951, 0.1311475410),
		(0.8442622951, 0.1229508197),
		(0.8442622951, 0.1147540984),
		(0.8524590164, 0.3032786885),
		(0.8524590164, 0.2950819672),
		(0.8524590164, 0.2868852459),
		(0.8524590164, 0.2786885246),
		(0.8524590164, 0.2704918033),
		(0.8524590164, 0.2622950820),
		(0.8524590164, 0.2540983607),
		(0.8524590164, 0.2459016393),
		(0.8524590164, 0.2377049180),
		(0.8524590164, 0.2295081967),
		(0.8524590164, 0.2213114754),
		(0.8524590164, 0.2131147541),
		(0.8524590164, 0.1557377049),
		(0.8524590164, 0.1475409836),
		(0.8524590164, 0.1393442623),
		(0.8524590164, 0.1311475410),
		(0.8524590164, 0.1229508197),
		(0.8524590164, 0.1147540984),
		(0.8606557377, 0.3032786885),
		(0.8606557377, 0.2950819672),
		(0.8606557377, 0.2868852459),
		(0.8606557377, 0.2786885246),
		(0.8606557377, 0.2704918033),
		(0.8606557377, 0.2622950820),
		(0.8606557377, 0.2540983607),
		(0.8606557377, 0.2459016393),
		(0.8606557377, 0.2377049180),
		(0.8606557377, 0.2295081967),
		(0.8606557377, 0.2213114754),
		(0.8606557377, 0.2131147541),
		(0.8606557377, 0.1557377049),
		(0.8606557377, 0.1475409836),
		(0.8606557377, 0.1393442623),
		(0.8606557377, 0.1311475410),
		(0.8606557377, 0.1229508197),
		(0.8606557377, 0.1147540984),
		(0.8688524590, 0.3032786885),
		(0.8688524590, 0.2950819672),
		(0.8688524590, 0.2868852459),
		(0.8688524590, 0.2786885246),
		(0.8688524590, 0.2704918033),
		(0.8688524590, 0.2622950820),
		(0.8688524590, 0.2540983607),
		(0.8688524590, 0.2459016393),
		(0.8688524590, 0.2377049180),
		(0.8688524590, 0.2295081967),
		(0.8688524590, 0.2213114754),
		(0.8688524590, 0.2131147541),
		(0.8688524590, 0.1557377049),
		(0.8688524590, 0.1475409836),
		(0.8688524590, 0.1393442623),
		(0.8688524590, 0.1311475410),
		(0.8688524590, 0.1229508197),
		(0.8688524590, 0.1147540984),
		(0.8770491803, 0.3032786885),
		(0.8770491803, 0.2950819672),
		(0.8770491803, 0.2868852459),
		(0.8770491803, 0.2786885246),
		(0.8770491803, 0.2704918033),
		(0.8770491803, 0.2622950820),
		(0.8770491803, 0.2540983607),
		(0.8770491803, 0.2459016393),
		(0.8770491803, 0.2377049180),
		(0.8770491803, 0.2295081967),
		(0.8770491803, 0.2213114754),
		(0.8770491803, 0.2131147541),
		(0.8770491803, 0.1557377049),
		(0.8770491803, 0.1475409836),
		(0.8770491803, 0.1393442623),
		(0.8770491803, 0.1311475410),
		(0.8770491803, 0.1229508197),
		(0.8770491803, 0.1147540984),
		(0.8852459016, 0.3032786885),
		(0.8852459016, 0.2950819672),
		(0.8852459016, 0.2868852459),
		(0.8852459016, 0.2786885246),
		(0.8852459016, 0.2704918033),
		(0.8852459016, 0.2622950820),
		(0.8852459016, 0.2540983607),
		(0.8852459016, 0.2459016393),
		(0.8852459016, 0.2377049180),
		(0.8852459016, 0.2295081967),
		(0.8852459016, 0.2213114754),
		(0.8852459016, 0.2131147541),
		(0.8852459016, 0.1557377049),
		(0.8852459016, 0.1475409836),
		(0.8852459016, 0.1393442623),
		(0.8852459016, 0.1311475410),
		(0.8852459016, 0.1229508197),
		(0.8852459016, 0.1147540984),
		(0.8934426230, 0.2049180328),
		(0.8934426230, 0.1967213115),
		(0.8934426230, 0.1885245902),
		(0.8934426230, 0.1803278689),
		(0.8934426230, 0.1721311475),
		(0.8934426230, 0.1639344262),
		(0.8934426230, 0.1065573770),
		(0.8934426230, 0.0983606557),
		(0.8934426230, 0.0901639344),
		(0.8934426230, 0.0819672131),
		(0.8934426230, 0.0737704918),
		(0.8934426230, 0.0655737705),
		(0.8934426230, 0.0573770492),
		(0.8934426230, 0.0491803279),
		(0.8934426230, 0.0409836066),
		(0.8934426230, 0.0327868852),
		(0.8934426230, 0.0245901639),
		(0.8934426230, 0.0163934426),
		(0.9016393443, 0.2049180328),
		(0.9016393443, 0.1967213115),
		(0.9016393443, 0.1885245902),
		(0.9016393443, 0.1803278689),
		(0.9016393443, 0.1721311475),
		(0.9016393443, 0.1639344262),
		(0.9016393443, 0.1065573770),
		(0.9016393443, 0.0983606557),
		(0.9016393443, 0.0901639344),
		(0.9016393443, 0.0819672131),
		(0.9016393443, 0.0737704918),
		(0.9016393443, 0.0655737705),
		(0.9016393443, 0.0573770492),
		(0.9016393443, 0.0491803279),
		(0.9016393443, 0.0409836066),
		(0.9016393443, 0.0327868852),
		(0.9016393443, 0.0245901639),
		(0.9016393443, 0.0163934426),
		(0.9098360656, 0.2049180328),
		(0.9098360656, 0.1967213115),
		(0.9098360656, 0.1885245902),
		(0.9098360656, 0.1803278689),
		(0.9098360656, 0.1721311475),
		(0.9098360656, 0.1639344262),
		(0.9098360656, 0.1065573770),
		(0.9098360656, 0.0983606557),
		(0.9098360656, 0.0901639344),
		(0.9098360656, 0.0819672131),
		(0.9098360656, 0.0737704918),
		(0.9098360656, 0.0655737705),
		(0.9098360656, 0.0573770492),
		(0.9098360656, 0.0491803279),
		(0.9098360656, 0.0409836066),
		(0.9098360656, 0.0327868852),
		(0.9098360656, 0.0245901639),
		(0.9098360656, 0.0163934426),
		(0.9180327869, 0.2049180328),
		(0.9180327869, 0.1967213115),
		(0.9180327869, 0.1885245902),
		(0.9180327869, 0.1803278689),
		(0.9180327869, 0.1721311475),
		(0.9180327869, 0.1639344262),
		(0.9180327869, 0.1065573770),
		(0.9180327869, 0.0983606557),
		(0.9180327869, 0.0901639344),
		(0.9180327869, 0.0819672131),
		(0.9180327869, 0.0737704918),
		(0.9180327869, 0.0655737705),
		(0.9180327869, 0.0573770492),
		(0.9180327869, 0.0491803279),
		(0.9180327869, 0.0409836066),
		(0.9180327869, 0.0327868852),
		(0.9180327869, 0.0245901639),
		(0.9180327869, 0.0163934426),
		(0.9262295082, 0.2049180328),
		(0.9262295082, 0.1967213115),
		(0.9262295082, 0.1885245902),
		(0.9262295082, 0.1803278689),
		(0.9262295082, 0.1721311475),
		(0.9262295082, 0.1639344262),
		(0.9262295082, 0.1065573770),
		(0.9262295082, 0.0983606557),
		(0.9262295082, 0.0901639344),
		(0.9262295082, 0.0819672131),
		(0.9262295082, 0.0737704918),
		(0.9262295082, 0.0655737705),
		(0.9262295082, 0.0573770492),
		(0.9262295082, 0.0491803279),
		(0.9262295082, 0.0409836066),
		(0.9262295082, 0.0327868852),
		(0.9262295082, 0.0245901639),
		(0.9262295082, 0.0163934426),
		(0.9344262295, 0.2049180328),
		(0.9344262295, 0.1967213115),
		(0.9344262295, 0.1885245902),
		(0.9344262295, 0.1803278689),
		(0.9344262295, 0.1721311475),
		(0.9344262295, 0.1639344262),
		(0.9344262295, 0.1065573770),
		(0.9344262295, 0.0983606557),
		(0.9344262295, 0.0901639344),
		(0.9344262295, 0.0819672131),
		(0.9344262295, 0.0737704918),
		(0.9344262295, 0.0655737705),
		(0.9344262295, 0.0573770492),
		(0.9344262295, 0.0491803279),
		(0.9344262295, 0.0409836066),
		(0.9344262295, 0.0327868852),
		(0.9344262295, 0.0245901639),
		(0.9344262295, 0.0163934426),
		(0.9426229508, 0.5000000000),
		(0.9426229508, 0.4918032787),
		(0.9426229508, 0.4836065574),
		(0.9426229508, 0.4754098361),
		(0.9426229508, 0.4672131148),
		(0.9426229508, 0.4590163934),
		(0.9426229508, 0.4016393443),
		(0.9426229508, 0.3934426230),
		(0.9426229508, 0.3852459016),
		(0.9426229508, 0.3770491803),
		(0.9426229508, 0.3688524590),
		(0.9426229508, 0.3606557377),
		(0.9426229508, 0.1065573770),
		(0.9426229508, 0.0983606557),
		(0.9426229508, 0.0901639344),
		(0.9426229508, 0.0819672131),
		(0.9426229508, 0.0737704918),
		(0.9426229508, 0.0655737705),
		(0.9426229508, 0.0573770492),
		(0.9426229508, 0.0491803279),
		(0.9426229508, 0.0409836066),
		(0.9426229508, 0.0327868852),
		(0.9426229508, 0.0245901639),
		(0.9426229508, 0.0163934426),
		(0.9508196721, 0.5000000000),
		(0.9508196721, 0.4918032787),
		(0.9508196721, 0.4836065574),
		(0.9508196721, 0.4754098361),
		(0.9508196721, 0.4672131148),
		(0.9508196721, 0.4590163934),
		(0.9508196721, 0.4016393443),
		(0.9508196721, 0.3934426230),
		(0.9508196721, 0.3852459016),
		(0.9508196721, 0.3770491803),
		(0.9508196721, 0.3688524590),
		(0.9508196721, 0.3606557377),
		(0.9508196721, 0.1065573770),
		(0.9508196721, 0.0983606557),
		(0.9508196721, 0.0901639344),
		(0.9508196721, 0.0819672131),
		(0.9508196721, 0.0737704918),
		(0.9508196721, 0.0655737705),
		(0.9508196721, 0.0573770492),
		(0.9508196721, 0.0491803279),
		(0.9508196721, 0.0409836066),
		(0.9508196721, 0.0327868852),
		(0.9508196721, 0.0245901639),
		(0.9508196721, 0.0163934426),
		(0.9590163934, 0.5000000000),
		(0.9590163934, 0.4918032787),
		(0.9590163934, 0.4836065574),
		(0.9590163934, 0.4754098361),
		(0.9590163934, 0.4672131148),
		(0.9590163934, 0.4590163934),
		(0.9590163934, 0.4016393443),
		(0.9590163934, 0.3934426230),
		(0.9590163934, 0.3852459016),
		(0.9590163934, 0.3770491803),
		(0.9590163934, 0.3688524590),
		(0.9590163934, 0.3606557377),
		(0.9590163934, 0.1065573770),
		(0.9590163934, 0.0983606557),
		(0.9590163934, 0.0901639344),
		(0.9590163934, 0.0819672131),
		(0.9590163934, 0.0737704918),
		(0.9590163934, 0.0655737705),
		(0.9590163934, 0.0573770492),
		(0.9590163934, 0.0491803279),
		(0.9590163934, 0.0409836066),
		(0.9590163934, 0.0327868852),
		(0.9590163934, 0.0245901639),
		(0.9590163934, 0.0163934426),
		(0.9672131148, 0.5000000000),
		(0.9672131148, 0.4918032787),
		(0.9672131148, 0.4836065574),
		(0.9672131148, 0.4754098361),
		(0.9672131148, 0.4672131148),
		(0.9672131148, 0.4590163934),
		(0.9672131148, 0.4016393443),
		(0.9672131148, 0.3934426230),
		(0.9672131148, 0.3852459016),
		(0.9672131148, 0.3770491803),
		(0.9672131148, 0.3688524590),
		(0.9672131148, 0.3606557377),
		(0.9672131148, 0.1065573770),
		(0.9672131148, 0.0983606557),
		(0.9672131148, 0.0901639344),
		(0.9672131148, 0.0819672131),
		(0.9672131148, 0.0737704918),
		(0.9672131148, 0.0655737705),
		(0.9672131148, 0.0573770492),
		(0.9672131148, 0.0491803279),
		(0.9672131148, 0.0409836066),
		(0.9672131148, 0.0327868852),
		(0.9672131148, 0.0245901639),
		(0.9672131148, 0.0163934426),
		(0.9754098361, 0.5000000000),
		(0.9754098361, 0.4918032787),
		(0.9754098361, 0.4836065574),
		(0.9754098361, 0.4754098361),
		(0.9754098361, 0.4672131148),
		(0.9754098361, 0.4590163934),
		(0.9754098361, 0.4016393443),
		(0.9754098361, 0.3934426230),
		(0.9754098361, 0.3852459016),
		(0.9754098361, 0.3770491803),
		(0.9754098361, 0.3688524590),
		(0.9754098361, 0.3606557377),
		(0.9754098361, 0.1065573770),
		(0.9754098361, 0.0983606557),
		(0.9754098361, 0.0901639344),
		(0.9754098361, 0.0819672131),
		(0.9754098361, 0.0737704918),
		(0.9754098361, 0.0655737705),
		(0.9754098361, 0.0573770492),
		(0.9754098361, 0.0491803279),
		(0.9754098361, 0.0409836066),
		(0.9754098361, 0.0327868852),
		(0.9754098361, 0.0245901639),
		(0.9754098361, 0.0163934426),
		(0.9836065574, 0.5000000000),
		(0.9836065574, 0.4918032787),
		(0.9836065574, 0.4836065574),
		(0.9836065574, 0.4754098361),
		(0.9836065574, 0.4672131148),
		(0.9836065574, 0.4590163934),
		(0.9836065574, 0.4016393443),
		(0.9836065574, 0.3934426230),
		(0.9836065574, 0.3852459016),
		(0.9836065574, 0.3770491803),
		(0.9836065574, 0.3688524590),
		(0.9836065574, 0.3606557377),
		(0.9836065574, 0.1065573770),
		(0.9836065574, 0.0983606557),
		(0.9836065574, 0.0901639344),
		(0.9836065574, 0.0819672131),
		(0.9836065574, 0.0737704918),
		(0.9836065574, 0.0655737705),
		(0.9836065574, 0.0573770492),
		(0.9836065574, 0.0491803279),
		(0.9836065574, 0.0409836066),
		(0.9836065574, 0.0327868852),
		(0.9836065574, 0.0245901639),
		(0.9836065574, 0.0163934426)
    } {
        \draw[color=\documentcolor, fill=\documentcolor] \point circle (0.001);
    }

	% Domain.
    \draw[thick, color=\documentcolor]
        (-0.0081967213,0) -- (1,0) -- (1,1.0081967213) -- (-0.0081967213,1.0081967213) -- cycle;

\end{tikzpicture}
            \end{figure}
        }

        \only<2>{
            \begin{figure}[!ht]
                \centering
                \begin{tikzpicture}[scale=4.0, line join=round]

	% Domain, filled.
    \draw[thick, color=\documentcolor, fill=white]
        (-0.0081967213,0) -- (1,0) -- (1,1.0081967213) -- (-0.0081967213,1.0081967213) -- cycle;
    
    % Points.
    \foreach \point in {
		(0.0081967213, 0.9672131148),
		(0.0081967213, 0.9590163934),
		(0.0081967213, 0.9508196721),
		(0.0163934426, 0.9672131148),
		(0.0163934426, 0.9590163934),
		(0.0163934426, 0.9508196721),
		(0.0245901639, 0.9672131148),
		(0.0245901639, 0.9590163934),
		(0.0245901639, 0.9508196721),
		(0.0327868852, 0.9672131148),
		(0.0327868852, 0.9590163934),
		(0.0327868852, 0.9508196721),
		(0.0409836066, 0.9672131148),
		(0.0409836066, 0.9590163934),
		(0.0409836066, 0.9508196721),
		(0.0491803279, 0.9672131148),
		(0.0491803279, 0.9590163934),
		(0.0491803279, 0.9508196721),
		(0.0573770492, 0.9180327869),
		(0.0573770492, 0.9098360656),
		(0.0573770492, 0.9016393443),
		(0.0655737705, 0.9180327869),
		(0.0655737705, 0.9098360656),
		(0.0655737705, 0.9016393443),
		(0.0737704918, 0.9180327869),
		(0.0737704918, 0.9098360656),
		(0.0737704918, 0.9016393443),
		(0.0819672131, 0.9180327869),
		(0.0819672131, 0.9098360656),
		(0.0819672131, 0.9016393443),
		(0.0901639344, 0.9180327869),
		(0.0901639344, 0.9098360656),
		(0.0901639344, 0.9016393443),
		(0.0983606557, 0.9180327869),
		(0.0983606557, 0.9098360656),
		(0.0983606557, 0.9016393443),
		(0.1065573770, 0.8688524590),
		(0.1065573770, 0.8606557377),
		(0.1065573770, 0.8524590164),
		(0.1147540984, 0.8688524590),
		(0.1147540984, 0.8606557377),
		(0.1147540984, 0.8524590164),
		(0.1229508197, 0.8688524590),
		(0.1229508197, 0.8606557377),
		(0.1229508197, 0.8524590164),
		(0.1311475410, 0.8688524590),
		(0.1311475410, 0.8606557377),
		(0.1311475410, 0.8524590164),
		(0.1393442623, 0.8688524590),
		(0.1393442623, 0.8606557377),
		(0.1393442623, 0.8524590164),
		(0.1475409836, 0.8688524590),
		(0.1475409836, 0.8606557377),
		(0.1475409836, 0.8524590164),
		(0.1557377049, 0.8196721311),
		(0.1557377049, 0.8114754098),
		(0.1557377049, 0.8032786885),
		(0.1639344262, 0.8196721311),
		(0.1639344262, 0.8114754098),
		(0.1639344262, 0.8032786885),
		(0.1721311475, 0.8196721311),
		(0.1721311475, 0.8114754098),
		(0.1721311475, 0.8032786885),
		(0.1803278689, 0.8196721311),
		(0.1803278689, 0.8114754098),
		(0.1803278689, 0.8032786885),
		(0.1885245902, 0.8196721311),
		(0.1885245902, 0.8114754098),
		(0.1885245902, 0.8032786885),
		(0.1967213115, 0.8196721311),
		(0.1967213115, 0.8114754098),
		(0.1967213115, 0.8032786885),
		(0.2049180328, 0.7704918033),
		(0.2049180328, 0.7622950820),
		(0.2049180328, 0.7540983607),
		(0.2131147541, 0.7704918033),
		(0.2131147541, 0.7622950820),
		(0.2131147541, 0.7540983607),
		(0.2213114754, 0.7704918033),
		(0.2213114754, 0.7622950820),
		(0.2213114754, 0.7540983607),
		(0.2295081967, 0.7704918033),
		(0.2295081967, 0.7622950820),
		(0.2295081967, 0.7540983607),
		(0.2377049180, 0.7704918033),
		(0.2377049180, 0.7622950820),
		(0.2377049180, 0.7540983607),
		(0.2459016393, 0.7704918033),
		(0.2459016393, 0.7622950820),
		(0.2459016393, 0.7540983607),
		(0.2540983607, 0.7213114754),
		(0.2540983607, 0.7131147541),
		(0.2540983607, 0.7049180328),
		(0.2622950820, 0.7213114754),
		(0.2622950820, 0.7131147541),
		(0.2622950820, 0.7049180328),
		(0.2704918033, 0.7213114754),
		(0.2704918033, 0.7131147541),
		(0.2704918033, 0.7049180328),
		(0.2786885246, 0.7213114754),
		(0.2786885246, 0.7131147541),
		(0.2786885246, 0.7049180328),
		(0.2868852459, 0.7213114754),
		(0.2868852459, 0.7131147541),
		(0.2868852459, 0.7049180328),
		(0.2950819672, 0.7213114754),
		(0.2950819672, 0.7131147541),
		(0.2950819672, 0.7049180328),
		(0.3032786885, 0.6721311475),
		(0.3032786885, 0.6639344262),
		(0.3032786885, 0.6557377049),
		(0.3114754098, 0.6721311475),
		(0.3114754098, 0.6639344262),
		(0.3114754098, 0.6557377049),
		(0.3196721311, 0.6721311475),
		(0.3196721311, 0.6639344262),
		(0.3196721311, 0.6557377049),
		(0.3278688525, 0.6721311475),
		(0.3278688525, 0.6639344262),
		(0.3278688525, 0.6557377049),
		(0.3360655738, 0.6721311475),
		(0.3360655738, 0.6639344262),
		(0.3360655738, 0.6557377049),
		(0.3442622951, 0.6721311475),
		(0.3442622951, 0.6639344262),
		(0.3442622951, 0.6557377049),
		(0.3524590164, 0.6229508197),
		(0.3524590164, 0.6147540984),
		(0.3524590164, 0.6065573770),
		(0.3606557377, 0.6229508197),
		(0.3606557377, 0.6147540984),
		(0.3606557377, 0.6065573770),
		(0.3688524590, 0.6229508197),
		(0.3688524590, 0.6147540984),
		(0.3688524590, 0.6065573770),
		(0.3770491803, 0.6229508197),
		(0.3770491803, 0.6147540984),
		(0.3770491803, 0.6065573770),
		(0.3852459016, 0.6229508197),
		(0.3852459016, 0.6147540984),
		(0.3852459016, 0.6065573770),
		(0.3934426230, 0.6229508197),
		(0.3934426230, 0.6147540984),
		(0.3934426230, 0.6065573770),
		(0.4016393443, 0.5737704918),
		(0.4016393443, 0.5655737705),
		(0.4016393443, 0.5573770492),
		(0.4098360656, 0.5737704918),
		(0.4098360656, 0.5655737705),
		(0.4098360656, 0.5573770492),
		(0.4180327869, 0.5737704918),
		(0.4180327869, 0.5655737705),
		(0.4180327869, 0.5573770492),
		(0.4262295082, 0.5737704918),
		(0.4262295082, 0.5655737705),
		(0.4262295082, 0.5573770492),
		(0.4344262295, 0.5737704918),
		(0.4344262295, 0.5655737705),
		(0.4344262295, 0.5573770492),
		(0.4426229508, 0.5737704918),
		(0.4426229508, 0.5655737705),
		(0.4426229508, 0.5573770492),
		(0.4508196721, 0.5245901639),
		(0.4508196721, 0.5163934426),
		(0.4508196721, 0.5081967213),
		(0.4590163934, 0.5245901639),
		(0.4590163934, 0.5163934426),
		(0.4590163934, 0.5081967213),
		(0.4672131148, 0.5245901639),
		(0.4672131148, 0.5163934426),
		(0.4672131148, 0.5081967213),
		(0.4754098361, 0.5245901639),
		(0.4754098361, 0.5163934426),
		(0.4754098361, 0.5081967213),
		(0.4836065574, 0.5245901639),
		(0.4836065574, 0.5163934426),
		(0.4836065574, 0.5081967213),
		(0.4918032787, 0.5245901639),
		(0.4918032787, 0.5163934426),
		(0.4918032787, 0.5081967213),
		(0.5000000000, 0.4754098361),
		(0.5000000000, 0.4672131148),
		(0.5000000000, 0.4590163934),
		(0.5081967213, 0.4754098361),
		(0.5081967213, 0.4672131148),
		(0.5081967213, 0.4590163934),
		(0.5163934426, 0.4754098361),
		(0.5163934426, 0.4672131148),
		(0.5163934426, 0.4590163934),
		(0.5245901639, 0.4754098361),
		(0.5245901639, 0.4672131148),
		(0.5245901639, 0.4590163934),
		(0.5327868852, 0.4754098361),
		(0.5327868852, 0.4672131148),
		(0.5327868852, 0.4590163934),
		(0.5409836066, 0.4754098361),
		(0.5409836066, 0.4672131148),
		(0.5409836066, 0.4590163934),
		(0.5491803279, 0.4262295082),
		(0.5491803279, 0.4180327869),
		(0.5491803279, 0.4098360656),
		(0.5573770492, 0.4262295082),
		(0.5573770492, 0.4180327869),
		(0.5573770492, 0.4098360656),
		(0.5655737705, 0.4262295082),
		(0.5655737705, 0.4180327869),
		(0.5655737705, 0.4098360656),
		(0.5737704918, 0.4262295082),
		(0.5737704918, 0.4180327869),
		(0.5737704918, 0.4098360656),
		(0.5819672131, 0.4262295082),
		(0.5819672131, 0.4180327869),
		(0.5819672131, 0.4098360656),
		(0.5901639344, 0.4262295082),
		(0.5901639344, 0.4180327869),
		(0.5901639344, 0.4098360656),
		(0.5983606557, 0.3770491803),
		(0.5983606557, 0.3688524590),
		(0.5983606557, 0.3606557377),
		(0.6065573770, 0.3770491803),
		(0.6065573770, 0.3688524590),
		(0.6065573770, 0.3606557377),
		(0.6147540984, 0.3770491803),
		(0.6147540984, 0.3688524590),
		(0.6147540984, 0.3606557377),
		(0.6229508197, 0.3770491803),
		(0.6229508197, 0.3688524590),
		(0.6229508197, 0.3606557377),
		(0.6311475410, 0.3770491803),
		(0.6311475410, 0.3688524590),
		(0.6311475410, 0.3606557377),
		(0.6393442623, 0.3770491803),
		(0.6393442623, 0.3688524590),
		(0.6393442623, 0.3606557377),
		(0.6475409836, 0.3278688525),
		(0.6475409836, 0.3196721311),
		(0.6475409836, 0.3114754098),
		(0.6557377049, 0.3278688525),
		(0.6557377049, 0.3196721311),
		(0.6557377049, 0.3114754098),
		(0.6639344262, 0.3278688525),
		(0.6639344262, 0.3196721311),
		(0.6639344262, 0.3114754098),
		(0.6721311475, 0.3278688525),
		(0.6721311475, 0.3196721311),
		(0.6721311475, 0.3114754098),
		(0.6803278689, 0.3278688525),
		(0.6803278689, 0.3196721311),
		(0.6803278689, 0.3114754098),
		(0.6885245902, 0.3278688525),
		(0.6885245902, 0.3196721311),
		(0.6885245902, 0.3114754098),
		(0.6967213115, 0.2786885246),
		(0.6967213115, 0.2704918033),
		(0.6967213115, 0.2622950820),
		(0.7049180328, 0.2786885246),
		(0.7049180328, 0.2704918033),
		(0.7049180328, 0.2622950820),
		(0.7131147541, 0.2786885246),
		(0.7131147541, 0.2704918033),
		(0.7131147541, 0.2622950820),
		(0.7213114754, 0.2786885246),
		(0.7213114754, 0.2704918033),
		(0.7213114754, 0.2622950820),
		(0.7295081967, 0.2786885246),
		(0.7295081967, 0.2704918033),
		(0.7295081967, 0.2622950820),
		(0.7377049180, 0.2786885246),
		(0.7377049180, 0.2704918033),
		(0.7377049180, 0.2622950820),
		(0.7459016393, 0.2295081967),
		(0.7459016393, 0.2213114754),
		(0.7459016393, 0.2131147541),
		(0.7540983607, 0.2295081967),
		(0.7540983607, 0.2213114754),
		(0.7540983607, 0.2131147541),
		(0.7622950820, 0.2295081967),
		(0.7622950820, 0.2213114754),
		(0.7622950820, 0.2131147541),
		(0.7704918033, 0.2295081967),
		(0.7704918033, 0.2213114754),
		(0.7704918033, 0.2131147541),
		(0.7786885246, 0.2295081967),
		(0.7786885246, 0.2213114754),
		(0.7786885246, 0.2131147541),
		(0.7868852459, 0.2295081967),
		(0.7868852459, 0.2213114754),
		(0.7868852459, 0.2131147541),
		(0.7950819672, 0.1803278689),
		(0.7950819672, 0.1721311475),
		(0.7950819672, 0.1639344262),
		(0.8032786885, 0.1803278689),
		(0.8032786885, 0.1721311475),
		(0.8032786885, 0.1639344262),
		(0.8114754098, 0.1803278689),
		(0.8114754098, 0.1721311475),
		(0.8114754098, 0.1639344262),
		(0.8196721311, 0.1803278689),
		(0.8196721311, 0.1721311475),
		(0.8196721311, 0.1639344262),
		(0.8278688525, 0.1803278689),
		(0.8278688525, 0.1721311475),
		(0.8278688525, 0.1639344262),
		(0.8360655738, 0.1803278689),
		(0.8360655738, 0.1721311475),
		(0.8360655738, 0.1639344262),
		(0.8442622951, 0.1311475410),
		(0.8442622951, 0.1229508197),
		(0.8442622951, 0.1147540984),
		(0.8524590164, 0.1311475410),
		(0.8524590164, 0.1229508197),
		(0.8524590164, 0.1147540984),
		(0.8606557377, 0.1311475410),
		(0.8606557377, 0.1229508197),
		(0.8606557377, 0.1147540984),
		(0.8688524590, 0.1311475410),
		(0.8688524590, 0.1229508197),
		(0.8688524590, 0.1147540984),
		(0.8770491803, 0.1311475410),
		(0.8770491803, 0.1229508197),
		(0.8770491803, 0.1147540984),
		(0.8852459016, 0.1311475410),
		(0.8852459016, 0.1229508197),
		(0.8852459016, 0.1147540984),
		(0.8934426230, 0.0819672131),
		(0.8934426230, 0.0737704918),
		(0.8934426230, 0.0655737705),
		(0.9016393443, 0.0819672131),
		(0.9016393443, 0.0737704918),
		(0.9016393443, 0.0655737705),
		(0.9098360656, 0.0819672131),
		(0.9098360656, 0.0737704918),
		(0.9098360656, 0.0655737705),
		(0.9180327869, 0.0819672131),
		(0.9180327869, 0.0737704918),
		(0.9180327869, 0.0655737705),
		(0.9262295082, 0.0819672131),
		(0.9262295082, 0.0737704918),
		(0.9262295082, 0.0655737705),
		(0.9344262295, 0.0819672131),
		(0.9344262295, 0.0737704918),
		(0.9344262295, 0.0655737705),
		(0.9426229508, 0.0327868852),
		(0.9426229508, 0.0245901639),
		(0.9426229508, 0.0163934426),
		(0.9508196721, 0.0327868852),
		(0.9508196721, 0.0245901639),
		(0.9508196721, 0.0163934426),
		(0.9590163934, 0.0327868852),
		(0.9590163934, 0.0245901639),
		(0.9590163934, 0.0163934426),
		(0.9672131148, 0.0327868852),
		(0.9672131148, 0.0245901639),
		(0.9672131148, 0.0163934426),
		(0.9754098361, 0.0327868852),
		(0.9754098361, 0.0245901639),
		(0.9754098361, 0.0163934426),
		(0.9836065574, 0.0327868852),
		(0.9836065574, 0.0245901639),
		(0.9836065574, 0.0163934426)
    } {
        \draw[color=\accentcolor, fill=\accentcolor] \point circle (0.001);
    }

	% Domain.
    \draw[thick, color=\documentcolor]
        (-0.0081967213,0) -- (1,0) -- (1,1.0081967213) -- (-0.0081967213,1.0081967213) -- cycle;

\end{tikzpicture}
            \end{figure}
        }

        \only<3>{
            \begin{figure}[!ht]
                \centering
                \begin{tikzpicture}[scale=4.0, line join=round]

	% Domain, filled.
    \draw[thick, color=\documentcolor, fill=white]
        (0,0.02) -- (0.98,0.02) -- (0.98,1) -- (0,1) -- cycle;
    
    % Points.
    \foreach \point in {
		(0.0200, 0.9200),
        (0.0200, 0.9000),
        (0.0200, 0.8800),
        (0.0400, 0.9200),
        (0.0400, 0.9000),
        (0.0400, 0.8800),
        (0.0600, 0.9200),
        (0.0600, 0.9000),
        (0.0600, 0.8800),
        (0.0800, 0.9200),
        (0.0800, 0.9000),
        (0.0800, 0.8800),
        (0.1000, 0.9200),
        (0.1000, 0.9000),
        (0.1000, 0.8800),
        (0.1200, 0.9200),
        (0.1200, 0.9000),
        (0.1200, 0.8800),
        (0.1400, 0.8000),
        (0.1400, 0.7800),
        (0.1400, 0.7600),
        (0.1600, 0.8000),
        (0.1600, 0.7800),
        (0.1600, 0.7600),
        (0.1800, 0.8000),
        (0.1800, 0.7800),
        (0.1800, 0.7600),
        (0.2000, 0.8000),
        (0.2000, 0.7800),
        (0.2000, 0.7600),
        (0.2200, 0.8000),
        (0.2200, 0.7800),
        (0.2200, 0.7600),
        (0.2400, 0.8000),
        (0.2400, 0.7800),
        (0.2400, 0.7600),
        (0.2600, 0.6800),
        (0.2600, 0.6600),
        (0.2600, 0.6400),
        (0.2800, 0.6800),
        (0.2800, 0.6600),
        (0.2800, 0.6400),
        (0.3000, 0.6800),
        (0.3000, 0.6600),
        (0.3000, 0.6400),
        (0.3200, 0.6800),
        (0.3200, 0.6600),
        (0.3200, 0.6400),
        (0.3400, 0.6800),
        (0.3400, 0.6600),
        (0.3400, 0.6400),
        (0.3600, 0.6800),
        (0.3600, 0.6600),
        (0.3600, 0.6400),
        (0.3800, 0.5600),
        (0.3800, 0.5400),
        (0.3800, 0.5200),
        (0.4000, 0.5600),
        (0.4000, 0.5400),
        (0.4000, 0.5200),
        (0.4200, 0.5600),
        (0.4200, 0.5400),
        (0.4200, 0.5200),
        (0.4400, 0.5600),
        (0.4400, 0.5400),
        (0.4400, 0.5200),
        (0.4600, 0.5600),
        (0.4600, 0.5400),
        (0.4600, 0.5200),
        (0.4800, 0.5600),
        (0.4800, 0.5400),
        (0.4800, 0.5200),
        (0.5000, 0.4400),
        (0.5000, 0.4200),
        (0.5000, 0.4000),
        (0.5200, 0.4400),
        (0.5200, 0.4200),
        (0.5200, 0.4000),
        (0.5400, 0.4400),
        (0.5400, 0.4200),
        (0.5400, 0.4000),
        (0.5600, 0.4400),
        (0.5600, 0.4200),
        (0.5600, 0.4000),
        (0.5800, 0.4400),
        (0.5800, 0.4200),
        (0.5800, 0.4000),
        (0.6000, 0.4400),
        (0.6000, 0.4200),
        (0.6000, 0.4000),
        (0.6200, 0.3200),
        (0.6200, 0.3000),
        (0.6200, 0.2800),
        (0.6400, 0.3200),
        (0.6400, 0.3000),
        (0.6400, 0.2800),
        (0.6600, 0.3200),
        (0.6600, 0.3000),
        (0.6600, 0.2800),
        (0.6800, 0.3200),
        (0.6800, 0.3000),
        (0.6800, 0.2800),
        (0.7000, 0.3200),
        (0.7000, 0.3000),
        (0.7000, 0.2800),
        (0.7200, 0.3200),
        (0.7200, 0.3000),
        (0.7200, 0.2800),
        (0.7400, 0.2000),
        (0.7400, 0.1800),
        (0.7400, 0.1600),
        (0.7600, 0.2000),
        (0.7600, 0.1800),
        (0.7600, 0.1600),
        (0.7800, 0.2000),
        (0.7800, 0.1800),
        (0.7800, 0.1600),
        (0.8000, 0.2000),
        (0.8000, 0.1800),
        (0.8000, 0.1600),
        (0.8200, 0.2000),
        (0.8200, 0.1800),
        (0.8200, 0.1600),
        (0.8400, 0.2000),
        (0.8400, 0.1800),
        (0.8400, 0.1600),
        (0.8600, 0.0800),
        (0.8600, 0.0600),
        (0.8600, 0.0400),
        (0.8800, 0.0800),
        (0.8800, 0.0600),
        (0.8800, 0.0400),
        (0.9000, 0.0800),
        (0.9000, 0.0600),
        (0.9000, 0.0400),
        (0.9200, 0.0800),
        (0.9200, 0.0600),
        (0.9200, 0.0400),
        (0.9400, 0.0800),
        (0.9400, 0.0600),
        (0.9400, 0.0400),
        (0.9600, 0.0800),
        (0.9600, 0.0600),
        (0.9600, 0.0400)
    } {
        \draw[color=\documentcolor, fill=\documentcolor] \point circle (0.005);
    }

    % Points.
    \foreach \point in {
		(0.0200, 0.9800),
        (0.0200, 0.9600),
        (0.0200, 0.9400),
        (0.0400, 0.9800),
        (0.0400, 0.9600),
        (0.0400, 0.9400),
        (0.0600, 0.9800),
        (0.0600, 0.9600),
        (0.0600, 0.9400),
        (0.0800, 0.9800),
        (0.0800, 0.9600),
        (0.0800, 0.9400),
        (0.1000, 0.9800),
        (0.1000, 0.9600),
        (0.1000, 0.9400),
        (0.1200, 0.9800),
        (0.1200, 0.9600),
        (0.1200, 0.9400),
        (0.1400, 0.8600),
        (0.1400, 0.8400),
        (0.1400, 0.8200),
        (0.1600, 0.8600),
        (0.1600, 0.8400),
        (0.1600, 0.8200),
        (0.1800, 0.8600),
        (0.1800, 0.8400),
        (0.1800, 0.8200),
        (0.2000, 0.8600),
        (0.2000, 0.8400),
        (0.2000, 0.8200),
        (0.2200, 0.8600),
        (0.2200, 0.8400),
        (0.2200, 0.8200),
        (0.2400, 0.8600),
        (0.2400, 0.8400),
        (0.2400, 0.8200),
        (0.2600, 0.7400),
        (0.2600, 0.7200),
        (0.2600, 0.7000),
        (0.2800, 0.7400),
        (0.2800, 0.7200),
        (0.2800, 0.7000),
        (0.3000, 0.7400),
        (0.3000, 0.7200),
        (0.3000, 0.7000),
        (0.3200, 0.7400),
        (0.3200, 0.7200),
        (0.3200, 0.7000),
        (0.3400, 0.7400),
        (0.3400, 0.7200),
        (0.3400, 0.7000),
        (0.3600, 0.7400),
        (0.3600, 0.7200),
        (0.3600, 0.7000),
        (0.3800, 0.6200),
        (0.3800, 0.6000),
        (0.3800, 0.5800),
        (0.4000, 0.6200),
        (0.4000, 0.6000),
        (0.4000, 0.5800),
        (0.4200, 0.6200),
        (0.4200, 0.6000),
        (0.4200, 0.5800),
        (0.4400, 0.6200),
        (0.4400, 0.6000),
        (0.4400, 0.5800),
        (0.4600, 0.6200),
        (0.4600, 0.6000),
        (0.4600, 0.5800),
        (0.4800, 0.6200),
        (0.4800, 0.6000),
        (0.4800, 0.5800),
        (0.5000, 0.5000),
        (0.5000, 0.4800),
        (0.5000, 0.4600),
        (0.5200, 0.5000),
        (0.5200, 0.4800),
        (0.5200, 0.4600),
        (0.5400, 0.5000),
        (0.5400, 0.4800),
        (0.5400, 0.4600),
        (0.5600, 0.5000),
        (0.5600, 0.4800),
        (0.5600, 0.4600),
        (0.5800, 0.5000),
        (0.5800, 0.4800),
        (0.5800, 0.4600),
        (0.6000, 0.5000),
        (0.6000, 0.4800),
        (0.6000, 0.4600),
        (0.6200, 0.3800),
        (0.6200, 0.3600),
        (0.6200, 0.3400),
        (0.6400, 0.3800),
        (0.6400, 0.3600),
        (0.6400, 0.3400),
        (0.6600, 0.3800),
        (0.6600, 0.3600),
        (0.6600, 0.3400),
        (0.6800, 0.3800),
        (0.6800, 0.3600),
        (0.6800, 0.3400),
        (0.7000, 0.3800),
        (0.7000, 0.3600),
        (0.7000, 0.3400),
        (0.7200, 0.3800),
        (0.7200, 0.3600),
        (0.7200, 0.3400),
        (0.7400, 0.2600),
        (0.7400, 0.2400),
        (0.7400, 0.2200),
        (0.7600, 0.2600),
        (0.7600, 0.2400),
        (0.7600, 0.2200),
        (0.7800, 0.2600),
        (0.7800, 0.2400),
        (0.7800, 0.2200),
        (0.8000, 0.2600),
        (0.8000, 0.2400),
        (0.8000, 0.2200),
        (0.8200, 0.2600),
        (0.8200, 0.2400),
        (0.8200, 0.2200),
        (0.8400, 0.2600),
        (0.8400, 0.2400),
        (0.8400, 0.2200),
        (0.8600, 0.1400),
        (0.8600, 0.1200),
        (0.8600, 0.1000),
        (0.8800, 0.1400),
        (0.8800, 0.1200),
        (0.8800, 0.1000),
        (0.9000, 0.1400),
        (0.9000, 0.1200),
        (0.9000, 0.1000),
        (0.9200, 0.1400),
        (0.9200, 0.1200),
        (0.9200, 0.1000),
        (0.9400, 0.1400),
        (0.9400, 0.1200),
        (0.9400, 0.1000),
        (0.9600, 0.1400),
        (0.9600, 0.1200),
        (0.9600, 0.1000)
    } {
        \draw[color=\accentcolor, fill=\accentcolor] \point circle (0.005);
    }

	% Domain.
    \draw[thick, color=\documentcolor]
        (0,0.02) -- (0.98,0.02) -- (0.98,1) -- (0,1) -- cycle;

\end{tikzpicture}
            \end{figure}
        }

        \only<4>{
            \begin{figure}[!ht]
                \centering
                \begin{tikzpicture}[scale=4.0, line join=round]

	% Domain, filled.
    \draw[thick, color=\documentcolor, fill=white]
        (0,0.02) -- (0.98,0.02) -- (0.98,1) -- (0,1) -- cycle;
    
    % Points.
    \foreach \point in {
		(0.0200, 0.9800),
        (0.0200, 0.9600),
        (0.0200, 0.9400),
        (0.0200, 0.9200),
        (0.0200, 0.9000),
        (0.0200, 0.8800),
        (0.0400, 0.9800),
        (0.0400, 0.9600),
        (0.0400, 0.9400),
        (0.0400, 0.9200),
        (0.0400, 0.9000),
        (0.0400, 0.8800),
        (0.0600, 0.9800),
        (0.0600, 0.9600),
        (0.0600, 0.9400),
        (0.0600, 0.9200),
        (0.0600, 0.9000),
        (0.0600, 0.8800),
        (0.0800, 0.9800),
        (0.0800, 0.9600),
        (0.0800, 0.9400),
        (0.0800, 0.9200),
        (0.0800, 0.9000),
        (0.0800, 0.8800),
        (0.1000, 0.9800),
        (0.1000, 0.9600),
        (0.1000, 0.9400),
        (0.1000, 0.9200),
        (0.1000, 0.9000),
        (0.1000, 0.8800),
        (0.1200, 0.9800),
        (0.1200, 0.9600),
        (0.1200, 0.9400),
        (0.1200, 0.9200),
        (0.1200, 0.9000),
        (0.1200, 0.8800),
        (0.1400, 0.8600),
        (0.1400, 0.8400),
        (0.1400, 0.8200),
        (0.1400, 0.8000),
        (0.1400, 0.7800),
        (0.1400, 0.7600),
        (0.1600, 0.8600),
        (0.1600, 0.8400),
        (0.1600, 0.8200),
        (0.1600, 0.8000),
        (0.1600, 0.7800),
        (0.1600, 0.7600),
        (0.1800, 0.8600),
        (0.1800, 0.8400),
        (0.1800, 0.8200),
        (0.1800, 0.8000),
        (0.1800, 0.7800),
        (0.1800, 0.7600),
        (0.2000, 0.8600),
        (0.2000, 0.8400),
        (0.2000, 0.8200),
        (0.2000, 0.8000),
        (0.2000, 0.7800),
        (0.2000, 0.7600),
        (0.2200, 0.8600),
        (0.2200, 0.8400),
        (0.2200, 0.8200),
        (0.2200, 0.8000),
        (0.2200, 0.7800),
        (0.2200, 0.7600),
        (0.2400, 0.8600),
        (0.2400, 0.8400),
        (0.2400, 0.8200),
        (0.2400, 0.8000),
        (0.2400, 0.7800),
        (0.2400, 0.7600),
        (0.2600, 0.7400),
        (0.2600, 0.7200),
        (0.2600, 0.7000),
        (0.2600, 0.6800),
        (0.2600, 0.6600),
        (0.2600, 0.6400),
        (0.2800, 0.7400),
        (0.2800, 0.7200),
        (0.2800, 0.7000),
        (0.2800, 0.6800),
        (0.2800, 0.6600),
        (0.2800, 0.6400),
        (0.3000, 0.7400),
        (0.3000, 0.7200),
        (0.3000, 0.7000),
        (0.3000, 0.6800),
        (0.3000, 0.6600),
        (0.3000, 0.6400),
        (0.3200, 0.7400),
        (0.3200, 0.7200),
        (0.3200, 0.7000),
        (0.3200, 0.6800),
        (0.3200, 0.6600),
        (0.3200, 0.6400),
        (0.3400, 0.7400),
        (0.3400, 0.7200),
        (0.3400, 0.7000),
        (0.3400, 0.6800),
        (0.3400, 0.6600),
        (0.3400, 0.6400),
        (0.3600, 0.7400),
        (0.3600, 0.7200),
        (0.3600, 0.7000),
        (0.3600, 0.6800),
        (0.3600, 0.6600),
        (0.3600, 0.6400),
        (0.3800, 0.6200),
        (0.3800, 0.6000),
        (0.3800, 0.5800),
        (0.3800, 0.5600),
        (0.3800, 0.5400),
        (0.3800, 0.5200),
        (0.4000, 0.6200),
        (0.4000, 0.6000),
        (0.4000, 0.5800),
        (0.4000, 0.5600),
        (0.4000, 0.5400),
        (0.4000, 0.5200),
        (0.4200, 0.6200),
        (0.4200, 0.6000),
        (0.4200, 0.5800),
        (0.4200, 0.5600),
        (0.4200, 0.5400),
        (0.4200, 0.5200),
        (0.4400, 0.6200),
        (0.4400, 0.6000),
        (0.4400, 0.5800),
        (0.4400, 0.5600),
        (0.4400, 0.5400),
        (0.4400, 0.5200),
        (0.4600, 0.6200),
        (0.4600, 0.6000),
        (0.4600, 0.5800),
        (0.4600, 0.5600),
        (0.4600, 0.5400),
        (0.4600, 0.5200),
        (0.4800, 0.6200),
        (0.4800, 0.6000),
        (0.4800, 0.5800),
        (0.4800, 0.5600),
        (0.4800, 0.5400),
        (0.4800, 0.5200),
        (0.5000, 0.5000),
        (0.5000, 0.4800),
        (0.5000, 0.4600),
        (0.5000, 0.4400),
        (0.5000, 0.4200),
        (0.5000, 0.4000),
        (0.5200, 0.5000),
        (0.5200, 0.4800),
        (0.5200, 0.4600),
        (0.5200, 0.4400),
        (0.5200, 0.4200),
        (0.5200, 0.4000),
        (0.5400, 0.5000),
        (0.5400, 0.4800),
        (0.5400, 0.4600),
        (0.5400, 0.4400),
        (0.5400, 0.4200),
        (0.5400, 0.4000),
        (0.5600, 0.5000),
        (0.5600, 0.4800),
        (0.5600, 0.4600),
        (0.5600, 0.4400),
        (0.5600, 0.4200),
        (0.5600, 0.4000),
        (0.5800, 0.5000),
        (0.5800, 0.4800),
        (0.5800, 0.4600),
        (0.5800, 0.4400),
        (0.5800, 0.4200),
        (0.5800, 0.4000),
        (0.6000, 0.5000),
        (0.6000, 0.4800),
        (0.6000, 0.4600),
        (0.6000, 0.4400),
        (0.6000, 0.4200),
        (0.6000, 0.4000),
        (0.6200, 0.3800),
        (0.6200, 0.3600),
        (0.6200, 0.3400),
        (0.6200, 0.3200),
        (0.6200, 0.3000),
        (0.6200, 0.2800),
        (0.6400, 0.3800),
        (0.6400, 0.3600),
        (0.6400, 0.3400),
        (0.6400, 0.3200),
        (0.6400, 0.3000),
        (0.6400, 0.2800),
        (0.6600, 0.3800),
        (0.6600, 0.3600),
        (0.6600, 0.3400),
        (0.6600, 0.3200),
        (0.6600, 0.3000),
        (0.6600, 0.2800),
        (0.6800, 0.3800),
        (0.6800, 0.3600),
        (0.6800, 0.3400),
        (0.6800, 0.3200),
        (0.6800, 0.3000),
        (0.6800, 0.2800),
        (0.7000, 0.3800),
        (0.7000, 0.3600),
        (0.7000, 0.3400),
        (0.7000, 0.3200),
        (0.7000, 0.3000),
        (0.7000, 0.2800),
        (0.7200, 0.3800),
        (0.7200, 0.3600),
        (0.7200, 0.3400),
        (0.7200, 0.3200),
        (0.7200, 0.3000),
        (0.7200, 0.2800),
        (0.7400, 0.2600),
        (0.7400, 0.2400),
        (0.7400, 0.2200),
        (0.7400, 0.2000),
        (0.7400, 0.1800),
        (0.7400, 0.1600),
        (0.7600, 0.2600),
        (0.7600, 0.2400),
        (0.7600, 0.2200),
        (0.7600, 0.2000),
        (0.7600, 0.1800),
        (0.7600, 0.1600),
        (0.7800, 0.2600),
        (0.7800, 0.2400),
        (0.7800, 0.2200),
        (0.7800, 0.2000),
        (0.7800, 0.1800),
        (0.7800, 0.1600),
        (0.8000, 0.2600),
        (0.8000, 0.2400),
        (0.8000, 0.2200),
        (0.8000, 0.2000),
        (0.8000, 0.1800),
        (0.8000, 0.1600),
        (0.8200, 0.2600),
        (0.8200, 0.2400),
        (0.8200, 0.2200),
        (0.8200, 0.2000),
        (0.8200, 0.1800),
        (0.8200, 0.1600),
        (0.8400, 0.2600),
        (0.8400, 0.2400),
        (0.8400, 0.2200),
        (0.8400, 0.2000),
        (0.8400, 0.1800),
        (0.8400, 0.1600),
        (0.8600, 0.1400),
        (0.8600, 0.1200),
        (0.8600, 0.1000),
        (0.8600, 0.0800),
        (0.8600, 0.0600),
        (0.8600, 0.0400),
        (0.8800, 0.1400),
        (0.8800, 0.1200),
        (0.8800, 0.1000),
        (0.8800, 0.0800),
        (0.8800, 0.0600),
        (0.8800, 0.0400),
        (0.9000, 0.1400),
        (0.9000, 0.1200),
        (0.9000, 0.1000),
        (0.9000, 0.0800),
        (0.9000, 0.0600),
        (0.9000, 0.0400),
        (0.9200, 0.1400),
        (0.9200, 0.1200),
        (0.9200, 0.1000),
        (0.9200, 0.0800),
        (0.9200, 0.0600),
        (0.9200, 0.0400),
        (0.9400, 0.1400),
        (0.9400, 0.1200),
        (0.9400, 0.1000),
        (0.9400, 0.0800),
        (0.9400, 0.0600),
        (0.9400, 0.0400),
        (0.9600, 0.1400),
        (0.9600, 0.1200),
        (0.9600, 0.1000),
        (0.9600, 0.0800),
        (0.9600, 0.0600),
        (0.9600, 0.0400)
    } {
        \draw[color=\documentcolor, fill=\documentcolor] \point circle (0.005);
    }

    % Points.
    \foreach \point in {
		(0.0200, 0.7400),
        (0.0200, 0.7200),
        (0.0200, 0.7000),
        (0.0200, 0.6800),
        (0.0200, 0.6600),
        (0.0200, 0.6400),
        (0.0200, 0.6200),
        (0.0200, 0.6000),
        (0.0200, 0.5800),
        (0.0200, 0.5600),
        (0.0200, 0.5400),
        (0.0200, 0.5200),
        (0.0400, 0.7400),
        (0.0400, 0.7200),
        (0.0400, 0.7000),
        (0.0400, 0.6800),
        (0.0400, 0.6600),
        (0.0400, 0.6400),
        (0.0400, 0.6200),
        (0.0400, 0.6000),
        (0.0400, 0.5800),
        (0.0400, 0.5600),
        (0.0400, 0.5400),
        (0.0400, 0.5200),
        (0.0600, 0.7400),
        (0.0600, 0.7200),
        (0.0600, 0.7000),
        (0.0600, 0.6800),
        (0.0600, 0.6600),
        (0.0600, 0.6400),
        (0.0600, 0.6200),
        (0.0600, 0.6000),
        (0.0600, 0.5800),
        (0.0600, 0.5600),
        (0.0600, 0.5400),
        (0.0600, 0.5200),
        (0.0800, 0.7400),
        (0.0800, 0.7200),
        (0.0800, 0.7000),
        (0.0800, 0.6800),
        (0.0800, 0.6600),
        (0.0800, 0.6400),
        (0.0800, 0.6200),
        (0.0800, 0.6000),
        (0.0800, 0.5800),
        (0.0800, 0.5600),
        (0.0800, 0.5400),
        (0.0800, 0.5200),
        (0.1000, 0.7400),
        (0.1000, 0.7200),
        (0.1000, 0.7000),
        (0.1000, 0.6800),
        (0.1000, 0.6600),
        (0.1000, 0.6400),
        (0.1000, 0.6200),
        (0.1000, 0.6000),
        (0.1000, 0.5800),
        (0.1000, 0.5600),
        (0.1000, 0.5400),
        (0.1000, 0.5200),
        (0.1200, 0.7400),
        (0.1200, 0.7200),
        (0.1200, 0.7000),
        (0.1200, 0.6800),
        (0.1200, 0.6600),
        (0.1200, 0.6400),
        (0.1200, 0.6200),
        (0.1200, 0.6000),
        (0.1200, 0.5800),
        (0.1200, 0.5600),
        (0.1200, 0.5400),
        (0.1200, 0.5200),
        (0.1400, 0.7400),
        (0.1400, 0.7200),
        (0.1400, 0.7000),
        (0.1400, 0.6800),
        (0.1400, 0.6600),
        (0.1400, 0.6400),
        (0.1400, 0.6200),
        (0.1400, 0.6000),
        (0.1400, 0.5800),
        (0.1400, 0.5600),
        (0.1400, 0.5400),
        (0.1400, 0.5200),
        (0.1600, 0.7400),
        (0.1600, 0.7200),
        (0.1600, 0.7000),
        (0.1600, 0.6800),
        (0.1600, 0.6600),
        (0.1600, 0.6400),
        (0.1600, 0.6200),
        (0.1600, 0.6000),
        (0.1600, 0.5800),
        (0.1600, 0.5600),
        (0.1600, 0.5400),
        (0.1600, 0.5200),
        (0.1800, 0.7400),
        (0.1800, 0.7200),
        (0.1800, 0.7000),
        (0.1800, 0.6800),
        (0.1800, 0.6600),
        (0.1800, 0.6400),
        (0.1800, 0.6200),
        (0.1800, 0.6000),
        (0.1800, 0.5800),
        (0.1800, 0.5600),
        (0.1800, 0.5400),
        (0.1800, 0.5200),
        (0.2000, 0.7400),
        (0.2000, 0.7200),
        (0.2000, 0.7000),
        (0.2000, 0.6800),
        (0.2000, 0.6600),
        (0.2000, 0.6400),
        (0.2000, 0.6200),
        (0.2000, 0.6000),
        (0.2000, 0.5800),
        (0.2000, 0.5600),
        (0.2000, 0.5400),
        (0.2000, 0.5200),
        (0.2200, 0.7400),
        (0.2200, 0.7200),
        (0.2200, 0.7000),
        (0.2200, 0.6800),
        (0.2200, 0.6600),
        (0.2200, 0.6400),
        (0.2200, 0.6200),
        (0.2200, 0.6000),
        (0.2200, 0.5800),
        (0.2200, 0.5600),
        (0.2200, 0.5400),
        (0.2200, 0.5200),
        (0.2400, 0.7400),
        (0.2400, 0.7200),
        (0.2400, 0.7000),
        (0.2400, 0.6800),
        (0.2400, 0.6600),
        (0.2400, 0.6400),
        (0.2400, 0.6200),
        (0.2400, 0.6000),
        (0.2400, 0.5800),
        (0.2400, 0.5600),
        (0.2400, 0.5400),
        (0.2400, 0.5200),
        (0.2600, 0.9800),
        (0.2600, 0.9600),
        (0.2600, 0.9400),
        (0.2600, 0.9200),
        (0.2600, 0.9000),
        (0.2600, 0.8800),
        (0.2600, 0.8600),
        (0.2600, 0.8400),
        (0.2600, 0.8200),
        (0.2600, 0.8000),
        (0.2600, 0.7800),
        (0.2600, 0.7600),
        (0.2800, 0.9800),
        (0.2800, 0.9600),
        (0.2800, 0.9400),
        (0.2800, 0.9200),
        (0.2800, 0.9000),
        (0.2800, 0.8800),
        (0.2800, 0.8600),
        (0.2800, 0.8400),
        (0.2800, 0.8200),
        (0.2800, 0.8000),
        (0.2800, 0.7800),
        (0.2800, 0.7600),
        (0.3000, 0.9800),
        (0.3000, 0.9600),
        (0.3000, 0.9400),
        (0.3000, 0.9200),
        (0.3000, 0.9000),
        (0.3000, 0.8800),
        (0.3000, 0.8600),
        (0.3000, 0.8400),
        (0.3000, 0.8200),
        (0.3000, 0.8000),
        (0.3000, 0.7800),
        (0.3000, 0.7600),
        (0.3200, 0.9800),
        (0.3200, 0.9600),
        (0.3200, 0.9400),
        (0.3200, 0.9200),
        (0.3200, 0.9000),
        (0.3200, 0.8800),
        (0.3200, 0.8600),
        (0.3200, 0.8400),
        (0.3200, 0.8200),
        (0.3200, 0.8000),
        (0.3200, 0.7800),
        (0.3200, 0.7600),
        (0.3400, 0.9800),
        (0.3400, 0.9600),
        (0.3400, 0.9400),
        (0.3400, 0.9200),
        (0.3400, 0.9000),
        (0.3400, 0.8800),
        (0.3400, 0.8600),
        (0.3400, 0.8400),
        (0.3400, 0.8200),
        (0.3400, 0.8000),
        (0.3400, 0.7800),
        (0.3400, 0.7600),
        (0.3600, 0.9800),
        (0.3600, 0.9600),
        (0.3600, 0.9400),
        (0.3600, 0.9200),
        (0.3600, 0.9000),
        (0.3600, 0.8800),
        (0.3600, 0.8600),
        (0.3600, 0.8400),
        (0.3600, 0.8200),
        (0.3600, 0.8000),
        (0.3600, 0.7800),
        (0.3600, 0.7600),
        (0.3800, 0.9800),
        (0.3800, 0.9600),
        (0.3800, 0.9400),
        (0.3800, 0.9200),
        (0.3800, 0.9000),
        (0.3800, 0.8800),
        (0.3800, 0.8600),
        (0.3800, 0.8400),
        (0.3800, 0.8200),
        (0.3800, 0.8000),
        (0.3800, 0.7800),
        (0.3800, 0.7600),
        (0.4000, 0.9800),
        (0.4000, 0.9600),
        (0.4000, 0.9400),
        (0.4000, 0.9200),
        (0.4000, 0.9000),
        (0.4000, 0.8800),
        (0.4000, 0.8600),
        (0.4000, 0.8400),
        (0.4000, 0.8200),
        (0.4000, 0.8000),
        (0.4000, 0.7800),
        (0.4000, 0.7600),
        (0.4200, 0.9800),
        (0.4200, 0.9600),
        (0.4200, 0.9400),
        (0.4200, 0.9200),
        (0.4200, 0.9000),
        (0.4200, 0.8800),
        (0.4200, 0.8600),
        (0.4200, 0.8400),
        (0.4200, 0.8200),
        (0.4200, 0.8000),
        (0.4200, 0.7800),
        (0.4200, 0.7600),
        (0.4400, 0.9800),
        (0.4400, 0.9600),
        (0.4400, 0.9400),
        (0.4400, 0.9200),
        (0.4400, 0.9000),
        (0.4400, 0.8800),
        (0.4400, 0.8600),
        (0.4400, 0.8400),
        (0.4400, 0.8200),
        (0.4400, 0.8000),
        (0.4400, 0.7800),
        (0.4400, 0.7600),
        (0.4600, 0.9800),
        (0.4600, 0.9600),
        (0.4600, 0.9400),
        (0.4600, 0.9200),
        (0.4600, 0.9000),
        (0.4600, 0.8800),
        (0.4600, 0.8600),
        (0.4600, 0.8400),
        (0.4600, 0.8200),
        (0.4600, 0.8000),
        (0.4600, 0.7800),
        (0.4600, 0.7600),
        (0.4800, 0.9800),
        (0.4800, 0.9600),
        (0.4800, 0.9400),
        (0.4800, 0.9200),
        (0.4800, 0.9000),
        (0.4800, 0.8800),
        (0.4800, 0.8600),
        (0.4800, 0.8400),
        (0.4800, 0.8200),
        (0.4800, 0.8000),
        (0.4800, 0.7800),
        (0.4800, 0.7600),
        (0.5000, 0.2600),
        (0.5000, 0.2400),
        (0.5000, 0.2200),
        (0.5000, 0.2000),
        (0.5000, 0.1800),
        (0.5000, 0.1600),
        (0.5000, 0.1400),
        (0.5000, 0.1200),
        (0.5000, 0.1000),
        (0.5000, 0.0800),
        (0.5000, 0.0600),
        (0.5000, 0.0400),
        (0.5200, 0.2600),
        (0.5200, 0.2400),
        (0.5200, 0.2200),
        (0.5200, 0.2000),
        (0.5200, 0.1800),
        (0.5200, 0.1600),
        (0.5200, 0.1400),
        (0.5200, 0.1200),
        (0.5200, 0.1000),
        (0.5200, 0.0800),
        (0.5200, 0.0600),
        (0.5200, 0.0400),
        (0.5400, 0.2600),
        (0.5400, 0.2400),
        (0.5400, 0.2200),
        (0.5400, 0.2000),
        (0.5400, 0.1800),
        (0.5400, 0.1600),
        (0.5400, 0.1400),
        (0.5400, 0.1200),
        (0.5400, 0.1000),
        (0.5400, 0.0800),
        (0.5400, 0.0600),
        (0.5400, 0.0400),
        (0.5600, 0.2600),
        (0.5600, 0.2400),
        (0.5600, 0.2200),
        (0.5600, 0.2000),
        (0.5600, 0.1800),
        (0.5600, 0.1600),
        (0.5600, 0.1400),
        (0.5600, 0.1200),
        (0.5600, 0.1000),
        (0.5600, 0.0800),
        (0.5600, 0.0600),
        (0.5600, 0.0400),
        (0.5800, 0.2600),
        (0.5800, 0.2400),
        (0.5800, 0.2200),
        (0.5800, 0.2000),
        (0.5800, 0.1800),
        (0.5800, 0.1600),
        (0.5800, 0.1400),
        (0.5800, 0.1200),
        (0.5800, 0.1000),
        (0.5800, 0.0800),
        (0.5800, 0.0600),
        (0.5800, 0.0400),
        (0.6000, 0.2600),
        (0.6000, 0.2400),
        (0.6000, 0.2200),
        (0.6000, 0.2000),
        (0.6000, 0.1800),
        (0.6000, 0.1600),
        (0.6000, 0.1400),
        (0.6000, 0.1200),
        (0.6000, 0.1000),
        (0.6000, 0.0800),
        (0.6000, 0.0600),
        (0.6000, 0.0400),
        (0.6200, 0.2600),
        (0.6200, 0.2400),
        (0.6200, 0.2200),
        (0.6200, 0.2000),
        (0.6200, 0.1800),
        (0.6200, 0.1600),
        (0.6200, 0.1400),
        (0.6200, 0.1200),
        (0.6200, 0.1000),
        (0.6200, 0.0800),
        (0.6200, 0.0600),
        (0.6200, 0.0400),
        (0.6400, 0.2600),
        (0.6400, 0.2400),
        (0.6400, 0.2200),
        (0.6400, 0.2000),
        (0.6400, 0.1800),
        (0.6400, 0.1600),
        (0.6400, 0.1400),
        (0.6400, 0.1200),
        (0.6400, 0.1000),
        (0.6400, 0.0800),
        (0.6400, 0.0600),
        (0.6400, 0.0400),
        (0.6600, 0.2600),
        (0.6600, 0.2400),
        (0.6600, 0.2200),
        (0.6600, 0.2000),
        (0.6600, 0.1800),
        (0.6600, 0.1600),
        (0.6600, 0.1400),
        (0.6600, 0.1200),
        (0.6600, 0.1000),
        (0.6600, 0.0800),
        (0.6600, 0.0600),
        (0.6600, 0.0400),
        (0.6800, 0.2600),
        (0.6800, 0.2400),
        (0.6800, 0.2200),
        (0.6800, 0.2000),
        (0.6800, 0.1800),
        (0.6800, 0.1600),
        (0.6800, 0.1400),
        (0.6800, 0.1200),
        (0.6800, 0.1000),
        (0.6800, 0.0800),
        (0.6800, 0.0600),
        (0.6800, 0.0400),
        (0.7000, 0.2600),
        (0.7000, 0.2400),
        (0.7000, 0.2200),
        (0.7000, 0.2000),
        (0.7000, 0.1800),
        (0.7000, 0.1600),
        (0.7000, 0.1400),
        (0.7000, 0.1200),
        (0.7000, 0.1000),
        (0.7000, 0.0800),
        (0.7000, 0.0600),
        (0.7000, 0.0400),
        (0.7200, 0.2600),
        (0.7200, 0.2400),
        (0.7200, 0.2200),
        (0.7200, 0.2000),
        (0.7200, 0.1800),
        (0.7200, 0.1600),
        (0.7200, 0.1400),
        (0.7200, 0.1200),
        (0.7200, 0.1000),
        (0.7200, 0.0800),
        (0.7200, 0.0600),
        (0.7200, 0.0400),
        (0.7400, 0.5000),
        (0.7400, 0.4800),
        (0.7400, 0.4600),
        (0.7400, 0.4400),
        (0.7400, 0.4200),
        (0.7400, 0.4000),
        (0.7400, 0.3800),
        (0.7400, 0.3600),
        (0.7400, 0.3400),
        (0.7400, 0.3200),
        (0.7400, 0.3000),
        (0.7400, 0.2800),
        (0.7600, 0.5000),
        (0.7600, 0.4800),
        (0.7600, 0.4600),
        (0.7600, 0.4400),
        (0.7600, 0.4200),
        (0.7600, 0.4000),
        (0.7600, 0.3800),
        (0.7600, 0.3600),
        (0.7600, 0.3400),
        (0.7600, 0.3200),
        (0.7600, 0.3000),
        (0.7600, 0.2800),
        (0.7800, 0.5000),
        (0.7800, 0.4800),
        (0.7800, 0.4600),
        (0.7800, 0.4400),
        (0.7800, 0.4200),
        (0.7800, 0.4000),
        (0.7800, 0.3800),
        (0.7800, 0.3600),
        (0.7800, 0.3400),
        (0.7800, 0.3200),
        (0.7800, 0.3000),
        (0.7800, 0.2800),
        (0.8000, 0.5000),
        (0.8000, 0.4800),
        (0.8000, 0.4600),
        (0.8000, 0.4400),
        (0.8000, 0.4200),
        (0.8000, 0.4000),
        (0.8000, 0.3800),
        (0.8000, 0.3600),
        (0.8000, 0.3400),
        (0.8000, 0.3200),
        (0.8000, 0.3000),
        (0.8000, 0.2800),
        (0.8200, 0.5000),
        (0.8200, 0.4800),
        (0.8200, 0.4600),
        (0.8200, 0.4400),
        (0.8200, 0.4200),
        (0.8200, 0.4000),
        (0.8200, 0.3800),
        (0.8200, 0.3600),
        (0.8200, 0.3400),
        (0.8200, 0.3200),
        (0.8200, 0.3000),
        (0.8200, 0.2800),
        (0.8400, 0.5000),
        (0.8400, 0.4800),
        (0.8400, 0.4600),
        (0.8400, 0.4400),
        (0.8400, 0.4200),
        (0.8400, 0.4000),
        (0.8400, 0.3800),
        (0.8400, 0.3600),
        (0.8400, 0.3400),
        (0.8400, 0.3200),
        (0.8400, 0.3000),
        (0.8400, 0.2800),
        (0.8600, 0.5000),
        (0.8600, 0.4800),
        (0.8600, 0.4600),
        (0.8600, 0.4400),
        (0.8600, 0.4200),
        (0.8600, 0.4000),
        (0.8600, 0.3800),
        (0.8600, 0.3600),
        (0.8600, 0.3400),
        (0.8600, 0.3200),
        (0.8600, 0.3000),
        (0.8600, 0.2800),
        (0.8800, 0.5000),
        (0.8800, 0.4800),
        (0.8800, 0.4600),
        (0.8800, 0.4400),
        (0.8800, 0.4200),
        (0.8800, 0.4000),
        (0.8800, 0.3800),
        (0.8800, 0.3600),
        (0.8800, 0.3400),
        (0.8800, 0.3200),
        (0.8800, 0.3000),
        (0.8800, 0.2800),
        (0.9000, 0.5000),
        (0.9000, 0.4800),
        (0.9000, 0.4600),
        (0.9000, 0.4400),
        (0.9000, 0.4200),
        (0.9000, 0.4000),
        (0.9000, 0.3800),
        (0.9000, 0.3600),
        (0.9000, 0.3400),
        (0.9000, 0.3200),
        (0.9000, 0.3000),
        (0.9000, 0.2800),
        (0.9200, 0.5000),
        (0.9200, 0.4800),
        (0.9200, 0.4600),
        (0.9200, 0.4400),
        (0.9200, 0.4200),
        (0.9200, 0.4000),
        (0.9200, 0.3800),
        (0.9200, 0.3600),
        (0.9200, 0.3400),
        (0.9200, 0.3200),
        (0.9200, 0.3000),
        (0.9200, 0.2800),
        (0.9400, 0.5000),
        (0.9400, 0.4800),
        (0.9400, 0.4600),
        (0.9400, 0.4400),
        (0.9400, 0.4200),
        (0.9400, 0.4000),
        (0.9400, 0.3800),
        (0.9400, 0.3600),
        (0.9400, 0.3400),
        (0.9400, 0.3200),
        (0.9400, 0.3000),
        (0.9400, 0.2800),
        (0.9600, 0.5000),
        (0.9600, 0.4800),
        (0.9600, 0.4600),
        (0.9600, 0.4400),
        (0.9600, 0.4200),
        (0.9600, 0.4000),
        (0.9600, 0.3800),
        (0.9600, 0.3600),
        (0.9600, 0.3400),
        (0.9600, 0.3200),
        (0.9600, 0.3000),
        (0.9600, 0.2800)
    } {
        \draw[color=\accentcolor, fill=\accentcolor] \point circle (0.005);
    }

	% Domain.
    \draw[thick, color=\documentcolor]
        (0,0.02) -- (0.98,0.02) -- (0.98,1) -- (0,1) -- cycle;

\end{tikzpicture}
            \end{figure}
        }

        \only<5>{
            \begin{figure}[!ht]
                \centering
                \begin{tikzpicture}[scale=4.0, line join=round]

	% Domain, filled.
    \draw[thick, color=\documentcolor, fill=white]
        (-0.0081967213,0) -- (1,0) -- (1,1.0081967213) -- (-0.0081967213,1.0081967213) -- cycle;

    % Points.
    \foreach \point in {
        (0.0081967213, 0.9918032787),
		(0.0081967213, 0.9836065574),
		(0.0081967213, 0.9754098361),
		(0.0081967213, 0.9672131148),
		(0.0081967213, 0.9590163934),
		(0.0081967213, 0.9508196721),
		(0.0081967213, 0.9426229508),
		(0.0081967213, 0.9344262295),
		(0.0081967213, 0.9262295082),
		(0.0081967213, 0.9180327869),
		(0.0081967213, 0.9098360656),
		(0.0081967213, 0.9016393443),
		(0.0081967213, 0.7950819672),
		(0.0081967213, 0.7868852459),
		(0.0081967213, 0.7786885246),
		(0.0081967213, 0.7704918033),
		(0.0081967213, 0.7622950820),
		(0.0081967213, 0.7540983607),
		(0.0081967213, 0.7459016393),
		(0.0081967213, 0.7377049180),
		(0.0081967213, 0.7295081967),
		(0.0081967213, 0.7213114754),
		(0.0081967213, 0.7131147541),
		(0.0081967213, 0.7049180328),
		(0.0081967213, 0.6967213115),
		(0.0081967213, 0.6885245902),
		(0.0081967213, 0.6803278689),
		(0.0081967213, 0.6721311475),
		(0.0081967213, 0.6639344262),
		(0.0081967213, 0.6557377049),
		(0.0081967213, 0.5491803279),
		(0.0081967213, 0.5409836066),
		(0.0081967213, 0.5327868852),
		(0.0081967213, 0.5245901639),
		(0.0081967213, 0.5163934426),
		(0.0081967213, 0.5081967213),
		(0.0163934426, 0.9918032787),
		(0.0163934426, 0.9836065574),
		(0.0163934426, 0.9754098361),
		(0.0163934426, 0.9672131148),
		(0.0163934426, 0.9590163934),
		(0.0163934426, 0.9508196721),
		(0.0163934426, 0.9426229508),
		(0.0163934426, 0.9344262295),
		(0.0163934426, 0.9262295082),
		(0.0163934426, 0.9180327869),
		(0.0163934426, 0.9098360656),
		(0.0163934426, 0.9016393443),
		(0.0163934426, 0.7950819672),
		(0.0163934426, 0.7868852459),
		(0.0163934426, 0.7786885246),
		(0.0163934426, 0.7704918033),
		(0.0163934426, 0.7622950820),
		(0.0163934426, 0.7540983607),
		(0.0163934426, 0.7459016393),
		(0.0163934426, 0.7377049180),
		(0.0163934426, 0.7295081967),
		(0.0163934426, 0.7213114754),
		(0.0163934426, 0.7131147541),
		(0.0163934426, 0.7049180328),
		(0.0163934426, 0.6967213115),
		(0.0163934426, 0.6885245902),
		(0.0163934426, 0.6803278689),
		(0.0163934426, 0.6721311475),
		(0.0163934426, 0.6639344262),
		(0.0163934426, 0.6557377049),
		(0.0163934426, 0.5491803279),
		(0.0163934426, 0.5409836066),
		(0.0163934426, 0.5327868852),
		(0.0163934426, 0.5245901639),
		(0.0163934426, 0.5163934426),
		(0.0163934426, 0.5081967213),
		(0.0245901639, 0.9918032787),
		(0.0245901639, 0.9836065574),
		(0.0245901639, 0.9754098361),
		(0.0245901639, 0.9672131148),
		(0.0245901639, 0.9590163934),
		(0.0245901639, 0.9508196721),
		(0.0245901639, 0.9426229508),
		(0.0245901639, 0.9344262295),
		(0.0245901639, 0.9262295082),
		(0.0245901639, 0.9180327869),
		(0.0245901639, 0.9098360656),
		(0.0245901639, 0.9016393443),
		(0.0245901639, 0.7950819672),
		(0.0245901639, 0.7868852459),
		(0.0245901639, 0.7786885246),
		(0.0245901639, 0.7704918033),
		(0.0245901639, 0.7622950820),
		(0.0245901639, 0.7540983607),
		(0.0245901639, 0.7459016393),
		(0.0245901639, 0.7377049180),
		(0.0245901639, 0.7295081967),
		(0.0245901639, 0.7213114754),
		(0.0245901639, 0.7131147541),
		(0.0245901639, 0.7049180328),
		(0.0245901639, 0.6967213115),
		(0.0245901639, 0.6885245902),
		(0.0245901639, 0.6803278689),
		(0.0245901639, 0.6721311475),
		(0.0245901639, 0.6639344262),
		(0.0245901639, 0.6557377049),
		(0.0245901639, 0.5491803279),
		(0.0245901639, 0.5409836066),
		(0.0245901639, 0.5327868852),
		(0.0245901639, 0.5245901639),
		(0.0245901639, 0.5163934426),
		(0.0245901639, 0.5081967213),
		(0.0327868852, 0.9918032787),
		(0.0327868852, 0.9836065574),
		(0.0327868852, 0.9754098361),
		(0.0327868852, 0.9672131148),
		(0.0327868852, 0.9590163934),
		(0.0327868852, 0.9508196721),
		(0.0327868852, 0.9426229508),
		(0.0327868852, 0.9344262295),
		(0.0327868852, 0.9262295082),
		(0.0327868852, 0.9180327869),
		(0.0327868852, 0.9098360656),
		(0.0327868852, 0.9016393443),
		(0.0327868852, 0.7950819672),
		(0.0327868852, 0.7868852459),
		(0.0327868852, 0.7786885246),
		(0.0327868852, 0.7704918033),
		(0.0327868852, 0.7622950820),
		(0.0327868852, 0.7540983607),
		(0.0327868852, 0.7459016393),
		(0.0327868852, 0.7377049180),
		(0.0327868852, 0.7295081967),
		(0.0327868852, 0.7213114754),
		(0.0327868852, 0.7131147541),
		(0.0327868852, 0.7049180328),
		(0.0327868852, 0.6967213115),
		(0.0327868852, 0.6885245902),
		(0.0327868852, 0.6803278689),
		(0.0327868852, 0.6721311475),
		(0.0327868852, 0.6639344262),
		(0.0327868852, 0.6557377049),
		(0.0327868852, 0.5491803279),
		(0.0327868852, 0.5409836066),
		(0.0327868852, 0.5327868852),
		(0.0327868852, 0.5245901639),
		(0.0327868852, 0.5163934426),
		(0.0327868852, 0.5081967213),
		(0.0409836066, 0.9918032787),
		(0.0409836066, 0.9836065574),
		(0.0409836066, 0.9754098361),
		(0.0409836066, 0.9672131148),
		(0.0409836066, 0.9590163934),
		(0.0409836066, 0.9508196721),
		(0.0409836066, 0.9426229508),
		(0.0409836066, 0.9344262295),
		(0.0409836066, 0.9262295082),
		(0.0409836066, 0.9180327869),
		(0.0409836066, 0.9098360656),
		(0.0409836066, 0.9016393443),
		(0.0409836066, 0.7950819672),
		(0.0409836066, 0.7868852459),
		(0.0409836066, 0.7786885246),
		(0.0409836066, 0.7704918033),
		(0.0409836066, 0.7622950820),
		(0.0409836066, 0.7540983607),
		(0.0409836066, 0.7459016393),
		(0.0409836066, 0.7377049180),
		(0.0409836066, 0.7295081967),
		(0.0409836066, 0.7213114754),
		(0.0409836066, 0.7131147541),
		(0.0409836066, 0.7049180328),
		(0.0409836066, 0.6967213115),
		(0.0409836066, 0.6885245902),
		(0.0409836066, 0.6803278689),
		(0.0409836066, 0.6721311475),
		(0.0409836066, 0.6639344262),
		(0.0409836066, 0.6557377049),
		(0.0409836066, 0.5491803279),
		(0.0409836066, 0.5409836066),
		(0.0409836066, 0.5327868852),
		(0.0409836066, 0.5245901639),
		(0.0409836066, 0.5163934426),
		(0.0409836066, 0.5081967213),
		(0.0491803279, 0.9918032787),
		(0.0491803279, 0.9836065574),
		(0.0491803279, 0.9754098361),
		(0.0491803279, 0.9672131148),
		(0.0491803279, 0.9590163934),
		(0.0491803279, 0.9508196721),
		(0.0491803279, 0.9426229508),
		(0.0491803279, 0.9344262295),
		(0.0491803279, 0.9262295082),
		(0.0491803279, 0.9180327869),
		(0.0491803279, 0.9098360656),
		(0.0491803279, 0.9016393443),
		(0.0491803279, 0.7950819672),
		(0.0491803279, 0.7868852459),
		(0.0491803279, 0.7786885246),
		(0.0491803279, 0.7704918033),
		(0.0491803279, 0.7622950820),
		(0.0491803279, 0.7540983607),
		(0.0491803279, 0.7459016393),
		(0.0491803279, 0.7377049180),
		(0.0491803279, 0.7295081967),
		(0.0491803279, 0.7213114754),
		(0.0491803279, 0.7131147541),
		(0.0491803279, 0.7049180328),
		(0.0491803279, 0.6967213115),
		(0.0491803279, 0.6885245902),
		(0.0491803279, 0.6803278689),
		(0.0491803279, 0.6721311475),
		(0.0491803279, 0.6639344262),
		(0.0491803279, 0.6557377049),
		(0.0491803279, 0.5491803279),
		(0.0491803279, 0.5409836066),
		(0.0491803279, 0.5327868852),
		(0.0491803279, 0.5245901639),
		(0.0491803279, 0.5163934426),
		(0.0491803279, 0.5081967213),
		(0.0573770492, 0.9918032787),
		(0.0573770492, 0.9836065574),
		(0.0573770492, 0.9754098361),
		(0.0573770492, 0.9672131148),
		(0.0573770492, 0.9590163934),
		(0.0573770492, 0.9508196721),
		(0.0573770492, 0.9426229508),
		(0.0573770492, 0.9344262295),
		(0.0573770492, 0.9262295082),
		(0.0573770492, 0.9180327869),
		(0.0573770492, 0.9098360656),
		(0.0573770492, 0.9016393443),
		(0.0573770492, 0.8442622951),
		(0.0573770492, 0.8360655738),
		(0.0573770492, 0.8278688525),
		(0.0573770492, 0.8196721311),
		(0.0573770492, 0.8114754098),
		(0.0573770492, 0.8032786885),
		(0.0573770492, 0.7459016393),
		(0.0573770492, 0.7377049180),
		(0.0573770492, 0.7295081967),
		(0.0573770492, 0.7213114754),
		(0.0573770492, 0.7131147541),
		(0.0573770492, 0.7049180328),
		(0.0573770492, 0.6967213115),
		(0.0573770492, 0.6885245902),
		(0.0573770492, 0.6803278689),
		(0.0573770492, 0.6721311475),
		(0.0573770492, 0.6639344262),
		(0.0573770492, 0.6557377049),
		(0.0655737705, 0.9918032787),
		(0.0655737705, 0.9836065574),
		(0.0655737705, 0.9754098361),
		(0.0655737705, 0.9672131148),
		(0.0655737705, 0.9590163934),
		(0.0655737705, 0.9508196721),
		(0.0655737705, 0.9426229508),
		(0.0655737705, 0.9344262295),
		(0.0655737705, 0.9262295082),
		(0.0655737705, 0.9180327869),
		(0.0655737705, 0.9098360656),
		(0.0655737705, 0.9016393443),
		(0.0655737705, 0.8442622951),
		(0.0655737705, 0.8360655738),
		(0.0655737705, 0.8278688525),
		(0.0655737705, 0.8196721311),
		(0.0655737705, 0.8114754098),
		(0.0655737705, 0.8032786885),
		(0.0655737705, 0.7459016393),
		(0.0655737705, 0.7377049180),
		(0.0655737705, 0.7295081967),
		(0.0655737705, 0.7213114754),
		(0.0655737705, 0.7131147541),
		(0.0655737705, 0.7049180328),
		(0.0655737705, 0.6967213115),
		(0.0655737705, 0.6885245902),
		(0.0655737705, 0.6803278689),
		(0.0655737705, 0.6721311475),
		(0.0655737705, 0.6639344262),
		(0.0655737705, 0.6557377049),
		(0.0737704918, 0.9918032787),
		(0.0737704918, 0.9836065574),
		(0.0737704918, 0.9754098361),
		(0.0737704918, 0.9672131148),
		(0.0737704918, 0.9590163934),
		(0.0737704918, 0.9508196721),
		(0.0737704918, 0.9426229508),
		(0.0737704918, 0.9344262295),
		(0.0737704918, 0.9262295082),
		(0.0737704918, 0.9180327869),
		(0.0737704918, 0.9098360656),
		(0.0737704918, 0.9016393443),
		(0.0737704918, 0.8442622951),
		(0.0737704918, 0.8360655738),
		(0.0737704918, 0.8278688525),
		(0.0737704918, 0.8196721311),
		(0.0737704918, 0.8114754098),
		(0.0737704918, 0.8032786885),
		(0.0737704918, 0.7459016393),
		(0.0737704918, 0.7377049180),
		(0.0737704918, 0.7295081967),
		(0.0737704918, 0.7213114754),
		(0.0737704918, 0.7131147541),
		(0.0737704918, 0.7049180328),
		(0.0737704918, 0.6967213115),
		(0.0737704918, 0.6885245902),
		(0.0737704918, 0.6803278689),
		(0.0737704918, 0.6721311475),
		(0.0737704918, 0.6639344262),
		(0.0737704918, 0.6557377049),
		(0.0819672131, 0.9918032787),
		(0.0819672131, 0.9836065574),
		(0.0819672131, 0.9754098361),
		(0.0819672131, 0.9672131148),
		(0.0819672131, 0.9590163934),
		(0.0819672131, 0.9508196721),
		(0.0819672131, 0.9426229508),
		(0.0819672131, 0.9344262295),
		(0.0819672131, 0.9262295082),
		(0.0819672131, 0.9180327869),
		(0.0819672131, 0.9098360656),
		(0.0819672131, 0.9016393443),
		(0.0819672131, 0.8442622951),
		(0.0819672131, 0.8360655738),
		(0.0819672131, 0.8278688525),
		(0.0819672131, 0.8196721311),
		(0.0819672131, 0.8114754098),
		(0.0819672131, 0.8032786885),
		(0.0819672131, 0.7459016393),
		(0.0819672131, 0.7377049180),
		(0.0819672131, 0.7295081967),
		(0.0819672131, 0.7213114754),
		(0.0819672131, 0.7131147541),
		(0.0819672131, 0.7049180328),
		(0.0819672131, 0.6967213115),
		(0.0819672131, 0.6885245902),
		(0.0819672131, 0.6803278689),
		(0.0819672131, 0.6721311475),
		(0.0819672131, 0.6639344262),
		(0.0819672131, 0.6557377049),
		(0.0901639344, 0.9918032787),
		(0.0901639344, 0.9836065574),
		(0.0901639344, 0.9754098361),
		(0.0901639344, 0.9672131148),
		(0.0901639344, 0.9590163934),
		(0.0901639344, 0.9508196721),
		(0.0901639344, 0.9426229508),
		(0.0901639344, 0.9344262295),
		(0.0901639344, 0.9262295082),
		(0.0901639344, 0.9180327869),
		(0.0901639344, 0.9098360656),
		(0.0901639344, 0.9016393443),
		(0.0901639344, 0.8442622951),
		(0.0901639344, 0.8360655738),
		(0.0901639344, 0.8278688525),
		(0.0901639344, 0.8196721311),
		(0.0901639344, 0.8114754098),
		(0.0901639344, 0.8032786885),
		(0.0901639344, 0.7459016393),
		(0.0901639344, 0.7377049180),
		(0.0901639344, 0.7295081967),
		(0.0901639344, 0.7213114754),
		(0.0901639344, 0.7131147541),
		(0.0901639344, 0.7049180328),
		(0.0901639344, 0.6967213115),
		(0.0901639344, 0.6885245902),
		(0.0901639344, 0.6803278689),
		(0.0901639344, 0.6721311475),
		(0.0901639344, 0.6639344262),
		(0.0901639344, 0.6557377049),
		(0.0983606557, 0.9918032787),
		(0.0983606557, 0.9836065574),
		(0.0983606557, 0.9754098361),
		(0.0983606557, 0.9672131148),
		(0.0983606557, 0.9590163934),
		(0.0983606557, 0.9508196721),
		(0.0983606557, 0.9426229508),
		(0.0983606557, 0.9344262295),
		(0.0983606557, 0.9262295082),
		(0.0983606557, 0.9180327869),
		(0.0983606557, 0.9098360656),
		(0.0983606557, 0.9016393443),
		(0.0983606557, 0.8442622951),
		(0.0983606557, 0.8360655738),
		(0.0983606557, 0.8278688525),
		(0.0983606557, 0.8196721311),
		(0.0983606557, 0.8114754098),
		(0.0983606557, 0.8032786885),
		(0.0983606557, 0.7459016393),
		(0.0983606557, 0.7377049180),
		(0.0983606557, 0.7295081967),
		(0.0983606557, 0.7213114754),
		(0.0983606557, 0.7131147541),
		(0.0983606557, 0.7049180328),
		(0.0983606557, 0.6967213115),
		(0.0983606557, 0.6885245902),
		(0.0983606557, 0.6803278689),
		(0.0983606557, 0.6721311475),
		(0.0983606557, 0.6639344262),
		(0.0983606557, 0.6557377049),
		(0.1065573770, 0.8934426230),
		(0.1065573770, 0.8852459016),
		(0.1065573770, 0.8770491803),
		(0.1065573770, 0.8688524590),
		(0.1065573770, 0.8606557377),
		(0.1065573770, 0.8524590164),
		(0.1065573770, 0.7950819672),
		(0.1065573770, 0.7868852459),
		(0.1065573770, 0.7786885246),
		(0.1065573770, 0.7704918033),
		(0.1065573770, 0.7622950820),
		(0.1065573770, 0.7540983607),
		(0.1065573770, 0.5491803279),
		(0.1065573770, 0.5409836066),
		(0.1065573770, 0.5327868852),
		(0.1065573770, 0.5245901639),
		(0.1065573770, 0.5163934426),
		(0.1065573770, 0.5081967213),
		(0.1147540984, 0.8934426230),
		(0.1147540984, 0.8852459016),
		(0.1147540984, 0.8770491803),
		(0.1147540984, 0.8688524590),
		(0.1147540984, 0.8606557377),
		(0.1147540984, 0.8524590164),
		(0.1147540984, 0.7950819672),
		(0.1147540984, 0.7868852459),
		(0.1147540984, 0.7786885246),
		(0.1147540984, 0.7704918033),
		(0.1147540984, 0.7622950820),
		(0.1147540984, 0.7540983607),
		(0.1147540984, 0.5491803279),
		(0.1147540984, 0.5409836066),
		(0.1147540984, 0.5327868852),
		(0.1147540984, 0.5245901639),
		(0.1147540984, 0.5163934426),
		(0.1147540984, 0.5081967213),
		(0.1229508197, 0.8934426230),
		(0.1229508197, 0.8852459016),
		(0.1229508197, 0.8770491803),
		(0.1229508197, 0.8688524590),
		(0.1229508197, 0.8606557377),
		(0.1229508197, 0.8524590164),
		(0.1229508197, 0.7950819672),
		(0.1229508197, 0.7868852459),
		(0.1229508197, 0.7786885246),
		(0.1229508197, 0.7704918033),
		(0.1229508197, 0.7622950820),
		(0.1229508197, 0.7540983607),
		(0.1229508197, 0.5491803279),
		(0.1229508197, 0.5409836066),
		(0.1229508197, 0.5327868852),
		(0.1229508197, 0.5245901639),
		(0.1229508197, 0.5163934426),
		(0.1229508197, 0.5081967213),
		(0.1311475410, 0.8934426230),
		(0.1311475410, 0.8852459016),
		(0.1311475410, 0.8770491803),
		(0.1311475410, 0.8688524590),
		(0.1311475410, 0.8606557377),
		(0.1311475410, 0.8524590164),
		(0.1311475410, 0.7950819672),
		(0.1311475410, 0.7868852459),
		(0.1311475410, 0.7786885246),
		(0.1311475410, 0.7704918033),
		(0.1311475410, 0.7622950820),
		(0.1311475410, 0.7540983607),
		(0.1311475410, 0.5491803279),
		(0.1311475410, 0.5409836066),
		(0.1311475410, 0.5327868852),
		(0.1311475410, 0.5245901639),
		(0.1311475410, 0.5163934426),
		(0.1311475410, 0.5081967213),
		(0.1393442623, 0.8934426230),
		(0.1393442623, 0.8852459016),
		(0.1393442623, 0.8770491803),
		(0.1393442623, 0.8688524590),
		(0.1393442623, 0.8606557377),
		(0.1393442623, 0.8524590164),
		(0.1393442623, 0.7950819672),
		(0.1393442623, 0.7868852459),
		(0.1393442623, 0.7786885246),
		(0.1393442623, 0.7704918033),
		(0.1393442623, 0.7622950820),
		(0.1393442623, 0.7540983607),
		(0.1393442623, 0.5491803279),
		(0.1393442623, 0.5409836066),
		(0.1393442623, 0.5327868852),
		(0.1393442623, 0.5245901639),
		(0.1393442623, 0.5163934426),
		(0.1393442623, 0.5081967213),
		(0.1475409836, 0.8934426230),
		(0.1475409836, 0.8852459016),
		(0.1475409836, 0.8770491803),
		(0.1475409836, 0.8688524590),
		(0.1475409836, 0.8606557377),
		(0.1475409836, 0.8524590164),
		(0.1475409836, 0.7950819672),
		(0.1475409836, 0.7868852459),
		(0.1475409836, 0.7786885246),
		(0.1475409836, 0.7704918033),
		(0.1475409836, 0.7622950820),
		(0.1475409836, 0.7540983607),
		(0.1475409836, 0.5491803279),
		(0.1475409836, 0.5409836066),
		(0.1475409836, 0.5327868852),
		(0.1475409836, 0.5245901639),
		(0.1475409836, 0.5163934426),
		(0.1475409836, 0.5081967213),
		(0.1557377049, 0.9426229508),
		(0.1557377049, 0.9344262295),
		(0.1557377049, 0.9262295082),
		(0.1557377049, 0.9180327869),
		(0.1557377049, 0.9098360656),
		(0.1557377049, 0.9016393443),
		(0.1557377049, 0.8442622951),
		(0.1557377049, 0.8360655738),
		(0.1557377049, 0.8278688525),
		(0.1557377049, 0.8196721311),
		(0.1557377049, 0.8114754098),
		(0.1557377049, 0.8032786885),
		(0.1557377049, 0.6967213115),
		(0.1557377049, 0.6885245902),
		(0.1557377049, 0.6803278689),
		(0.1557377049, 0.6721311475),
		(0.1557377049, 0.6639344262),
		(0.1557377049, 0.6557377049),
		(0.1639344262, 0.9426229508),
		(0.1639344262, 0.9344262295),
		(0.1639344262, 0.9262295082),
		(0.1639344262, 0.9180327869),
		(0.1639344262, 0.9098360656),
		(0.1639344262, 0.9016393443),
		(0.1639344262, 0.8442622951),
		(0.1639344262, 0.8360655738),
		(0.1639344262, 0.8278688525),
		(0.1639344262, 0.8196721311),
		(0.1639344262, 0.8114754098),
		(0.1639344262, 0.8032786885),
		(0.1639344262, 0.6967213115),
		(0.1639344262, 0.6885245902),
		(0.1639344262, 0.6803278689),
		(0.1639344262, 0.6721311475),
		(0.1639344262, 0.6639344262),
		(0.1639344262, 0.6557377049),
		(0.1721311475, 0.9426229508),
		(0.1721311475, 0.9344262295),
		(0.1721311475, 0.9262295082),
		(0.1721311475, 0.9180327869),
		(0.1721311475, 0.9098360656),
		(0.1721311475, 0.9016393443),
		(0.1721311475, 0.8442622951),
		(0.1721311475, 0.8360655738),
		(0.1721311475, 0.8278688525),
		(0.1721311475, 0.8196721311),
		(0.1721311475, 0.8114754098),
		(0.1721311475, 0.8032786885),
		(0.1721311475, 0.6967213115),
		(0.1721311475, 0.6885245902),
		(0.1721311475, 0.6803278689),
		(0.1721311475, 0.6721311475),
		(0.1721311475, 0.6639344262),
		(0.1721311475, 0.6557377049),
		(0.1803278689, 0.9426229508),
		(0.1803278689, 0.9344262295),
		(0.1803278689, 0.9262295082),
		(0.1803278689, 0.9180327869),
		(0.1803278689, 0.9098360656),
		(0.1803278689, 0.9016393443),
		(0.1803278689, 0.8442622951),
		(0.1803278689, 0.8360655738),
		(0.1803278689, 0.8278688525),
		(0.1803278689, 0.8196721311),
		(0.1803278689, 0.8114754098),
		(0.1803278689, 0.8032786885),
		(0.1803278689, 0.6967213115),
		(0.1803278689, 0.6885245902),
		(0.1803278689, 0.6803278689),
		(0.1803278689, 0.6721311475),
		(0.1803278689, 0.6639344262),
		(0.1803278689, 0.6557377049),
		(0.1885245902, 0.9426229508),
		(0.1885245902, 0.9344262295),
		(0.1885245902, 0.9262295082),
		(0.1885245902, 0.9180327869),
		(0.1885245902, 0.9098360656),
		(0.1885245902, 0.9016393443),
		(0.1885245902, 0.8442622951),
		(0.1885245902, 0.8360655738),
		(0.1885245902, 0.8278688525),
		(0.1885245902, 0.8196721311),
		(0.1885245902, 0.8114754098),
		(0.1885245902, 0.8032786885),
		(0.1885245902, 0.6967213115),
		(0.1885245902, 0.6885245902),
		(0.1885245902, 0.6803278689),
		(0.1885245902, 0.6721311475),
		(0.1885245902, 0.6639344262),
		(0.1885245902, 0.6557377049),
		(0.1967213115, 0.9426229508),
		(0.1967213115, 0.9344262295),
		(0.1967213115, 0.9262295082),
		(0.1967213115, 0.9180327869),
		(0.1967213115, 0.9098360656),
		(0.1967213115, 0.9016393443),
		(0.1967213115, 0.8442622951),
		(0.1967213115, 0.8360655738),
		(0.1967213115, 0.8278688525),
		(0.1967213115, 0.8196721311),
		(0.1967213115, 0.8114754098),
		(0.1967213115, 0.8032786885),
		(0.1967213115, 0.6967213115),
		(0.1967213115, 0.6885245902),
		(0.1967213115, 0.6803278689),
		(0.1967213115, 0.6721311475),
		(0.1967213115, 0.6639344262),
		(0.1967213115, 0.6557377049),
		(0.2049180328, 0.9918032787),
		(0.2049180328, 0.9836065574),
		(0.2049180328, 0.9754098361),
		(0.2049180328, 0.9672131148),
		(0.2049180328, 0.9590163934),
		(0.2049180328, 0.9508196721),
		(0.2049180328, 0.8934426230),
		(0.2049180328, 0.8852459016),
		(0.2049180328, 0.8770491803),
		(0.2049180328, 0.8688524590),
		(0.2049180328, 0.8606557377),
		(0.2049180328, 0.8524590164),
		(0.2049180328, 0.7950819672),
		(0.2049180328, 0.7868852459),
		(0.2049180328, 0.7786885246),
		(0.2049180328, 0.7704918033),
		(0.2049180328, 0.7622950820),
		(0.2049180328, 0.7540983607),
		(0.2049180328, 0.7459016393),
		(0.2049180328, 0.7377049180),
		(0.2049180328, 0.7295081967),
		(0.2049180328, 0.7213114754),
		(0.2049180328, 0.7131147541),
		(0.2049180328, 0.7049180328),
		(0.2049180328, 0.6475409836),
		(0.2049180328, 0.6393442623),
		(0.2049180328, 0.6311475410),
		(0.2049180328, 0.6229508197),
		(0.2049180328, 0.6147540984),
		(0.2049180328, 0.6065573770),
		(0.2131147541, 0.9918032787),
		(0.2131147541, 0.9836065574),
		(0.2131147541, 0.9754098361),
		(0.2131147541, 0.9672131148),
		(0.2131147541, 0.9590163934),
		(0.2131147541, 0.9508196721),
		(0.2131147541, 0.8934426230),
		(0.2131147541, 0.8852459016),
		(0.2131147541, 0.8770491803),
		(0.2131147541, 0.8688524590),
		(0.2131147541, 0.8606557377),
		(0.2131147541, 0.8524590164),
		(0.2131147541, 0.7950819672),
		(0.2131147541, 0.7868852459),
		(0.2131147541, 0.7786885246),
		(0.2131147541, 0.7704918033),
		(0.2131147541, 0.7622950820),
		(0.2131147541, 0.7540983607),
		(0.2131147541, 0.7459016393),
		(0.2131147541, 0.7377049180),
		(0.2131147541, 0.7295081967),
		(0.2131147541, 0.7213114754),
		(0.2131147541, 0.7131147541),
		(0.2131147541, 0.7049180328),
		(0.2131147541, 0.6475409836),
		(0.2131147541, 0.6393442623),
		(0.2131147541, 0.6311475410),
		(0.2131147541, 0.6229508197),
		(0.2131147541, 0.6147540984),
		(0.2131147541, 0.6065573770),
		(0.2213114754, 0.9918032787),
		(0.2213114754, 0.9836065574),
		(0.2213114754, 0.9754098361),
		(0.2213114754, 0.9672131148),
		(0.2213114754, 0.9590163934),
		(0.2213114754, 0.9508196721),
		(0.2213114754, 0.8934426230),
		(0.2213114754, 0.8852459016),
		(0.2213114754, 0.8770491803),
		(0.2213114754, 0.8688524590),
		(0.2213114754, 0.8606557377),
		(0.2213114754, 0.8524590164),
		(0.2213114754, 0.7950819672),
		(0.2213114754, 0.7868852459),
		(0.2213114754, 0.7786885246),
		(0.2213114754, 0.7704918033),
		(0.2213114754, 0.7622950820),
		(0.2213114754, 0.7540983607),
		(0.2213114754, 0.7459016393),
		(0.2213114754, 0.7377049180),
		(0.2213114754, 0.7295081967),
		(0.2213114754, 0.7213114754),
		(0.2213114754, 0.7131147541),
		(0.2213114754, 0.7049180328),
		(0.2213114754, 0.6475409836),
		(0.2213114754, 0.6393442623),
		(0.2213114754, 0.6311475410),
		(0.2213114754, 0.6229508197),
		(0.2213114754, 0.6147540984),
		(0.2213114754, 0.6065573770),
		(0.2295081967, 0.9918032787),
		(0.2295081967, 0.9836065574),
		(0.2295081967, 0.9754098361),
		(0.2295081967, 0.9672131148),
		(0.2295081967, 0.9590163934),
		(0.2295081967, 0.9508196721),
		(0.2295081967, 0.8934426230),
		(0.2295081967, 0.8852459016),
		(0.2295081967, 0.8770491803),
		(0.2295081967, 0.8688524590),
		(0.2295081967, 0.8606557377),
		(0.2295081967, 0.8524590164),
		(0.2295081967, 0.7950819672),
		(0.2295081967, 0.7868852459),
		(0.2295081967, 0.7786885246),
		(0.2295081967, 0.7704918033),
		(0.2295081967, 0.7622950820),
		(0.2295081967, 0.7540983607),
		(0.2295081967, 0.7459016393),
		(0.2295081967, 0.7377049180),
		(0.2295081967, 0.7295081967),
		(0.2295081967, 0.7213114754),
		(0.2295081967, 0.7131147541),
		(0.2295081967, 0.7049180328),
		(0.2295081967, 0.6475409836),
		(0.2295081967, 0.6393442623),
		(0.2295081967, 0.6311475410),
		(0.2295081967, 0.6229508197),
		(0.2295081967, 0.6147540984),
		(0.2295081967, 0.6065573770),
		(0.2377049180, 0.9918032787),
		(0.2377049180, 0.9836065574),
		(0.2377049180, 0.9754098361),
		(0.2377049180, 0.9672131148),
		(0.2377049180, 0.9590163934),
		(0.2377049180, 0.9508196721),
		(0.2377049180, 0.8934426230),
		(0.2377049180, 0.8852459016),
		(0.2377049180, 0.8770491803),
		(0.2377049180, 0.8688524590),
		(0.2377049180, 0.8606557377),
		(0.2377049180, 0.8524590164),
		(0.2377049180, 0.7950819672),
		(0.2377049180, 0.7868852459),
		(0.2377049180, 0.7786885246),
		(0.2377049180, 0.7704918033),
		(0.2377049180, 0.7622950820),
		(0.2377049180, 0.7540983607),
		(0.2377049180, 0.7459016393),
		(0.2377049180, 0.7377049180),
		(0.2377049180, 0.7295081967),
		(0.2377049180, 0.7213114754),
		(0.2377049180, 0.7131147541),
		(0.2377049180, 0.7049180328),
		(0.2377049180, 0.6475409836),
		(0.2377049180, 0.6393442623),
		(0.2377049180, 0.6311475410),
		(0.2377049180, 0.6229508197),
		(0.2377049180, 0.6147540984),
		(0.2377049180, 0.6065573770),
		(0.2459016393, 0.9918032787),
		(0.2459016393, 0.9836065574),
		(0.2459016393, 0.9754098361),
		(0.2459016393, 0.9672131148),
		(0.2459016393, 0.9590163934),
		(0.2459016393, 0.9508196721),
		(0.2459016393, 0.8934426230),
		(0.2459016393, 0.8852459016),
		(0.2459016393, 0.8770491803),
		(0.2459016393, 0.8688524590),
		(0.2459016393, 0.8606557377),
		(0.2459016393, 0.8524590164),
		(0.2459016393, 0.7950819672),
		(0.2459016393, 0.7868852459),
		(0.2459016393, 0.7786885246),
		(0.2459016393, 0.7704918033),
		(0.2459016393, 0.7622950820),
		(0.2459016393, 0.7540983607),
		(0.2459016393, 0.7459016393),
		(0.2459016393, 0.7377049180),
		(0.2459016393, 0.7295081967),
		(0.2459016393, 0.7213114754),
		(0.2459016393, 0.7131147541),
		(0.2459016393, 0.7049180328),
		(0.2459016393, 0.6475409836),
		(0.2459016393, 0.6393442623),
		(0.2459016393, 0.6311475410),
		(0.2459016393, 0.6229508197),
		(0.2459016393, 0.6147540984),
		(0.2459016393, 0.6065573770),
		(0.2540983607, 0.9918032787),
		(0.2540983607, 0.9836065574),
		(0.2540983607, 0.9754098361),
		(0.2540983607, 0.9672131148),
		(0.2540983607, 0.9590163934),
		(0.2540983607, 0.9508196721),
		(0.2540983607, 0.9426229508),
		(0.2540983607, 0.9344262295),
		(0.2540983607, 0.9262295082),
		(0.2540983607, 0.9180327869),
		(0.2540983607, 0.9098360656),
		(0.2540983607, 0.9016393443),
		(0.2540983607, 0.7950819672),
		(0.2540983607, 0.7868852459),
		(0.2540983607, 0.7786885246),
		(0.2540983607, 0.7704918033),
		(0.2540983607, 0.7622950820),
		(0.2540983607, 0.7540983607),
		(0.2540983607, 0.7459016393),
		(0.2540983607, 0.7377049180),
		(0.2540983607, 0.7295081967),
		(0.2540983607, 0.7213114754),
		(0.2540983607, 0.7131147541),
		(0.2540983607, 0.7049180328),
		(0.2540983607, 0.6475409836),
		(0.2540983607, 0.6393442623),
		(0.2540983607, 0.6311475410),
		(0.2540983607, 0.6229508197),
		(0.2540983607, 0.6147540984),
		(0.2540983607, 0.6065573770),
		(0.2622950820, 0.9918032787),
		(0.2622950820, 0.9836065574),
		(0.2622950820, 0.9754098361),
		(0.2622950820, 0.9672131148),
		(0.2622950820, 0.9590163934),
		(0.2622950820, 0.9508196721),
		(0.2622950820, 0.9426229508),
		(0.2622950820, 0.9344262295),
		(0.2622950820, 0.9262295082),
		(0.2622950820, 0.9180327869),
		(0.2622950820, 0.9098360656),
		(0.2622950820, 0.9016393443),
		(0.2622950820, 0.7950819672),
		(0.2622950820, 0.7868852459),
		(0.2622950820, 0.7786885246),
		(0.2622950820, 0.7704918033),
		(0.2622950820, 0.7622950820),
		(0.2622950820, 0.7540983607),
		(0.2622950820, 0.7459016393),
		(0.2622950820, 0.7377049180),
		(0.2622950820, 0.7295081967),
		(0.2622950820, 0.7213114754),
		(0.2622950820, 0.7131147541),
		(0.2622950820, 0.7049180328),
		(0.2622950820, 0.6475409836),
		(0.2622950820, 0.6393442623),
		(0.2622950820, 0.6311475410),
		(0.2622950820, 0.6229508197),
		(0.2622950820, 0.6147540984),
		(0.2622950820, 0.6065573770),
		(0.2704918033, 0.9918032787),
		(0.2704918033, 0.9836065574),
		(0.2704918033, 0.9754098361),
		(0.2704918033, 0.9672131148),
		(0.2704918033, 0.9590163934),
		(0.2704918033, 0.9508196721),
		(0.2704918033, 0.9426229508),
		(0.2704918033, 0.9344262295),
		(0.2704918033, 0.9262295082),
		(0.2704918033, 0.9180327869),
		(0.2704918033, 0.9098360656),
		(0.2704918033, 0.9016393443),
		(0.2704918033, 0.7950819672),
		(0.2704918033, 0.7868852459),
		(0.2704918033, 0.7786885246),
		(0.2704918033, 0.7704918033),
		(0.2704918033, 0.7622950820),
		(0.2704918033, 0.7540983607),
		(0.2704918033, 0.7459016393),
		(0.2704918033, 0.7377049180),
		(0.2704918033, 0.7295081967),
		(0.2704918033, 0.7213114754),
		(0.2704918033, 0.7131147541),
		(0.2704918033, 0.7049180328),
		(0.2704918033, 0.6475409836),
		(0.2704918033, 0.6393442623),
		(0.2704918033, 0.6311475410),
		(0.2704918033, 0.6229508197),
		(0.2704918033, 0.6147540984),
		(0.2704918033, 0.6065573770),
		(0.2786885246, 0.9918032787),
		(0.2786885246, 0.9836065574),
		(0.2786885246, 0.9754098361),
		(0.2786885246, 0.9672131148),
		(0.2786885246, 0.9590163934),
		(0.2786885246, 0.9508196721),
		(0.2786885246, 0.9426229508),
		(0.2786885246, 0.9344262295),
		(0.2786885246, 0.9262295082),
		(0.2786885246, 0.9180327869),
		(0.2786885246, 0.9098360656),
		(0.2786885246, 0.9016393443),
		(0.2786885246, 0.7950819672),
		(0.2786885246, 0.7868852459),
		(0.2786885246, 0.7786885246),
		(0.2786885246, 0.7704918033),
		(0.2786885246, 0.7622950820),
		(0.2786885246, 0.7540983607),
		(0.2786885246, 0.7459016393),
		(0.2786885246, 0.7377049180),
		(0.2786885246, 0.7295081967),
		(0.2786885246, 0.7213114754),
		(0.2786885246, 0.7131147541),
		(0.2786885246, 0.7049180328),
		(0.2786885246, 0.6475409836),
		(0.2786885246, 0.6393442623),
		(0.2786885246, 0.6311475410),
		(0.2786885246, 0.6229508197),
		(0.2786885246, 0.6147540984),
		(0.2786885246, 0.6065573770),
		(0.2868852459, 0.9918032787),
		(0.2868852459, 0.9836065574),
		(0.2868852459, 0.9754098361),
		(0.2868852459, 0.9672131148),
		(0.2868852459, 0.9590163934),
		(0.2868852459, 0.9508196721),
		(0.2868852459, 0.9426229508),
		(0.2868852459, 0.9344262295),
		(0.2868852459, 0.9262295082),
		(0.2868852459, 0.9180327869),
		(0.2868852459, 0.9098360656),
		(0.2868852459, 0.9016393443),
		(0.2868852459, 0.7950819672),
		(0.2868852459, 0.7868852459),
		(0.2868852459, 0.7786885246),
		(0.2868852459, 0.7704918033),
		(0.2868852459, 0.7622950820),
		(0.2868852459, 0.7540983607),
		(0.2868852459, 0.7459016393),
		(0.2868852459, 0.7377049180),
		(0.2868852459, 0.7295081967),
		(0.2868852459, 0.7213114754),
		(0.2868852459, 0.7131147541),
		(0.2868852459, 0.7049180328),
		(0.2868852459, 0.6475409836),
		(0.2868852459, 0.6393442623),
		(0.2868852459, 0.6311475410),
		(0.2868852459, 0.6229508197),
		(0.2868852459, 0.6147540984),
		(0.2868852459, 0.6065573770),
		(0.2950819672, 0.9918032787),
		(0.2950819672, 0.9836065574),
		(0.2950819672, 0.9754098361),
		(0.2950819672, 0.9672131148),
		(0.2950819672, 0.9590163934),
		(0.2950819672, 0.9508196721),
		(0.2950819672, 0.9426229508),
		(0.2950819672, 0.9344262295),
		(0.2950819672, 0.9262295082),
		(0.2950819672, 0.9180327869),
		(0.2950819672, 0.9098360656),
		(0.2950819672, 0.9016393443),
		(0.2950819672, 0.7950819672),
		(0.2950819672, 0.7868852459),
		(0.2950819672, 0.7786885246),
		(0.2950819672, 0.7704918033),
		(0.2950819672, 0.7622950820),
		(0.2950819672, 0.7540983607),
		(0.2950819672, 0.7459016393),
		(0.2950819672, 0.7377049180),
		(0.2950819672, 0.7295081967),
		(0.2950819672, 0.7213114754),
		(0.2950819672, 0.7131147541),
		(0.2950819672, 0.7049180328),
		(0.2950819672, 0.6475409836),
		(0.2950819672, 0.6393442623),
		(0.2950819672, 0.6311475410),
		(0.2950819672, 0.6229508197),
		(0.2950819672, 0.6147540984),
		(0.2950819672, 0.6065573770),
		(0.3032786885, 0.9918032787),
		(0.3032786885, 0.9836065574),
		(0.3032786885, 0.9754098361),
		(0.3032786885, 0.9672131148),
		(0.3032786885, 0.9590163934),
		(0.3032786885, 0.9508196721),
		(0.3032786885, 0.9426229508),
		(0.3032786885, 0.9344262295),
		(0.3032786885, 0.9262295082),
		(0.3032786885, 0.9180327869),
		(0.3032786885, 0.9098360656),
		(0.3032786885, 0.9016393443),
		(0.3032786885, 0.8442622951),
		(0.3032786885, 0.8360655738),
		(0.3032786885, 0.8278688525),
		(0.3032786885, 0.8196721311),
		(0.3032786885, 0.8114754098),
		(0.3032786885, 0.8032786885),
		(0.3032786885, 0.6967213115),
		(0.3032786885, 0.6885245902),
		(0.3032786885, 0.6803278689),
		(0.3032786885, 0.6721311475),
		(0.3032786885, 0.6639344262),
		(0.3032786885, 0.6557377049),
		(0.3032786885, 0.5983606557),
		(0.3032786885, 0.5901639344),
		(0.3032786885, 0.5819672131),
		(0.3032786885, 0.5737704918),
		(0.3032786885, 0.5655737705),
		(0.3032786885, 0.5573770492),
		(0.3114754098, 0.9918032787),
		(0.3114754098, 0.9836065574),
		(0.3114754098, 0.9754098361),
		(0.3114754098, 0.9672131148),
		(0.3114754098, 0.9590163934),
		(0.3114754098, 0.9508196721),
		(0.3114754098, 0.9426229508),
		(0.3114754098, 0.9344262295),
		(0.3114754098, 0.9262295082),
		(0.3114754098, 0.9180327869),
		(0.3114754098, 0.9098360656),
		(0.3114754098, 0.9016393443),
		(0.3114754098, 0.8442622951),
		(0.3114754098, 0.8360655738),
		(0.3114754098, 0.8278688525),
		(0.3114754098, 0.8196721311),
		(0.3114754098, 0.8114754098),
		(0.3114754098, 0.8032786885),
		(0.3114754098, 0.6967213115),
		(0.3114754098, 0.6885245902),
		(0.3114754098, 0.6803278689),
		(0.3114754098, 0.6721311475),
		(0.3114754098, 0.6639344262),
		(0.3114754098, 0.6557377049),
		(0.3114754098, 0.5983606557),
		(0.3114754098, 0.5901639344),
		(0.3114754098, 0.5819672131),
		(0.3114754098, 0.5737704918),
		(0.3114754098, 0.5655737705),
		(0.3114754098, 0.5573770492),
		(0.3196721311, 0.9918032787),
		(0.3196721311, 0.9836065574),
		(0.3196721311, 0.9754098361),
		(0.3196721311, 0.9672131148),
		(0.3196721311, 0.9590163934),
		(0.3196721311, 0.9508196721),
		(0.3196721311, 0.9426229508),
		(0.3196721311, 0.9344262295),
		(0.3196721311, 0.9262295082),
		(0.3196721311, 0.9180327869),
		(0.3196721311, 0.9098360656),
		(0.3196721311, 0.9016393443),
		(0.3196721311, 0.8442622951),
		(0.3196721311, 0.8360655738),
		(0.3196721311, 0.8278688525),
		(0.3196721311, 0.8196721311),
		(0.3196721311, 0.8114754098),
		(0.3196721311, 0.8032786885),
		(0.3196721311, 0.6967213115),
		(0.3196721311, 0.6885245902),
		(0.3196721311, 0.6803278689),
		(0.3196721311, 0.6721311475),
		(0.3196721311, 0.6639344262),
		(0.3196721311, 0.6557377049),
		(0.3196721311, 0.5983606557),
		(0.3196721311, 0.5901639344),
		(0.3196721311, 0.5819672131),
		(0.3196721311, 0.5737704918),
		(0.3196721311, 0.5655737705),
		(0.3196721311, 0.5573770492),
		(0.3278688525, 0.9918032787),
		(0.3278688525, 0.9836065574),
		(0.3278688525, 0.9754098361),
		(0.3278688525, 0.9672131148),
		(0.3278688525, 0.9590163934),
		(0.3278688525, 0.9508196721),
		(0.3278688525, 0.9426229508),
		(0.3278688525, 0.9344262295),
		(0.3278688525, 0.9262295082),
		(0.3278688525, 0.9180327869),
		(0.3278688525, 0.9098360656),
		(0.3278688525, 0.9016393443),
		(0.3278688525, 0.8442622951),
		(0.3278688525, 0.8360655738),
		(0.3278688525, 0.8278688525),
		(0.3278688525, 0.8196721311),
		(0.3278688525, 0.8114754098),
		(0.3278688525, 0.8032786885),
		(0.3278688525, 0.6967213115),
		(0.3278688525, 0.6885245902),
		(0.3278688525, 0.6803278689),
		(0.3278688525, 0.6721311475),
		(0.3278688525, 0.6639344262),
		(0.3278688525, 0.6557377049),
		(0.3278688525, 0.5983606557),
		(0.3278688525, 0.5901639344),
		(0.3278688525, 0.5819672131),
		(0.3278688525, 0.5737704918),
		(0.3278688525, 0.5655737705),
		(0.3278688525, 0.5573770492),
		(0.3360655738, 0.9918032787),
		(0.3360655738, 0.9836065574),
		(0.3360655738, 0.9754098361),
		(0.3360655738, 0.9672131148),
		(0.3360655738, 0.9590163934),
		(0.3360655738, 0.9508196721),
		(0.3360655738, 0.9426229508),
		(0.3360655738, 0.9344262295),
		(0.3360655738, 0.9262295082),
		(0.3360655738, 0.9180327869),
		(0.3360655738, 0.9098360656),
		(0.3360655738, 0.9016393443),
		(0.3360655738, 0.8442622951),
		(0.3360655738, 0.8360655738),
		(0.3360655738, 0.8278688525),
		(0.3360655738, 0.8196721311),
		(0.3360655738, 0.8114754098),
		(0.3360655738, 0.8032786885),
		(0.3360655738, 0.6967213115),
		(0.3360655738, 0.6885245902),
		(0.3360655738, 0.6803278689),
		(0.3360655738, 0.6721311475),
		(0.3360655738, 0.6639344262),
		(0.3360655738, 0.6557377049),
		(0.3360655738, 0.5983606557),
		(0.3360655738, 0.5901639344),
		(0.3360655738, 0.5819672131),
		(0.3360655738, 0.5737704918),
		(0.3360655738, 0.5655737705),
		(0.3360655738, 0.5573770492),
		(0.3442622951, 0.9918032787),
		(0.3442622951, 0.9836065574),
		(0.3442622951, 0.9754098361),
		(0.3442622951, 0.9672131148),
		(0.3442622951, 0.9590163934),
		(0.3442622951, 0.9508196721),
		(0.3442622951, 0.9426229508),
		(0.3442622951, 0.9344262295),
		(0.3442622951, 0.9262295082),
		(0.3442622951, 0.9180327869),
		(0.3442622951, 0.9098360656),
		(0.3442622951, 0.9016393443),
		(0.3442622951, 0.8442622951),
		(0.3442622951, 0.8360655738),
		(0.3442622951, 0.8278688525),
		(0.3442622951, 0.8196721311),
		(0.3442622951, 0.8114754098),
		(0.3442622951, 0.8032786885),
		(0.3442622951, 0.6967213115),
		(0.3442622951, 0.6885245902),
		(0.3442622951, 0.6803278689),
		(0.3442622951, 0.6721311475),
		(0.3442622951, 0.6639344262),
		(0.3442622951, 0.6557377049),
		(0.3442622951, 0.5983606557),
		(0.3442622951, 0.5901639344),
		(0.3442622951, 0.5819672131),
		(0.3442622951, 0.5737704918),
		(0.3442622951, 0.5655737705),
		(0.3442622951, 0.5573770492),
		(0.3524590164, 0.7950819672),
		(0.3524590164, 0.7868852459),
		(0.3524590164, 0.7786885246),
		(0.3524590164, 0.7704918033),
		(0.3524590164, 0.7622950820),
		(0.3524590164, 0.7540983607),
		(0.3524590164, 0.7459016393),
		(0.3524590164, 0.7377049180),
		(0.3524590164, 0.7295081967),
		(0.3524590164, 0.7213114754),
		(0.3524590164, 0.7131147541),
		(0.3524590164, 0.7049180328),
		(0.3524590164, 0.6475409836),
		(0.3524590164, 0.6393442623),
		(0.3524590164, 0.6311475410),
		(0.3524590164, 0.6229508197),
		(0.3524590164, 0.6147540984),
		(0.3524590164, 0.6065573770),
		(0.3606557377, 0.7950819672),
		(0.3606557377, 0.7868852459),
		(0.3606557377, 0.7786885246),
		(0.3606557377, 0.7704918033),
		(0.3606557377, 0.7622950820),
		(0.3606557377, 0.7540983607),
		(0.3606557377, 0.7459016393),
		(0.3606557377, 0.7377049180),
		(0.3606557377, 0.7295081967),
		(0.3606557377, 0.7213114754),
		(0.3606557377, 0.7131147541),
		(0.3606557377, 0.7049180328),
		(0.3606557377, 0.6475409836),
		(0.3606557377, 0.6393442623),
		(0.3606557377, 0.6311475410),
		(0.3606557377, 0.6229508197),
		(0.3606557377, 0.6147540984),
		(0.3606557377, 0.6065573770),
		(0.3688524590, 0.7950819672),
		(0.3688524590, 0.7868852459),
		(0.3688524590, 0.7786885246),
		(0.3688524590, 0.7704918033),
		(0.3688524590, 0.7622950820),
		(0.3688524590, 0.7540983607),
		(0.3688524590, 0.7459016393),
		(0.3688524590, 0.7377049180),
		(0.3688524590, 0.7295081967),
		(0.3688524590, 0.7213114754),
		(0.3688524590, 0.7131147541),
		(0.3688524590, 0.7049180328),
		(0.3688524590, 0.6475409836),
		(0.3688524590, 0.6393442623),
		(0.3688524590, 0.6311475410),
		(0.3688524590, 0.6229508197),
		(0.3688524590, 0.6147540984),
		(0.3688524590, 0.6065573770),
		(0.3770491803, 0.7950819672),
		(0.3770491803, 0.7868852459),
		(0.3770491803, 0.7786885246),
		(0.3770491803, 0.7704918033),
		(0.3770491803, 0.7622950820),
		(0.3770491803, 0.7540983607),
		(0.3770491803, 0.7459016393),
		(0.3770491803, 0.7377049180),
		(0.3770491803, 0.7295081967),
		(0.3770491803, 0.7213114754),
		(0.3770491803, 0.7131147541),
		(0.3770491803, 0.7049180328),
		(0.3770491803, 0.6475409836),
		(0.3770491803, 0.6393442623),
		(0.3770491803, 0.6311475410),
		(0.3770491803, 0.6229508197),
		(0.3770491803, 0.6147540984),
		(0.3770491803, 0.6065573770),
		(0.3852459016, 0.7950819672),
		(0.3852459016, 0.7868852459),
		(0.3852459016, 0.7786885246),
		(0.3852459016, 0.7704918033),
		(0.3852459016, 0.7622950820),
		(0.3852459016, 0.7540983607),
		(0.3852459016, 0.7459016393),
		(0.3852459016, 0.7377049180),
		(0.3852459016, 0.7295081967),
		(0.3852459016, 0.7213114754),
		(0.3852459016, 0.7131147541),
		(0.3852459016, 0.7049180328),
		(0.3852459016, 0.6475409836),
		(0.3852459016, 0.6393442623),
		(0.3852459016, 0.6311475410),
		(0.3852459016, 0.6229508197),
		(0.3852459016, 0.6147540984),
		(0.3852459016, 0.6065573770),
		(0.3934426230, 0.7950819672),
		(0.3934426230, 0.7868852459),
		(0.3934426230, 0.7786885246),
		(0.3934426230, 0.7704918033),
		(0.3934426230, 0.7622950820),
		(0.3934426230, 0.7540983607),
		(0.3934426230, 0.7459016393),
		(0.3934426230, 0.7377049180),
		(0.3934426230, 0.7295081967),
		(0.3934426230, 0.7213114754),
		(0.3934426230, 0.7131147541),
		(0.3934426230, 0.7049180328),
		(0.3934426230, 0.6475409836),
		(0.3934426230, 0.6393442623),
		(0.3934426230, 0.6311475410),
		(0.3934426230, 0.6229508197),
		(0.3934426230, 0.6147540984),
		(0.3934426230, 0.6065573770),
		(0.4016393443, 0.6967213115),
		(0.4016393443, 0.6885245902),
		(0.4016393443, 0.6803278689),
		(0.4016393443, 0.6721311475),
		(0.4016393443, 0.6639344262),
		(0.4016393443, 0.6557377049),
		(0.4016393443, 0.5983606557),
		(0.4016393443, 0.5901639344),
		(0.4016393443, 0.5819672131),
		(0.4016393443, 0.5737704918),
		(0.4016393443, 0.5655737705),
		(0.4016393443, 0.5573770492),
		(0.4016393443, 0.5491803279),
		(0.4016393443, 0.5409836066),
		(0.4016393443, 0.5327868852),
		(0.4016393443, 0.5245901639),
		(0.4016393443, 0.5163934426),
		(0.4016393443, 0.5081967213),
		(0.4098360656, 0.6967213115),
		(0.4098360656, 0.6885245902),
		(0.4098360656, 0.6803278689),
		(0.4098360656, 0.6721311475),
		(0.4098360656, 0.6639344262),
		(0.4098360656, 0.6557377049),
		(0.4098360656, 0.5983606557),
		(0.4098360656, 0.5901639344),
		(0.4098360656, 0.5819672131),
		(0.4098360656, 0.5737704918),
		(0.4098360656, 0.5655737705),
		(0.4098360656, 0.5573770492),
		(0.4098360656, 0.5491803279),
		(0.4098360656, 0.5409836066),
		(0.4098360656, 0.5327868852),
		(0.4098360656, 0.5245901639),
		(0.4098360656, 0.5163934426),
		(0.4098360656, 0.5081967213),
		(0.4180327869, 0.6967213115),
		(0.4180327869, 0.6885245902),
		(0.4180327869, 0.6803278689),
		(0.4180327869, 0.6721311475),
		(0.4180327869, 0.6639344262),
		(0.4180327869, 0.6557377049),
		(0.4180327869, 0.5983606557),
		(0.4180327869, 0.5901639344),
		(0.4180327869, 0.5819672131),
		(0.4180327869, 0.5737704918),
		(0.4180327869, 0.5655737705),
		(0.4180327869, 0.5573770492),
		(0.4180327869, 0.5491803279),
		(0.4180327869, 0.5409836066),
		(0.4180327869, 0.5327868852),
		(0.4180327869, 0.5245901639),
		(0.4180327869, 0.5163934426),
		(0.4180327869, 0.5081967213),
		(0.4262295082, 0.6967213115),
		(0.4262295082, 0.6885245902),
		(0.4262295082, 0.6803278689),
		(0.4262295082, 0.6721311475),
		(0.4262295082, 0.6639344262),
		(0.4262295082, 0.6557377049),
		(0.4262295082, 0.5983606557),
		(0.4262295082, 0.5901639344),
		(0.4262295082, 0.5819672131),
		(0.4262295082, 0.5737704918),
		(0.4262295082, 0.5655737705),
		(0.4262295082, 0.5573770492),
		(0.4262295082, 0.5491803279),
		(0.4262295082, 0.5409836066),
		(0.4262295082, 0.5327868852),
		(0.4262295082, 0.5245901639),
		(0.4262295082, 0.5163934426),
		(0.4262295082, 0.5081967213),
		(0.4344262295, 0.6967213115),
		(0.4344262295, 0.6885245902),
		(0.4344262295, 0.6803278689),
		(0.4344262295, 0.6721311475),
		(0.4344262295, 0.6639344262),
		(0.4344262295, 0.6557377049),
		(0.4344262295, 0.5983606557),
		(0.4344262295, 0.5901639344),
		(0.4344262295, 0.5819672131),
		(0.4344262295, 0.5737704918),
		(0.4344262295, 0.5655737705),
		(0.4344262295, 0.5573770492),
		(0.4344262295, 0.5491803279),
		(0.4344262295, 0.5409836066),
		(0.4344262295, 0.5327868852),
		(0.4344262295, 0.5245901639),
		(0.4344262295, 0.5163934426),
		(0.4344262295, 0.5081967213),
		(0.4426229508, 0.6967213115),
		(0.4426229508, 0.6885245902),
		(0.4426229508, 0.6803278689),
		(0.4426229508, 0.6721311475),
		(0.4426229508, 0.6639344262),
		(0.4426229508, 0.6557377049),
		(0.4426229508, 0.5983606557),
		(0.4426229508, 0.5901639344),
		(0.4426229508, 0.5819672131),
		(0.4426229508, 0.5737704918),
		(0.4426229508, 0.5655737705),
		(0.4426229508, 0.5573770492),
		(0.4426229508, 0.5491803279),
		(0.4426229508, 0.5409836066),
		(0.4426229508, 0.5327868852),
		(0.4426229508, 0.5245901639),
		(0.4426229508, 0.5163934426),
		(0.4426229508, 0.5081967213),
		(0.4508196721, 0.9918032787),
		(0.4508196721, 0.9836065574),
		(0.4508196721, 0.9754098361),
		(0.4508196721, 0.9672131148),
		(0.4508196721, 0.9590163934),
		(0.4508196721, 0.9508196721),
		(0.4508196721, 0.8934426230),
		(0.4508196721, 0.8852459016),
		(0.4508196721, 0.8770491803),
		(0.4508196721, 0.8688524590),
		(0.4508196721, 0.8606557377),
		(0.4508196721, 0.8524590164),
		(0.4508196721, 0.5983606557),
		(0.4508196721, 0.5901639344),
		(0.4508196721, 0.5819672131),
		(0.4508196721, 0.5737704918),
		(0.4508196721, 0.5655737705),
		(0.4508196721, 0.5573770492),
		(0.4508196721, 0.5491803279),
		(0.4508196721, 0.5409836066),
		(0.4508196721, 0.5327868852),
		(0.4508196721, 0.5245901639),
		(0.4508196721, 0.5163934426),
		(0.4508196721, 0.5081967213),
		(0.4590163934, 0.9918032787),
		(0.4590163934, 0.9836065574),
		(0.4590163934, 0.9754098361),
		(0.4590163934, 0.9672131148),
		(0.4590163934, 0.9590163934),
		(0.4590163934, 0.9508196721),
		(0.4590163934, 0.8934426230),
		(0.4590163934, 0.8852459016),
		(0.4590163934, 0.8770491803),
		(0.4590163934, 0.8688524590),
		(0.4590163934, 0.8606557377),
		(0.4590163934, 0.8524590164),
		(0.4590163934, 0.5983606557),
		(0.4590163934, 0.5901639344),
		(0.4590163934, 0.5819672131),
		(0.4590163934, 0.5737704918),
		(0.4590163934, 0.5655737705),
		(0.4590163934, 0.5573770492),
		(0.4590163934, 0.5491803279),
		(0.4590163934, 0.5409836066),
		(0.4590163934, 0.5327868852),
		(0.4590163934, 0.5245901639),
		(0.4590163934, 0.5163934426),
		(0.4590163934, 0.5081967213),
		(0.4672131148, 0.9918032787),
		(0.4672131148, 0.9836065574),
		(0.4672131148, 0.9754098361),
		(0.4672131148, 0.9672131148),
		(0.4672131148, 0.9590163934),
		(0.4672131148, 0.9508196721),
		(0.4672131148, 0.8934426230),
		(0.4672131148, 0.8852459016),
		(0.4672131148, 0.8770491803),
		(0.4672131148, 0.8688524590),
		(0.4672131148, 0.8606557377),
		(0.4672131148, 0.8524590164),
		(0.4672131148, 0.5983606557),
		(0.4672131148, 0.5901639344),
		(0.4672131148, 0.5819672131),
		(0.4672131148, 0.5737704918),
		(0.4672131148, 0.5655737705),
		(0.4672131148, 0.5573770492),
		(0.4672131148, 0.5491803279),
		(0.4672131148, 0.5409836066),
		(0.4672131148, 0.5327868852),
		(0.4672131148, 0.5245901639),
		(0.4672131148, 0.5163934426),
		(0.4672131148, 0.5081967213),
		(0.4754098361, 0.9918032787),
		(0.4754098361, 0.9836065574),
		(0.4754098361, 0.9754098361),
		(0.4754098361, 0.9672131148),
		(0.4754098361, 0.9590163934),
		(0.4754098361, 0.9508196721),
		(0.4754098361, 0.8934426230),
		(0.4754098361, 0.8852459016),
		(0.4754098361, 0.8770491803),
		(0.4754098361, 0.8688524590),
		(0.4754098361, 0.8606557377),
		(0.4754098361, 0.8524590164),
		(0.4754098361, 0.5983606557),
		(0.4754098361, 0.5901639344),
		(0.4754098361, 0.5819672131),
		(0.4754098361, 0.5737704918),
		(0.4754098361, 0.5655737705),
		(0.4754098361, 0.5573770492),
		(0.4754098361, 0.5491803279),
		(0.4754098361, 0.5409836066),
		(0.4754098361, 0.5327868852),
		(0.4754098361, 0.5245901639),
		(0.4754098361, 0.5163934426),
		(0.4754098361, 0.5081967213),
		(0.4836065574, 0.9918032787),
		(0.4836065574, 0.9836065574),
		(0.4836065574, 0.9754098361),
		(0.4836065574, 0.9672131148),
		(0.4836065574, 0.9590163934),
		(0.4836065574, 0.9508196721),
		(0.4836065574, 0.8934426230),
		(0.4836065574, 0.8852459016),
		(0.4836065574, 0.8770491803),
		(0.4836065574, 0.8688524590),
		(0.4836065574, 0.8606557377),
		(0.4836065574, 0.8524590164),
		(0.4836065574, 0.5983606557),
		(0.4836065574, 0.5901639344),
		(0.4836065574, 0.5819672131),
		(0.4836065574, 0.5737704918),
		(0.4836065574, 0.5655737705),
		(0.4836065574, 0.5573770492),
		(0.4836065574, 0.5491803279),
		(0.4836065574, 0.5409836066),
		(0.4836065574, 0.5327868852),
		(0.4836065574, 0.5245901639),
		(0.4836065574, 0.5163934426),
		(0.4836065574, 0.5081967213),
		(0.4918032787, 0.9918032787),
		(0.4918032787, 0.9836065574),
		(0.4918032787, 0.9754098361),
		(0.4918032787, 0.9672131148),
		(0.4918032787, 0.9590163934),
		(0.4918032787, 0.9508196721),
		(0.4918032787, 0.8934426230),
		(0.4918032787, 0.8852459016),
		(0.4918032787, 0.8770491803),
		(0.4918032787, 0.8688524590),
		(0.4918032787, 0.8606557377),
		(0.4918032787, 0.8524590164),
		(0.4918032787, 0.5983606557),
		(0.4918032787, 0.5901639344),
		(0.4918032787, 0.5819672131),
		(0.4918032787, 0.5737704918),
		(0.4918032787, 0.5655737705),
		(0.4918032787, 0.5573770492),
		(0.4918032787, 0.5491803279),
		(0.4918032787, 0.5409836066),
		(0.4918032787, 0.5327868852),
		(0.4918032787, 0.5245901639),
		(0.4918032787, 0.5163934426),
		(0.4918032787, 0.5081967213),
		(0.5000000000, 0.5000000000),
		(0.5000000000, 0.4918032787),
		(0.5000000000, 0.4836065574),
		(0.5000000000, 0.4754098361),
		(0.5000000000, 0.4672131148),
		(0.5000000000, 0.4590163934),
		(0.5000000000, 0.4508196721),
		(0.5000000000, 0.4426229508),
		(0.5000000000, 0.4344262295),
		(0.5000000000, 0.4262295082),
		(0.5000000000, 0.4180327869),
		(0.5000000000, 0.4098360656),
		(0.5000000000, 0.3032786885),
		(0.5000000000, 0.2950819672),
		(0.5000000000, 0.2868852459),
		(0.5000000000, 0.2786885246),
		(0.5000000000, 0.2704918033),
		(0.5000000000, 0.2622950820),
		(0.5000000000, 0.2540983607),
		(0.5000000000, 0.2459016393),
		(0.5000000000, 0.2377049180),
		(0.5000000000, 0.2295081967),
		(0.5000000000, 0.2213114754),
		(0.5000000000, 0.2131147541),
		(0.5000000000, 0.2049180328),
		(0.5000000000, 0.1967213115),
		(0.5000000000, 0.1885245902),
		(0.5000000000, 0.1803278689),
		(0.5000000000, 0.1721311475),
		(0.5000000000, 0.1639344262),
		(0.5000000000, 0.0573770492),
		(0.5000000000, 0.0491803279),
		(0.5000000000, 0.0409836066),
		(0.5000000000, 0.0327868852),
		(0.5000000000, 0.0245901639),
		(0.5000000000, 0.0163934426),
		(0.5081967213, 0.5000000000),
		(0.5081967213, 0.4918032787),
		(0.5081967213, 0.4836065574),
		(0.5081967213, 0.4754098361),
		(0.5081967213, 0.4672131148),
		(0.5081967213, 0.4590163934),
		(0.5081967213, 0.4508196721),
		(0.5081967213, 0.4426229508),
		(0.5081967213, 0.4344262295),
		(0.5081967213, 0.4262295082),
		(0.5081967213, 0.4180327869),
		(0.5081967213, 0.4098360656),
		(0.5081967213, 0.3032786885),
		(0.5081967213, 0.2950819672),
		(0.5081967213, 0.2868852459),
		(0.5081967213, 0.2786885246),
		(0.5081967213, 0.2704918033),
		(0.5081967213, 0.2622950820),
		(0.5081967213, 0.2540983607),
		(0.5081967213, 0.2459016393),
		(0.5081967213, 0.2377049180),
		(0.5081967213, 0.2295081967),
		(0.5081967213, 0.2213114754),
		(0.5081967213, 0.2131147541),
		(0.5081967213, 0.2049180328),
		(0.5081967213, 0.1967213115),
		(0.5081967213, 0.1885245902),
		(0.5081967213, 0.1803278689),
		(0.5081967213, 0.1721311475),
		(0.5081967213, 0.1639344262),
		(0.5081967213, 0.0573770492),
		(0.5081967213, 0.0491803279),
		(0.5081967213, 0.0409836066),
		(0.5081967213, 0.0327868852),
		(0.5081967213, 0.0245901639),
		(0.5081967213, 0.0163934426),
		(0.5163934426, 0.5000000000),
		(0.5163934426, 0.4918032787),
		(0.5163934426, 0.4836065574),
		(0.5163934426, 0.4754098361),
		(0.5163934426, 0.4672131148),
		(0.5163934426, 0.4590163934),
		(0.5163934426, 0.4508196721),
		(0.5163934426, 0.4426229508),
		(0.5163934426, 0.4344262295),
		(0.5163934426, 0.4262295082),
		(0.5163934426, 0.4180327869),
		(0.5163934426, 0.4098360656),
		(0.5163934426, 0.3032786885),
		(0.5163934426, 0.2950819672),
		(0.5163934426, 0.2868852459),
		(0.5163934426, 0.2786885246),
		(0.5163934426, 0.2704918033),
		(0.5163934426, 0.2622950820),
		(0.5163934426, 0.2540983607),
		(0.5163934426, 0.2459016393),
		(0.5163934426, 0.2377049180),
		(0.5163934426, 0.2295081967),
		(0.5163934426, 0.2213114754),
		(0.5163934426, 0.2131147541),
		(0.5163934426, 0.2049180328),
		(0.5163934426, 0.1967213115),
		(0.5163934426, 0.1885245902),
		(0.5163934426, 0.1803278689),
		(0.5163934426, 0.1721311475),
		(0.5163934426, 0.1639344262),
		(0.5163934426, 0.0573770492),
		(0.5163934426, 0.0491803279),
		(0.5163934426, 0.0409836066),
		(0.5163934426, 0.0327868852),
		(0.5163934426, 0.0245901639),
		(0.5163934426, 0.0163934426),
		(0.5245901639, 0.5000000000),
		(0.5245901639, 0.4918032787),
		(0.5245901639, 0.4836065574),
		(0.5245901639, 0.4754098361),
		(0.5245901639, 0.4672131148),
		(0.5245901639, 0.4590163934),
		(0.5245901639, 0.4508196721),
		(0.5245901639, 0.4426229508),
		(0.5245901639, 0.4344262295),
		(0.5245901639, 0.4262295082),
		(0.5245901639, 0.4180327869),
		(0.5245901639, 0.4098360656),
		(0.5245901639, 0.3032786885),
		(0.5245901639, 0.2950819672),
		(0.5245901639, 0.2868852459),
		(0.5245901639, 0.2786885246),
		(0.5245901639, 0.2704918033),
		(0.5245901639, 0.2622950820),
		(0.5245901639, 0.2540983607),
		(0.5245901639, 0.2459016393),
		(0.5245901639, 0.2377049180),
		(0.5245901639, 0.2295081967),
		(0.5245901639, 0.2213114754),
		(0.5245901639, 0.2131147541),
		(0.5245901639, 0.2049180328),
		(0.5245901639, 0.1967213115),
		(0.5245901639, 0.1885245902),
		(0.5245901639, 0.1803278689),
		(0.5245901639, 0.1721311475),
		(0.5245901639, 0.1639344262),
		(0.5245901639, 0.0573770492),
		(0.5245901639, 0.0491803279),
		(0.5245901639, 0.0409836066),
		(0.5245901639, 0.0327868852),
		(0.5245901639, 0.0245901639),
		(0.5245901639, 0.0163934426),
		(0.5327868852, 0.5000000000),
		(0.5327868852, 0.4918032787),
		(0.5327868852, 0.4836065574),
		(0.5327868852, 0.4754098361),
		(0.5327868852, 0.4672131148),
		(0.5327868852, 0.4590163934),
		(0.5327868852, 0.4508196721),
		(0.5327868852, 0.4426229508),
		(0.5327868852, 0.4344262295),
		(0.5327868852, 0.4262295082),
		(0.5327868852, 0.4180327869),
		(0.5327868852, 0.4098360656),
		(0.5327868852, 0.3032786885),
		(0.5327868852, 0.2950819672),
		(0.5327868852, 0.2868852459),
		(0.5327868852, 0.2786885246),
		(0.5327868852, 0.2704918033),
		(0.5327868852, 0.2622950820),
		(0.5327868852, 0.2540983607),
		(0.5327868852, 0.2459016393),
		(0.5327868852, 0.2377049180),
		(0.5327868852, 0.2295081967),
		(0.5327868852, 0.2213114754),
		(0.5327868852, 0.2131147541),
		(0.5327868852, 0.2049180328),
		(0.5327868852, 0.1967213115),
		(0.5327868852, 0.1885245902),
		(0.5327868852, 0.1803278689),
		(0.5327868852, 0.1721311475),
		(0.5327868852, 0.1639344262),
		(0.5327868852, 0.0573770492),
		(0.5327868852, 0.0491803279),
		(0.5327868852, 0.0409836066),
		(0.5327868852, 0.0327868852),
		(0.5327868852, 0.0245901639),
		(0.5327868852, 0.0163934426),
		(0.5409836066, 0.5000000000),
		(0.5409836066, 0.4918032787),
		(0.5409836066, 0.4836065574),
		(0.5409836066, 0.4754098361),
		(0.5409836066, 0.4672131148),
		(0.5409836066, 0.4590163934),
		(0.5409836066, 0.4508196721),
		(0.5409836066, 0.4426229508),
		(0.5409836066, 0.4344262295),
		(0.5409836066, 0.4262295082),
		(0.5409836066, 0.4180327869),
		(0.5409836066, 0.4098360656),
		(0.5409836066, 0.3032786885),
		(0.5409836066, 0.2950819672),
		(0.5409836066, 0.2868852459),
		(0.5409836066, 0.2786885246),
		(0.5409836066, 0.2704918033),
		(0.5409836066, 0.2622950820),
		(0.5409836066, 0.2540983607),
		(0.5409836066, 0.2459016393),
		(0.5409836066, 0.2377049180),
		(0.5409836066, 0.2295081967),
		(0.5409836066, 0.2213114754),
		(0.5409836066, 0.2131147541),
		(0.5409836066, 0.2049180328),
		(0.5409836066, 0.1967213115),
		(0.5409836066, 0.1885245902),
		(0.5409836066, 0.1803278689),
		(0.5409836066, 0.1721311475),
		(0.5409836066, 0.1639344262),
		(0.5409836066, 0.0573770492),
		(0.5409836066, 0.0491803279),
		(0.5409836066, 0.0409836066),
		(0.5409836066, 0.0327868852),
		(0.5409836066, 0.0245901639),
		(0.5409836066, 0.0163934426),
		(0.5491803279, 0.5000000000),
		(0.5491803279, 0.4918032787),
		(0.5491803279, 0.4836065574),
		(0.5491803279, 0.4754098361),
		(0.5491803279, 0.4672131148),
		(0.5491803279, 0.4590163934),
		(0.5491803279, 0.4508196721),
		(0.5491803279, 0.4426229508),
		(0.5491803279, 0.4344262295),
		(0.5491803279, 0.4262295082),
		(0.5491803279, 0.4180327869),
		(0.5491803279, 0.4098360656),
		(0.5491803279, 0.3524590164),
		(0.5491803279, 0.3442622951),
		(0.5491803279, 0.3360655738),
		(0.5491803279, 0.3278688525),
		(0.5491803279, 0.3196721311),
		(0.5491803279, 0.3114754098),
		(0.5491803279, 0.2540983607),
		(0.5491803279, 0.2459016393),
		(0.5491803279, 0.2377049180),
		(0.5491803279, 0.2295081967),
		(0.5491803279, 0.2213114754),
		(0.5491803279, 0.2131147541),
		(0.5491803279, 0.2049180328),
		(0.5491803279, 0.1967213115),
		(0.5491803279, 0.1885245902),
		(0.5491803279, 0.1803278689),
		(0.5491803279, 0.1721311475),
		(0.5491803279, 0.1639344262),
		(0.5573770492, 0.5000000000),
		(0.5573770492, 0.4918032787),
		(0.5573770492, 0.4836065574),
		(0.5573770492, 0.4754098361),
		(0.5573770492, 0.4672131148),
		(0.5573770492, 0.4590163934),
		(0.5573770492, 0.4508196721),
		(0.5573770492, 0.4426229508),
		(0.5573770492, 0.4344262295),
		(0.5573770492, 0.4262295082),
		(0.5573770492, 0.4180327869),
		(0.5573770492, 0.4098360656),
		(0.5573770492, 0.3524590164),
		(0.5573770492, 0.3442622951),
		(0.5573770492, 0.3360655738),
		(0.5573770492, 0.3278688525),
		(0.5573770492, 0.3196721311),
		(0.5573770492, 0.3114754098),
		(0.5573770492, 0.2540983607),
		(0.5573770492, 0.2459016393),
		(0.5573770492, 0.2377049180),
		(0.5573770492, 0.2295081967),
		(0.5573770492, 0.2213114754),
		(0.5573770492, 0.2131147541),
		(0.5573770492, 0.2049180328),
		(0.5573770492, 0.1967213115),
		(0.5573770492, 0.1885245902),
		(0.5573770492, 0.1803278689),
		(0.5573770492, 0.1721311475),
		(0.5573770492, 0.1639344262),
		(0.5655737705, 0.5000000000),
		(0.5655737705, 0.4918032787),
		(0.5655737705, 0.4836065574),
		(0.5655737705, 0.4754098361),
		(0.5655737705, 0.4672131148),
		(0.5655737705, 0.4590163934),
		(0.5655737705, 0.4508196721),
		(0.5655737705, 0.4426229508),
		(0.5655737705, 0.4344262295),
		(0.5655737705, 0.4262295082),
		(0.5655737705, 0.4180327869),
		(0.5655737705, 0.4098360656),
		(0.5655737705, 0.3524590164),
		(0.5655737705, 0.3442622951),
		(0.5655737705, 0.3360655738),
		(0.5655737705, 0.3278688525),
		(0.5655737705, 0.3196721311),
		(0.5655737705, 0.3114754098),
		(0.5655737705, 0.2540983607),
		(0.5655737705, 0.2459016393),
		(0.5655737705, 0.2377049180),
		(0.5655737705, 0.2295081967),
		(0.5655737705, 0.2213114754),
		(0.5655737705, 0.2131147541),
		(0.5655737705, 0.2049180328),
		(0.5655737705, 0.1967213115),
		(0.5655737705, 0.1885245902),
		(0.5655737705, 0.1803278689),
		(0.5655737705, 0.1721311475),
		(0.5655737705, 0.1639344262),
		(0.5737704918, 0.5000000000),
		(0.5737704918, 0.4918032787),
		(0.5737704918, 0.4836065574),
		(0.5737704918, 0.4754098361),
		(0.5737704918, 0.4672131148),
		(0.5737704918, 0.4590163934),
		(0.5737704918, 0.4508196721),
		(0.5737704918, 0.4426229508),
		(0.5737704918, 0.4344262295),
		(0.5737704918, 0.4262295082),
		(0.5737704918, 0.4180327869),
		(0.5737704918, 0.4098360656),
		(0.5737704918, 0.3524590164),
		(0.5737704918, 0.3442622951),
		(0.5737704918, 0.3360655738),
		(0.5737704918, 0.3278688525),
		(0.5737704918, 0.3196721311),
		(0.5737704918, 0.3114754098),
		(0.5737704918, 0.2540983607),
		(0.5737704918, 0.2459016393),
		(0.5737704918, 0.2377049180),
		(0.5737704918, 0.2295081967),
		(0.5737704918, 0.2213114754),
		(0.5737704918, 0.2131147541),
		(0.5737704918, 0.2049180328),
		(0.5737704918, 0.1967213115),
		(0.5737704918, 0.1885245902),
		(0.5737704918, 0.1803278689),
		(0.5737704918, 0.1721311475),
		(0.5737704918, 0.1639344262),
		(0.5819672131, 0.5000000000),
		(0.5819672131, 0.4918032787),
		(0.5819672131, 0.4836065574),
		(0.5819672131, 0.4754098361),
		(0.5819672131, 0.4672131148),
		(0.5819672131, 0.4590163934),
		(0.5819672131, 0.4508196721),
		(0.5819672131, 0.4426229508),
		(0.5819672131, 0.4344262295),
		(0.5819672131, 0.4262295082),
		(0.5819672131, 0.4180327869),
		(0.5819672131, 0.4098360656),
		(0.5819672131, 0.3524590164),
		(0.5819672131, 0.3442622951),
		(0.5819672131, 0.3360655738),
		(0.5819672131, 0.3278688525),
		(0.5819672131, 0.3196721311),
		(0.5819672131, 0.3114754098),
		(0.5819672131, 0.2540983607),
		(0.5819672131, 0.2459016393),
		(0.5819672131, 0.2377049180),
		(0.5819672131, 0.2295081967),
		(0.5819672131, 0.2213114754),
		(0.5819672131, 0.2131147541),
		(0.5819672131, 0.2049180328),
		(0.5819672131, 0.1967213115),
		(0.5819672131, 0.1885245902),
		(0.5819672131, 0.1803278689),
		(0.5819672131, 0.1721311475),
		(0.5819672131, 0.1639344262),
		(0.5901639344, 0.5000000000),
		(0.5901639344, 0.4918032787),
		(0.5901639344, 0.4836065574),
		(0.5901639344, 0.4754098361),
		(0.5901639344, 0.4672131148),
		(0.5901639344, 0.4590163934),
		(0.5901639344, 0.4508196721),
		(0.5901639344, 0.4426229508),
		(0.5901639344, 0.4344262295),
		(0.5901639344, 0.4262295082),
		(0.5901639344, 0.4180327869),
		(0.5901639344, 0.4098360656),
		(0.5901639344, 0.3524590164),
		(0.5901639344, 0.3442622951),
		(0.5901639344, 0.3360655738),
		(0.5901639344, 0.3278688525),
		(0.5901639344, 0.3196721311),
		(0.5901639344, 0.3114754098),
		(0.5901639344, 0.2540983607),
		(0.5901639344, 0.2459016393),
		(0.5901639344, 0.2377049180),
		(0.5901639344, 0.2295081967),
		(0.5901639344, 0.2213114754),
		(0.5901639344, 0.2131147541),
		(0.5901639344, 0.2049180328),
		(0.5901639344, 0.1967213115),
		(0.5901639344, 0.1885245902),
		(0.5901639344, 0.1803278689),
		(0.5901639344, 0.1721311475),
		(0.5901639344, 0.1639344262),
		(0.5983606557, 0.4016393443),
		(0.5983606557, 0.3934426230),
		(0.5983606557, 0.3852459016),
		(0.5983606557, 0.3770491803),
		(0.5983606557, 0.3688524590),
		(0.5983606557, 0.3606557377),
		(0.5983606557, 0.3032786885),
		(0.5983606557, 0.2950819672),
		(0.5983606557, 0.2868852459),
		(0.5983606557, 0.2786885246),
		(0.5983606557, 0.2704918033),
		(0.5983606557, 0.2622950820),
		(0.5983606557, 0.0573770492),
		(0.5983606557, 0.0491803279),
		(0.5983606557, 0.0409836066),
		(0.5983606557, 0.0327868852),
		(0.5983606557, 0.0245901639),
		(0.5983606557, 0.0163934426),
		(0.6065573770, 0.4016393443),
		(0.6065573770, 0.3934426230),
		(0.6065573770, 0.3852459016),
		(0.6065573770, 0.3770491803),
		(0.6065573770, 0.3688524590),
		(0.6065573770, 0.3606557377),
		(0.6065573770, 0.3032786885),
		(0.6065573770, 0.2950819672),
		(0.6065573770, 0.2868852459),
		(0.6065573770, 0.2786885246),
		(0.6065573770, 0.2704918033),
		(0.6065573770, 0.2622950820),
		(0.6065573770, 0.0573770492),
		(0.6065573770, 0.0491803279),
		(0.6065573770, 0.0409836066),
		(0.6065573770, 0.0327868852),
		(0.6065573770, 0.0245901639),
		(0.6065573770, 0.0163934426),
		(0.6147540984, 0.4016393443),
		(0.6147540984, 0.3934426230),
		(0.6147540984, 0.3852459016),
		(0.6147540984, 0.3770491803),
		(0.6147540984, 0.3688524590),
		(0.6147540984, 0.3606557377),
		(0.6147540984, 0.3032786885),
		(0.6147540984, 0.2950819672),
		(0.6147540984, 0.2868852459),
		(0.6147540984, 0.2786885246),
		(0.6147540984, 0.2704918033),
		(0.6147540984, 0.2622950820),
		(0.6147540984, 0.0573770492),
		(0.6147540984, 0.0491803279),
		(0.6147540984, 0.0409836066),
		(0.6147540984, 0.0327868852),
		(0.6147540984, 0.0245901639),
		(0.6147540984, 0.0163934426),
		(0.6229508197, 0.4016393443),
		(0.6229508197, 0.3934426230),
		(0.6229508197, 0.3852459016),
		(0.6229508197, 0.3770491803),
		(0.6229508197, 0.3688524590),
		(0.6229508197, 0.3606557377),
		(0.6229508197, 0.3032786885),
		(0.6229508197, 0.2950819672),
		(0.6229508197, 0.2868852459),
		(0.6229508197, 0.2786885246),
		(0.6229508197, 0.2704918033),
		(0.6229508197, 0.2622950820),
		(0.6229508197, 0.0573770492),
		(0.6229508197, 0.0491803279),
		(0.6229508197, 0.0409836066),
		(0.6229508197, 0.0327868852),
		(0.6229508197, 0.0245901639),
		(0.6229508197, 0.0163934426),
		(0.6311475410, 0.4016393443),
		(0.6311475410, 0.3934426230),
		(0.6311475410, 0.3852459016),
		(0.6311475410, 0.3770491803),
		(0.6311475410, 0.3688524590),
		(0.6311475410, 0.3606557377),
		(0.6311475410, 0.3032786885),
		(0.6311475410, 0.2950819672),
		(0.6311475410, 0.2868852459),
		(0.6311475410, 0.2786885246),
		(0.6311475410, 0.2704918033),
		(0.6311475410, 0.2622950820),
		(0.6311475410, 0.0573770492),
		(0.6311475410, 0.0491803279),
		(0.6311475410, 0.0409836066),
		(0.6311475410, 0.0327868852),
		(0.6311475410, 0.0245901639),
		(0.6311475410, 0.0163934426),
		(0.6393442623, 0.4016393443),
		(0.6393442623, 0.3934426230),
		(0.6393442623, 0.3852459016),
		(0.6393442623, 0.3770491803),
		(0.6393442623, 0.3688524590),
		(0.6393442623, 0.3606557377),
		(0.6393442623, 0.3032786885),
		(0.6393442623, 0.2950819672),
		(0.6393442623, 0.2868852459),
		(0.6393442623, 0.2786885246),
		(0.6393442623, 0.2704918033),
		(0.6393442623, 0.2622950820),
		(0.6393442623, 0.0573770492),
		(0.6393442623, 0.0491803279),
		(0.6393442623, 0.0409836066),
		(0.6393442623, 0.0327868852),
		(0.6393442623, 0.0245901639),
		(0.6393442623, 0.0163934426),
		(0.6475409836, 0.4508196721),
		(0.6475409836, 0.4426229508),
		(0.6475409836, 0.4344262295),
		(0.6475409836, 0.4262295082),
		(0.6475409836, 0.4180327869),
		(0.6475409836, 0.4098360656),
		(0.6475409836, 0.3524590164),
		(0.6475409836, 0.3442622951),
		(0.6475409836, 0.3360655738),
		(0.6475409836, 0.3278688525),
		(0.6475409836, 0.3196721311),
		(0.6475409836, 0.3114754098),
		(0.6475409836, 0.2049180328),
		(0.6475409836, 0.1967213115),
		(0.6475409836, 0.1885245902),
		(0.6475409836, 0.1803278689),
		(0.6475409836, 0.1721311475),
		(0.6475409836, 0.1639344262),
		(0.6557377049, 0.4508196721),
		(0.6557377049, 0.4426229508),
		(0.6557377049, 0.4344262295),
		(0.6557377049, 0.4262295082),
		(0.6557377049, 0.4180327869),
		(0.6557377049, 0.4098360656),
		(0.6557377049, 0.3524590164),
		(0.6557377049, 0.3442622951),
		(0.6557377049, 0.3360655738),
		(0.6557377049, 0.3278688525),
		(0.6557377049, 0.3196721311),
		(0.6557377049, 0.3114754098),
		(0.6557377049, 0.2049180328),
		(0.6557377049, 0.1967213115),
		(0.6557377049, 0.1885245902),
		(0.6557377049, 0.1803278689),
		(0.6557377049, 0.1721311475),
		(0.6557377049, 0.1639344262),
		(0.6639344262, 0.4508196721),
		(0.6639344262, 0.4426229508),
		(0.6639344262, 0.4344262295),
		(0.6639344262, 0.4262295082),
		(0.6639344262, 0.4180327869),
		(0.6639344262, 0.4098360656),
		(0.6639344262, 0.3524590164),
		(0.6639344262, 0.3442622951),
		(0.6639344262, 0.3360655738),
		(0.6639344262, 0.3278688525),
		(0.6639344262, 0.3196721311),
		(0.6639344262, 0.3114754098),
		(0.6639344262, 0.2049180328),
		(0.6639344262, 0.1967213115),
		(0.6639344262, 0.1885245902),
		(0.6639344262, 0.1803278689),
		(0.6639344262, 0.1721311475),
		(0.6639344262, 0.1639344262),
		(0.6721311475, 0.4508196721),
		(0.6721311475, 0.4426229508),
		(0.6721311475, 0.4344262295),
		(0.6721311475, 0.4262295082),
		(0.6721311475, 0.4180327869),
		(0.6721311475, 0.4098360656),
		(0.6721311475, 0.3524590164),
		(0.6721311475, 0.3442622951),
		(0.6721311475, 0.3360655738),
		(0.6721311475, 0.3278688525),
		(0.6721311475, 0.3196721311),
		(0.6721311475, 0.3114754098),
		(0.6721311475, 0.2049180328),
		(0.6721311475, 0.1967213115),
		(0.6721311475, 0.1885245902),
		(0.6721311475, 0.1803278689),
		(0.6721311475, 0.1721311475),
		(0.6721311475, 0.1639344262),
		(0.6803278689, 0.4508196721),
		(0.6803278689, 0.4426229508),
		(0.6803278689, 0.4344262295),
		(0.6803278689, 0.4262295082),
		(0.6803278689, 0.4180327869),
		(0.6803278689, 0.4098360656),
		(0.6803278689, 0.3524590164),
		(0.6803278689, 0.3442622951),
		(0.6803278689, 0.3360655738),
		(0.6803278689, 0.3278688525),
		(0.6803278689, 0.3196721311),
		(0.6803278689, 0.3114754098),
		(0.6803278689, 0.2049180328),
		(0.6803278689, 0.1967213115),
		(0.6803278689, 0.1885245902),
		(0.6803278689, 0.1803278689),
		(0.6803278689, 0.1721311475),
		(0.6803278689, 0.1639344262),
		(0.6885245902, 0.4508196721),
		(0.6885245902, 0.4426229508),
		(0.6885245902, 0.4344262295),
		(0.6885245902, 0.4262295082),
		(0.6885245902, 0.4180327869),
		(0.6885245902, 0.4098360656),
		(0.6885245902, 0.3524590164),
		(0.6885245902, 0.3442622951),
		(0.6885245902, 0.3360655738),
		(0.6885245902, 0.3278688525),
		(0.6885245902, 0.3196721311),
		(0.6885245902, 0.3114754098),
		(0.6885245902, 0.2049180328),
		(0.6885245902, 0.1967213115),
		(0.6885245902, 0.1885245902),
		(0.6885245902, 0.1803278689),
		(0.6885245902, 0.1721311475),
		(0.6885245902, 0.1639344262),
		(0.6967213115, 0.5000000000),
		(0.6967213115, 0.4918032787),
		(0.6967213115, 0.4836065574),
		(0.6967213115, 0.4754098361),
		(0.6967213115, 0.4672131148),
		(0.6967213115, 0.4590163934),
		(0.6967213115, 0.4016393443),
		(0.6967213115, 0.3934426230),
		(0.6967213115, 0.3852459016),
		(0.6967213115, 0.3770491803),
		(0.6967213115, 0.3688524590),
		(0.6967213115, 0.3606557377),
		(0.6967213115, 0.3032786885),
		(0.6967213115, 0.2950819672),
		(0.6967213115, 0.2868852459),
		(0.6967213115, 0.2786885246),
		(0.6967213115, 0.2704918033),
		(0.6967213115, 0.2622950820),
		(0.6967213115, 0.2540983607),
		(0.6967213115, 0.2459016393),
		(0.6967213115, 0.2377049180),
		(0.6967213115, 0.2295081967),
		(0.6967213115, 0.2213114754),
		(0.6967213115, 0.2131147541),
		(0.6967213115, 0.1557377049),
		(0.6967213115, 0.1475409836),
		(0.6967213115, 0.1393442623),
		(0.6967213115, 0.1311475410),
		(0.6967213115, 0.1229508197),
		(0.6967213115, 0.1147540984),
		(0.7049180328, 0.5000000000),
		(0.7049180328, 0.4918032787),
		(0.7049180328, 0.4836065574),
		(0.7049180328, 0.4754098361),
		(0.7049180328, 0.4672131148),
		(0.7049180328, 0.4590163934),
		(0.7049180328, 0.4016393443),
		(0.7049180328, 0.3934426230),
		(0.7049180328, 0.3852459016),
		(0.7049180328, 0.3770491803),
		(0.7049180328, 0.3688524590),
		(0.7049180328, 0.3606557377),
		(0.7049180328, 0.3032786885),
		(0.7049180328, 0.2950819672),
		(0.7049180328, 0.2868852459),
		(0.7049180328, 0.2786885246),
		(0.7049180328, 0.2704918033),
		(0.7049180328, 0.2622950820),
		(0.7049180328, 0.2540983607),
		(0.7049180328, 0.2459016393),
		(0.7049180328, 0.2377049180),
		(0.7049180328, 0.2295081967),
		(0.7049180328, 0.2213114754),
		(0.7049180328, 0.2131147541),
		(0.7049180328, 0.1557377049),
		(0.7049180328, 0.1475409836),
		(0.7049180328, 0.1393442623),
		(0.7049180328, 0.1311475410),
		(0.7049180328, 0.1229508197),
		(0.7049180328, 0.1147540984),
		(0.7131147541, 0.5000000000),
		(0.7131147541, 0.4918032787),
		(0.7131147541, 0.4836065574),
		(0.7131147541, 0.4754098361),
		(0.7131147541, 0.4672131148),
		(0.7131147541, 0.4590163934),
		(0.7131147541, 0.4016393443),
		(0.7131147541, 0.3934426230),
		(0.7131147541, 0.3852459016),
		(0.7131147541, 0.3770491803),
		(0.7131147541, 0.3688524590),
		(0.7131147541, 0.3606557377),
		(0.7131147541, 0.3032786885),
		(0.7131147541, 0.2950819672),
		(0.7131147541, 0.2868852459),
		(0.7131147541, 0.2786885246),
		(0.7131147541, 0.2704918033),
		(0.7131147541, 0.2622950820),
		(0.7131147541, 0.2540983607),
		(0.7131147541, 0.2459016393),
		(0.7131147541, 0.2377049180),
		(0.7131147541, 0.2295081967),
		(0.7131147541, 0.2213114754),
		(0.7131147541, 0.2131147541),
		(0.7131147541, 0.1557377049),
		(0.7131147541, 0.1475409836),
		(0.7131147541, 0.1393442623),
		(0.7131147541, 0.1311475410),
		(0.7131147541, 0.1229508197),
		(0.7131147541, 0.1147540984),
		(0.7213114754, 0.5000000000),
		(0.7213114754, 0.4918032787),
		(0.7213114754, 0.4836065574),
		(0.7213114754, 0.4754098361),
		(0.7213114754, 0.4672131148),
		(0.7213114754, 0.4590163934),
		(0.7213114754, 0.4016393443),
		(0.7213114754, 0.3934426230),
		(0.7213114754, 0.3852459016),
		(0.7213114754, 0.3770491803),
		(0.7213114754, 0.3688524590),
		(0.7213114754, 0.3606557377),
		(0.7213114754, 0.3032786885),
		(0.7213114754, 0.2950819672),
		(0.7213114754, 0.2868852459),
		(0.7213114754, 0.2786885246),
		(0.7213114754, 0.2704918033),
		(0.7213114754, 0.2622950820),
		(0.7213114754, 0.2540983607),
		(0.7213114754, 0.2459016393),
		(0.7213114754, 0.2377049180),
		(0.7213114754, 0.2295081967),
		(0.7213114754, 0.2213114754),
		(0.7213114754, 0.2131147541),
		(0.7213114754, 0.1557377049),
		(0.7213114754, 0.1475409836),
		(0.7213114754, 0.1393442623),
		(0.7213114754, 0.1311475410),
		(0.7213114754, 0.1229508197),
		(0.7213114754, 0.1147540984),
		(0.7295081967, 0.5000000000),
		(0.7295081967, 0.4918032787),
		(0.7295081967, 0.4836065574),
		(0.7295081967, 0.4754098361),
		(0.7295081967, 0.4672131148),
		(0.7295081967, 0.4590163934),
		(0.7295081967, 0.4016393443),
		(0.7295081967, 0.3934426230),
		(0.7295081967, 0.3852459016),
		(0.7295081967, 0.3770491803),
		(0.7295081967, 0.3688524590),
		(0.7295081967, 0.3606557377),
		(0.7295081967, 0.3032786885),
		(0.7295081967, 0.2950819672),
		(0.7295081967, 0.2868852459),
		(0.7295081967, 0.2786885246),
		(0.7295081967, 0.2704918033),
		(0.7295081967, 0.2622950820),
		(0.7295081967, 0.2540983607),
		(0.7295081967, 0.2459016393),
		(0.7295081967, 0.2377049180),
		(0.7295081967, 0.2295081967),
		(0.7295081967, 0.2213114754),
		(0.7295081967, 0.2131147541),
		(0.7295081967, 0.1557377049),
		(0.7295081967, 0.1475409836),
		(0.7295081967, 0.1393442623),
		(0.7295081967, 0.1311475410),
		(0.7295081967, 0.1229508197),
		(0.7295081967, 0.1147540984),
		(0.7377049180, 0.5000000000),
		(0.7377049180, 0.4918032787),
		(0.7377049180, 0.4836065574),
		(0.7377049180, 0.4754098361),
		(0.7377049180, 0.4672131148),
		(0.7377049180, 0.4590163934),
		(0.7377049180, 0.4016393443),
		(0.7377049180, 0.3934426230),
		(0.7377049180, 0.3852459016),
		(0.7377049180, 0.3770491803),
		(0.7377049180, 0.3688524590),
		(0.7377049180, 0.3606557377),
		(0.7377049180, 0.3032786885),
		(0.7377049180, 0.2950819672),
		(0.7377049180, 0.2868852459),
		(0.7377049180, 0.2786885246),
		(0.7377049180, 0.2704918033),
		(0.7377049180, 0.2622950820),
		(0.7377049180, 0.2540983607),
		(0.7377049180, 0.2459016393),
		(0.7377049180, 0.2377049180),
		(0.7377049180, 0.2295081967),
		(0.7377049180, 0.2213114754),
		(0.7377049180, 0.2131147541),
		(0.7377049180, 0.1557377049),
		(0.7377049180, 0.1475409836),
		(0.7377049180, 0.1393442623),
		(0.7377049180, 0.1311475410),
		(0.7377049180, 0.1229508197),
		(0.7377049180, 0.1147540984),
		(0.7459016393, 0.5000000000),
		(0.7459016393, 0.4918032787),
		(0.7459016393, 0.4836065574),
		(0.7459016393, 0.4754098361),
		(0.7459016393, 0.4672131148),
		(0.7459016393, 0.4590163934),
		(0.7459016393, 0.4508196721),
		(0.7459016393, 0.4426229508),
		(0.7459016393, 0.4344262295),
		(0.7459016393, 0.4262295082),
		(0.7459016393, 0.4180327869),
		(0.7459016393, 0.4098360656),
		(0.7459016393, 0.3032786885),
		(0.7459016393, 0.2950819672),
		(0.7459016393, 0.2868852459),
		(0.7459016393, 0.2786885246),
		(0.7459016393, 0.2704918033),
		(0.7459016393, 0.2622950820),
		(0.7459016393, 0.2540983607),
		(0.7459016393, 0.2459016393),
		(0.7459016393, 0.2377049180),
		(0.7459016393, 0.2295081967),
		(0.7459016393, 0.2213114754),
		(0.7459016393, 0.2131147541),
		(0.7459016393, 0.1557377049),
		(0.7459016393, 0.1475409836),
		(0.7459016393, 0.1393442623),
		(0.7459016393, 0.1311475410),
		(0.7459016393, 0.1229508197),
		(0.7459016393, 0.1147540984),
		(0.7540983607, 0.5000000000),
		(0.7540983607, 0.4918032787),
		(0.7540983607, 0.4836065574),
		(0.7540983607, 0.4754098361),
		(0.7540983607, 0.4672131148),
		(0.7540983607, 0.4590163934),
		(0.7540983607, 0.4508196721),
		(0.7540983607, 0.4426229508),
		(0.7540983607, 0.4344262295),
		(0.7540983607, 0.4262295082),
		(0.7540983607, 0.4180327869),
		(0.7540983607, 0.4098360656),
		(0.7540983607, 0.3032786885),
		(0.7540983607, 0.2950819672),
		(0.7540983607, 0.2868852459),
		(0.7540983607, 0.2786885246),
		(0.7540983607, 0.2704918033),
		(0.7540983607, 0.2622950820),
		(0.7540983607, 0.2540983607),
		(0.7540983607, 0.2459016393),
		(0.7540983607, 0.2377049180),
		(0.7540983607, 0.2295081967),
		(0.7540983607, 0.2213114754),
		(0.7540983607, 0.2131147541),
		(0.7540983607, 0.1557377049),
		(0.7540983607, 0.1475409836),
		(0.7540983607, 0.1393442623),
		(0.7540983607, 0.1311475410),
		(0.7540983607, 0.1229508197),
		(0.7540983607, 0.1147540984),
		(0.7622950820, 0.5000000000),
		(0.7622950820, 0.4918032787),
		(0.7622950820, 0.4836065574),
		(0.7622950820, 0.4754098361),
		(0.7622950820, 0.4672131148),
		(0.7622950820, 0.4590163934),
		(0.7622950820, 0.4508196721),
		(0.7622950820, 0.4426229508),
		(0.7622950820, 0.4344262295),
		(0.7622950820, 0.4262295082),
		(0.7622950820, 0.4180327869),
		(0.7622950820, 0.4098360656),
		(0.7622950820, 0.3032786885),
		(0.7622950820, 0.2950819672),
		(0.7622950820, 0.2868852459),
		(0.7622950820, 0.2786885246),
		(0.7622950820, 0.2704918033),
		(0.7622950820, 0.2622950820),
		(0.7622950820, 0.2540983607),
		(0.7622950820, 0.2459016393),
		(0.7622950820, 0.2377049180),
		(0.7622950820, 0.2295081967),
		(0.7622950820, 0.2213114754),
		(0.7622950820, 0.2131147541),
		(0.7622950820, 0.1557377049),
		(0.7622950820, 0.1475409836),
		(0.7622950820, 0.1393442623),
		(0.7622950820, 0.1311475410),
		(0.7622950820, 0.1229508197),
		(0.7622950820, 0.1147540984),
		(0.7704918033, 0.5000000000),
		(0.7704918033, 0.4918032787),
		(0.7704918033, 0.4836065574),
		(0.7704918033, 0.4754098361),
		(0.7704918033, 0.4672131148),
		(0.7704918033, 0.4590163934),
		(0.7704918033, 0.4508196721),
		(0.7704918033, 0.4426229508),
		(0.7704918033, 0.4344262295),
		(0.7704918033, 0.4262295082),
		(0.7704918033, 0.4180327869),
		(0.7704918033, 0.4098360656),
		(0.7704918033, 0.3032786885),
		(0.7704918033, 0.2950819672),
		(0.7704918033, 0.2868852459),
		(0.7704918033, 0.2786885246),
		(0.7704918033, 0.2704918033),
		(0.7704918033, 0.2622950820),
		(0.7704918033, 0.2540983607),
		(0.7704918033, 0.2459016393),
		(0.7704918033, 0.2377049180),
		(0.7704918033, 0.2295081967),
		(0.7704918033, 0.2213114754),
		(0.7704918033, 0.2131147541),
		(0.7704918033, 0.1557377049),
		(0.7704918033, 0.1475409836),
		(0.7704918033, 0.1393442623),
		(0.7704918033, 0.1311475410),
		(0.7704918033, 0.1229508197),
		(0.7704918033, 0.1147540984),
		(0.7786885246, 0.5000000000),
		(0.7786885246, 0.4918032787),
		(0.7786885246, 0.4836065574),
		(0.7786885246, 0.4754098361),
		(0.7786885246, 0.4672131148),
		(0.7786885246, 0.4590163934),
		(0.7786885246, 0.4508196721),
		(0.7786885246, 0.4426229508),
		(0.7786885246, 0.4344262295),
		(0.7786885246, 0.4262295082),
		(0.7786885246, 0.4180327869),
		(0.7786885246, 0.4098360656),
		(0.7786885246, 0.3032786885),
		(0.7786885246, 0.2950819672),
		(0.7786885246, 0.2868852459),
		(0.7786885246, 0.2786885246),
		(0.7786885246, 0.2704918033),
		(0.7786885246, 0.2622950820),
		(0.7786885246, 0.2540983607),
		(0.7786885246, 0.2459016393),
		(0.7786885246, 0.2377049180),
		(0.7786885246, 0.2295081967),
		(0.7786885246, 0.2213114754),
		(0.7786885246, 0.2131147541),
		(0.7786885246, 0.1557377049),
		(0.7786885246, 0.1475409836),
		(0.7786885246, 0.1393442623),
		(0.7786885246, 0.1311475410),
		(0.7786885246, 0.1229508197),
		(0.7786885246, 0.1147540984),
		(0.7868852459, 0.5000000000),
		(0.7868852459, 0.4918032787),
		(0.7868852459, 0.4836065574),
		(0.7868852459, 0.4754098361),
		(0.7868852459, 0.4672131148),
		(0.7868852459, 0.4590163934),
		(0.7868852459, 0.4508196721),
		(0.7868852459, 0.4426229508),
		(0.7868852459, 0.4344262295),
		(0.7868852459, 0.4262295082),
		(0.7868852459, 0.4180327869),
		(0.7868852459, 0.4098360656),
		(0.7868852459, 0.3032786885),
		(0.7868852459, 0.2950819672),
		(0.7868852459, 0.2868852459),
		(0.7868852459, 0.2786885246),
		(0.7868852459, 0.2704918033),
		(0.7868852459, 0.2622950820),
		(0.7868852459, 0.2540983607),
		(0.7868852459, 0.2459016393),
		(0.7868852459, 0.2377049180),
		(0.7868852459, 0.2295081967),
		(0.7868852459, 0.2213114754),
		(0.7868852459, 0.2131147541),
		(0.7868852459, 0.1557377049),
		(0.7868852459, 0.1475409836),
		(0.7868852459, 0.1393442623),
		(0.7868852459, 0.1311475410),
		(0.7868852459, 0.1229508197),
		(0.7868852459, 0.1147540984),
		(0.7950819672, 0.5000000000),
		(0.7950819672, 0.4918032787),
		(0.7950819672, 0.4836065574),
		(0.7950819672, 0.4754098361),
		(0.7950819672, 0.4672131148),
		(0.7950819672, 0.4590163934),
		(0.7950819672, 0.4508196721),
		(0.7950819672, 0.4426229508),
		(0.7950819672, 0.4344262295),
		(0.7950819672, 0.4262295082),
		(0.7950819672, 0.4180327869),
		(0.7950819672, 0.4098360656),
		(0.7950819672, 0.3524590164),
		(0.7950819672, 0.3442622951),
		(0.7950819672, 0.3360655738),
		(0.7950819672, 0.3278688525),
		(0.7950819672, 0.3196721311),
		(0.7950819672, 0.3114754098),
		(0.7950819672, 0.2049180328),
		(0.7950819672, 0.1967213115),
		(0.7950819672, 0.1885245902),
		(0.7950819672, 0.1803278689),
		(0.7950819672, 0.1721311475),
		(0.7950819672, 0.1639344262),
		(0.7950819672, 0.1065573770),
		(0.7950819672, 0.0983606557),
		(0.7950819672, 0.0901639344),
		(0.7950819672, 0.0819672131),
		(0.7950819672, 0.0737704918),
		(0.7950819672, 0.0655737705),
		(0.8032786885, 0.5000000000),
		(0.8032786885, 0.4918032787),
		(0.8032786885, 0.4836065574),
		(0.8032786885, 0.4754098361),
		(0.8032786885, 0.4672131148),
		(0.8032786885, 0.4590163934),
		(0.8032786885, 0.4508196721),
		(0.8032786885, 0.4426229508),
		(0.8032786885, 0.4344262295),
		(0.8032786885, 0.4262295082),
		(0.8032786885, 0.4180327869),
		(0.8032786885, 0.4098360656),
		(0.8032786885, 0.3524590164),
		(0.8032786885, 0.3442622951),
		(0.8032786885, 0.3360655738),
		(0.8032786885, 0.3278688525),
		(0.8032786885, 0.3196721311),
		(0.8032786885, 0.3114754098),
		(0.8032786885, 0.2049180328),
		(0.8032786885, 0.1967213115),
		(0.8032786885, 0.1885245902),
		(0.8032786885, 0.1803278689),
		(0.8032786885, 0.1721311475),
		(0.8032786885, 0.1639344262),
		(0.8032786885, 0.1065573770),
		(0.8032786885, 0.0983606557),
		(0.8032786885, 0.0901639344),
		(0.8032786885, 0.0819672131),
		(0.8032786885, 0.0737704918),
		(0.8032786885, 0.0655737705),
		(0.8114754098, 0.5000000000),
		(0.8114754098, 0.4918032787),
		(0.8114754098, 0.4836065574),
		(0.8114754098, 0.4754098361),
		(0.8114754098, 0.4672131148),
		(0.8114754098, 0.4590163934),
		(0.8114754098, 0.4508196721),
		(0.8114754098, 0.4426229508),
		(0.8114754098, 0.4344262295),
		(0.8114754098, 0.4262295082),
		(0.8114754098, 0.4180327869),
		(0.8114754098, 0.4098360656),
		(0.8114754098, 0.3524590164),
		(0.8114754098, 0.3442622951),
		(0.8114754098, 0.3360655738),
		(0.8114754098, 0.3278688525),
		(0.8114754098, 0.3196721311),
		(0.8114754098, 0.3114754098),
		(0.8114754098, 0.2049180328),
		(0.8114754098, 0.1967213115),
		(0.8114754098, 0.1885245902),
		(0.8114754098, 0.1803278689),
		(0.8114754098, 0.1721311475),
		(0.8114754098, 0.1639344262),
		(0.8114754098, 0.1065573770),
		(0.8114754098, 0.0983606557),
		(0.8114754098, 0.0901639344),
		(0.8114754098, 0.0819672131),
		(0.8114754098, 0.0737704918),
		(0.8114754098, 0.0655737705),
		(0.8196721311, 0.5000000000),
		(0.8196721311, 0.4918032787),
		(0.8196721311, 0.4836065574),
		(0.8196721311, 0.4754098361),
		(0.8196721311, 0.4672131148),
		(0.8196721311, 0.4590163934),
		(0.8196721311, 0.4508196721),
		(0.8196721311, 0.4426229508),
		(0.8196721311, 0.4344262295),
		(0.8196721311, 0.4262295082),
		(0.8196721311, 0.4180327869),
		(0.8196721311, 0.4098360656),
		(0.8196721311, 0.3524590164),
		(0.8196721311, 0.3442622951),
		(0.8196721311, 0.3360655738),
		(0.8196721311, 0.3278688525),
		(0.8196721311, 0.3196721311),
		(0.8196721311, 0.3114754098),
		(0.8196721311, 0.2049180328),
		(0.8196721311, 0.1967213115),
		(0.8196721311, 0.1885245902),
		(0.8196721311, 0.1803278689),
		(0.8196721311, 0.1721311475),
		(0.8196721311, 0.1639344262),
		(0.8196721311, 0.1065573770),
		(0.8196721311, 0.0983606557),
		(0.8196721311, 0.0901639344),
		(0.8196721311, 0.0819672131),
		(0.8196721311, 0.0737704918),
		(0.8196721311, 0.0655737705),
		(0.8278688525, 0.5000000000),
		(0.8278688525, 0.4918032787),
		(0.8278688525, 0.4836065574),
		(0.8278688525, 0.4754098361),
		(0.8278688525, 0.4672131148),
		(0.8278688525, 0.4590163934),
		(0.8278688525, 0.4508196721),
		(0.8278688525, 0.4426229508),
		(0.8278688525, 0.4344262295),
		(0.8278688525, 0.4262295082),
		(0.8278688525, 0.4180327869),
		(0.8278688525, 0.4098360656),
		(0.8278688525, 0.3524590164),
		(0.8278688525, 0.3442622951),
		(0.8278688525, 0.3360655738),
		(0.8278688525, 0.3278688525),
		(0.8278688525, 0.3196721311),
		(0.8278688525, 0.3114754098),
		(0.8278688525, 0.2049180328),
		(0.8278688525, 0.1967213115),
		(0.8278688525, 0.1885245902),
		(0.8278688525, 0.1803278689),
		(0.8278688525, 0.1721311475),
		(0.8278688525, 0.1639344262),
		(0.8278688525, 0.1065573770),
		(0.8278688525, 0.0983606557),
		(0.8278688525, 0.0901639344),
		(0.8278688525, 0.0819672131),
		(0.8278688525, 0.0737704918),
		(0.8278688525, 0.0655737705),
		(0.8360655738, 0.5000000000),
		(0.8360655738, 0.4918032787),
		(0.8360655738, 0.4836065574),
		(0.8360655738, 0.4754098361),
		(0.8360655738, 0.4672131148),
		(0.8360655738, 0.4590163934),
		(0.8360655738, 0.4508196721),
		(0.8360655738, 0.4426229508),
		(0.8360655738, 0.4344262295),
		(0.8360655738, 0.4262295082),
		(0.8360655738, 0.4180327869),
		(0.8360655738, 0.4098360656),
		(0.8360655738, 0.3524590164),
		(0.8360655738, 0.3442622951),
		(0.8360655738, 0.3360655738),
		(0.8360655738, 0.3278688525),
		(0.8360655738, 0.3196721311),
		(0.8360655738, 0.3114754098),
		(0.8360655738, 0.2049180328),
		(0.8360655738, 0.1967213115),
		(0.8360655738, 0.1885245902),
		(0.8360655738, 0.1803278689),
		(0.8360655738, 0.1721311475),
		(0.8360655738, 0.1639344262),
		(0.8360655738, 0.1065573770),
		(0.8360655738, 0.0983606557),
		(0.8360655738, 0.0901639344),
		(0.8360655738, 0.0819672131),
		(0.8360655738, 0.0737704918),
		(0.8360655738, 0.0655737705),
		(0.8442622951, 0.3032786885),
		(0.8442622951, 0.2950819672),
		(0.8442622951, 0.2868852459),
		(0.8442622951, 0.2786885246),
		(0.8442622951, 0.2704918033),
		(0.8442622951, 0.2622950820),
		(0.8442622951, 0.2540983607),
		(0.8442622951, 0.2459016393),
		(0.8442622951, 0.2377049180),
		(0.8442622951, 0.2295081967),
		(0.8442622951, 0.2213114754),
		(0.8442622951, 0.2131147541),
		(0.8442622951, 0.1557377049),
		(0.8442622951, 0.1475409836),
		(0.8442622951, 0.1393442623),
		(0.8442622951, 0.1311475410),
		(0.8442622951, 0.1229508197),
		(0.8442622951, 0.1147540984),
		(0.8524590164, 0.3032786885),
		(0.8524590164, 0.2950819672),
		(0.8524590164, 0.2868852459),
		(0.8524590164, 0.2786885246),
		(0.8524590164, 0.2704918033),
		(0.8524590164, 0.2622950820),
		(0.8524590164, 0.2540983607),
		(0.8524590164, 0.2459016393),
		(0.8524590164, 0.2377049180),
		(0.8524590164, 0.2295081967),
		(0.8524590164, 0.2213114754),
		(0.8524590164, 0.2131147541),
		(0.8524590164, 0.1557377049),
		(0.8524590164, 0.1475409836),
		(0.8524590164, 0.1393442623),
		(0.8524590164, 0.1311475410),
		(0.8524590164, 0.1229508197),
		(0.8524590164, 0.1147540984),
		(0.8606557377, 0.3032786885),
		(0.8606557377, 0.2950819672),
		(0.8606557377, 0.2868852459),
		(0.8606557377, 0.2786885246),
		(0.8606557377, 0.2704918033),
		(0.8606557377, 0.2622950820),
		(0.8606557377, 0.2540983607),
		(0.8606557377, 0.2459016393),
		(0.8606557377, 0.2377049180),
		(0.8606557377, 0.2295081967),
		(0.8606557377, 0.2213114754),
		(0.8606557377, 0.2131147541),
		(0.8606557377, 0.1557377049),
		(0.8606557377, 0.1475409836),
		(0.8606557377, 0.1393442623),
		(0.8606557377, 0.1311475410),
		(0.8606557377, 0.1229508197),
		(0.8606557377, 0.1147540984),
		(0.8688524590, 0.3032786885),
		(0.8688524590, 0.2950819672),
		(0.8688524590, 0.2868852459),
		(0.8688524590, 0.2786885246),
		(0.8688524590, 0.2704918033),
		(0.8688524590, 0.2622950820),
		(0.8688524590, 0.2540983607),
		(0.8688524590, 0.2459016393),
		(0.8688524590, 0.2377049180),
		(0.8688524590, 0.2295081967),
		(0.8688524590, 0.2213114754),
		(0.8688524590, 0.2131147541),
		(0.8688524590, 0.1557377049),
		(0.8688524590, 0.1475409836),
		(0.8688524590, 0.1393442623),
		(0.8688524590, 0.1311475410),
		(0.8688524590, 0.1229508197),
		(0.8688524590, 0.1147540984),
		(0.8770491803, 0.3032786885),
		(0.8770491803, 0.2950819672),
		(0.8770491803, 0.2868852459),
		(0.8770491803, 0.2786885246),
		(0.8770491803, 0.2704918033),
		(0.8770491803, 0.2622950820),
		(0.8770491803, 0.2540983607),
		(0.8770491803, 0.2459016393),
		(0.8770491803, 0.2377049180),
		(0.8770491803, 0.2295081967),
		(0.8770491803, 0.2213114754),
		(0.8770491803, 0.2131147541),
		(0.8770491803, 0.1557377049),
		(0.8770491803, 0.1475409836),
		(0.8770491803, 0.1393442623),
		(0.8770491803, 0.1311475410),
		(0.8770491803, 0.1229508197),
		(0.8770491803, 0.1147540984),
		(0.8852459016, 0.3032786885),
		(0.8852459016, 0.2950819672),
		(0.8852459016, 0.2868852459),
		(0.8852459016, 0.2786885246),
		(0.8852459016, 0.2704918033),
		(0.8852459016, 0.2622950820),
		(0.8852459016, 0.2540983607),
		(0.8852459016, 0.2459016393),
		(0.8852459016, 0.2377049180),
		(0.8852459016, 0.2295081967),
		(0.8852459016, 0.2213114754),
		(0.8852459016, 0.2131147541),
		(0.8852459016, 0.1557377049),
		(0.8852459016, 0.1475409836),
		(0.8852459016, 0.1393442623),
		(0.8852459016, 0.1311475410),
		(0.8852459016, 0.1229508197),
		(0.8852459016, 0.1147540984),
		(0.8934426230, 0.2049180328),
		(0.8934426230, 0.1967213115),
		(0.8934426230, 0.1885245902),
		(0.8934426230, 0.1803278689),
		(0.8934426230, 0.1721311475),
		(0.8934426230, 0.1639344262),
		(0.8934426230, 0.1065573770),
		(0.8934426230, 0.0983606557),
		(0.8934426230, 0.0901639344),
		(0.8934426230, 0.0819672131),
		(0.8934426230, 0.0737704918),
		(0.8934426230, 0.0655737705),
		(0.8934426230, 0.0573770492),
		(0.8934426230, 0.0491803279),
		(0.8934426230, 0.0409836066),
		(0.8934426230, 0.0327868852),
		(0.8934426230, 0.0245901639),
		(0.8934426230, 0.0163934426),
		(0.9016393443, 0.2049180328),
		(0.9016393443, 0.1967213115),
		(0.9016393443, 0.1885245902),
		(0.9016393443, 0.1803278689),
		(0.9016393443, 0.1721311475),
		(0.9016393443, 0.1639344262),
		(0.9016393443, 0.1065573770),
		(0.9016393443, 0.0983606557),
		(0.9016393443, 0.0901639344),
		(0.9016393443, 0.0819672131),
		(0.9016393443, 0.0737704918),
		(0.9016393443, 0.0655737705),
		(0.9016393443, 0.0573770492),
		(0.9016393443, 0.0491803279),
		(0.9016393443, 0.0409836066),
		(0.9016393443, 0.0327868852),
		(0.9016393443, 0.0245901639),
		(0.9016393443, 0.0163934426),
		(0.9098360656, 0.2049180328),
		(0.9098360656, 0.1967213115),
		(0.9098360656, 0.1885245902),
		(0.9098360656, 0.1803278689),
		(0.9098360656, 0.1721311475),
		(0.9098360656, 0.1639344262),
		(0.9098360656, 0.1065573770),
		(0.9098360656, 0.0983606557),
		(0.9098360656, 0.0901639344),
		(0.9098360656, 0.0819672131),
		(0.9098360656, 0.0737704918),
		(0.9098360656, 0.0655737705),
		(0.9098360656, 0.0573770492),
		(0.9098360656, 0.0491803279),
		(0.9098360656, 0.0409836066),
		(0.9098360656, 0.0327868852),
		(0.9098360656, 0.0245901639),
		(0.9098360656, 0.0163934426),
		(0.9180327869, 0.2049180328),
		(0.9180327869, 0.1967213115),
		(0.9180327869, 0.1885245902),
		(0.9180327869, 0.1803278689),
		(0.9180327869, 0.1721311475),
		(0.9180327869, 0.1639344262),
		(0.9180327869, 0.1065573770),
		(0.9180327869, 0.0983606557),
		(0.9180327869, 0.0901639344),
		(0.9180327869, 0.0819672131),
		(0.9180327869, 0.0737704918),
		(0.9180327869, 0.0655737705),
		(0.9180327869, 0.0573770492),
		(0.9180327869, 0.0491803279),
		(0.9180327869, 0.0409836066),
		(0.9180327869, 0.0327868852),
		(0.9180327869, 0.0245901639),
		(0.9180327869, 0.0163934426),
		(0.9262295082, 0.2049180328),
		(0.9262295082, 0.1967213115),
		(0.9262295082, 0.1885245902),
		(0.9262295082, 0.1803278689),
		(0.9262295082, 0.1721311475),
		(0.9262295082, 0.1639344262),
		(0.9262295082, 0.1065573770),
		(0.9262295082, 0.0983606557),
		(0.9262295082, 0.0901639344),
		(0.9262295082, 0.0819672131),
		(0.9262295082, 0.0737704918),
		(0.9262295082, 0.0655737705),
		(0.9262295082, 0.0573770492),
		(0.9262295082, 0.0491803279),
		(0.9262295082, 0.0409836066),
		(0.9262295082, 0.0327868852),
		(0.9262295082, 0.0245901639),
		(0.9262295082, 0.0163934426),
		(0.9344262295, 0.2049180328),
		(0.9344262295, 0.1967213115),
		(0.9344262295, 0.1885245902),
		(0.9344262295, 0.1803278689),
		(0.9344262295, 0.1721311475),
		(0.9344262295, 0.1639344262),
		(0.9344262295, 0.1065573770),
		(0.9344262295, 0.0983606557),
		(0.9344262295, 0.0901639344),
		(0.9344262295, 0.0819672131),
		(0.9344262295, 0.0737704918),
		(0.9344262295, 0.0655737705),
		(0.9344262295, 0.0573770492),
		(0.9344262295, 0.0491803279),
		(0.9344262295, 0.0409836066),
		(0.9344262295, 0.0327868852),
		(0.9344262295, 0.0245901639),
		(0.9344262295, 0.0163934426),
		(0.9426229508, 0.5000000000),
		(0.9426229508, 0.4918032787),
		(0.9426229508, 0.4836065574),
		(0.9426229508, 0.4754098361),
		(0.9426229508, 0.4672131148),
		(0.9426229508, 0.4590163934),
		(0.9426229508, 0.4016393443),
		(0.9426229508, 0.3934426230),
		(0.9426229508, 0.3852459016),
		(0.9426229508, 0.3770491803),
		(0.9426229508, 0.3688524590),
		(0.9426229508, 0.3606557377),
		(0.9426229508, 0.1065573770),
		(0.9426229508, 0.0983606557),
		(0.9426229508, 0.0901639344),
		(0.9426229508, 0.0819672131),
		(0.9426229508, 0.0737704918),
		(0.9426229508, 0.0655737705),
		(0.9426229508, 0.0573770492),
		(0.9426229508, 0.0491803279),
		(0.9426229508, 0.0409836066),
		(0.9426229508, 0.0327868852),
		(0.9426229508, 0.0245901639),
		(0.9426229508, 0.0163934426),
		(0.9508196721, 0.5000000000),
		(0.9508196721, 0.4918032787),
		(0.9508196721, 0.4836065574),
		(0.9508196721, 0.4754098361),
		(0.9508196721, 0.4672131148),
		(0.9508196721, 0.4590163934),
		(0.9508196721, 0.4016393443),
		(0.9508196721, 0.3934426230),
		(0.9508196721, 0.3852459016),
		(0.9508196721, 0.3770491803),
		(0.9508196721, 0.3688524590),
		(0.9508196721, 0.3606557377),
		(0.9508196721, 0.1065573770),
		(0.9508196721, 0.0983606557),
		(0.9508196721, 0.0901639344),
		(0.9508196721, 0.0819672131),
		(0.9508196721, 0.0737704918),
		(0.9508196721, 0.0655737705),
		(0.9508196721, 0.0573770492),
		(0.9508196721, 0.0491803279),
		(0.9508196721, 0.0409836066),
		(0.9508196721, 0.0327868852),
		(0.9508196721, 0.0245901639),
		(0.9508196721, 0.0163934426),
		(0.9590163934, 0.5000000000),
		(0.9590163934, 0.4918032787),
		(0.9590163934, 0.4836065574),
		(0.9590163934, 0.4754098361),
		(0.9590163934, 0.4672131148),
		(0.9590163934, 0.4590163934),
		(0.9590163934, 0.4016393443),
		(0.9590163934, 0.3934426230),
		(0.9590163934, 0.3852459016),
		(0.9590163934, 0.3770491803),
		(0.9590163934, 0.3688524590),
		(0.9590163934, 0.3606557377),
		(0.9590163934, 0.1065573770),
		(0.9590163934, 0.0983606557),
		(0.9590163934, 0.0901639344),
		(0.9590163934, 0.0819672131),
		(0.9590163934, 0.0737704918),
		(0.9590163934, 0.0655737705),
		(0.9590163934, 0.0573770492),
		(0.9590163934, 0.0491803279),
		(0.9590163934, 0.0409836066),
		(0.9590163934, 0.0327868852),
		(0.9590163934, 0.0245901639),
		(0.9590163934, 0.0163934426),
		(0.9672131148, 0.5000000000),
		(0.9672131148, 0.4918032787),
		(0.9672131148, 0.4836065574),
		(0.9672131148, 0.4754098361),
		(0.9672131148, 0.4672131148),
		(0.9672131148, 0.4590163934),
		(0.9672131148, 0.4016393443),
		(0.9672131148, 0.3934426230),
		(0.9672131148, 0.3852459016),
		(0.9672131148, 0.3770491803),
		(0.9672131148, 0.3688524590),
		(0.9672131148, 0.3606557377),
		(0.9672131148, 0.1065573770),
		(0.9672131148, 0.0983606557),
		(0.9672131148, 0.0901639344),
		(0.9672131148, 0.0819672131),
		(0.9672131148, 0.0737704918),
		(0.9672131148, 0.0655737705),
		(0.9672131148, 0.0573770492),
		(0.9672131148, 0.0491803279),
		(0.9672131148, 0.0409836066),
		(0.9672131148, 0.0327868852),
		(0.9672131148, 0.0245901639),
		(0.9672131148, 0.0163934426),
		(0.9754098361, 0.5000000000),
		(0.9754098361, 0.4918032787),
		(0.9754098361, 0.4836065574),
		(0.9754098361, 0.4754098361),
		(0.9754098361, 0.4672131148),
		(0.9754098361, 0.4590163934),
		(0.9754098361, 0.4016393443),
		(0.9754098361, 0.3934426230),
		(0.9754098361, 0.3852459016),
		(0.9754098361, 0.3770491803),
		(0.9754098361, 0.3688524590),
		(0.9754098361, 0.3606557377),
		(0.9754098361, 0.1065573770),
		(0.9754098361, 0.0983606557),
		(0.9754098361, 0.0901639344),
		(0.9754098361, 0.0819672131),
		(0.9754098361, 0.0737704918),
		(0.9754098361, 0.0655737705),
		(0.9754098361, 0.0573770492),
		(0.9754098361, 0.0491803279),
		(0.9754098361, 0.0409836066),
		(0.9754098361, 0.0327868852),
		(0.9754098361, 0.0245901639),
		(0.9754098361, 0.0163934426),
		(0.9836065574, 0.5000000000),
		(0.9836065574, 0.4918032787),
		(0.9836065574, 0.4836065574),
		(0.9836065574, 0.4754098361),
		(0.9836065574, 0.4672131148),
		(0.9836065574, 0.4590163934),
		(0.9836065574, 0.4016393443),
		(0.9836065574, 0.3934426230),
		(0.9836065574, 0.3852459016),
		(0.9836065574, 0.3770491803),
		(0.9836065574, 0.3688524590),
		(0.9836065574, 0.3606557377),
		(0.9836065574, 0.1065573770),
		(0.9836065574, 0.0983606557),
		(0.9836065574, 0.0901639344),
		(0.9836065574, 0.0819672131),
		(0.9836065574, 0.0737704918),
		(0.9836065574, 0.0655737705),
		(0.9836065574, 0.0573770492),
		(0.9836065574, 0.0491803279),
		(0.9836065574, 0.0409836066),
		(0.9836065574, 0.0327868852),
		(0.9836065574, 0.0245901639),
		(0.9836065574, 0.0163934426)
    } {
        \draw[color=\documentcolor, fill=\documentcolor] \point circle (0.001);
    }
    
    % Points.
    \foreach \point in {
        (0.0081967213, 0.9918032787),
		(0.0081967213, 0.9836065574),
		(0.0081967213, 0.9754098361),
		(0.0081967213, 0.9672131148),
		(0.0081967213, 0.9590163934),
		(0.0081967213, 0.9508196721),
		(0.0163934426, 0.9918032787),
		(0.0163934426, 0.9836065574),
		(0.0163934426, 0.9754098361),
		(0.0163934426, 0.9672131148),
		(0.0163934426, 0.9590163934),
		(0.0163934426, 0.9508196721),
		(0.0245901639, 0.9918032787),
		(0.0245901639, 0.9836065574),
		(0.0245901639, 0.9754098361),
		(0.0245901639, 0.9672131148),
		(0.0245901639, 0.9590163934),
		(0.0245901639, 0.9508196721),
		(0.0327868852, 0.9918032787),
		(0.0327868852, 0.9836065574),
		(0.0327868852, 0.9754098361),
		(0.0327868852, 0.9672131148),
		(0.0327868852, 0.9590163934),
		(0.0327868852, 0.9508196721),
		(0.0409836066, 0.9918032787),
		(0.0409836066, 0.9836065574),
		(0.0409836066, 0.9754098361),
		(0.0409836066, 0.9672131148),
		(0.0409836066, 0.9590163934),
		(0.0409836066, 0.9508196721),
		(0.0491803279, 0.9918032787),
		(0.0491803279, 0.9836065574),
		(0.0491803279, 0.9754098361),
		(0.0491803279, 0.9672131148),
		(0.0491803279, 0.9590163934),
		(0.0491803279, 0.9508196721),
		(0.0573770492, 0.9426229508),
		(0.0573770492, 0.9344262295),
		(0.0573770492, 0.9262295082),
		(0.0573770492, 0.9180327869),
		(0.0573770492, 0.9098360656),
		(0.0573770492, 0.9016393443),
		(0.0655737705, 0.9426229508),
		(0.0655737705, 0.9344262295),
		(0.0655737705, 0.9262295082),
		(0.0655737705, 0.9180327869),
		(0.0655737705, 0.9098360656),
		(0.0655737705, 0.9016393443),
		(0.0737704918, 0.9426229508),
		(0.0737704918, 0.9344262295),
		(0.0737704918, 0.9262295082),
		(0.0737704918, 0.9180327869),
		(0.0737704918, 0.9098360656),
		(0.0737704918, 0.9016393443),
		(0.0819672131, 0.9426229508),
		(0.0819672131, 0.9344262295),
		(0.0819672131, 0.9262295082),
		(0.0819672131, 0.9180327869),
		(0.0819672131, 0.9098360656),
		(0.0819672131, 0.9016393443),
		(0.0901639344, 0.9426229508),
		(0.0901639344, 0.9344262295),
		(0.0901639344, 0.9262295082),
		(0.0901639344, 0.9180327869),
		(0.0901639344, 0.9098360656),
		(0.0901639344, 0.9016393443),
		(0.0983606557, 0.9426229508),
		(0.0983606557, 0.9344262295),
		(0.0983606557, 0.9262295082),
		(0.0983606557, 0.9180327869),
		(0.0983606557, 0.9098360656),
		(0.0983606557, 0.9016393443),
		(0.1065573770, 0.8934426230),
		(0.1065573770, 0.8852459016),
		(0.1065573770, 0.8770491803),
		(0.1065573770, 0.8688524590),
		(0.1065573770, 0.8606557377),
		(0.1065573770, 0.8524590164),
		(0.1147540984, 0.8934426230),
		(0.1147540984, 0.8852459016),
		(0.1147540984, 0.8770491803),
		(0.1147540984, 0.8688524590),
		(0.1147540984, 0.8606557377),
		(0.1147540984, 0.8524590164),
		(0.1229508197, 0.8934426230),
		(0.1229508197, 0.8852459016),
		(0.1229508197, 0.8770491803),
		(0.1229508197, 0.8688524590),
		(0.1229508197, 0.8606557377),
		(0.1229508197, 0.8524590164),
		(0.1311475410, 0.8934426230),
		(0.1311475410, 0.8852459016),
		(0.1311475410, 0.8770491803),
		(0.1311475410, 0.8688524590),
		(0.1311475410, 0.8606557377),
		(0.1311475410, 0.8524590164),
		(0.1393442623, 0.8934426230),
		(0.1393442623, 0.8852459016),
		(0.1393442623, 0.8770491803),
		(0.1393442623, 0.8688524590),
		(0.1393442623, 0.8606557377),
		(0.1393442623, 0.8524590164),
		(0.1475409836, 0.8934426230),
		(0.1475409836, 0.8852459016),
		(0.1475409836, 0.8770491803),
		(0.1475409836, 0.8688524590),
		(0.1475409836, 0.8606557377),
		(0.1475409836, 0.8524590164),
		(0.1557377049, 0.8442622951),
		(0.1557377049, 0.8360655738),
		(0.1557377049, 0.8278688525),
		(0.1557377049, 0.8196721311),
		(0.1557377049, 0.8114754098),
		(0.1557377049, 0.8032786885),
		(0.1639344262, 0.8442622951),
		(0.1639344262, 0.8360655738),
		(0.1639344262, 0.8278688525),
		(0.1639344262, 0.8196721311),
		(0.1639344262, 0.8114754098),
		(0.1639344262, 0.8032786885),
		(0.1721311475, 0.8442622951),
		(0.1721311475, 0.8360655738),
		(0.1721311475, 0.8278688525),
		(0.1721311475, 0.8196721311),
		(0.1721311475, 0.8114754098),
		(0.1721311475, 0.8032786885),
		(0.1803278689, 0.8442622951),
		(0.1803278689, 0.8360655738),
		(0.1803278689, 0.8278688525),
		(0.1803278689, 0.8196721311),
		(0.1803278689, 0.8114754098),
		(0.1803278689, 0.8032786885),
		(0.1885245902, 0.8442622951),
		(0.1885245902, 0.8360655738),
		(0.1885245902, 0.8278688525),
		(0.1885245902, 0.8196721311),
		(0.1885245902, 0.8114754098),
		(0.1885245902, 0.8032786885),
		(0.1967213115, 0.8442622951),
		(0.1967213115, 0.8360655738),
		(0.1967213115, 0.8278688525),
		(0.1967213115, 0.8196721311),
		(0.1967213115, 0.8114754098),
		(0.1967213115, 0.8032786885),
		(0.2049180328, 0.7950819672),
		(0.2049180328, 0.7868852459),
		(0.2049180328, 0.7786885246),
		(0.2049180328, 0.7704918033),
		(0.2049180328, 0.7622950820),
		(0.2049180328, 0.7540983607),
		(0.2131147541, 0.7950819672),
		(0.2131147541, 0.7868852459),
		(0.2131147541, 0.7786885246),
		(0.2131147541, 0.7704918033),
		(0.2131147541, 0.7622950820),
		(0.2131147541, 0.7540983607),
		(0.2213114754, 0.7950819672),
		(0.2213114754, 0.7868852459),
		(0.2213114754, 0.7786885246),
		(0.2213114754, 0.7704918033),
		(0.2213114754, 0.7622950820),
		(0.2213114754, 0.7540983607),
		(0.2295081967, 0.7950819672),
		(0.2295081967, 0.7868852459),
		(0.2295081967, 0.7786885246),
		(0.2295081967, 0.7704918033),
		(0.2295081967, 0.7622950820),
		(0.2295081967, 0.7540983607),
		(0.2377049180, 0.7950819672),
		(0.2377049180, 0.7868852459),
		(0.2377049180, 0.7786885246),
		(0.2377049180, 0.7704918033),
		(0.2377049180, 0.7622950820),
		(0.2377049180, 0.7540983607),
		(0.2459016393, 0.7950819672),
		(0.2459016393, 0.7868852459),
		(0.2459016393, 0.7786885246),
		(0.2459016393, 0.7704918033),
		(0.2459016393, 0.7622950820),
		(0.2459016393, 0.7540983607),
		(0.2540983607, 0.7459016393),
		(0.2540983607, 0.7377049180),
		(0.2540983607, 0.7295081967),
		(0.2540983607, 0.7213114754),
		(0.2540983607, 0.7131147541),
		(0.2540983607, 0.7049180328),
		(0.2622950820, 0.7459016393),
		(0.2622950820, 0.7377049180),
		(0.2622950820, 0.7295081967),
		(0.2622950820, 0.7213114754),
		(0.2622950820, 0.7131147541),
		(0.2622950820, 0.7049180328),
		(0.2704918033, 0.7459016393),
		(0.2704918033, 0.7377049180),
		(0.2704918033, 0.7295081967),
		(0.2704918033, 0.7213114754),
		(0.2704918033, 0.7131147541),
		(0.2704918033, 0.7049180328),
		(0.2786885246, 0.7459016393),
		(0.2786885246, 0.7377049180),
		(0.2786885246, 0.7295081967),
		(0.2786885246, 0.7213114754),
		(0.2786885246, 0.7131147541),
		(0.2786885246, 0.7049180328),
		(0.2868852459, 0.7459016393),
		(0.2868852459, 0.7377049180),
		(0.2868852459, 0.7295081967),
		(0.2868852459, 0.7213114754),
		(0.2868852459, 0.7131147541),
		(0.2868852459, 0.7049180328),
		(0.2950819672, 0.7459016393),
		(0.2950819672, 0.7377049180),
		(0.2950819672, 0.7295081967),
		(0.2950819672, 0.7213114754),
		(0.2950819672, 0.7131147541),
		(0.2950819672, 0.7049180328),
		(0.3032786885, 0.6967213115),
		(0.3032786885, 0.6885245902),
		(0.3032786885, 0.6803278689),
		(0.3032786885, 0.6721311475),
		(0.3032786885, 0.6639344262),
		(0.3032786885, 0.6557377049),
		(0.3114754098, 0.6967213115),
		(0.3114754098, 0.6885245902),
		(0.3114754098, 0.6803278689),
		(0.3114754098, 0.6721311475),
		(0.3114754098, 0.6639344262),
		(0.3114754098, 0.6557377049),
		(0.3196721311, 0.6967213115),
		(0.3196721311, 0.6885245902),
		(0.3196721311, 0.6803278689),
		(0.3196721311, 0.6721311475),
		(0.3196721311, 0.6639344262),
		(0.3196721311, 0.6557377049),
		(0.3278688525, 0.6967213115),
		(0.3278688525, 0.6885245902),
		(0.3278688525, 0.6803278689),
		(0.3278688525, 0.6721311475),
		(0.3278688525, 0.6639344262),
		(0.3278688525, 0.6557377049),
		(0.3360655738, 0.6967213115),
		(0.3360655738, 0.6885245902),
		(0.3360655738, 0.6803278689),
		(0.3360655738, 0.6721311475),
		(0.3360655738, 0.6639344262),
		(0.3360655738, 0.6557377049),
		(0.3442622951, 0.6967213115),
		(0.3442622951, 0.6885245902),
		(0.3442622951, 0.6803278689),
		(0.3442622951, 0.6721311475),
		(0.3442622951, 0.6639344262),
		(0.3442622951, 0.6557377049),
		(0.3524590164, 0.6475409836),
		(0.3524590164, 0.6393442623),
		(0.3524590164, 0.6311475410),
		(0.3524590164, 0.6229508197),
		(0.3524590164, 0.6147540984),
		(0.3524590164, 0.6065573770),
		(0.3606557377, 0.6475409836),
		(0.3606557377, 0.6393442623),
		(0.3606557377, 0.6311475410),
		(0.3606557377, 0.6229508197),
		(0.3606557377, 0.6147540984),
		(0.3606557377, 0.6065573770),
		(0.3688524590, 0.6475409836),
		(0.3688524590, 0.6393442623),
		(0.3688524590, 0.6311475410),
		(0.3688524590, 0.6229508197),
		(0.3688524590, 0.6147540984),
		(0.3688524590, 0.6065573770),
		(0.3770491803, 0.6475409836),
		(0.3770491803, 0.6393442623),
		(0.3770491803, 0.6311475410),
		(0.3770491803, 0.6229508197),
		(0.3770491803, 0.6147540984),
		(0.3770491803, 0.6065573770),
		(0.3852459016, 0.6475409836),
		(0.3852459016, 0.6393442623),
		(0.3852459016, 0.6311475410),
		(0.3852459016, 0.6229508197),
		(0.3852459016, 0.6147540984),
		(0.3852459016, 0.6065573770),
		(0.3934426230, 0.6475409836),
		(0.3934426230, 0.6393442623),
		(0.3934426230, 0.6311475410),
		(0.3934426230, 0.6229508197),
		(0.3934426230, 0.6147540984),
		(0.3934426230, 0.6065573770),
		(0.4016393443, 0.5983606557),
		(0.4016393443, 0.5901639344),
		(0.4016393443, 0.5819672131),
		(0.4016393443, 0.5737704918),
		(0.4016393443, 0.5655737705),
		(0.4016393443, 0.5573770492),
		(0.4098360656, 0.5983606557),
		(0.4098360656, 0.5901639344),
		(0.4098360656, 0.5819672131),
		(0.4098360656, 0.5737704918),
		(0.4098360656, 0.5655737705),
		(0.4098360656, 0.5573770492),
		(0.4180327869, 0.5983606557),
		(0.4180327869, 0.5901639344),
		(0.4180327869, 0.5819672131),
		(0.4180327869, 0.5737704918),
		(0.4180327869, 0.5655737705),
		(0.4180327869, 0.5573770492),
		(0.4262295082, 0.5983606557),
		(0.4262295082, 0.5901639344),
		(0.4262295082, 0.5819672131),
		(0.4262295082, 0.5737704918),
		(0.4262295082, 0.5655737705),
		(0.4262295082, 0.5573770492),
		(0.4344262295, 0.5983606557),
		(0.4344262295, 0.5901639344),
		(0.4344262295, 0.5819672131),
		(0.4344262295, 0.5737704918),
		(0.4344262295, 0.5655737705),
		(0.4344262295, 0.5573770492),
		(0.4426229508, 0.5983606557),
		(0.4426229508, 0.5901639344),
		(0.4426229508, 0.5819672131),
		(0.4426229508, 0.5737704918),
		(0.4426229508, 0.5655737705),
		(0.4426229508, 0.5573770492),
		(0.4508196721, 0.5491803279),
		(0.4508196721, 0.5409836066),
		(0.4508196721, 0.5327868852),
		(0.4508196721, 0.5245901639),
		(0.4508196721, 0.5163934426),
		(0.4508196721, 0.5081967213),
		(0.4590163934, 0.5491803279),
		(0.4590163934, 0.5409836066),
		(0.4590163934, 0.5327868852),
		(0.4590163934, 0.5245901639),
		(0.4590163934, 0.5163934426),
		(0.4590163934, 0.5081967213),
		(0.4672131148, 0.5491803279),
		(0.4672131148, 0.5409836066),
		(0.4672131148, 0.5327868852),
		(0.4672131148, 0.5245901639),
		(0.4672131148, 0.5163934426),
		(0.4672131148, 0.5081967213),
		(0.4754098361, 0.5491803279),
		(0.4754098361, 0.5409836066),
		(0.4754098361, 0.5327868852),
		(0.4754098361, 0.5245901639),
		(0.4754098361, 0.5163934426),
		(0.4754098361, 0.5081967213),
		(0.4836065574, 0.5491803279),
		(0.4836065574, 0.5409836066),
		(0.4836065574, 0.5327868852),
		(0.4836065574, 0.5245901639),
		(0.4836065574, 0.5163934426),
		(0.4836065574, 0.5081967213),
		(0.4918032787, 0.5491803279),
		(0.4918032787, 0.5409836066),
		(0.4918032787, 0.5327868852),
		(0.4918032787, 0.5245901639),
		(0.4918032787, 0.5163934426),
		(0.4918032787, 0.5081967213),
		(0.5000000000, 0.5000000000),
		(0.5000000000, 0.4918032787),
		(0.5000000000, 0.4836065574),
		(0.5000000000, 0.4754098361),
		(0.5000000000, 0.4672131148),
		(0.5000000000, 0.4590163934),
		(0.5081967213, 0.5000000000),
		(0.5081967213, 0.4918032787),
		(0.5081967213, 0.4836065574),
		(0.5081967213, 0.4754098361),
		(0.5081967213, 0.4672131148),
		(0.5081967213, 0.4590163934),
		(0.5163934426, 0.5000000000),
		(0.5163934426, 0.4918032787),
		(0.5163934426, 0.4836065574),
		(0.5163934426, 0.4754098361),
		(0.5163934426, 0.4672131148),
		(0.5163934426, 0.4590163934),
		(0.5245901639, 0.5000000000),
		(0.5245901639, 0.4918032787),
		(0.5245901639, 0.4836065574),
		(0.5245901639, 0.4754098361),
		(0.5245901639, 0.4672131148),
		(0.5245901639, 0.4590163934),
		(0.5327868852, 0.5000000000),
		(0.5327868852, 0.4918032787),
		(0.5327868852, 0.4836065574),
		(0.5327868852, 0.4754098361),
		(0.5327868852, 0.4672131148),
		(0.5327868852, 0.4590163934),
		(0.5409836066, 0.5000000000),
		(0.5409836066, 0.4918032787),
		(0.5409836066, 0.4836065574),
		(0.5409836066, 0.4754098361),
		(0.5409836066, 0.4672131148),
		(0.5409836066, 0.4590163934),
		(0.5491803279, 0.4508196721),
		(0.5491803279, 0.4426229508),
		(0.5491803279, 0.4344262295),
		(0.5491803279, 0.4262295082),
		(0.5491803279, 0.4180327869),
		(0.5491803279, 0.4098360656),
		(0.5573770492, 0.4508196721),
		(0.5573770492, 0.4426229508),
		(0.5573770492, 0.4344262295),
		(0.5573770492, 0.4262295082),
		(0.5573770492, 0.4180327869),
		(0.5573770492, 0.4098360656),
		(0.5655737705, 0.4508196721),
		(0.5655737705, 0.4426229508),
		(0.5655737705, 0.4344262295),
		(0.5655737705, 0.4262295082),
		(0.5655737705, 0.4180327869),
		(0.5655737705, 0.4098360656),
		(0.5737704918, 0.4508196721),
		(0.5737704918, 0.4426229508),
		(0.5737704918, 0.4344262295),
		(0.5737704918, 0.4262295082),
		(0.5737704918, 0.4180327869),
		(0.5737704918, 0.4098360656),
		(0.5819672131, 0.4508196721),
		(0.5819672131, 0.4426229508),
		(0.5819672131, 0.4344262295),
		(0.5819672131, 0.4262295082),
		(0.5819672131, 0.4180327869),
		(0.5819672131, 0.4098360656),
		(0.5901639344, 0.4508196721),
		(0.5901639344, 0.4426229508),
		(0.5901639344, 0.4344262295),
		(0.5901639344, 0.4262295082),
		(0.5901639344, 0.4180327869),
		(0.5901639344, 0.4098360656),
		(0.5983606557, 0.4016393443),
		(0.5983606557, 0.3934426230),
		(0.5983606557, 0.3852459016),
		(0.5983606557, 0.3770491803),
		(0.5983606557, 0.3688524590),
		(0.5983606557, 0.3606557377),
		(0.6065573770, 0.4016393443),
		(0.6065573770, 0.3934426230),
		(0.6065573770, 0.3852459016),
		(0.6065573770, 0.3770491803),
		(0.6065573770, 0.3688524590),
		(0.6065573770, 0.3606557377),
		(0.6147540984, 0.4016393443),
		(0.6147540984, 0.3934426230),
		(0.6147540984, 0.3852459016),
		(0.6147540984, 0.3770491803),
		(0.6147540984, 0.3688524590),
		(0.6147540984, 0.3606557377),
		(0.6229508197, 0.4016393443),
		(0.6229508197, 0.3934426230),
		(0.6229508197, 0.3852459016),
		(0.6229508197, 0.3770491803),
		(0.6229508197, 0.3688524590),
		(0.6229508197, 0.3606557377),
		(0.6311475410, 0.4016393443),
		(0.6311475410, 0.3934426230),
		(0.6311475410, 0.3852459016),
		(0.6311475410, 0.3770491803),
		(0.6311475410, 0.3688524590),
		(0.6311475410, 0.3606557377),
		(0.6393442623, 0.4016393443),
		(0.6393442623, 0.3934426230),
		(0.6393442623, 0.3852459016),
		(0.6393442623, 0.3770491803),
		(0.6393442623, 0.3688524590),
		(0.6393442623, 0.3606557377),
		(0.6475409836, 0.3524590164),
		(0.6475409836, 0.3442622951),
		(0.6475409836, 0.3360655738),
		(0.6475409836, 0.3278688525),
		(0.6475409836, 0.3196721311),
		(0.6475409836, 0.3114754098),
		(0.6557377049, 0.3524590164),
		(0.6557377049, 0.3442622951),
		(0.6557377049, 0.3360655738),
		(0.6557377049, 0.3278688525),
		(0.6557377049, 0.3196721311),
		(0.6557377049, 0.3114754098),
		(0.6639344262, 0.3524590164),
		(0.6639344262, 0.3442622951),
		(0.6639344262, 0.3360655738),
		(0.6639344262, 0.3278688525),
		(0.6639344262, 0.3196721311),
		(0.6639344262, 0.3114754098),
		(0.6721311475, 0.3524590164),
		(0.6721311475, 0.3442622951),
		(0.6721311475, 0.3360655738),
		(0.6721311475, 0.3278688525),
		(0.6721311475, 0.3196721311),
		(0.6721311475, 0.3114754098),
		(0.6803278689, 0.3524590164),
		(0.6803278689, 0.3442622951),
		(0.6803278689, 0.3360655738),
		(0.6803278689, 0.3278688525),
		(0.6803278689, 0.3196721311),
		(0.6803278689, 0.3114754098),
		(0.6885245902, 0.3524590164),
		(0.6885245902, 0.3442622951),
		(0.6885245902, 0.3360655738),
		(0.6885245902, 0.3278688525),
		(0.6885245902, 0.3196721311),
		(0.6885245902, 0.3114754098),
		(0.6967213115, 0.3032786885),
		(0.6967213115, 0.2950819672),
		(0.6967213115, 0.2868852459),
		(0.6967213115, 0.2786885246),
		(0.6967213115, 0.2704918033),
		(0.6967213115, 0.2622950820),
		(0.7049180328, 0.3032786885),
		(0.7049180328, 0.2950819672),
		(0.7049180328, 0.2868852459),
		(0.7049180328, 0.2786885246),
		(0.7049180328, 0.2704918033),
		(0.7049180328, 0.2622950820),
		(0.7131147541, 0.3032786885),
		(0.7131147541, 0.2950819672),
		(0.7131147541, 0.2868852459),
		(0.7131147541, 0.2786885246),
		(0.7131147541, 0.2704918033),
		(0.7131147541, 0.2622950820),
		(0.7213114754, 0.3032786885),
		(0.7213114754, 0.2950819672),
		(0.7213114754, 0.2868852459),
		(0.7213114754, 0.2786885246),
		(0.7213114754, 0.2704918033),
		(0.7213114754, 0.2622950820),
		(0.7295081967, 0.3032786885),
		(0.7295081967, 0.2950819672),
		(0.7295081967, 0.2868852459),
		(0.7295081967, 0.2786885246),
		(0.7295081967, 0.2704918033),
		(0.7295081967, 0.2622950820),
		(0.7377049180, 0.3032786885),
		(0.7377049180, 0.2950819672),
		(0.7377049180, 0.2868852459),
		(0.7377049180, 0.2786885246),
		(0.7377049180, 0.2704918033),
		(0.7377049180, 0.2622950820),
		(0.7459016393, 0.2540983607),
		(0.7459016393, 0.2459016393),
		(0.7459016393, 0.2377049180),
		(0.7459016393, 0.2295081967),
		(0.7459016393, 0.2213114754),
		(0.7459016393, 0.2131147541),
		(0.7540983607, 0.2540983607),
		(0.7540983607, 0.2459016393),
		(0.7540983607, 0.2377049180),
		(0.7540983607, 0.2295081967),
		(0.7540983607, 0.2213114754),
		(0.7540983607, 0.2131147541),
		(0.7622950820, 0.2540983607),
		(0.7622950820, 0.2459016393),
		(0.7622950820, 0.2377049180),
		(0.7622950820, 0.2295081967),
		(0.7622950820, 0.2213114754),
		(0.7622950820, 0.2131147541),
		(0.7704918033, 0.2540983607),
		(0.7704918033, 0.2459016393),
		(0.7704918033, 0.2377049180),
		(0.7704918033, 0.2295081967),
		(0.7704918033, 0.2213114754),
		(0.7704918033, 0.2131147541),
		(0.7786885246, 0.2540983607),
		(0.7786885246, 0.2459016393),
		(0.7786885246, 0.2377049180),
		(0.7786885246, 0.2295081967),
		(0.7786885246, 0.2213114754),
		(0.7786885246, 0.2131147541),
		(0.7868852459, 0.2540983607),
		(0.7868852459, 0.2459016393),
		(0.7868852459, 0.2377049180),
		(0.7868852459, 0.2295081967),
		(0.7868852459, 0.2213114754),
		(0.7868852459, 0.2131147541),
		(0.7950819672, 0.2049180328),
		(0.7950819672, 0.1967213115),
		(0.7950819672, 0.1885245902),
		(0.7950819672, 0.1803278689),
		(0.7950819672, 0.1721311475),
		(0.7950819672, 0.1639344262),
		(0.8032786885, 0.2049180328),
		(0.8032786885, 0.1967213115),
		(0.8032786885, 0.1885245902),
		(0.8032786885, 0.1803278689),
		(0.8032786885, 0.1721311475),
		(0.8032786885, 0.1639344262),
		(0.8114754098, 0.2049180328),
		(0.8114754098, 0.1967213115),
		(0.8114754098, 0.1885245902),
		(0.8114754098, 0.1803278689),
		(0.8114754098, 0.1721311475),
		(0.8114754098, 0.1639344262),
		(0.8196721311, 0.2049180328),
		(0.8196721311, 0.1967213115),
		(0.8196721311, 0.1885245902),
		(0.8196721311, 0.1803278689),
		(0.8196721311, 0.1721311475),
		(0.8196721311, 0.1639344262),
		(0.8278688525, 0.2049180328),
		(0.8278688525, 0.1967213115),
		(0.8278688525, 0.1885245902),
		(0.8278688525, 0.1803278689),
		(0.8278688525, 0.1721311475),
		(0.8278688525, 0.1639344262),
		(0.8360655738, 0.2049180328),
		(0.8360655738, 0.1967213115),
		(0.8360655738, 0.1885245902),
		(0.8360655738, 0.1803278689),
		(0.8360655738, 0.1721311475),
		(0.8360655738, 0.1639344262),
		(0.8442622951, 0.1557377049),
		(0.8442622951, 0.1475409836),
		(0.8442622951, 0.1393442623),
		(0.8442622951, 0.1311475410),
		(0.8442622951, 0.1229508197),
		(0.8442622951, 0.1147540984),
		(0.8524590164, 0.1557377049),
		(0.8524590164, 0.1475409836),
		(0.8524590164, 0.1393442623),
		(0.8524590164, 0.1311475410),
		(0.8524590164, 0.1229508197),
		(0.8524590164, 0.1147540984),
		(0.8606557377, 0.1557377049),
		(0.8606557377, 0.1475409836),
		(0.8606557377, 0.1393442623),
		(0.8606557377, 0.1311475410),
		(0.8606557377, 0.1229508197),
		(0.8606557377, 0.1147540984),
		(0.8688524590, 0.1557377049),
		(0.8688524590, 0.1475409836),
		(0.8688524590, 0.1393442623),
		(0.8688524590, 0.1311475410),
		(0.8688524590, 0.1229508197),
		(0.8688524590, 0.1147540984),
		(0.8770491803, 0.1557377049),
		(0.8770491803, 0.1475409836),
		(0.8770491803, 0.1393442623),
		(0.8770491803, 0.1311475410),
		(0.8770491803, 0.1229508197),
		(0.8770491803, 0.1147540984),
		(0.8852459016, 0.1557377049),
		(0.8852459016, 0.1475409836),
		(0.8852459016, 0.1393442623),
		(0.8852459016, 0.1311475410),
		(0.8852459016, 0.1229508197),
		(0.8852459016, 0.1147540984),
		(0.8934426230, 0.1065573770),
		(0.8934426230, 0.0983606557),
		(0.8934426230, 0.0901639344),
		(0.8934426230, 0.0819672131),
		(0.8934426230, 0.0737704918),
		(0.8934426230, 0.0655737705),
		(0.9016393443, 0.1065573770),
		(0.9016393443, 0.0983606557),
		(0.9016393443, 0.0901639344),
		(0.9016393443, 0.0819672131),
		(0.9016393443, 0.0737704918),
		(0.9016393443, 0.0655737705),
		(0.9098360656, 0.1065573770),
		(0.9098360656, 0.0983606557),
		(0.9098360656, 0.0901639344),
		(0.9098360656, 0.0819672131),
		(0.9098360656, 0.0737704918),
		(0.9098360656, 0.0655737705),
		(0.9180327869, 0.1065573770),
		(0.9180327869, 0.0983606557),
		(0.9180327869, 0.0901639344),
		(0.9180327869, 0.0819672131),
		(0.9180327869, 0.0737704918),
		(0.9180327869, 0.0655737705),
		(0.9262295082, 0.1065573770),
		(0.9262295082, 0.0983606557),
		(0.9262295082, 0.0901639344),
		(0.9262295082, 0.0819672131),
		(0.9262295082, 0.0737704918),
		(0.9262295082, 0.0655737705),
		(0.9344262295, 0.1065573770),
		(0.9344262295, 0.0983606557),
		(0.9344262295, 0.0901639344),
		(0.9344262295, 0.0819672131),
		(0.9344262295, 0.0737704918),
		(0.9344262295, 0.0655737705),
		(0.9426229508, 0.0573770492),
		(0.9426229508, 0.0491803279),
		(0.9426229508, 0.0409836066),
		(0.9426229508, 0.0327868852),
		(0.9426229508, 0.0245901639),
		(0.9426229508, 0.0163934426),
		(0.9508196721, 0.0573770492),
		(0.9508196721, 0.0491803279),
		(0.9508196721, 0.0409836066),
		(0.9508196721, 0.0327868852),
		(0.9508196721, 0.0245901639),
		(0.9508196721, 0.0163934426),
		(0.9590163934, 0.0573770492),
		(0.9590163934, 0.0491803279),
		(0.9590163934, 0.0409836066),
		(0.9590163934, 0.0327868852),
		(0.9590163934, 0.0245901639),
		(0.9590163934, 0.0163934426),
		(0.9672131148, 0.0573770492),
		(0.9672131148, 0.0491803279),
		(0.9672131148, 0.0409836066),
		(0.9672131148, 0.0327868852),
		(0.9672131148, 0.0245901639),
		(0.9672131148, 0.0163934426),
		(0.9754098361, 0.0573770492),
		(0.9754098361, 0.0491803279),
		(0.9754098361, 0.0409836066),
		(0.9754098361, 0.0327868852),
		(0.9754098361, 0.0245901639),
		(0.9754098361, 0.0163934426),
		(0.9836065574, 0.0573770492),
		(0.9836065574, 0.0491803279),
		(0.9836065574, 0.0409836066),
		(0.9836065574, 0.0327868852),
		(0.9836065574, 0.0245901639),
		(0.9836065574, 0.0163934426)
    } {
        \draw[color=\accentcolor, fill=\accentcolor] \point circle (0.001);
    }

	% Domain.
    \draw[thick, color=\documentcolor]
        (-0.0081967213,0) -- (1,0) -- (1,1.0081967213) -- (-0.0081967213,1.0081967213) -- cycle;

\end{tikzpicture}
            \end{figure}
        }
        \vspace*{\fill}

    \end{multicols}
    \vspace*{\fill}
    
\end{frame}

% \begin{frame}
%     \frametitle{Solution of a \acrshort{cdr} Problem}

%     % [!]
    
% \end{frame}

\subsection{Testing the Implementation}

\begin{frame}
    \frametitle{Overview of the Testing Conditions}

    \vspace*{\fill}
    \begin{center}
        {\color{\accentcolor} \Large \textbf{Exact Solution}}
        \vspace*{0.25cm}

        \begin{minipage}{0.75\textwidth}
            \begin{align*}
                u(t, \Vektor{x}) &= \left( 1 - e^{-t} \right) \left( 2 x_1 + 2 x_2 - x_1 x_2 + 2 \left( 1 - e^{\frac{\ConvectionNoVector_1 \left( x_1 - 1 \right) }{\Boundary}} \right) \left( 1 - e^{\frac{\ConvectionNoVector_2 \left( x_2 - 1 \right) }{\Boundary}} \right) \right)
            \end{align*}
        \end{minipage}
    \end{center}

    \vspace*{\fill}

    \begin{center}
        {\color{\accentcolor} \Large \textbf{Meshes}}
        \vspace*{0.25cm}

        \begin{minipage}{0.75\textwidth}
            \begin{table}[!ht]
                \def\arraystretch{1.25}
                \begin{tabular}{ | C{0.75cm} || C{1.75cm} | C{1.75cm} | C{1.75cm} | C{1.75cm} | }
                    \hline
                     & $\nSpaceIndices$ & $\nTimeIndices$ & $h$ & $\tau$ \\
                    \hline
                    \hline
                    1 & $125$ & $7$ & $\num{0.136}$ & $\num{0.143}$ \\
                    2 & $250$ & $10$ & $\num{0.093}$ & $\num{0.100}$ \\
                    3 & $500$ & $14$ & $\num{0.069}$ & $\num{0.071}$ \\
                    4 & $1000$ & $20$ & $\num{0.049}$ & $\num{0.050}$ \\
                    5 & $2000$ & $28$ & $\num{0.038}$ & $\num{0.036}$ \\
                    \hline
                \end{tabular}
            \end{table}
        \end{minipage}
    \end{center}
    \vspace*{\fill}
    
\end{frame}

\begin{frame}
    \frametitle{Overview of the Testing Conditions}

    \vspace*{\fill}
    \begin{center}
        {\color{\accentcolor} \Large \textbf{Coefficients}}
        \vspace*{0.5cm}

        \begin{minipage}{0.75\textwidth}
            \begin{description}
                \item[Convection] $\ConvectionNoVector_1(t, \Vektor{x}) = 1$ and $\ConvectionNoVector_2(t, \Vektor{x}) = 1$
                \item[Reaction] $\Reaction(t, \Vektor{x}) = 0.5$ 
                \item[Diffusion, parabolic] $\Diffusion = 0.005$
                \item[Diffusion, hyperbolic] $\Diffusion = 0$
            \end{description}
        \end{minipage}
    \end{center}

    \vspace*{\fill}

    \begin{center}
        {\color{\accentcolor} \Large \textbf{Norms}}
    \end{center}

    \begin{multicols}{2}

        % Literature Norms.
        \begin{center}
            {\color{\accentcolor} \Large \textbf{Literature Norms}}
            \vspace*{0.5cm}

            \begin{minipage}{0.4\textwidth}
                \begin{itemize}
                    \item $\Norm{\Error}_{\SpaceLp{2}(I; \SpaceLp{2}(\Omega))}$
                    \item {\color{\accentcolor} $\Norm{\sqrt{\Diffusion} \Error}_{\SpaceLp{2}(I; \SpaceHk{1}(\Omega))}$}
                \end{itemize}
            \end{minipage}
        \end{center}

        \vfill\null
        \columnbreak

        % New Norms.
        \begin{center}
            {\color{\accentcolor} \Large \textbf{New Norms}}
            \vspace*{0.5cm}

            \begin{minipage}{0.4\textwidth}
                \begin{itemize}
                    \item $\Norm{\Error}_{\SpaceLp{2}(\left\{ T \right\} \times \Omega)}$
                    \item $\Norm{\Error}_{\SpaceLp{\infty}(I; \SpaceLp{2}(\Omega))}$
                \end{itemize}
            \end{minipage}
        \end{center}
    \end{multicols}
    \vspace*{\fill}
    
\end{frame}

\subsection{Results}

% "Old" norms.

\begin{frame}
    \frametitle{Numerical Results in Literature Norms}

    \vspace*{\fill}
    \begin{center}
        {\color{\accentcolor} \Large \textbf{Parabolic Case, $\SpaceLp{2}(I; \SpaceLp{2}(\Omega))$ Norm}}
    \end{center}

    \vspace*{\fill}

    \begin{multicols}{2}

        \begin{center}
            \begin{minipage}{0.4\textwidth}
                \begin{figure}[!ht]
                    \caption{$\left( p, q  \right) = \left( 1, 1  \right)$}
                    \label{fig:p_1_1_par_l2l2}
                    \begin{tikzpicture}[scale=.75]
\begin{loglogaxis}[
xlabel={$DoFs$},
% xtick={},
% xticklabels={},
% ylabel={ _LABERROR },
% grid=both,
legend pos=north east,
axis background/.style={fill=white!10},
yticklabel pos=right
]
\addplot[color=\documentcolor, dashed, semithick] coordinates {(5250, 0.008158029807153699) (15000, 0.003778963652924082) (42000, 0.0021149591552920614) (120000, 0.0010673187064462785) (336000, 0.000643075)};
\addplot[color=\documentcolor, dotted, semithick] coordinates {(5250, 0.010289171190271694) (15000, 0.005041703966636022) (42000, 0.0025723) (120000, 0.0012604259916590055) (336000, 0.000643075)};
\addplot[color=\documentcolor, solid, thick] coordinates {(5250, 0.009223600498712696) (15000, 0.004410333809780053) (42000, 0.0023436295776460307) (120000, 0.0011638723490526419) (336000, 0.000643075)};
\addplot[color=\accentcolor, mark=triangle, very thick] coordinates {(5250, 0.00843558) (15000, 0.00461507) (42000, 0.00236626) (120000, 0.00113055) (336000, 0.000643075)};
\legend{$\BigO{h^2}$,$\BigO{\tau^2}$,$\BigO{h^2 + \tau^2}$}
\end{loglogaxis}
\end{tikzpicture}
                \end{figure}
            \end{minipage}
        \end{center}

        \vfill\null
        \columnbreak

        \begin{center}
            \begin{minipage}{0.4\textwidth}
                \begin{figure}[!ht]
                    \caption{$\left( p, q  \right) = \left( 2, 2  \right)$}
                    \label{fig:p_2_2_par_l2l2}
                    \begin{tikzpicture}[scale=.75]
\begin{loglogaxis}[
xlabel={$DoFs$},
% xtick={},
% xticklabels={},
% ylabel={ _LABERROR },
% grid=both,
axis background/.style={fill=white!10},
legend pos=north east
]
\addplot[color=\documentcolor, dashed, semithick] coordinates {(15750, 0.001746168162036185) (45000, 0.0005505125271136028) (126000, 0.00023049478861934987) (360000, 8.26322e-05)};
\addplot[color=\documentcolor, dotted, semithick] coordinates {(15750, 0.0019272758508139458) (45000, 0.0006610576) (126000, 0.00024091049317399307) (360000, 8.26322e-05)};
\addplot[color=\documentcolor, solid, thick] coordinates {(15750, 0.0018367220064250652) (45000, 0.0006057850635568014) (126000, 0.00023570264089667147) (360000, 8.26322e-05)};
\addplot[color=\accentcolor, mark=triangle, very thick] coordinates {(15750, 0.0014825) (45000, 0.000617285) (126000, 0.000246978) (360000, 8.26322e-05)};
\legend{$\BigO{h^3}$,$\BigO{\tau^3}$,$\BigO{h^3 + \tau^3}$}
\end{loglogaxis}
\end{tikzpicture}
                \end{figure}
            \end{minipage}
        \end{center}

    \end{multicols}
    \vspace*{\fill}
    
\end{frame}

\begin{frame}
    \frametitle{Numerical Results in Literature Norms}

    \vspace*{\fill}
    \begin{center}
        {\color{\accentcolor} \Large \textbf{Parabolic Case, $\SpaceLp{2}(I; \SpaceHk{1}(\Omega))$ Norm}}
    \end{center}

    \vspace*{\fill}

    \begin{multicols}{2}

        \begin{center}
            \begin{minipage}{0.4\textwidth}
                \begin{figure}[!ht]
                    \caption{$\left( p, q  \right) = \left( 1, 1  \right)$}
                    \label{fig:p_1_1_par_l2h1}
                    \begin{tikzpicture}[scale=.75]
\begin{loglogaxis}[
xlabel={$DoFs$},
% xtick={},
% xticklabels={},
% ylabel={ _LABERROR },
% grid=both,
legend pos=north east,
axis background/.style={fill=white!10},
yticklabel pos=right
]
\addplot[color=\documentcolor, dashed, semithick] coordinates {(5250, 0.036223220032647725) (15000, 0.02465362396030036) (42000, 0.01844358758028077) (120000, 0.013102116919882468) (336000, 0.0101701)};
\addplot[color=\documentcolor, dotted, semithick] coordinates {(5250, 0.040680343047462784) (15000, 0.02847626860949256) (42000, 0.0203402) (120000, 0.01423813430474628) (336000, 0.0101701)};
\addplot[color=\documentcolor, solid, thick] coordinates {(5250, 0.038451781540055255) (15000, 0.026564946284896463) (42000, 0.019391893790140384) (120000, 0.013670125612314375) (336000, 0.0101701)};
\addplot[color=\accentcolor, mark=triangle, very thick] coordinates {(5250, 0.0347433) (15000, 0.0264992) (42000, 0.019354) (120000, 0.0135571) (336000, 0.0101701)};
\legend{$\BigO{h}$,$\BigO{\tau}$,$\BigO{h + \tau}$}
\end{loglogaxis}
\end{tikzpicture}
                \end{figure}
            \end{minipage}
        \end{center}

        \vfill\null
        \columnbreak

        \begin{center}
            \begin{minipage}{0.4\textwidth}
                \begin{figure}[!ht]
                    \caption{$\left( p, q  \right) = \left( 1, 1  \right)$}
                    \label{fig:p_2_2_par_l2h1}
                    \begin{tikzpicture}[scale=.75]
\begin{loglogaxis}[
xlabel={$DoFs$},
% xtick={},
% xticklabels={},
% ylabel={ _LABERROR },
% grid=both,
axis background/.style={fill=white!10},
legend pos=north east
]
\addplot[color=\documentcolor, dashed, semithick] coordinates {(15750, 0.0106346038376668) (45000, 0.0049261626049158964) (126000, 0.0027570079150305003) (360000, 0.00139133)};
\addplot[color=\documentcolor, dotted, semithick] coordinates {(15750, 0.011357773202786869) (45000, 0.00556532) (126000, 0.002839451251151474) (360000, 0.00139133)};
\addplot[color=\documentcolor, solid, thick] coordinates {(15750, 0.010996188520226834) (45000, 0.005245741302457948) (126000, 0.002798229583090987) (360000, 0.00139133)};
\addplot[color=\accentcolor, mark=triangle, very thick] coordinates {(15750, 0.00961274) (45000, 0.00543306) (126000, 0.00296394) (360000, 0.00139133)};
\legend{$\BigO{h^2}$,$\BigO{\tau^2}$,$\BigO{h^2 + \tau^2}$}
\end{loglogaxis}
\end{tikzpicture}
                \end{figure}
            \end{minipage}
        \end{center}

    \end{multicols}
    \vspace*{\fill}
    
\end{frame}

\begin{frame}
    \frametitle{Numerical Results in Literature Norms}

    \vspace*{\fill}
    \begin{center}
        {\color{\accentcolor} \Large \textbf{Hyperbolic Case, $\SpaceLp{2}(I; \SpaceLp{2}(\Omega))$ Norm}}
    \end{center}

    \vspace*{\fill}

    \begin{multicols}{2}

        \begin{center}
            \begin{minipage}{0.4\textwidth}
                \begin{figure}[!ht]
                    \caption{$\left( p, q  \right) = \left( 1, 1  \right)$}
                    \label{fig:p_1_1_hyp_l2l2}
                    \begin{tikzpicture}[scale=.7]
\begin{loglogaxis}[
xlabel={$DoFs$},
% xtick={},
% xticklabels={},
% ylabel={ _LABERROR },
% grid=both,
legend pos=north east,
axis background/.style={fill=white!10},
yticklabel pos=right
]
\addplot[color=\documentcolor, dashed, semithick] coordinates {(5250, 0.009322969696219697) (15000, 0.004318587263365112) (42000, 0.00241696838325352) (120000, 0.001219728315736416) (336000, 0.000734904)};
\addplot[color=\documentcolor, dotted, semithick] coordinates {(5250, 0.01175843107633702) (15000, 0.00576164275068488) (42000, 0.002939616) (120000, 0.00144041068767122) (336000, 0.000734904)};
\addplot[color=\documentcolor, solid, thick] coordinates {(5250, 0.010540700386278358) (15000, 0.005040115007024996) (42000, 0.0026782921916267597) (120000, 0.001330069501703818) (336000, 0.000734904)};
\addplot[color=\accentcolor, mark=triangle, very thick] coordinates {(5250, 0.00900924) (15000, 0.00504002) (42000, 0.00264629) (120000, 0.00130295) (336000, 0.000734904)};
\legend{$\BigO{h^2}$,$\BigO{\tau^2}$,$\BigO{h^2 + \tau^2}$}
\end{loglogaxis}
\end{tikzpicture}
                \end{figure}
            \end{minipage}
        \end{center}

        \vfill\null
        \columnbreak

        \begin{center}
            \begin{minipage}{0.4\textwidth}
                \begin{figure}[!ht]
                    \caption{$\left( p, q  \right) = \left( 2, 2  \right)$}
                    \label{fig:p_2_2_hyp_l2l2}
                    \begin{tikzpicture}[scale=.7]
\begin{loglogaxis}[
xlabel={$DoFs$},
% xtick={},
% xticklabels={},
% ylabel={ _LABERROR },
% grid=both,
axis background/.style={fill=white!10},
legend pos=north east
]
\addplot[color=\documentcolor, dashed, semithick] coordinates {(15750, 0.001981815343465629) (45000, 0.0006248047563366166) (126000, 0.00026160029635520775) (360000, 9.37835e-05)};
\addplot[color=\documentcolor, dotted, semithick] coordinates {(15750, 0.002187363700286446) (45000, 0.000750268) (126000, 0.00027342161090450427) (360000, 9.37835e-05)};
\addplot[color=\documentcolor, solid, thick] coordinates {(15750, 0.0020845895218760376) (45000, 0.0006875363781683082) (126000, 0.000267510953629856) (360000, 9.37835e-05)};
\addplot[color=\accentcolor, mark=triangle, very thick] coordinates {(15750, 0.00161122) (45000, 0.00068533) (126000, 0.000278694) (360000, 9.37835e-05)};
\legend{$\BigO{h^3}$,$\BigO{\tau^3}$,$\BigO{h^3 + \tau^3}$}
\end{loglogaxis}
\end{tikzpicture}
                \end{figure}
            \end{minipage}
        \end{center}

    \end{multicols}
    \vspace*{\fill}
    
\end{frame}

% New norms.

\begin{frame}
    \frametitle{Numerical Results in New Norms}

    \vspace*{\fill}
    \begin{center}
        {\color{\accentcolor} \Large \textbf{Parabolic Case, $\SpaceLp{2}(\left\{ T \right\} \times \Omega)$ and $\SpaceLp{\infty}(I; \SpaceLp{2}(\Omega))$ Norms}}
    \end{center}

    \vspace*{\fill}

    \begin{multicols}{2}

        \begin{center}
            \begin{minipage}{0.4\textwidth}
                \begin{figure}[!ht]
                    \caption{$\left( p, q  \right) = \left( 2, 1  \right)$}
                    \label{fig:p_2_1_par_l2T}
                    \begin{tikzpicture}[scale=.75]
\begin{loglogaxis}[
xlabel={$DoFs$},
% xtick={},
% xticklabels={},
% ylabel={ _LABERROR },
% grid=both,
axis background/.style={fill=white!10},
legend pos=north east,
yticklabel pos=right
]
\addplot[color=\documentcolor, dashed, semithick] coordinates {(10500, 0.0030302661423184823) (30000, 0.0009553486932720556) (84000, 0.0003999961567960357) (240000, 0.00014339830685797405) (672000, 6.70649e-05)};
\addplot[color=\documentcolor, dotted, semithick] coordinates {(10500, 0.0010730353954957854) (30000, 0.0005257883953691999) (84000, 0.0002682596) (240000, 0.00013144709884229996) (672000, 6.70649e-05)};
\addplot[color=\documentcolor, solid, thick] coordinates {(10500, 0.002051650768907134) (30000, 0.0007405685443206277) (84000, 0.0003341278783980178) (240000, 0.00013742270285013698) (672000, 6.70649e-05)};
\addplot[color=\accentcolor, mark=triangle, very thick] coordinates {(10500, 0.00229703) (30000, 0.00095478) (84000, 0.00038154) (240000, 0.000127705) (672000, 6.70649e-05)};
\legend{$\BigO{h^3}$,$\BigO{\tau^2}$,$\BigO{h^3 + \tau^2}$}
\end{loglogaxis}
\end{tikzpicture}
                \end{figure}
            \end{minipage}
        \end{center}

        \vfill\null
        \columnbreak

        \begin{center}
            \begin{minipage}{0.4\textwidth}
                \begin{figure}[!ht]
                    \caption{$\left( p, q  \right) = \left( 2, 1  \right)$}
                    \label{fig:p_2_1_par_linfl2}
                    \begin{tikzpicture}[scale=.7]
\begin{loglogaxis}[
xlabel={$DoFs$},
% xtick={},
% xticklabels={},
% ylabel={ _LABERROR },
% grid=both,
axis background/.style={fill=white!10},
legend pos=north east
]
\addplot[color=\documentcolor, dashed, semithick] coordinates {(10500, 0.017782512941530295) (30000, 0.0056062734109507415) (84000, 0.0023472977292171822) (240000, 0.0008415043853357522) (672000, 0.000393557)};
\addplot[color=\documentcolor, dotted, semithick] coordinates {(10500, 0.006296894368665797) (30000, 0.0030854844116119783) (84000, 0.001574228) (240000, 0.0007713711029029946) (672000, 0.000393557)};
\addplot[color=\documentcolor, solid, thick] coordinates {(10500, 0.012039703655098046) (30000, 0.00434587891128136) (84000, 0.0019607628646085913) (240000, 0.0008064377441193733) (672000, 0.000393557)};
\addplot[color=\accentcolor, mark=triangle, very thick] coordinates {(10500, 0.00568751) (30000, 0.00289989) (84000, 0.00152017) (240000, 0.000760563) (672000, 0.000393557)};
\legend{$\BigO{h^3}$,$\BigO{\tau^2}$,$\BigO{h^3 + \tau^2}$}
\end{loglogaxis}
\end{tikzpicture}
                \end{figure}
            \end{minipage}
        \end{center}

    \end{multicols}
    \vspace*{\fill}
    
\end{frame}

\begin{frame}
    \frametitle{Numerical Results in New Norms}

    \vspace*{\fill}
    \begin{center}
        {\color{\accentcolor} \Large \textbf{Hyperbolic Case, $\SpaceLp{2}(\left\{ T \right\} \times \Omega)$ and $\SpaceLp{\infty}(I; \SpaceLp{2}(\Omega))$ Norms}}
    \end{center}

    \vspace*{\fill}

    \begin{multicols}{2}

        \begin{center}
            \begin{minipage}{0.4\textwidth}
                \begin{figure}[!ht]
                    \caption{$\left( p, q  \right) = \left( 2, 1  \right)$}
                    \label{fig:p_2_1_hyp_l2T}
                    \begin{tikzpicture}[scale=.7]
\begin{loglogaxis}[
xlabel={$DoFs$},
% xtick={},
% xticklabels={},
% ylabel={ _LABERROR },
% grid=both,
axis background/.style={fill=white!10},
yticklabel pos=right,
legend pos=north east
]
\addplot[color=\documentcolor, dashed, semithick] coordinates {(10500, 0.0030750526073734675) (30000, 0.0009694684731385916) (84000, 0.00040590798534742735) (240000, 0.00014551769273281502) (672000, 6.80561e-05)};
\addplot[color=\documentcolor, dotted, semithick] coordinates {(10500, 0.0010888945510900742) (30000, 0.0005335593971523971) (84000, 0.0002722244) (240000, 0.00013338984928809927) (672000, 6.80561e-05)};
\addplot[color=\documentcolor, solid, thick] coordinates {(10500, 0.0020819735792317706) (30000, 0.0007515139351454943) (84000, 0.00033906619267371364) (240000, 0.00013945377101045713) (672000, 6.80561e-05)};
\addplot[color=\accentcolor, mark=triangle, very thick] coordinates {(10500, 0.00250241) (30000, 0.00106238) (84000, 0.000431468) (240000, 0.000145193) (672000, 6.80561e-05)};
\legend{$\BigO{h^3}$,$\BigO{\tau^2}$,$\BigO{h^3 + \tau^2}$}
\end{loglogaxis}
\end{tikzpicture}
                \end{figure}
            \end{minipage}
        \end{center}

        \vfill\null
        \columnbreak

        \begin{center}
            \begin{minipage}{0.4\textwidth}
                \begin{figure}[!ht]
                    \caption{$\left( p, q  \right) = \left( 2, 1  \right)$}
                    \label{fig:p_2_1_hyp_linfl2}
                    \begin{tikzpicture}[scale=.7]
\begin{loglogaxis}[
xlabel={$DoFs$},
% xtick={},
% xticklabels={},
% ylabel={ _LABERROR },
% grid=both,
axis background/.style={fill=white!10},
legend pos=north east
]
\addplot[color=\documentcolor, dashed, semithick] coordinates {(10500, 0.017798417739451355) (30000, 0.005611287699200359) (84000, 0.002349397167927927) (240000, 0.0008422570324583753) (672000, 0.000393909)};
\addplot[color=\documentcolor, dotted, semithick] coordinates {(10500, 0.006302526352896214) (30000, 0.0030882440894042357) (84000, 0.001575636) (240000, 0.0007720610223510589) (672000, 0.000393909)};
\addplot[color=\documentcolor, solid, thick] coordinates {(10500, 0.012050472046173785) (30000, 0.004349765894302297) (84000, 0.001962516583963963) (240000, 0.0008071590274047171) (672000, 0.000393909)};
\addplot[color=\accentcolor, mark=triangle, very thick] coordinates {(10500, 0.00568812) (30000, 0.00290104) (84000, 0.00152116) (240000, 0.00076117) (672000, 0.000393909)};
\legend{$\BigO{h^3}$,$\BigO{\tau^2}$,$\BigO{h^3 + \tau^2}$}
\end{loglogaxis}
\end{tikzpicture}
                \end{figure}
            \end{minipage}
        \end{center}

    \end{multicols}
    \vspace*{\fill}
    
\end{frame}
\subsection{\texorpdfstring{\acrlong{pdes}}{}}

\begin{frame}
    \frametitle{Relevance and Examples of \acrfull{pdes}}

    \vspace*{\fill}
    \begin{center}
        {\color{\accentcolor} \Large \textbf{Relevance}}
    \end{center}

    \vspace*{0.125cm}

    \begin{center}
        \begin{minipage}{0.75\textwidth}
            \begin{description}
                \item[Description of key real-world phenomena] Robust mathematical framework for accurately modeling and describing complex physical, biological, and engineering phenomena
                \item[Predictive power and simulation] Numerical simulations enable precise predictions of real-world systems by incorporating underlying physical laws
                \item[Bridging theoretical and applied mathematics] Developments in PDE theory frequently motivate innovations in numerical methods and computational techniques
            \end{description}
        \end{minipage}
    \end{center}

    \vspace*{\fill}

    \begin{center}
        {\color{\accentcolor} \Large \textbf{Examples}}
    \end{center}

    \begin{multicols}{3}

        % Quantum Mechanics.
        \begin{center}
            \begin{align*}
                i \hbar \partial_t u = - \frac{\hbar^2}{2 m} \Laplacian u + V u
            \end{align*}
            
            \vspace*{0.125cm}
            {\large \textbf{Schrödinger}} \\
            Quantum Mechanics
        \end{center}

        % Fluid Dynamics.
        \begin{center}
            \begin{align*}
                \partial_t u + \Convection \cdot \Gradient u - \Diffusion \Laplacian u + \Reaction u &= g
            \end{align*}

            \vspace*{0.125cm}
            {\large \textbf{\acrshort{cdr}}} \\
            Fluid Dynamics
        \end{center}

        % Gas Flow.
        \begin{center}
            \begin{align*}
                \partial_t u = \Laplacian \left( u^m \right)
            \end{align*}
            
            \vspace*{0.125cm}
            {\large \textbf{Porous Medium}} \\
            Gas Flow
        \end{center}

    \end{multicols}
    \vspace*{\fill}
    
\end{frame}

\subsection{Numerical Methods for \texorpdfstring{\acrlong{pdes}}{}}

\begin{frame}
    \frametitle{Importance of Numerical Methods for \acrshort{pdes}}

    \vspace*{\fill}
    \begin{center}
        {\color{\accentcolor} \Large \textbf{Challenges}}
    \end{center}

    \vspace*{0.125cm}

    \begin{center}
        \begin{minipage}{0.75\textwidth}
            \begin{description}
                \item[Complex domains] Realistic problems often involve irregular geometries and boundary conditions, where standard solution methods fail
                \item[Coupled and multi-physics problems] Multiple interacting \acrshort{pdes} rarely have straightforward closed-form solutions
                \item[Nonlinear complexity] Many \acrshort{pdes} are nonlinear, rendering classical analytical techniques insufficient
            \end{description}
        \end{minipage}
    \end{center}
    \vspace*{\fill}
    
\end{frame}

% \begin{frame}
%     \frametitle{Two Examples of Numerical Methods for \acrshort{pdes}}

%     \vspace*{\fill}
%     \begin{multicols}{2}

%         % Finite Difference.
%         \begin{center}
%             {\color{\accentcolor} \Large \textbf{Finite Difference Methods}}
%             \vspace*{0.5cm}

%             \begin{minipage}{0.4\textwidth}
%                 \begin{itemize}
%                     \color{\procolor}
%                     \item Simplicity and ease of implementation
%                     \item Computational efficiency on structured grids
%                 \end{itemize}
    
%                 \begin{itemize}
%                     \color{\concolor}
%                     \item Limited geometric flexibility
%                     \item Difficulty in handling boundary conditions
%                 \end{itemize}
%             \end{minipage}
%         \end{center}

%         \vfill\null
%         \columnbreak

%         % Finite Element.
%         \begin{center}
%             {\color{\accentcolor} \Large \textbf{Finite Element Methods}}
%             \vspace*{0.5cm}

%             \begin{minipage}{0.4\textwidth}
%                 \begin{itemize}
%                     \color{\procolor}
%                     \item Flexibility in handling complex geometries
%                     \item Better handling of boundary conditions and higher-order accuracy
%                 \end{itemize}

%                 \begin{itemize}
%                     \color{\concolor}
%                     \item Higher computational cost and complexity
%                     \item More involved implementation
%                 \end{itemize}
%             \end{minipage}
%         \end{center}
%     \end{multicols}
%     \vspace*{\fill}
    
% \end{frame}

\begin{frame}
    \frametitle{An Example of a Numerical Method for \acrshort{pdes}}

    \vspace*{\fill}
    \begin{center}
        {\color{\accentcolor} \Large \textbf{Finite Element Method}}
    \end{center}

    \vspace*{0.125cm}

    \begin{center}
        \begin{minipage}{0.75\textwidth}
            \begin{itemize}
                \color{\procolor}
                \item Flexibility in handling complex geometries and boundary conditions
                \item Adaptivity through mesh refinement and higher-order basis functions
                \item Naturally suited for variational formulations and weak solutions
            \end{itemize}

            \begin{itemize}
                \color{\concolor}
                \item Higher computational cost and memory requirements
                \item Complex mesh generation, especially in three dimensions
                \item More involved implementation, requiring variational analysis and numerical integration
            \end{itemize}
        \end{minipage}
    \end{center}
    \vspace*{\fill}
    
\end{frame}
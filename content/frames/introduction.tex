\subsection{\texorpdfstring{\acrlong{pdes}}{}}

\begin{frame}
    \frametitle{Relevance and Examples of \acrfull{pdes}}

    \vspace*{\fill}
    \begin{center}
        {\color{\accentcolor} \Large \textbf{Relevance}}
    \end{center}

    \vspace*{0.125cm}

    \begin{center}
        \begin{minipage}{0.75\textwidth}
            \begin{description}
                \item[Description of key real-world phenomena] Mathematical backbone for describing a wide range of physical, biological, and engineering phenomena
                \item[Predictive power and simulation] Accurate simulations of real-world systems by incorporating physical laws
                \item[Bridging theoretical and applied mathematics] Advances in \acrshort{pdes} theory often drive innovations in numerical methods and computational techniques
            \end{description}
        \end{minipage}
    \end{center}

    \vspace*{\fill}

    \begin{center}
        {\color{\accentcolor} \Large \textbf{Examples}}
    \end{center}

    \begin{multicols}{3}

        % Quantum Mechanics.
        \begin{center}
            \begin{align*}
                i \hbar \partial_t u = - \frac{\hbar^2}{2 m} \Laplacian u + V u
            \end{align*}
            
            \vspace*{0.125cm}
            {\large \textbf{Schrödinger}} \\
            Quantum Mechanics
        \end{center}

        % Fluid Dynamics.
        \begin{center}
            \begin{align*}
                \partial_t u + \Convection \cdot \Gradient u - \Diffusion \Laplacian u + \Reaction u &= g
            \end{align*}

            \vspace*{0.125cm}
            {\large \textbf{\acrshort{cdr}}} \\
            Fluid Dynamics
        \end{center}

        % Gas Flow.
        \begin{center}
            \begin{align*}
                \partial_t u = \Laplacian \left( u^m \right)
            \end{align*}
            
            \vspace*{0.125cm}
            {\large \textbf{Porous Medium}} \\
            Gas Flow
        \end{center}

    \end{multicols}
    \vspace*{\fill}
    
\end{frame}

\subsection{Numerical Methods for \texorpdfstring{\acrlong{pdes}}{}}

\begin{frame}
    \frametitle{Importance of Numerical Methods for \acrshort{pdes}}

    \vspace*{\fill}
    \begin{center}
        {\color{\accentcolor} \Large \textbf{Challenges}}
    \end{center}

    \vspace*{0.125cm}

    \begin{center}
        \begin{minipage}{0.75\textwidth}
            \begin{description}
                \item[Complex domains] Realistic problems often involve irregular geometries and boundary conditions, where standard solution methods fail
                \item[Coupled and multi-physics problems] Multiple interacting \acrshort{pdes} rarely have straightforward closed-form solutions
                \item[Nonlinear complexity] Many \acrshort{pdes} are nonlinear, rendering classical analytical techniques insufficient
            \end{description}
        \end{minipage}
    \end{center}

    \vspace*{\fill}
    
\end{frame}

\begin{frame}
    \frametitle{Examples of Numerical Methods for \acrshort{pdes}}

    \vspace*{\fill}
    \begin{multicols}{2}

        % Finite Difference.
        \begin{center}
            {\color{\accentcolor} \Large \textbf{Finite Difference Methods}}
            \vspace*{0.5cm}

            \begin{minipage}{0.4\textwidth}
                \begin{itemize}
                    \color{\procolor}
                    \item Simplicity and ease of implementation
                    \item Computational efficiency on structured grids
                \end{itemize}
    
                \begin{itemize}
                    \color{\concolor}
                    \item Limited geometric flexibility
                    \item Difficulty in handling boundary conditions
                \end{itemize}
            \end{minipage}
        \end{center}

        % Finite Element.
        \begin{center}
            {\color{\accentcolor} \Large \textbf{Finite Element Methods}}
            \vspace*{0.5cm}

            \begin{minipage}{0.4\textwidth}
                \begin{itemize}
                    \color{\procolor}
                    \item Flexibility in handling complex geometries
                    \item Better handling of boundary conditions and higher-order accuracy
                \end{itemize}

                \begin{itemize}
                    \color{\concolor}
                    \item Higher computational cost and complexity
                    \item More involved implementation
                \end{itemize}
            \end{minipage}
        \end{center}
    \end{multicols}
    \vspace*{\fill}
    
\end{frame}

% \begin{frame}
%     \frametitle{Finite Element Methods}

%     % [!]
    
% \end{frame}
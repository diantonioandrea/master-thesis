\section{Bochner Spaces}

The Bochner spaces introduced in this section provide an ideal setting for studying and characterizing time-dependent \acrshort{pdes}, by considering, following \cite[p.~111]{Ern2021}, functions defined on a bounded time interval $I \subset \RealNumbers$, with values in a Banach (or Hilbert) space of functions defined on the space domain $\Omega \subset \RealNumbersTo{d}$.

\subsection{Bochner Integral}

Let $I \subset \RealNumbers$ be open, nonempty and bounded and let $V$ be a Banach space. Moreover, equip $I$ with its own $\sigma$-algebra $\SigmaAlgebraI$ and with the Lebesgue measure $\lambda$.

\begin{definition}[Simple functions]
    A function $u\colon I \rightarrow V$ is said to be a simple function if there exist $N \in \NaturalNumbers$, $\left\{ v_n \right\}_{n = 1}^N \subset V$ and $\left\{ A_n \right\}_{n = 1}^N \subset \SigmaAlgebraI$ disjoint such that:
    \begin{align}
        u(t) = \sum_{n = 1}^N v_n \Characteristic{A_n}(t) &\text{ for all } t \in I.
    \end{align}

    Moreover, the Bochner integral of a simple function is defined as:
    \begin{align}
        \int_I u(t) ~ d \lambda(t)  = \sum_{n = 1}^N v_n \lambda(A_n) \in V.
    \end{align}
\end{definition}

\begin{lemma} % [!] Ugly \left and \right with \lVert and \rVert.
    \begin{align}
        \left\lVert \int_I u(t) ~ d \lambda(t) \right\rVert_V &\leq \int_I \left\lVert u(t) \right\rVert_V ~ d \lambda(t) &\text{ for all simple functions } u.
    \end{align}
\end{lemma}

\begin{definition}[Strongly measurable function]
    A function $u\colon I \rightarrow V$ is said to be strongly measurable if there exist $\left\{ u_n \right\}_{n \in N}$ simple functions such that:
    \begin{align}
        \lim_{n \rightarrow \infty} \lVert u(t) - u_n(t) \rVert_V &= 0 &\text{ for } \lambda \text{-a.e. } t \in I.
&    \end{align}
\end{definition}

\begin{lemma}
    Let $u\colon I \rightarrow V$ be a strongly measurable function, then $\lVert u(\cdot) \rVert_V \colon I \rightarrow \RealNumbers$ is Lebesgue-measurable.
\end{lemma}

\begin{definition}[Bochner integrable function] \label{definition:bochner_integrable}
    A function $u\colon I \rightarrow V$ is said to be Bochner integrable if there exist $\left\{ u_n \right\}_{n \in N}$ simple functions such that:
    \begin{align}
        \lim_{n \rightarrow \infty} \lVert u(t) - u_n(t) \rVert_V &= 0 &\text{ for } \lambda \text{-a.e. } t \in I,
&    \end{align}
    that is  $u$ is strongly measurable, and:
    \begin{align}
        \lim_{n \rightarrow \infty} \int_I \lVert u(t) - u_n(t) \rVert_V ~ d \lambda(t) = 0.
    \end{align}
\end{definition}

\begin{lemma}
    Let $u\colon I \rightarrow V$ be a Bochner integrable function and $\left\{ u_n \right\}_{n \in \NaturalNumbers}$ as in \cref{definition:bochner_integrable}, then $\left\{ \int_I u_n(t) ~ d \lambda(t) \right\}_{n \in \NaturalNumbers}$ converges in $V$ with respect to $\lVert \cdot \rVert_V$.
\end{lemma}

\begin{definition}[Bochner integral]
    Let $u\colon I \rightarrow V$ be a Bochner integrable function and $\left\{ u_n \right\}_{n \in \NaturalNumbers}$ as in \cref{definition:bochner_integrable}, then the Bochner integral of $u$ is defined as:
    \begin{align}
        \int_I u(t) ~ d \lambda(t) = \lim_{n \rightarrow \infty} \int_I u_n(t) ~ d \lambda(t).
    \end{align}
\end{definition}

\begin{theorem}[Bochner]
    Let $u\colon I \rightarrow V$ be a strongly measurable function, then $u$ is Bochner integrable if and only if:
    \begin{align}
        \int_I \lVert u(t) \rVert_V ~ d \lambda(t) < + \infty.
    \end{align}
\end{theorem}

\newpage
\subsection{Definition and Main Properties}

\begin{corollary}[Linear maps]
    Let $W$ be a Banach space and $F$ be a bounded linear operator from $V$ to $W$. Let $u\colon I \rightarrow V$ be a Bochner integrable function and define $F(u) \colon I \rightarrow W$ for which:
    \begin{align}
        F(u)(t) &= F(u(t)) &\text{ for } \lambda \text{-a.e. } t \in I,
    \end{align}
    then $F(u)$ is a Bochner integrable function and:
    \begin{align}
        \int_I F(u)(t) ~ d \lambda(t) = F\left( \int_I u(t) ~ d \lambda(t) \right).
    \end{align}
\end{corollary}

\begin{remark} % [!] Ugly \left and \right with \langle and \rangle.
    Let $u\colon I \rightarrow V$ be a Bochner integrable function and let $\varphi \in V^*$, then:
    \begin{align}
        \int_I \left\langle \varphi, (u)(t) \right\rangle_{V^*, V} ~ d \lambda(t) = \left\langle \varphi, \int_I u(t) ~ d \lambda(t) \right\rangle_{V^*, V}.
    \end{align}
\end{remark}

\begin{remark}[Embedding]
    Let $u\colon I \rightarrow V$ be a Bochner integrable function and let $W$ be a Banach space such that $V \hookrightarrow W$. Denote by $F_{V \rightarrow W} \colon V \rightarrow W$ the canonical embedding, then:
    \begin{align}
        \int_I F_{V \rightarrow W}(u)(t) ~ d \lambda(t) = F_{V \rightarrow W}\left( \int_I u(t) ~ d \lambda(t) \right),
    \end{align}
    leading to a useful identification between the $V$-valued and the $W$-valued integrals.
\end{remark}

\begin{definition}[$\SpaceLp{p}(I; V)$]
    Let $1 \leq p \leq +\infty$, then:
    \begin{align}
        \SpaceLp{p}(I; V) = \left\{ u \colon I \rightarrow V \text{ strongly measurable such that } \lVert u \rVert_{\SpaceLp{p}(I; V)} < +\infty \right\},
    \end{align}
    where:
    \begin{align}
        & \lVert u \rVert_{\SpaceLp{p}(I; V)}^p = \int_I \lVert u(t) \rVert_V^p ~ d \lambda(t) & \text{ if } p < +\infty, \\
        & \lVert u \rVert_{\SpaceLp{\infty}(I; V)} = \esssup_{t \in I} \left\{ \lVert u(t) \rVert_V \right\} & \text{ if } p = +\infty.
    \end{align}
\end{definition}

\begin{theorem}
    Let $1 \leq p \leq +\infty$, then $\SpaceLp{p}(I; V)$ is a Banach space.
\end{theorem}

\begin{theorem} % [!] Check.
    Let $1 \leq p \leq +\infty$ and let $V$ be reflexive, then $\left( \SpaceLp{p}(I; V) \right)^*$ is isometrically isomorphic to $\SpaceLp{p^{\prime}}(I; V^*)$.

    Furthermore, $\SpaceLp{p}(I; V)$ is reflexive for all $1 < p < +\infty$.
\end{theorem}

\newpage
\subsection{Weak Time Derivative}

Set $I = (0, T)$ for $T > 0$ for the entirety of this section.

\begin{definition}[Continuity]
    A function $u\colon I \rightarrow V$ is said to be continuous at $t \in I$ if for every $\left\{ t_n \right\}_{n \in \NaturalNumbers}$ which converges to $t$, the sequence $\left\{ u(t_n) \right\}_{n \in \NaturalNumbers}$ converges to $u(t)$ in $V$.

    Moreover, $u$ is said to be continuous if it is continuous for all $t \in I$.
\end{definition}

\begin{definition}[Strongly differentiable function]
    A function $u\colon I \rightarrow V$ is said to be strongly differentiable at $t \in I$ if it is continuous in a neighbourhood of $t$ and the ratio:
    \begin{align}
        \frac{u(t + \tau) - u(t)}{\tau}
    \end{align}
    converges in $V$ as $\tau \rightarrow 0$.

    The limit is then denoted by $\partial_t u(t) \in V$.
\end{definition}

\begin{theorem}[Lebesgue's differentiation]
    Let $u \in \SpaceLp{1}(I; V)$ and:
    \begin{align}
        F(t) &= \int_0^t u(s) ~ d \lambda(t) &\text{ for all } t \in I,
    \end{align}
    then $F$ is strongly differentiable for $\lambda$-a.e. $t \in I$ and $\partial_t F(t) = u(t)$ for $\lambda$-a.e. $t \in I$.
\end{theorem}

\begin{definition}[$\SpaceLp{1}_{loc}(I; V)$]
    \begin{align}
        \SpaceLp{1}_{loc}(I; V) = \left\{ u\colon I \rightarrow V  \text{ such that } u \in \SpaceLp{1}(K; V) \text{ for all } K \subset \subset I \right\}.
    \end{align}
\end{definition}

\begin{corollary}[Vanishing integrals]
    Let $u \in \SpaceLp{1}_{loc}(I; V)$ be such that:
    \begin{align}
        \int_I u(t) \varphi(t) ~ d \lambda(t) &= 0 &\text{ for all } \varphi \in C_c^{\infty}(I; \RealNumbers),
    \end{align}
    then $u(t) = 0$ for $\lambda$-a.e. $t \in I$.
\end{corollary}

\begin{definition}[Weakly differentiable function]
    A function $u \in \SpaceLp{1}_{loc}(I; V)$ is said to be weakly differentiable if there exists $v \in \SpaceLp{1}_{loc}(I; V)$ such that:
    \begin{align}
        - \int_I \varphi^{\prime}(t) u(t) &= \int_I \varphi(t) v(t) &\text{ for all } \varphi \in C_c^{\infty}(I; \RealNumbers).
    \end{align}
    Its weak derivative is denoted by $\partial_t u$ when unambiguous.
\end{definition}

\begin{lemma}
    Let $u \in \SpaceLp{1}_{loc}(I; V)$ weakly differentiable be such that $\partial_t u(t) = 0$ for $\lambda$-a.e. $t \in I$, then there exists $a \in V$ such that $u(t) = a$ for $\lambda$-a.e. $t \in I$.
\end{lemma}

\begin{theorem}[Fundamental theorem of calculus]
    Let $u, v \in \SpaceLp{1}(I; V)$, then $u$ is weakly differentiable with $v = \partial_t u$ if and only if there exists $a \in V$ such that:
    \begin{align}
        u(t) &= a + \int_0^t v(s) ~ d \lambda(t) &\text{ for } \lambda \text{-a.e.} t \in I.
    \end{align}
\end{theorem}

\begin{corollary}
    Let $u \in \SpaceLp{1}(I; V)$ be weakly differentiable, then it is strongly differentiable for $\lambda$-a.e. $t \in I$ and its strong and weak derivatives coincide.
\end{corollary}

\begin{lemma}[Linear maps]
    Let $W$ be a Banach space and $F$ be a linear operator from $V$ to $W$, then for all $u \in \SpaceLp{1}_{loc}(I; V)$ weakly differentiable, $F(u) \in \SpaceLp{1}_{loc}(I; W)$ is weakly differentiable and:
    \begin{align}
        F(\partial_t u) = \partial_t F(u) \in \SpaceLp{1}_{loc}(I; W).
    \end{align}
\end{lemma}

\newpage
\subsection{Function Spaces with Weak Time Derivative}

Let $V, W$ Banach spaces such that $V \hookrightarrow W$ with $F_{V \rightarrow W}$ canonical embedding.

\begin{definition}[$\SpaceXpq{p}{q}(I; V, W)$]
    Let $1 \leq p, q \leq +\infty$, then:
    \begin{align}
        \SpaceXpq{p}{q}(I; V, W) = \left\{ u \in \SpaceLp{p}(I; V) \text{ for which } \partial_t u \in \SpaceLp{q}(I; W) \right\}
    \end{align}

    Moreover, $\SpaceXpq{}{q}(I; V, W)$ is equipped with the norm $\lVert \cdot \rVert_{\SpaceXpq{}{q}(I; V, W)}$, such that for all \\ $u \in \SpaceXpq{P}{q}(I; V, W)$:
    \begin{align}
        \lVert u \rVert_{\SpaceXpq{}{q}(I; V, W)} = \lVert u \rVert_{\SpaceLp{p}(I; V)} + \iota_{V, W}^{-1}T^{1 + \frac{1}{p} - \frac{1}{q}} \lVert \partial_t u \rVert_{\SpaceLp{q}(I; W)},
    \end{align}
    where:
    \begin{align}
        \iota_{V, W} = \lVert F_{V \rightarrow W} \rVert_{\Lagrange(V;W)}.
    \end{align}
\end{definition}

\begin{theorem}
    $C^{\infty}(\overline{I}; V)$ is dense in $\SpaceXpq{p}{q}(I; V, W)$.
\end{theorem}

\begin{lemma}
    \begin{align}
        & \SpaceXpq{p}{q}(I; V, W) \hookrightarrow C^{0, 1 - \frac{1}{q}}(\overline{I}; W) & \text{ if q > 1}, \\
        & \SpaceXpq{p}{q}(I; V, W) \hookrightarrow C^{0}(\overline{I}; W) & \text{ if q = 1}.
    \end{align}
\end{lemma}

\begin{theorem}
    Let $Y$ be a Banach space such that $V \hookrightarrow Y \hookrightarrow W$ and assume that $V \hookrightarrow \hookrightarrow Y$, then:
    \begin{align}
        \SpaceXpq{p}{q}(I; V, W) &\hookrightarrow \hookrightarrow \SpaceLp{p}(I; Y) &\text{ for all } 1 \leq p, q < +\infty,\\
        \SpaceXpq{\infty}{q}(I; V, W) &\hookrightarrow \hookrightarrow C^0(\overline{I}; Y) &\text{ for all } q > 1.
    \end{align}
\end{theorem}

\begin{definition}[Gelfand triple] \label{definition:gelfand}
    Let now $V, W$ be separable Hilbert spaces such that $V \hookrightarrow W$ and $V$ is dense in $W$. Moreover, identify $W$ with its dual space $W^*$. A triple $\left( V, W \equiv W^*, V^* \right)$ is said to be a Gelfand triple.
\end{definition}

\begin{definition}[$\SpaceX(I; V, V^*)$] \label{definition:x}
    \begin{align}
        \SpaceX(I; V, V^*) = \SpaceXpq{2}{2}(I; V, V^*) = \left\{ u \in \SpaceLp{2}(I; V) \text{ for which } \partial_t u \in \SpaceLp{2}(I; V^*) \right\}
    \end{align}
\end{definition}

\begin{theorem}[Time trace and integration by parts]
    Let $V, W$ as in \cref{definition:gelfand}, then \newline \nobreak $\SpaceX(I; V, V^*) \hookrightarrow C^0(\overline{I};W)$, and the map that associates $u(0) \in W$ to each $u \in \SpaceX(I; V, V^*)$ is surjective.

    Moreover, for all $u, v \in \SpaceX(I; V, V^*)$:
    \begin{align}
        \int_I \langle \partial_t u(t), v(t) \rangle_{V^*, V} ~ d \lambda(t) = - \int_I \langle \partial_t v(t), u(t) \rangle_{V^*, V} ~ d \lambda(t) + \left( u(T), v(T) \right)_W - \left( u(0), v(0) \right)_W.
    \end{align}
\end{theorem}

\newpage
\section{Weak Formulation of the Model Parabolic Problem}

Following \cite[p.~124]{Ern2021}, let $V, W$ and $\SpaceX$ as in \cref{definition:gelfand} and \cref{definition:x}.

\subsection{Model Problem}

\begin{definition}[$A$] \label{definition:A} % [!] Check instances of \ContC{a}
    Let $A \colon I \rightarrow \Lagrange(V;V^*)$ be an operator such that the map that associates $\langle A(t)(u), v \rangle_{V^*, V}$ to $t \in I$ is measurable for all $u, v \in V$. Moreover there exist $\ContC{a}, \CoerC{a} > 0$ such that:
    \begin{align} \label{equation:a}
        \lVert A(t)(u) \rVert_{V^*} & \leq \ContC{a} \lVert u \rVert_V &\text{ for all } u \in V \text{ and for } \lambda \text{-a.e. } t \in I, \\
        \langle A(t)(u), u \rangle_{V^*, V} & \geq \CoerC{a} \lVert u \rVert_V^2 &\text{ for all } u \in V \text{ and for } \lambda \text{-a.e. } t \in I.
    \end{align}
\end{definition}

\begin{lemma}
    Let $1 \leq p \leq +\infty$ and  $A$ as in \cref{definition:A}, then for all $u \in \SpaceLp{p}(I; V)$ the function $A(u) \colon I \rightarrow V^*$ such that $A(u)(t) = A(t)(u(t))$ is strongly measurable. Moreover $A(u) \in \SpaceLp{p}(I; V^*)$ with:
    \begin{align}
        \lVert A(u) \rVert_{\SpaceLp{p}(I; V^*)} \leq \ContC{a} \lVert u \rVert_{\SpaceLp{p}(I; V)}.
    \end{align}
\end{lemma}

\begin{definition}[Model problem]
    Let $f \in \SpaceLp{2}(I; V^*)$, the model problem is to find $u \in \SpaceX(I; V, V^*)$ such that:
    \begin{align}
        \begin{cases} \label{equation:model}
            \partial_t u(t) + A(u)(t) = f(t) & \text{ in } \SpaceLp{2}(I; V^*), \\
            u(0) = u_0 & \text{ in } W.
        \end{cases}
    \end{align}
\end{definition}

\newpage
\subsection{Weak Formulation}

\begin{definition}[Trial and test spaces]
    Let $\SpaceTrial = \SpaceX(I; V, V^*)$ be the trial space and let $\SpaceTest = \SpaceTest_0 \times \SpaceTest_1$ be the test space where $\SpaceTest_0 = W$ and $\SpaceTest_1 = \SpaceLp{2}(I; V)$.
\end{definition}

\begin{definition}[$b$ and $l$]
    Let $b \colon \SpaceTrial \times \SpaceTest \rightarrow \RealNumbers$ and $l \colon \SpaceTest \rightarrow \RealNumbers$ be such that for all $u \in \SpaceTrial$ and all $y = (y_0, y_1) \in \SpaceTest$:
    \begin{align}
        & b(u, y) = \left( u(0), y_0 \right)_W + \int_I \langle \partial_t u(t) + A(u)(t), y_1(t) \rangle_{V^*, V} ~ d \lambda(t), \\
        & l(y) = \left( u_0, y_0 \right)_W + \int_I \langle f(t), y_1(t) \rangle_{V^*, V} ~ d \lambda(t).
    \end{align}
\end{definition}

\begin{definition}[Weak formulation]
    The weak formulation of \cref{equation:model} is to find $u \in \SpaceTrial$ such that:
    \begin{align}
        b(u, y) = l(y) \text{ for all } y \in \SpaceTest. \label{equation:weak}
    \end{align}
\end{definition}

\begin{definition}[Parabolic equation]
    Let $f \in \SpaceLp{2}(I; V^*)$ and $u_0 \in W$, then \cref{equation:weak} is said to be parabolic if $A$ satisfies \cref{equation:a}.
\end{definition}

\begin{lemma}[Weak solution]
    Let $u \in \SpaceTrial$ solution of \cref{equation:weak}, then $\partial_t u(t) + A(u)(t) = f(t)$ for $\lambda$-a.e. $t \in I$ and $u(0) = u_0$ in $W$.
\end{lemma}

\newpage
\subsection{Well-Posedeness}

The objective of this section is to establish the well-posedness of the parabolic model problem \cref{equation:weak}.

\subsubsection{Uniqueness}

\begin{lemma}[Uniqueness and a priori estimate]
    Let $u \in \SpaceTrial$ solution of \cref{equation:weak}, then \cref{equation:weak} admits at most one solution and:
    \begin{align}
        \CoerC{a} \lVert u \rVert_{\SpaceLp{2}(I; V)}^2 + \lVert u(T) \rVert_W^2 \leq \frac{1}{\CoerC{a}} \lVert f \rVert_{\SpaceLp{2}(I; V^*)}^2 + \lVert u_0 \rVert_W^2.
    \end{align}
\end{lemma}

% [!] Proof to be included.

% \begin{lemma}[$W$-norm estimate]
%     % [!]
% \end{lemma}

\subsubsection{Existence}

\begin{lemma}[Existence]
    There exists $u \in \SpaceTrial$ solution of \cref{equation:weak}.
\end{lemma}

% [!] Proof to be included.

\newpage
\section{Discretization of the Model Parabolic Problem}

\subsection{Conforming Semi-Discretization in Space}

Following \cite[p.~135]{Ern2021}, set $a(t; u, v) = \langle A(t)(u), v \rangle_{V^*, V}$ for all $u, v \in V$ and for $\lambda$-a.e. $t \in I$.

\begin{definition}[$b$ and $l$]
    Let $b \colon \SpaceTrial \times \SpaceTest \rightarrow \RealNumbers$ and $l \colon \SpaceTest \rightarrow \RealNumbers$ be such that for all $u \in \SpaceTrial$ and all $y = (y_0, y_1) \in \SpaceTest$:
    \begin{align}
        b(u, y) &= \left( u(0), y_0 \right)_W + \int_I \left( \langle \partial_t u(t), y_1(t) \rangle_{V^*, V} + a(t; u(t), y_1(t)) \right)~ d \lambda(t), \\
        l(y) &= \left( u_0, y_0 \right)_W + \int_I \langle f(t), y_1(t) \rangle_{V^*, V} ~ d \lambda(t).
    \end{align}
\end{definition}

Let $\left\{ V_h \right\}_{h \in \hIndices}$ be a sequence of finite-dimensional subspaces of $V$ constructed using a finite element and a shape-regular mesh family $\left\{ \SpaceMesh \right\}_{h \in \hIndices}$, which is fixed in time, such that each mesh exactly covers $\Omega$.

Furthermore, let $\left\{ \varphi_j \right\}_{j \in \SpaceIndices}$ be a basis for $V_h$.

\begin{definition}[Semi-discrete trial and test spaces]
    Let $\SpaceTrial_h = \SpaceHk{1}(I; V_h) = \SpaceX(I; V_h, V_h)$ be the semi-discrete trial space and let $\SpaceTest_h = V_h \times \SpaceLp{2}(I; V_h)$ be the semi-discrete test space.
\end{definition}

Observe that $\SpaceTrial_h \subset \SpaceTrial$, $\SpaceTest_h \subset \SpaceTest$, and a generic $u_h \in \SpaceTrial_h$ is of the form of:
\begin{align}
    u_h(t, \Vector{x}) = \sum_{j \in \SpaceIndices} u_j(t) \varphi_j(\Vector{x}),
\end{align}
and:
\begin{align}
    \partial_t u_h(t, \Vector{x}) = \sum_{j \in \SpaceIndices} u_j^{\prime}(t) \varphi_j(\Vector{x}),
\end{align}
with $u_j \in \SpaceHk{1}(I)$ for all $j \in \SpaceIndices$. Similarly, a generic $y_h \in \SpaceTest_h$ is a pair $(y_{0h}, y_{1h})$ where $y_{0h} \in V_h$ and $y_{1h}$ is of the form of:
\begin{align}
    y_{1h}(t, \Vector{x}) = \sum_{j \in \SpaceIndices} y_j(t) \varphi_j(\Vector{x}),
\end{align}
with $y_j \in \SpaceLp{2}(I)$ for all $j \in \SpaceIndices$.

\begin{definition}[Semi-discrete weak formulation]
    The semi-discrete weak formulation of \cref{equation:model} is to find $u_h \in \SpaceTrial_h$ such that:
    \begin{align} \label{equation:weak_h1}
        b(u_h, y_h) &= l(y_h) &\text{ for all } y_h \in \SpaceTest_h.
    \end{align}
\end{definition}

\begin{definition}[$\ProjectionOnto{V_h}$]
    Let $\ProjectionOnto{V_h} \colon W \rightarrow V_h$ be the $W$-orthogonal projection, that is, for all $w \in W$, $\ProjectionOnto{V_h}(w)$ is the unique element in $V_h$ such that:
    \begin{align}
        \left( w - \ProjectionOnto{V_h}(w), v_h \right)_W &= 0 &\text{ for all } v_h \in V_h.
    \end{align}
\end{definition}

\begin{proposition}[Equivalence]
    $u_h \in \SpaceTrial_h$ solves \cref{equation:weak_h1} if and only if for all $v_h \in V_h$:
    \begin{align}
        \begin{cases} \label{equation:weak_h2}
            \left( \partial_t u_h(t), v_h \right)_W + a(t; u_h(t), v_h) = \langle f(t), v_h \rangle_{V^*, V} \text{ in } \SpaceLp{2}(I), \\
            u_h(0) = \ProjectionOnto{V_h}(u_0),
        \end{cases}
    \end{align}
    and both \cref{equation:weak_h1} and \cref{equation:weak_h2} are well-posed.

    Moreover, if $f \in C^0(\overline{I}; V^*)$ and $A \in C^0(\overline{I}; \Linear(V; V^*))$, $u_h \in C^1(\overline{I}; V_h)$.
\end{proposition}

\subsubsection{Error Analysis}

% [!] Expand here.



\begin{theorem}[$\SpaceLp{2}(I; V)$-estimate] \label{theorem:estimate_h}
    Let $u \in \SpaceX$ be the solution of \cref{equation:model} and $u_h \in \SpaceX_h$ be the solution of \cref{equation:weak_h1}. Let $\InterpolantH$ be any approximation operator and set $\eta(t) = u(t) - \InterpolantH(u(t))$ for all $t \in I$. Assume that $u \in \SpaceHk{1}(I; V)$, then:
    \begin{align}
        \lVert u - u_{h \tau} \rVert_{\SpaceLp{2}(I; V)} \leq \left( 1 + \CoCoR{a} \right) \lVert \eta \rVert_{\SpaceLp{2}(I; V)} + \frac{1}{\CoerC{a}} \lVert \partial_t \eta \rVert_{\SpaceLp{2}(I; V^*)} + \frac{1}{\sqrt{\CoerC{a}}} \lVert \eta(0) \rVert_W,
    \end{align}
    where:
    \begin{align}
        \CoCoR{a} = \frac{\ContC{a}}{\CoerC{a}}.
    \end{align}
\end{theorem}

\newpage
\subsection{Discontinuous Galërkin Discretization in Time} \label{subsection:time_dg}

Following \cite[p.~177]{Ern2021}, let $N > 0$ be a positive natural number. Fix $\TimeMesh = \left\{ I_n \right\}_{n \in \TimeIndices}$ partition of $I$ where $\TimeIndices = \left\{ 1, \dots, N \right\}$, $I_n = (t_{n - 1}, t_n]$ and $\tau = \frac{T}{N}$.

\begin{definition}[Mapping]
    Let $T_n \colon \hat{I} \rightarrow I_n$ for $\hat{I} = (-1, 1]$ be such that:
    \begin{align}
        T_n(t) &= \frac{1}{2}(t_{n - 1} + t_n) + \frac{1}{2} \tau t &\text{ for all } t \in \hat{I}.
    \end{align}
\end{definition}

Let $H$ be a real Hilbert space composed of functions defined on the space domain $\Omega \subset \RealNumbersTo{d}$, and let $q \geq 0$ the polynomial degree for the time approximation of the functions in $\SpaceLp{1}(I; H)$.

\begin{definition}[$\SpacePolynomials{q}(I_n; H)$]
    \begin{align}
        \SpacePolynomials{q}(I_n; H) = \SpacePolynomials{q}(I_n; \RealNumbers) \otimes H,
    \end{align}
    that is $u \in \SpacePolynomials{q}(I_n; H)$ if there exist $M \in \NaturalNumbers$ and $\left\{ (u_j, p_j) \right\}_{j = 1}^M \subset H \times \SpacePolynomials{q}(I_n; \RealNumbers)$ such that:
    \begin{align}
        u(t) = \sum_{j = 1}^M u_j p_j(t).
    \end{align}
\end{definition}

Let $\left\{ \psi_i \right\}_{i = 0}^q$ be a basis for $\SpacePolynomials{q}(\hat{I}; \RealNumbers)$, then $u \in \SpacePolynomials{q}(I_n; H)$ if there exist $\left\{ u_i \right\}_{i = 0}^q \subset H$ such that:
\begin{align} % [!] Check indices (i).
    u = \sum_{i = 0}^q u_i (\psi_i \circ T_n^{-1}).
\end{align}

\begin{definition}[$\SpacePolynomials{q}^b(\TimeMesh; H)$]
    \begin{align}
        \SpacePolynomials{q}^b(\TimeMesh; H) = \left\{ u_{\tau} \colon I \rightarrow H \text{ such that } u_{\tau \mid I_n} \in \SpacePolynomials{q}(I_n; H) \text{ for all } n \in \TimeIndices \right\},
    \end{align}
    that is the space of the $H$-valued functions that are piecewise polynomials of degree at most $q$ on the time mesh $\TimeMesh$.
\end{definition}

\begin{definition}[$\SpacePolynomials{q}^b(\ClosedTimeMesh; H)$]
    \begin{align}
        \SpacePolynomials{q}^b(\ClosedTimeMesh; H) = \left\{ u_{\tau} \colon \overline{I} \rightarrow H \text{ such that } u_{\tau \mid (0, T]} \in \SpacePolynomials{q}^b(\TimeMesh; H) \right\}.
    \end{align}
\end{definition}

% Note that every function $u_{\tau} \in \SpacePolynomials{q}^b(\ClosedTimeMesh; H)$ can be represented by the pair $(u_{\tau}(0), u_{\tau \mid (0, T]}) \in H \times \SpacePolynomials{q}^b(\TimeMesh; H)$, meaning that $\SpacePolynomials{q}^b(\ClosedTimeMesh; H)$ is isomorphic to $H \times \SpacePolynomials{q}^b(\TimeMesh; H)$.

By definition, every $u_{\tau} \in \SpacePolynomials{q}^b(\ClosedTimeMesh; H)$ is left-continuous at the discrete time nodes $t_n$.

\begin{definition}[Time jumps] \label{def:time_jumps}
    \begin{align}
        \llbracket u_{\tau} \rrbracket_{n - 1} = u_{\tau}(t_{n - 1}^+) - u_{\tau}(t_{n - 1}),
    \end{align}
    where:
    \begin{align}
        u_{\tau}(t_{n - 1}^+) = \lim_{\varepsilon \rightarrow 0^+} u_{\tau}(t_{n - 1} + \varepsilon).
    \end{align}
\end{definition}

Another useful space is the subspace of $\SpacePolynomials{q}^b(\ClosedTimeMesh; H)$ composed of the $H$-valued functions that are continuous in time.

\begin{definition}[$\SpacePolynomials{q}^g(\ClosedTimeMesh; H)$]
    \begin{align}
        \SpacePolynomials{q}^g(\TimeMesh; H) = \SpacePolynomials{q}^b(\ClosedTimeMesh; H) \cap C^0(\overline{I}; H).
    \end{align}
\end{definition}

\newpage
\subsection{Method Formulation}

\subsubsection{Quadratures and Interpolation}

Let $\left\{ \zeta_i \right\}_{i = 0}^{q}$ and $\left\{ \omega_i \right\}_{i = 0}^{q}$ be the \acrfull{gr} quadrature nodes and corresponding weights on $\hat{I}$, thus:
\begin{align}
    t_{n, i} &= T_n(\zeta_i) &\text{ for all } i \in \left\{ 0, \dots, q \right\}, \\
    \omega_{n, i} &= \frac{\tau}{2} \omega_i &\text{ for all } i \in \left\{ 0, \dots, q \right\},
\end{align}
are the quadrature nodes and corresponding weights on the $n$-th time interval $I_n$.

\begin{definition}[$\lambdaGR_q$]
    For all $u \in C^0(\overline{I}; \RealNumbers)$:
    \begin{align}
        \int_I u(t) ~ d \lambdaGR_q(t) = \sum_{n \in \TimeIndices} \int_{I_n} u(t) ~ d \lambdaGR_q(t) = \sum_{n \in \TimeIndices} \sum_{i = 0}^q \omega_{n, i} u(t_{n, i}).
    \end{align}
\end{definition}

\begin{definition}[Lagrange polynomials]
    Let $\Lagrange$ defined as:
    \begin{align}
        \Lagrange_i (\zeta) = \prod_{\substack{l = 0 \\ l \neq i}}^q \frac{\zeta - \zeta_l}{\zeta_i - \zeta_l} \in \SpacePolynomials{q}(\hat{I}; \RealNumbers),
    \end{align}
    be the Lagrange polynomial based on the \acrshort{gr} nodes and associated with the $i$-th node, so that $\Lagrange_i (\zeta_l) = \delta_{i l}$ for all $i, l \in \left\{ 0, \dots, q\right\}$.
\end{definition}

\begin{definition}[$\InterpolantGR_q$] \label{def:interpolant_gr}
    Let $Z \in \left\{V_h, W, V^* \right\}$. Define $\InterpolantGR_q \colon \SpaceHk{1}(I; Z) \rightarrow \SpacePolynomials{q}^b(\TimeMesh; Z)$ as the Lagrange interpolation operator associated with the \acrshort{gr} nodes, such that for all $u \in \SpaceHk{1}(I; Z) \hookrightarrow C^0(\overline{I}; Z)$:
    \begin{align}
        \InterpolantGR_q(u) \Restriction_{I_n} = \sum_{i = 0}^q u(t_{n, i}) \left( \Lagrange_i \circ T_n^{-1} \right).
    \end{align}

    Moreover, there exists $C \geq 0$ such that:
    \begin{align}
        \lVert u - \InterpolantGR_q(u) \rVert_{\SpaceLp{2}(I; Z)} \leq C \tau^{q + 1} \lvert u \rvert_{\SpaceHk{q + 1}(I; Z)}.
    \end{align}
\end{definition}

\begin{proposition}
    For all $p \in \SpacePolynomials{q}^b(\TimeMesh; W)$ and for all $u, v \in \SpaceHk{1}(I; W)$:
    \begin{align}
        & \int_I \left( u, \InterpolantGR_q(v) \right)_W ~ d \lambdaGR_q(t) = \int_I \left( u, v \right)_W ~ d \lambdaGR_q(t), \\
        & \int_I \left( p, \InterpolantGR_q(v) \right)_W ~ d \lambda(t) = \int_I \left( p, v \right)_W ~ d \lambdaGR_q(t)
    \end{align}
\end{proposition}

\subsubsection{Discretization}

\begin{definition}[Discrete trial and test spaces]
    Let $\SpaceTrial_{h \tau} = \SpacePolynomials{q}^b(\ClosedTimeMesh; V_h)$ be the discrete trial space and let $\SpaceTest_{h \tau} = \SpaceTrial_{h \tau}$ be the discrete test space.
\end{definition}

\begin{definition}[$b_{\tau}$ and $l_{\tau}$]
    Let $b_{\tau} \colon \SpaceTrial_{h \tau} \times \SpaceTest_{h \tau} \rightarrow \RealNumbers$ and $l_{\tau} \colon \SpaceTest_{h \tau} \rightarrow \RealNumbers$ be such that for all $u_{h \tau} \in \SpaceTrial_{h \tau}$ and all $y_{h \tau} \in \SpaceTest_{h \tau}$:
    \begin{align}
        b_{\tau}(u_{h \tau}, y_{h \tau}) & = \left( u_{h \tau}(0), y_{h \tau}(0) \right)_W + \sum_{n \in \TimeIndices} \int_{I_n} \left( \partial_t u_{h \tau}(t), y_{h \tau}(t) \right)_W ~ d \lambdaGR_q(t) \\
        & + \sum_{n \in \TimeIndices} \left( \llbracket u_{h \tau} \rrbracket_{n - 1}, y_{h \tau}(t_{n - 1}^+) \right)_W + \int_I a(t; u_{h \tau}(t), y_{h \tau}(t)) ~ d \lambdaGR_q(t), \notag \\
        l_{\tau}(y_{h \tau}) & = \left( u_0, y_{h \tau}(0) \right)_W + \int_I \langle f(t), y_{h \tau}(t) \rangle_{V^*, V} ~ d \lambdaGR_q(t).
    \end{align}
\end{definition}

\begin{remark}
    \acrshort{gr} gives a quadrature of order $2q$, hence:
    \begin{align}
        \int_{I_n} \left( \partial_t u_{h \tau}(t), y_{h \tau}(t) \right)_W ~ d \lambdaGR_q(t) &= \int_{I_n} \left( \partial_t u_{h \tau}(t), y_{h \tau}(t) \right)_W ~ d \lambda(t) &\text{ for all } n \in \TimeIndices.
    \end{align}
    
    The same applies to:
    \begin{align}
        \int_I a(t; u_{h \tau}(t), y_{h \tau}(t)) ~ d \lambdaGR_q(t),
    \end{align}
    if $a$ is time-independent.
\end{remark}
In \cref{chapter:cdr}, a sufficiently accurate quadrature order will be assumed, ensuring that only exact quadratures are employed.

\begin{definition}[$dG(q)$ weak formulation]
    The discrete weak formulation of \cref{equation:model} is to find $u_{h \tau} \in \SpaceTrial_{h \tau}$ such that:
    \begin{align} \label{equation:weak_ht}
        b_{\tau}(u_{h \tau}, y_{h \tau}) &= l_{\tau}(y_{h \tau}) &\text{ for all } y_{h \tau} \in \SpaceTest_{h \tau}.
&    \end{align}
\end{definition}

\begin{proposition}[Localization]
    The $dG(q)$ solution $u_{h \tau}$ of \cref{equation:weak_ht}, if it exists, is such that $u_{h \tau}(0) = \ProjectionOnto{V_h}(u_0)$ and for all $p \in \SpacePolynomials{q}(I_n; V_h)$:
    \begin{align}
        \int_{I_n} \left( \partial_t u_{h \tau}(t), p(t) \right)_W ~ d \lambda(t) & + \sum_{n \in \TimeIndices} \left( \llbracket u_{h \tau} \rrbracket_{n - 1}, p(t_{n - 1}^+) \right)_W \notag \\
        + \int_{I_n} a(t; u_{h \tau}(t), y_{h \tau}(t)) ~ d \lambdaGR_q(t) & = \int_I \langle f(t), p(t) \rangle_{V^*, V} ~ d \lambdaGR_q(t) &\text{ for all } n \in \TimeIndices.
    \end{align}
\end{proposition}

\newpage
\subsection{Stability and Error Analysis}

Following \cite[p.~186]{Ern2021}, recall that $\CoerC{a}$ is the coercivity constant of the bilinear form $a$ and equip $\SpaceX_{h \tau}$ with the following norm such that for all $u_{h \tau} \in \SpaceX_{h \tau}$:
\begin{align}
    \lVert u_{h \tau} \rVert_{\SpaceX_{h \tau}}^2 = \lVert u_{h \tau} \rVert_{\SpaceLp{2}(I; V)}^2 + \frac{1}{2 \CoerC{a}} \left( \lVert u_{h \tau}(0) \rVert_W^2 + \lVert u_{h \tau}(T) \rVert_W^2 + \sum_{n \in \TimeIndices} \lVert \llbracket u_{h \tau} \rrbracket_{n - 1} \rVert_W^2 \right). \label{eq:norm_ht}
\end{align}

\subsubsection{Stability}

\begin{lemma}[Coercivity]
    The discrete problem \cref{equation:weak_ht} is well posed by:
    \begin{align}
        b_{\tau}(u_{h \tau}, u_{h \tau}) &\geq \CoerC{a} \lVert u_{h \tau} \rVert_{\SpaceX_{h \tau}}^2 &\text{ for all } u_{h \tau} \in \SpaceX_{h \tau}. \label{eq:b_coercivity}
    \end{align}
\end{lemma}

\begin{proof}
    By the coercivity of $a$ at the $q + 1$ \acrshort{gr} nodes and the positivity of the weights, it follows that:
    \begin{align*}
        \int_{I_n} a(t; u_{h \tau}(t), u_{h \tau}(t)) ~ d \lambdaGR_{q + 1}(t) &= \sum_{i = 1}^{q + 1} \omega_{n, i} a(t_{n, i}; u_{h \tau}(t_{n, i}), u_{h \tau}(t_{n, i})) \\
        &\geq \CoerC{a} \sum_{i = 1}^{q + 1} \omega_{n, i} \Norm{ u_{h \tau}(t_{n, i}) }_V^2 \\
        &= \CoerC{a} \int_{I_n} \Norm{ u_{h \tau}(t) }_V^2 ~ d \lambda(t),
    \end{align*}
    given that the quadrature is of order $2q$.

    Moreover, by using the facts that:
    \begin{align*}
        \frac{d}{dt} \Norm{ u_{h \tau} }_W^2 &= 2 \left( \partial_t u_{h \tau}, u_{h \tau} \right)_W, \\
        2 \left( u_{h \tau}(t_{n - 1}^+) - u_{h \tau}(t_{n - 1}) \right) u_{h \tau}(t_{n - 1}^+) &= u_{h \tau}^2(t_{n - 1}^+) - u_{h \tau}^2(t_{n - 1}) + \left( u_{h \tau}(t_{n - 1}^+) - u_{h \tau}(t_{n - 1}) \right)^2,
    \end{align*}
    together with \cref{def:time_jumps}:
    \begin{align*}
        \sum_{n \in \TimeIndices} &\left( \int_{I_n} \left( \partial_t u_{h \tau}(t), u_{h \tau}(t) \right)_W ~ d \lambda(t) + \left( \llbracket u_{h \tau} \rrbracket_{n - 1}, u_{h \tau}(t_{n - 1}^+) \right)_W \right) \\
        &= \frac12 \sum_{n \in \TimeIndices} \left( \Norm{ u_{h \tau}(t_n) }_W^2 - \Norm{ u_{h \tau}(t_{n - 1}^+) }_W^2 + 2 \left( \llbracket u_{h \tau} \rrbracket_{n - 1}, u_{h \tau}(t_{n - 1}^+) \right)_W \right) \\
        &= \frac12 \sum_{n \in \TimeIndices} \left( \Norm{ u_{h \tau}(t_n) }_W^2 + \Norm{ u_{h \tau}(t_{n - 1}) }_W^2 + \Norm{ \llbracket u_{h \tau} \rrbracket_{n - 1} }_W^2 \right) \\
        &= \frac12 \Norm{ u_{h \tau}(T) }_W^2 - \frac12 \Norm{ u_{h \tau}(0) }_W^2 + \frac12 \sum_{n \in \TimeIndices} \Norm{ \llbracket u_{h \tau} \rrbracket_{n - 1} }_W^2.
    \end{align*}

    This, together with the lower bound on $a$, proves \cref{eq:b_coercivity}, while the well-posedness of \cref{equation:weak_ht} follows from \cref{theorem:lm}.
\end{proof}

\newpage
\subsubsection{Error Analysis}

\begin{definition}[$\InterpolantQN$] \label{def:interpolant_qn}
    Let $Z \in \left\{V_h, W\right\}$. Define $\InterpolantQN \colon \SpaceHk{1}(I_n; Z) \rightarrow \SpacePolynomials{q}(I_n; Z)$ such that for all $u \in \SpaceHk{1}(I_n; Z)$ and for all $n \in \TimeIndices$:
    \begin{align}
        \InterpolantQN(u)(t_n) & = u(t_n), \\
        \int_{I_n} \left( \InterpolantQN(u) - u, p \right)_W ~ d \lambda(t) &= 0 &\text{ for all } p \in \SpacePolynomials{q - 1}(I_n; Z)
    \end{align}
\end{definition}

\begin{definition}[$\InterpolantQT$] \label{def:interpolant_qt}
    Let $Z \in \left\{V_h, W\right\}$. Define $\InterpolantQT \colon \SpaceHk{1}(I; Z) \rightarrow \SpacePolynomials{q}^b(\ClosedTimeMesh; Z)$ such that for all $u \in \SpaceHk{1}(I_n; Z)$ and for all $n \in \TimeIndices$:
    \begin{align}
        & \InterpolantQT(u)(0) = u(0), \\
        & \InterpolantQT(u) \Restriction_{I_n} = \InterpolantQN(u \Restriction_{I_n}).
    \end{align}
\end{definition}

\begin{proposition} \label{prop:interpolant_qt}
    Set $Z$ as in \cref{def:interpolant_qt}. There exists $C > 0$ such that for all $u \in \SpaceHk{1}(I_n; Z)$ and for all $\tau > 0$:
    \begin{align}
        \Norm{ u - \InterpolantQT \left( u \right) }_{\SpaceLp{\infty}(I; Z)} \leq C \tau^{q + 1} \Seminorm{ u }_{\SpaceWkp{q + 1}{\infty}(I; Z)}.
    \end{align}
\end{proposition}

\begin{lemma}[Orthogonality] \label{lemma:orthogonality}
    For all $u \in \SpaceHk{1}(I; W)$, $y_{\tau} \in \SpacePolynomials{q}^b(\ClosedTimeMesh; W)$ and all $n \in \TimeIndices$:
    \begin{align}
        \int_{I_n} \left( \partial_t \left( u - \InterpolantQT(u) \right), y_{\tau} \right)_W ~ d \lambda(t) - \left( \llbracket \InterpolantQT(u) \rrbracket_{n - 1}, y_{\tau}(t_{n - 1}^+) \right)_W = 0.
    \end{align}
\end{lemma}

\begin{proof}
    Let:
    \begin{align*}
        \delta = u - \InterpolantQT(u),
    \end{align*}
    then, integrating by parts and using \cref{def:interpolant_qn}, together with the facts that $\delta()$ :
    \begin{align*}
        \delta(t_n) &= 0, \\
        \partial_t y_{\tau}\Restriction_{I_n} &\in \SpacePolynomials{q - 1}(I_n; W),
    \end{align*}
    for all $n \in \TimeIndices$, it follows that:
    \begin{align*}
        \int_{I_n} \left( \partial_t \delta, y_{\tau} \right)_W ~ d \lambda(t) &= - \int_{I_n} \left( \delta, \partial_t y_{\tau} \right)_W ~ d \lambda(t) + \left( \delta(t_n), y_{\tau}(t_n) \right)_W - \left( \delta(t_{n - 1}^+), y_{\tau}(t_{n - 1}^+) \right)_W \\
        &= - \left( \delta(t_{n - 1}^+), y_{\tau}(t_{n - 1}^+) \right)_W.
    \end{align*}

    Since, by \cref{def:interpolant_qt}:
    \begin{align*}
        \InterpolantQT(u)(t_{n - 1}) = u(t_{n - 1}) &= u(t_{n - 1}^+),
    \end{align*}
    as $u \in C^0(\overline{I}; W)$, it follows that:
    \begin{align*}
        \llbracket \InterpolantQT(u) \rrbracket_{n - 1} &= - \delta(t_{n - 1}^+),
    \end{align*}
    hence:
    \begin{align*}
        \int_{I_n} \left( \partial_t \delta, y_{\tau} \right)_W ~ d \lambda(t) &= - \left( \delta(t_{n - 1}^+), y_{\tau}(t_{n - 1}^+) \right)_W = \left( \llbracket \InterpolantQT(u) \rrbracket_{n - 1}, y_{\tau}(t_{n - 1}^+) \right)_W.
    \end{align*}
\end{proof}

Consider a time-independent space approximation operator $\InterpolantH$ to separate the time and space approximations.

\begin{theorem}[$\SpaceLp{2}(I; V)$-estimate] \label{theorem:estimate_ht}
    Let $u \in \SpaceX$ be the solution of \cref{equation:model} and $u_{h \tau} \in \SpaceX_{h \tau}$ be the solution of \cref{equation:weak_ht}. Assume that $u \in \SpaceHk{q + 2}(I; V^*) \cap \SpaceWkp{q + 1}{\infty}(I; V)$, then there exists $C \geq 0$ such that for all $h \in \hIndices$ and $T > 0$:
    \begin{align}
        \lVert u - u_{h \tau} \rVert_{\SpaceX_{h \tau}} \leq & \frac{\sqrt{2}}{\sqrt{\CoerC{a}}} \lVert u_0 - \InterpolantH \left( u_0 \right) \rVert_W + C C_1(u) \tau^{q + 1} \label{eq:estimates_ht} \\
        & + \frac{C}{\CoerC{a}} \lVert \partial_t u - \InterpolantH \left( \partial_t u \right) \rVert_{\SpaceLp{2}(I; V^*)} + C \CoCoR{a} C_2 \lVert u - \InterpolantH \left( u \right) \rVert_{\SpaceLp{\infty}(I; V)}, \notag
    \end{align}
    where:
    \begin{align}
        C_1(u) & = \frac{1}{\CoerC{a}} \lvert u \rvert_{\SpaceHk{q + 2}(I; V^*)} + \CoCoR{a} \sqrt{T} \lvert u \rvert_{\SpaceWkp{q + 1}{\infty}(I; V)}, \\
        C_2^2 & = \max \left\{ 2 \frac{\iota_{V, W}^2}{\CoerC{a}}, T \right\}.
    \end{align}
\end{theorem}

\begin{proof}
    Let $y_{h \tau} \in \SpaceX_{h \tau}$, then, by the regularity assumptions on $u$, it follows that:
    \begin{align*}
        \left( \partial_t u(t_{n, i}), y_{h \tau} \right)_W + a \left( t_{n, i}; u(t_{n, i}), y_{h \tau} \right) &= \left\langle f(t_{n, i}), y_{h \tau} \right\rangle_{V^*, V},
    \end{align*}
    for all $n \in \TimeIndices$ and for all $i \in \left\{ 1, \dots, q + 1\right\}$.

    Hence:
    \begin{align*}
        b(u_{h \tau}, y_{h \tau}) &= \left( u_0, y_{h \tau}(0) \right)_W + \int_I \left\langle f, y_{h \tau} \right\rangle_{V^*, V} ~ d \lambdaGR_{q + 1}(t) \\
        &= \left( u_0, y_{h \tau}(0) \right)_W + \Remainder \left( y_{h \tau} \right) + \int_I \left( \partial_t u, y_{h \tau}\right)_W ~ d \lambda(t) + \int_I a \left( t; u, y_{h \tau}\right) ~ d \lambdaGR_{q + 1}(t),
    \end{align*}
    where:
    \begin{align*}
        \Remainder \left( y_{h \tau} \right) &= \int_I \left( \partial_t u, y_{h \tau}\right)_W ~ d \lambdaGR_{q + 1}(t) - \int_I \left( \partial_t u, y_{h \tau}\right)_W ~ d \lambda(t).
    \end{align*}

    Introduce $w_{h \tau} = \InterpolantQT \left( \InterpolantH \left( u \right) \right) \in \SpaceX_{h \tau}$, and set $e_{h \tau} = u_{h \tau} - w_{h \tau}$ and $\eta = u - w_{h \tau}$, then:
    \begin{align*}
        b(u_{h \tau} - w_{h \tau}, y_{h \tau}) &= \left( u_0 - w_{h \tau}(0), y_{h \tau}(0) \right)_W + \Remainder \left( y_{h \tau} \right) \\
        &+ \sum_{n \in \TimeIndices} \int_{I_n} \left( \partial_t \left( u - w_{h \tau} \right), y_{h \tau}\right)_W ~ d \lambda(t) - \sum_{n \in \TimeIndices}  \left( \llbracket w_{h \tau} \rrbracket_{n - 1}, y_{h \tau}(t_{n - 1}^+) \right)_W \\
        &+ \int_I a \left( t; u - w_{h \tau}, y_{h \tau}\right) ~ d \lambdaGR_{q + 1}(t).
    \end{align*}

    By \cref{def:interpolant_qt}, $w_{h \tau}(0) = \InterpolantH \left( u_0 \right) \in V$, and by \cref{lemma:orthogonality} it follows that, for all $n \in \TimeIndices$:
    \begin{align*}
        \int_{I_n} &\left( \partial_t w_{h \tau}, y_{h \tau}\right)_W ~ d \lambda(t) + \left( \llbracket w_{h \tau} \rrbracket_{n - 1}, y_{h \tau}(t_{n - 1}^+) \right)_W \\
        &= \int_{I_n} \left( \partial_t \left( \InterpolantQT \left( \InterpolantH \left( u \right) \right) \right), y_{h \tau}\right)_W ~ d \lambda(t) + \left( \llbracket \InterpolantQT \left( \InterpolantH \left( u \right) \right) \rrbracket_{n - 1}, y_{h \tau}(t_{n - 1}^+) \right)_W \\
        &= \int_{I_n} \left( \partial_t \left( \InterpolantH \left( u \right) \right), y_{h \tau}\right)_W ~ d \lambda(t) = \int_{I_n} \left( \InterpolantH \left( \partial_t u \right), y_{h \tau}\right)_W ~ d \lambda(t).
    \end{align*}

    Hence:
    \begin{align*}
        b(e_{h \tau}, y_{h \tau}) &= \left( u_0 - \InterpolantH \left( u_0 \right), y_{h \tau}(0) \right)_W + \Remainder \left( y_{h \tau} \right) \\
        &+ \int_I \left( \partial_t u - \InterpolantH \left( \partial_t u \right), y_{h \tau}\right)_W ~ d \lambda(t) + \int_I a \left( t; \eta, y_{h \tau}\right) ~ d \lambdaGR_{q + 1}(t).
    \end{align*}

    Regarding the terms in the right-hand side, it holds that:
    \begin{align*}
        \left\lvert \left( u_0 - \InterpolantH \left( u_0 \right), y_{h \tau}(0) \right)_W \right\rvert \leq \Norm{ u_0 - \InterpolantH \left( u_0 \right) }_W \Norm{ y_{h \tau}(0) }_W.
    \end{align*}

    Moreover, since:
    \begin{align*}
        \int_I \left( \partial_t u, y_{h \tau}\right)_W ~ d \lambdaGR_{q + 1}(t) = \int_I \left( \InterpolantGR_q \left( \partial_t u \right), y_{h \tau}\right)_W ~ d \lambda(t),
    \end{align*}
    by \cref{def:interpolant_gr}, it follows that:
    \begin{align*}
        \Remainder \left( y_{h \tau} \right) = \int_I \left( \InterpolantGR_q \left( \partial_t u \right) - \partial_t u, y_{h \tau}\right)_W ~ d \lambda(t),
    \end{align*}
    thus:
    \begin{align*}
        \left\lvert \Remainder \left( y_{h \tau} \right) \right\rvert &\leq \Norm{ \InterpolantGR_q \left( \partial_t u \right) - \partial_t u }_{\SpaceLp{2}(I; V^*)} \Norm{ y_{h \tau} }_{\SpaceLp{2}(I; V)} \\
        &\leq C \tau^{q + 1} \Seminorm{u}_{\SpaceHk{q + 2}(I; V^*)} \Norm{ y_{h \tau} }_{\SpaceLp{2}(I; V)}.
    \end{align*}

    Furthermore:
    \begin{align*}
        \left\lvert \int_I \left( \partial_t u - \InterpolantH \left( \partial_t u \right), y_{h \tau}\right)_W ~ d \lambda(t) \right\rvert &\leq \Norm{ \partial_t u - \InterpolantH \left( \partial_t u \right) }_{\SpaceLp{2}(I; V^*)} \Norm{ y_{h \tau} }_{\SpaceLp{2}(I; V)}.
    \end{align*}

    Additionally, by the commutation properties of $\InterpolantQT$ and $\InterpolantH$, the stability of $\InterpolantH$, and by \cref{prop:interpolant_qt}, it follows that for all $n \in \TimeIndices$ and for all $i \in \left\{ 1, \dots, q + 1\right\}$:
    \begin{align*}
        \Norm{ \eta(t_{n, i}) }_V &\leq \Norm{ \left( u - \InterpolantH \left( u \right) \right) \left( t_{n, i} \right) }_V + \Norm{ \InterpolantH }_{\Linear \left(V; V_h \right)} \Norm{ \left( u - \InterpolantQT \left( u \right) \right) \left( t_{n, i} \right) }_V \\
        &\leq \Norm{ u - \InterpolantH \left( u \right) }_{\SpaceLp{\infty}(I; V)} + C \tau^{q + 1} \Seminorm{ u }_{\SpaceWkp{q + 1}{\infty}(I; V)} = C(u).
    \end{align*}

    Thus, by the boundedness of $a$, the Cauchy-Schwarz inequality, the facts that the quadrature is of order $2q$ and:
    \begin{align*}
        \sum_{n \in \TimeIndices} \sum_{i = 1}^{q + 1} \omega_{n, i} = T,
    \end{align*}
    it follows that:
    \begin{align*}
        \left\lvert \int_I a \left( t; \eta, y_{h \tau}\right) ~ d \lambdaGR_{q + 1}(t) \right\rvert &\leq \sum_{n \in \TimeIndices} \sum_{i = 1}^{q + 1} \omega_{n, i} \ContC{a} C(u) \Norm{ y_{h \tau} \left( t_{n, i} \right)}_V \\
        &\leq \ContC{a} T^{\nicefrac12} C(u) \Norm{ y_{h \tau} }_{\SpaceLp{2}(I; V)}.
    \end{align*}

    Combining these estimates, and recalling \cref{eq:norm_ht} and the definition of $C_1(u)$:
    \begin{align*}
        \sup_{y_{h \tau} \in \SpaceX_{h \tau}} \frac{ \left\lvert b(e_{h \tau}, y_{h \tau}) \right\rvert }{\Norm{y_{h \tau}}_{\SpaceX_{h \tau}}} &\leq \sqrt{2 \CoerC{a}} \Norm{ u_0 - \InterpolantH \left( u_0 \right) }_W + C \CoerC{a} \tau^{q + 1} C_1(u) \\
        &+ \Norm{ \partial_t u - \InterpolantH \left( \partial_t u \right) }_{\SpaceLp{2}(I; V^*)} + \ContC{a} T^{\nicefrac12} \Norm{ u - \InterpolantH \left( u \right) }_{\SpaceLp{\infty}(I; V)}.
    \end{align*}

    Furthermore, by \cref{eq:b_coercivity}, it follows that:
    \begin{align*}
        \CoerC{a} \Norm{e_{h \tau}}_{\SpaceX_{h \tau}} \leq \sup_{y_{h \tau} \in \SpaceX_{h \tau}} \frac{ \left\lvert b(e_{h \tau}, y_{h \tau}) \right\rvert }{\Norm{y_{h \tau}}_{\SpaceX_{h \tau}}},
    \end{align*}
    that is, together with the definition of $\CoCoR{a}$:
    \begin{align*}
        \Norm{e_{h \tau}}_{\SpaceX_{h \tau}} &\leq \frac{\sqrt{2}}{\sqrt{\CoerC{a}}} \Norm{ u_0 - \InterpolantH \left( u_0 \right) }_W + C \CoerC{a} \tau^{q + 1} C_1(u) \\
        &+ \frac{1}{\CoerC{a}} \Norm{ \partial_t u - \InterpolantH \left( \partial_t u \right) }_{\SpaceLp{2}(I; V^*)} + \CoCoR{a} T^{\nicefrac12} \Norm{ u - \InterpolantH \left( u \right) }_{\SpaceLp{\infty}(I; V)}.
    \end{align*}

    Regarding $\eta$, the definition of the time scales, yields:
    \begin{align*}
        \Norm{ \eta }_{\SpaceX_{h \tau}} \leq \Norm{ \eta }_{\SpaceLp{2}(I; V)} + C \frac{\iota_{V, W}}{\sqrt{\CoerC{a}}} \Norm{ \eta }_{\SpaceLp{\infty}(I; V)} \leq C C_2 \Norm{ \eta }_{\SpaceLp{\infty}(I; V)}.
    \end{align*}

    Similarly:
    \begin{align*}
        \Norm{ \eta }_{\SpaceLp{\infty}(I; V)} \leq \Norm{ u - \InterpolantH \left( u \right) }_{\SpaceLp{\infty}(I; V)} + C \tau^{q + 1} \Seminorm{u}_{\SpaceWkp{q + 1}{\infty}(I; V)}.
    \end{align*}

    Finally, triangle inequality implies that:
    \begin{align*}
        \Norm{u - u_{h \tau}}_{\SpaceX_{h \tau}} &\leq \Norm{e_{h \tau}}_{\SpaceX_{h \tau}} + \Norm{ \eta }_{\SpaceX_{h \tau}},
    \end{align*}
    concluding the proof by recovering \cref{eq:estimates_ht}.
\end{proof}
\section{Bochner Spaces}

The Bochner spaces introduced in this section provide an ideal setting for studying and characterizing time-dependent \acrshort{pdes}, by considering, following \cite[p. 111]{Ern2021}, functions defined on a bounded time interval $I \subset \R$, with values in a Banach (or Hilbert) space of functions defined on the space domain $\Omega \subset \Rd{d}$.

\subsection{Bochner Integral}

Let $I \subset \R$ be open, nonempty and bounded and let $V$ be a Banach space. Moreover, equip $I$ with its own $\sigma$-algebra $\SI$ and with the Lebesgue measure $\lambda$.

\begin{definition}[Simple functions]
    A function $f\colon I \rightarrow V$ is said to be a simple function if there exist $N \in \N$, $\left\{ v_n \right\}_{n = 1}^N \subset V$ and $\left\{ A_n \right\}_{n = 1}^N \subset \SI$ disjoint such that:
    \begin{gather}
        f(t) = \sum_{n = 1}^N v_n \characteristic{A_n} \text{ for all } t \in I.
    \end{gather}

    Moreover, the Bochner integral of a simple function is defined as:
    \begin{gather}
        \int_I f(t) ~ dt  = \sum_{n = 1}^N v_n \lambda(A_n) \in V.
    \end{gather}
\end{definition}

\begin{lemma}
    \begin{gather}
        \lVert \int_I f(t) ~ dt \rVert_V \leq \int_I \lVert f(t) \rVert_V ~ dt \text{ for all simple functions } f.
    \end{gather}
\end{lemma}

\begin{definition}[Strongly measurable function]
    A function $f\colon I \rightarrow V$ is said to be strongly measurable if there exist $\left\{ f_n \right\}_{n \in N}$ simple functions such that:
    \begin{gather}
        \lim_{n \rightarrow \infty} \lVert f(t) - f_n(t) \rVert_V = 0 \text{ for } \lambda \text{-a.e. } t \in I.
    \end{gather}
\end{definition}

\begin{lemma}
    Let $f\colon I \rightarrow V$ be a strongly measurable function, then $\lVert f(\cdot) \rVert_V \colon I \rightarrow \R$ is Lebesgue-measurable.
\end{lemma}

\begin{definition}[Bochner integrable function] \label{definition:bochner_integrable}
    A function $f\colon I \rightarrow V$ is said to be Bochner integrable if there exist $\left\{ f_n \right\}_{n \in N}$ simple functions such that:
    \begin{gather}
        \lim_{n \rightarrow \infty} \lVert f(t) - f_n(t) \rVert_V = 0 \text{ for } \lambda \text{-a.e. } t \in I,
    \end{gather}
    that is  $f$ is strongly measurable, and:
    \begin{gather}
        \lim_{n \rightarrow \infty} \int_I \lVert f(t) - f_n(t) \rVert_V ~ dt = 0.
    \end{gather}
\end{definition}

\begin{lemma}
    Let $f\colon I \rightarrow V$ be a Bochner integrable function and $\left\{ f_n \right\}_{n \in \N}$ as in \ref{definition:bochner_integrable}, then $\left\{ \int_I f_n(t) ~ dt \right\}_{n \in \N}$ converges in $V$ with respect to $\lVert \cdot \rVert_V$.
\end{lemma}

\begin{definition}[Bochner integral]
    Let $f\colon I \rightarrow V$ be a Bochner integrable function and $\left\{ f_n \right\}_{n \in \N}$ as in \ref{definition:bochner_integrable}, then the Bochner integral of $f$ is defined as:
    \begin{gather}
        \int_I f(t) ~ dt = \lim_{n \rightarrow \infty} \int_I f_n(t) ~ dt.
    \end{gather}
\end{definition}

\begin{theorem}[Bochner]
    Let $f\colon I \rightarrow V$ be a strongly measurable function, then $f$ is Bochner integrable if and only if:
    \begin{gather}
        \int_I \lVert f(t) \rVert_V ~ dt < + \infty.
    \end{gather}
\end{theorem}

\subsection{Definition and Main Properties}

\begin{corollary}[Linear maps]
    Let $V, W$ be Banach spaces and $F$ be a bounded linear operator from $V$ to $W$. Let $f\colon I \rightarrow V$ be a Bochner integrable function and define $F(f) \colon I \rightarrow W$ for which:
    \begin{gather}
        F(f)(t) = F(f(t)) \text{ for } \lambda \text{-a.e. } t \in I,
    \end{gather}
    then $F(f)$ is a Bochner integrable function and:
    \begin{gather}
        \int_I F(f)(t) ~ dt = F\left( \int_I f(t) ~ dt \right).
    \end{gather}
\end{corollary}

\begin{definition}[$\Lp{p}(I; V)$]
    Let $1 \leq p \leq +\infty$, then:
    \begin{gather}
        \Lp{p}(I; V) = \left\{ f \colon I \rightarrow V \text{ strongly measurable such that } \lVert f \rVert_{\Lp{p}(I; V)} < +\infty \right\},
    \end{gather}
    where:
    \begin{align}
        & \lVert f \rVert_{\Lp{p}(I; V)} = \left( \int_I \lVert f(t) \rVert_V^p ~ dt \right)^{1/p} & \text{ if } p < +\infty, \\
        & \lVert f \rVert_{\Lp{\infty}(I; V)} = \esssup_{t \in I} \lVert f(t) \rVert_V & \text{ if } p = +\infty.
    \end{align}
\end{definition}

\begin{theorem}
    Let $1 \leq p \leq +\infty$, then $\Lp{p}(I; V)$ is a Banach space.
\end{theorem}

\begin{theorem}
    Let $1 \leq p \leq +\infty$ and let $V$ be reflexive, then $\left( \Lp{p}(I; V) \right)^{*}$ is a isometrically isomorphic to $\Lp{p^{\prime}}(I; V^{*})$.

    Furthermore, $\Lp{p}(I; V)$ is reflexive for all $1 < p < +\infty$.
\end{theorem}
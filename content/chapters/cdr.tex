\section{Setting and Strong Formulation}

Following \cite{Feistauer2007}, let $\Omega \in \RealNumbersTo{d}$ be a bounded domain\footnote{Consider the case where $d \in \left\{ 2, 3 \right\}$ and $\Omega$ is a bounded polytopal domain.} and $T > 0$ so that $I = \left( 0, T\right)$.

Consider the following problem, that is to find $u \colon I \times \Omega \rightarrow \RealNumbers$ such that:
\begin{gather}
    \partial_t u + \Convection \cdot \Gradient u - \Diffusion \Laplacian u + \Reaction u = g \quad \text{ in } I \times \Omega, \label{equation:cdr}
\end{gather}
with the following initial-boundary conditions:
\begin{align}
    u(t, \Vector{x}) &= u_D(t, \Vector{x}) \quad \text{for all } t, \Vector{x} \in I \times \partial \Omega^-, \label{equation:cdr_dirichlet} \\
    \Diffusion \partial_{\Vector{n}} u(t, \Vector{x}) &= u_N(t, \Vector{x}) \quad \text{for all } t, \Vector{x} \in I \times \partial \Omega^+, \label{equation:cdr_neumann} \\
    u(0, \Vector{x}) &= u_0(\Vector{x}) \quad \text{for all } \Vector{x} \in \Omega, \label{equation:cdr_initial}
\end{align}
and where:
\begin{align}
    \Convection & \colon I \times \Omega \rightarrow \RealNumbersTo{d}, \\
    \Diffusion & \geq 0, \\
    \Reaction & \colon I \times \Omega \rightarrow \RealNumbers,
\end{align}
are respectively the convection, diffusion and reaction coefficients, and:
\begin{gather}
    \Vector{n} \colon \partial \Omega \rightarrow \RealNumbersTo{d},
\end{gather}
is the unit outer normal with respect to $\partial \Omega$.

Assume that $\partial \Omega = \partial \Omega^- \cup \partial \Omega^+$ where:
\begin{align}
    \partial \Omega^- = \left\{ \Vector{x} \in \partial \Omega \text{ such that } \Convection(t, \Vector{x}) \cdot \Vector{n}(\Vector{x}) < 0 \text{ for all } t \in \overline{I} \right\}, \\
    \partial \Omega^+ = \left\{ \Vector{x} \in \partial \Omega \text{ such that } \Convection(t, \Vector{x}) \cdot \Vector{n}(\Vector{x}) \geq 0 \text{ for all } t \in \overline{I} \right\}.
\end{align}

\newpage
\section{Weak Formulation}

\subsection{Data Assumptions} \label{assumptions}

Assume that:
\begin{align}
    g & \in C^0(\overline{I}; \SpaceLp{2}(\Omega)), \\
    u_0 & \in \SpaceLp{2}(\Omega), \\
    u_N & \in C^0(\overline{I}; \SpaceLp{2}(\partial \Omega^+)),
\end{align}
and that there exists $u^* \in C^0(\overline{I}; \SpaceHk{1}(\Omega)) \cap \SpaceLp{\infty}(I \times \Omega)$ such that $u_D$ is the trace of $u^*$ on $I \times \partial \Omega^-$.

Moreover, assume that:
\begin{align}
    \Convection & \in \left[ C^0(\overline{I}; \SpaceWkp{1}{\infty}(\Omega)) \right]^2, \\
    \Reaction & \in C^0(\overline{I}; \SpaceLp{\infty}(\Omega)),
\end{align}
and that there exist $\BoundC{\Convection}, \BoundC{\Reaction} > 0$ such that:
\begin{align}
    \lVert \Convection(t, \Vector{x}) \rVert & \leq \BoundC{\Convection} \quad \text{ for a.e. } t, \Vector{x} \in I \times \Omega, \\
    \lvert \Divergence \Convection(t, \Vector{x}) \rvert & \leq \BoundC{\Convection} \quad \text{ for a.e. } t, \Vector{x} \in I \times \Omega, \\
    \lvert \Reaction(t, \Vector{x}) \rvert & \leq \BoundC{\Reaction} \quad \text{ for a.e. } t, \Vector{x} \in I \times \Omega.
\end{align}

Finally, assume that there exists $\BoundC{\Reaction \Convection} > 0$ such that:
\begin{align}
    \Reaction(t, \Vector{x}) - \frac{1}{2} \Divergence \Convection(t, \Vector{x}) & \geq \BoundC{\Reaction \Convection} \quad \text{ for a.e. } t, \Vector{x} \in I \times \Omega.
\end{align}

\newpage
\subsection{Weak Solution}

Following \cite{Feistauer2004}, set $V = \left\{ u \in \SpaceHk{1}(\Omega) \text{ such that } u_{\mid \partial \Omega^-} = 0 \right\}$.

\begin{definition}[Weak solution]
    The weak solution $u \in \SpaceLp{\infty}(I \times \Omega)$ is such that $u - u^* \in \SpaceLp{2}(I; V)$ and $u(0) = u_0$ in $\Omega$. Moreover:
    \begin{align}
        \partial_t \int_{\Omega} u \varphi ~ d \omega(\Vector{x}) &+ \int_{\partial \Omega^+} \left( \Convection \cdot \Vector{n} \right) u \varphi ~ d \sigma(\Vector{x}) - \int_{\Omega} u \Divergence \left( \Convection \varphi \right) ~ d \omega(\Vector{x}) + \Diffusion \int_{\Omega} \Gradient u \cdot \Gradient \varphi ~ d \omega(\Vector{x}) \notag \\
        &+ \int_{\Omega} \Reaction u \varphi ~ d \omega(\Vector{x}) = \int_{\Omega} g \varphi ~ d \omega(\Vector{x}) + \int_{\partial \Omega^+} u_N \varphi ~ d \sigma(\Vector{x}) \quad \text{ for all } \varphi \in V.
    \end{align}
\end{definition}
This weak formulation is derived by multiplying \eqref{equation:cdr} by any $\varphi \in V$, applying Green's theorem and using \eqref{equation:cdr_neumann}. % [!] Expand.

The existence of the weak solution $u$ is assumed with the following regularity\footnote{It is possible to show that such $u$ satisfies \eqref{equation:cdr} pointwise and that $u \in C^0(\left[ 0, T \right); \SpaceHk{p + 1}(\Omega))$.}\footnote{If $\Diffusion > 0$ it is possible to show the existence and uniqueness of $u$. Moreover, it can be shown that $\partial_t u \in \SpaceLp{2}(I \times \Omega)$.}: % [!] Citation needed.
\begin{align}
    \partial_t u & \in \SpaceLp{1}(I; \SpaceHk{p + 1}(\Omega)), \label{equation:assumption_regularity_1} \\
    u & \in \SpaceLp{1}(I; \SpaceHk{p + 1}(\Omega)) \cap \SpaceLp{2}(I; \SpaceHk{p + 1}(\Omega)). \label{equation:assumption_regularity_2}
\end{align}
where $p \geq 1$ integer will denote the space degree of approximation.

\newpage
\section{Discretization}

\subsection{Discontinuous Galërkin Semi-Discretization in Space}

Following \cite{Feistauer2004}, for the semi-discretization in space of this problem, consider $\left\{ \SpaceMesh \right\}_{h \in \hIndices}$ to be a family of standard triangulations of $\Omega$, each consisting of closed $d$-simplices $K$. % [!] d (Dimension).

If $K_i \cap K_j \neq \emptyset$, define $\Gamma_{ij} = K_i \cap K_j$ for $i, j \in \SpatialIndices$, and refer to $K_i$ and $K_j$ as neighbours. Moreover, define:
\begin{gather}
    \Neighbours{i} = \left\{ j \in \SpatialIndices \text{ such that } K_j \text{ is a neighbour of } K_i \right\} \quad \text{ for all } i \in \SpatialIndices.
\end{gather}

Furthermore, for $K \in \Omega_h$, introduce $h_K$ and $\rho_K$ as the diameter of $K$ and the diameter of the largest ball inscribed in $K$, respectively. It follows that:
\begin{gather}
    h = \max_{K \in \Omega_h} \left\{ h_K \right\}.
\end{gather}

Assume that all triangulations are shape-regular, that is there exists $\BoundC{\Omega_h} > 0$ such that:
\begin{gather}
    \frac{h_K}{\rho_K} \leq \BoundC{\Omega_h} \quad \text{ for all } K \in \Omega_h \text{ and for all } h \in \hIndices.
\end{gather}

Consider now the broken Sobolev space.
\begin{definition}[$\SpaceHk{k}(\Omega, \Omega_h)$] % [!] Check u, break.
    Let $k \geq 1$ integer, then:
    \begin{gather}
        \SpaceHk{k}(\Omega, \Omega_h) = \left\{ u \in \SpaceLp{2}(\Omega) \text{ such that } u_{\mid K} \in \SpaceHk{k}(K) \text{ for all } K \in \Omega_h \right\}.
    \end{gather}
    Moreover, $\SpaceHk{k}(\Omega, \Omega_h)$ is equipped with the seminorm $\lvert \cdot \rvert_{\SpaceHk{k}(\Omega, \Omega_h)}$, such that for all \\ $u \in \SpaceHk{k}(\Omega, \Omega_h)$:
    \begin{gather}
        \lvert u \rvert_{\SpaceHk{k}(\Omega, \Omega_h)}^2 = \sum_{K \in \Omega_h} \lvert u \rvert_{\SpaceHk{k}(K)}^2.
    \end{gather}
\end{definition}

For $u \in \SpaceHk{1}(\Omega, \Omega_h)$, $u_{\mid \Gamma_{ij}}$ corresponds to the trace of $u_{\mid K_i}$ on $\Gamma_{ij}$ while $u_{\mid \Gamma_{ji}}$ corresponds to the trace of $u_{\mid K_j}$ on $\Gamma_{ji}$\footnote{$\Gamma_{ji} = \Gamma_{ij}$.}. Moreover:
\begin{align}
    \langle u \rangle_{\Gamma_{ij}} &= \frac{1}{2} \left( u_{\mid \Gamma_{ij}} + u_{\mid \Gamma_{ji}} \right), \\
    \left[ u \right]_{\Gamma_{ij}} &= u_{\mid \Gamma_{ij}} - u_{\mid \Gamma_{ji}},
\end{align}
and $\Vector{n}_{ij}$ denotes the unit outer normal with respect to $\partial K_i$ on $\Gamma_{ij}$.

Finally, for $i \in \SpatialIndices$\footnote{The dependence of $\partial K_i^-(t)$ and $\partial K_i^+(t)$ on $t$ will be implicit in what follows.}:
\begin{align}
    \partial K_i^-(t) = \left\{ \Vector{x} \in \partial K_i \text{ such that } \Convection(t, \Vector{x}) \cdot \Vector{n}_i(\Vector{x}) < 0 \right\}, \\
    \partial K_i^+(t) = \left\{ \Vector{x} \in \partial K_i \text{ such that } \Convection(t, \Vector{x}) \cdot \Vector{n}_i(\Vector{x}) \geq 0 \right\},
\end{align}
where $\Vector{n}_i$ denotes the unit outer normal with respect to $\partial K_i$.

\newpage
Under assumptions \eqref{equation:assumption_regularity_1} and \eqref{equation:assumption_regularity_2}, the exact solution $u$ satisfies the following identity:
\begin{align}
    \left( \partial_t u(t), \varphi \right)_{\SpaceLp{2}(\Omega)} &+ a_h \left( u(t), \varphi \right) + b_h \left( u(t), \varphi \right) + c_h \left( u(t), \varphi \right) + \Diffusion J_h \left( u(t), \varphi \right) \notag \\ 
    &= l_h \left( \varphi \right) (t) \quad \text{ for all } \varphi \in \SpaceHk{2}(\Omega, \Omega_h) \text{ and for } \lambda \text{-a.e. } t \in I.
\end{align}

\newpage
\subsection{Discontinuous Galërkin Discretization in Time}
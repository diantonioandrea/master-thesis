\section{Setting and Strong Formulation}

Following \cite{Feistauer2007}, let $\Omega \in \RealNumbersTo{d}$ be a bounded domain\footnote{Consider the case where $d \in \left\{ 2, 3 \right\}$ and $\Omega$ is a bounded polytopal domain.} and $T > 0$ so that $I = \left( 0, T\right)$.

Consider the following problem, that is to find $u \colon I \times \Omega \rightarrow \RealNumbers$ such that:
\begin{gather}
    \partial_t u + \Convection \cdot \Gradient u - \Diffusion \Laplacian u + \Reaction u = g \quad \text{ in } I \times \Omega, \label{equation:cdr}
\end{gather}
with the following initial-boundary conditions:
\begin{align}
    u(t, \Vector{x}) &= u_D(t, \Vector{x}) \quad \text{for all } t, \Vector{x} \in I \times \partial \Omega^-, \label{equation:cdr_dirichlet} \\
    \Diffusion \partial_{\Vector{n}} u(t, \Vector{x}) &= u_N(t, \Vector{x}) \quad \text{for all } t, \Vector{x} \in I \times \partial \Omega^+, \label{equation:cdr_neumann} \\
    u(0, \Vector{x}) &= u_0(\Vector{x}) \quad \text{for all } \Vector{x} \in \Omega, \label{equation:cdr_initial}
\end{align}
and where:
\begin{align}
    \Convection & \colon I \times \Omega \rightarrow \RealNumbersTo{d}, \\
    \Diffusion & \geq 0, \\
    \Reaction & \colon I \times \Omega \rightarrow \RealNumbers,
\end{align}
are respectively the convection, diffusion and reaction coefficients, and:
\begin{gather}
    \Vector{n} \colon \partial \Omega \rightarrow \RealNumbersTo{d},
\end{gather}
is the unit outer normal with respect to $\partial \Omega$.

Assume that $\partial \Omega = \partial \Omega^- \cup \partial \Omega^+$ where:
\begin{align}
    \partial \Omega^- = \left\{ \Vector{x} \in \partial \Omega \text{ such that } \Convection(t, \Vector{x}) \cdot \Vector{n}(\Vector{x}) < 0 \text{ for all } t \in \overline{I} \right\}, \\
    \partial \Omega^+ = \left\{ \Vector{x} \in \partial \Omega \text{ such that } \Convection(t, \Vector{x}) \cdot \Vector{n}(\Vector{x}) \geq 0 \text{ for all } t \in \overline{I} \right\},
\end{align}
namely the inflow and outflow parts of $\partial \Omega$.

\newpage
\section{Weak Formulation}

\subsection{Data Assumptions} \label{assumptions}

Assume that:
\begin{align}
    g & \in C^0(\overline{I}; \SpaceLp{2}(\Omega)), \\
    u_0 & \in \SpaceLp{2}(\Omega), \\
    u_N & \in C^0(\overline{I}; \SpaceLp{2}(\partial \Omega^+)),
\end{align}
and that there exists $u^* \in C^0(\overline{I}; \SpaceHk{1}(\Omega)) \cap \SpaceLp{\infty}(I \times \Omega)$ such that $u_D$ is the trace of $u^*$ on $I \times \partial \Omega^-$.

Moreover, assume that:
\begin{align}
    \Convection & \in \left[ C^0(\overline{I}; \SpaceWkp{1}{\infty}(\Omega)) \right]^2, \\
    \Reaction & \in C^0(\overline{I}; \SpaceLp{\infty}(\Omega)),
\end{align}
and that there exist $\BoundC{\Convection}, \BoundC{\Reaction} > 0$ such that:
\begin{align}
    \lVert \Convection(t, \Vector{x}) \rVert & \leq \BoundC{\Convection} \quad \text{ for } \lambda \otimes \omega \text{-a.e. } t, \Vector{x} \in I \times \Omega, \\
    \lvert \Divergence \Convection(t, \Vector{x}) \rvert & \leq \BoundC{\Convection} \quad \text{ for } \lambda \otimes \omega \text{-a.e. } t, \Vector{x} \in I \times \Omega, \\
    \lvert \Reaction(t, \Vector{x}) \rvert & \leq \BoundC{\Reaction} \quad \text{ for } \lambda \otimes \omega \text{-a.e. } t, \Vector{x} \in I \times \Omega.
\end{align}

Finally, assume that there exists $\BoundC{\Reaction \Convection} > 0$ such that:
\begin{align}
    \Reaction(t, \Vector{x}) - \frac{1}{2} \Divergence \Convection(t, \Vector{x}) & \geq \BoundC{\Reaction \Convection} \quad \text{ for} \lambda \text{-a.e. } t, \Vector{x} \in I \times \Omega.
\end{align}

\newpage
\subsection{Weak Solution}

Following \cite{Feistauer2004}, set $V = \left\{ u \in \SpaceHk{1}(\Omega) \text{ such that } u_{\mid \partial \Omega^-} = 0 \right\}$.

\begin{definition}[Weak solution]
    The weak solution $u \in \SpaceLp{\infty}(I \times \Omega)$ is such that $u - u^* \in \SpaceLp{2}(I; V)$ and $u(0) = u_0$ in $\Omega$. Moreover:
    \begin{align}
        \partial_t \int_{\Omega} u \varphi ~ d \omega(\Vector{x}) &+ \Diffusion \int_{\Omega} \Gradient u \cdot \Gradient \varphi ~ d \omega(\Vector{x}) + \int_{\partial \Omega^+} \left( \Convection \cdot \Vector{n} \right) u \varphi ~ d \sigma(\Vector{x}) - \int_{\Omega} u \Divergence \left( \Convection \varphi \right) ~ d \omega(\Vector{x}) \notag \\
        &+ \int_{\Omega} \Reaction u \varphi ~ d \omega(\Vector{x}) = \int_{\Omega} g \varphi ~ d \omega(\Vector{x}) + \int_{\partial \Omega^+} u_N \varphi ~ d \sigma(\Vector{x}) \quad \text{ for all } \varphi \in V.
    \end{align}
\end{definition}
This weak formulation is derived by multiplying \eqref{equation:cdr} by any $\varphi \in V$, applying Green's theorem and using \eqref{equation:cdr_neumann}. % [!] Expand.

The existence of the weak solution $u$ is assumed with the following regularity\footnote{It is possible to show that such $u$ satisfies \eqref{equation:cdr} pointwise and that $u \in C^0(\left[ 0, T \right); \SpaceHk{p + 1}(\Omega))$.}\footnote{If $\Diffusion > 0$ it is possible to show the existence and uniqueness of $u$. Moreover, it can be shown that $\partial_t u \in \SpaceLp{2}(I \times \Omega)$.}: % [!] Citation needed.
\begin{align}
    \partial_t u & \in \SpaceLp{1}(I; \SpaceHk{p + 1}(\Omega)), \label{equation:assumption_regularity_1} \\
    u & \in \SpaceLp{1}(I; \SpaceHk{p + 1}(\Omega)) \cap \SpaceLp{2}(I; \SpaceHk{p + 1}(\Omega)). \label{equation:assumption_regularity_2}
\end{align}
where $p \geq 1$ integer will denote the space degree of approximation.

\newpage
\section{Discretization}

\subsection{Space Discretization Setting}

Following \cite{Feistauer2004}, for the semi-discretization in space of this problem, consider $\left\{ \SpaceMesh \right\}_{h \in \hIndices}$ to be a family of standard triangulations of $\Omega$, each consisting of closed $d$-simplices $K_j$\footnote{Denoted by $K$ when unambiguous.} indexed by $j \in \SpaceIndices$. % [!] d (Dimension).

If $K_i \cap K_j \neq \emptyset$, define $\Gamma_{ij} = K_i \cap K_j$ for $i, j \in \SpaceIndices$, and refer to $K_i$ and $K_j$ as neighbours. Moreover, define:
\begin{gather}
    \Neighbours{i} = \left\{ j \in \SpaceIndices \text{ such that } K_j \text{ is a neighbour of } K_i \right\} \quad \text{ for all } i \in \SpaceIndices.
\end{gather}

Furthermore, for $K \in \Omega_h$, introduce $h_K$ and $\rho_K$ as the diameter of $K$ and the diameter of the largest ball inscribed in $K$, respectively. It follows that:
\begin{gather}
    h = \max_{K \in \Omega_h} \left\{ h_K \right\}.
\end{gather}

Assume that all triangulations are shape-regular, that is there exists $\BoundC{\Omega_h} > 0$ such that:
\begin{gather}
    \frac{h_K}{\rho_K} \leq \BoundC{\Omega_h} \quad \text{ for all } K \in \Omega_h \text{ and for all } h \in \hIndices.
\end{gather}

Consider now the broken Sobolev space.
\begin{definition}[$\SpaceHk{k}(\Omega, \Omega_h)$] % [!] Check u, break.
    Let $k \geq 1$ integer, then:
    \begin{gather}
        \SpaceHk{k}(\Omega, \Omega_h) = \left\{ u \in \SpaceLp{2}(\Omega) \text{ such that } u_{\mid K} \in \SpaceHk{k}(K) \text{ for all } K \in \Omega_h \right\}.
    \end{gather}
    Moreover, $\SpaceHk{k}(\Omega, \Omega_h)$ is equipped with the seminorm $\lvert \cdot \rvert_{\SpaceHk{k}(\Omega, \Omega_h)}$, such that for all \\ $u \in \SpaceHk{k}(\Omega, \Omega_h)$:
    \begin{gather}
        \lvert u \rvert_{\SpaceHk{k}(\Omega, \Omega_h)}^2 = \sum_{K \in \Omega_h} \lvert u \rvert_{\SpaceHk{k}(K)}^2.
    \end{gather}
\end{definition}

For $u \in \SpaceHk{1}(\Omega, \Omega_h)$, $u_{\mid \Gamma_{ij}}$ corresponds to the trace of $u_{\mid K_i}$ on $\Gamma_{ij}$ while $u_{\mid \Gamma_{ji}}$ corresponds to the trace of $u_{\mid K_j}$ on $\Gamma_{ji}$\footnote{$\Gamma_{ji} = \Gamma_{ij}$.}. Moreover:
\begin{align}
    \langle u \rangle_{\Gamma_{ij}} &= \frac{1}{2} \left( u_{\mid \Gamma_{ij}} + u_{\mid \Gamma_{ji}} \right), \\
    \left[ u \right]_{\Gamma_{ij}} &= u_{\mid \Gamma_{ij}} - u_{\mid \Gamma_{ji}},
\end{align}
and $\Vector{n}_{ij}$ denotes the unit outer normal with respect to $\partial K_i$ on $\Gamma_{ij}$.

Finally, for $i \in \SpaceIndices$\footnote{The dependence of $\partial K_i^-(t)$ and $\partial K_i^+(t)$ on $t$ will be implicit in what follows.}:
\begin{align}
    \partial K_i^-(t) = \left\{ \Vector{x} \in \partial K_i \text{ such that } \Convection(t, \Vector{x}) \cdot \Vector{n}_i(\Vector{x}) < 0 \right\}, \\
    \partial K_i^+(t) = \left\{ \Vector{x} \in \partial K_i \text{ such that } \Convection(t, \Vector{x}) \cdot \Vector{n}_i(\Vector{x}) \geq 0 \right\},
\end{align}
namely the inflow and outflow parts of $\partial K_i$, where $\Vector{n}_i$ denotes the unit outer normal with respect to $\partial K_i$.

\newpage
\subsection{Discontinuous Galërkin Semi-Discretization in Space}

Under assumptions \eqref{equation:assumption_regularity_1} and \eqref{equation:assumption_regularity_2}, the exact solution $u$ satisfies what follows for all $\varphi \in \SpaceHk{2}(\Omega, \Omega_h)$ and for $\lambda$-a.e. $t \in I$:
\begin{align}
    \left( \partial_t u(t), \varphi \right)_{\SpaceLp{2}(\Omega)} &+ A_h(u(t), \varphi) = l_h \left( \varphi \right) (t),
\end{align}
where:
\begin{gather}
    A_h(u, \varphi) = a_h \left( u(t), \varphi \right) + b_h \left( u(t), \varphi \right) + c_h \left( u(t), \varphi \right) + \Diffusion J^{\kappa}_h \left( u(t), \varphi \right) \label{equation:cdr_Ah},
\end{gather}
and:
\begin{align}
    a_h \left( u, \varphi \right) &= \Diffusion \sum_{i \in \SpaceIndices} \int_{K_i} \Gradient u \cdot \Gradient \varphi ~ d \omega(\Vector{x}) \label{equation:cdr_ah} \\
    &- \Diffusion \sum_{i \in \SpaceIndices} \sum_{\substack{j \in \Neighbours{i} \\ j < i}} \int_{\Gamma_{ij}} \left( \left\langle \Gradient u \right\rangle \cdot \Vector{n}_{ij} \left[ \varphi \right] - \left\langle \Gradient \varphi \right\rangle \cdot \Vector{n}_{ij} \left[ u \right] \right) ~ d \sigma(\Vector{x}) \notag \\
    &- \Diffusion \sum_{i \in \SpaceIndices} \int_{\partial K_i^- \cap \partial \Omega} \left( \left( \Gradient u \cdot \Vector{n}_i \right) \varphi - \left( \Gradient \varphi \cdot \Vector{n}_i \right) u \right) ~ d \sigma(\Vector{x}), \notag \\
    b_h \left( u, \varphi \right) &= \sum_{i \in \SpaceIndices} \int_{K_i} \left( \Convection \cdot \Gradient u \right) \varphi ~ d \omega(\Vector{x}) \label{equation:cdr_bh} - \Diffusion \sum_{i \in \SpaceIndices} \int_{\partial K_i^- \setminus \partial \Omega} \left( \Convection \cdot \Vector{n}_i \right) \left[ u \right] \varphi ~ d \sigma(\Vector{x}) \\
    &- \Diffusion \sum_{i \in \SpaceIndices} \int_{\partial K_i^- \cap \partial \Omega} \left( \Convection \cdot \Vector{n}_i \right) u \varphi ~ d \sigma(\Vector{x}), \notag \\
    c_h \left( u, \varphi \right) &= \Reaction \left( u, \varphi \right)_{\SpaceLp{2}(\Omega)}, \label{equation:cdr_ch} \\
    J^{\kappa}_h \left( u, \varphi \right) &= \sum_{i \in \SpaceIndices} \sum_{j \in \Neighbours{i}} \int_{\Gamma_{ij}} \kappa \left[ u \right] \left[ \varphi \right] ~ d \sigma(\Vector{x}) + \sum_{i \in \SpaceIndices} \int_{\partial K_i^- \cap \partial \Omega} \kappa u \varphi ~ d \sigma(\Vector{x}), \label{equation:cdr_jh} \\
    l_h \left( \varphi \right)  &= \left( g, \varphi \right)_{\SpaceLp{2}(\Omega)} + \sum_{i \in \SpaceIndices} \int_{\partial K_i^+ \cap \partial \Omega} u_N \varphi ~ d \sigma(\Vector{x}) \label{equation:cdr_lh} \\
    &+ \Diffusion \sum_{i \in \SpaceIndices} \int_{\partial K_i^- \cap \partial \Omega} \kappa u_D \varphi ~ d \sigma(\Vector{x}) + \Diffusion \sum_{i \in \SpaceIndices} \int_{\partial K_i^- \cap \partial \Omega} u_D \left( \Gradient \varphi \cdot \Vector{n}_i \right) ~ d \sigma(\Vector{x}) \notag \\
    &- \sum_{i \in \SpaceIndices} \int_{\partial K_i^- \cap \partial \Omega} \left( \Convection \cdot \Vector{n}_i \right) u_D \varphi ~ d \sigma(\Vector{x}) \notag,
\end{align}
with:
\begin{gather}
    \kappa_{\mid \Gamma_{ij}} = \left( \Diameter \Gamma_{ij} \right)^{-1}.
\end{gather}

This formulation is derived by multiplying \eqref{equation:cdr} by any $\varphi \in \SpaceHk{2}(\Omega, \Omega_h)$, integrating over each $K \in \Omega_h$, applying Green's theorem, summing over all $K \in \Omega_h$, adding terms that either appear on both sides or vanish, and using \eqref{equation:cdr_dirichlet} and \eqref{equation:cdr_neumann}. % [!] Expand.

\newpage
\subsubsection{Diffusion Term Discretization}

The bilinear form $a_h$ represents the discretization of the diffusion term in the nonsymmetric formulation. Green's theorem is used and the following terms are added:
\begin{gather}
    \int_{\Gamma_{ij}} \left\langle \Gradient \varphi \right\rangle \cdot \Vector{n}_{ij} \left[ u \right] ~ d \sigma(\Vector{x}),
\end{gather}
which vanish under assumptions \eqref{equation:assumption_regularity_1} and \eqref{equation:assumption_regularity_2} as $\left[ u \right]_{\Gamma_{ij}} = 0$ for all $i \in \SpaceIndices$ and for all $j \in \Neighbours{i}$. Note that, due to $\left\langle \Gradient u \right\rangle_{\Gamma_{ij}} = \Gradient u_{\mid \Gamma{ij}} = \Gradient u_{\mid \Gamma{ji}}$ for all $i \in \SpaceIndices$ and for all $j \in \Neighbours{i}$, the following terms:
\begin{gather}
    \sum_{\substack{j \in \Neighbours{i} \\ j < i}} \int_{\Gamma_{ij}} \left\langle \Gradient u \right\rangle \cdot \Vector{n}_{ij} \left[ \varphi \right] ~ d \sigma(\Vector{x}),
\end{gather}
represent the sum of integrals of $\varphi \partial_{\Vector{n}} u$ over faces of $\partial K_i$ in $\Omega$ for all $i \in \SpaceIndices$. Finally, the remaining integrals are represented by the following terms:
\begin{gather}
    \int_{\partial K_i^- \cap \partial \Omega} \left( \left( \Gradient u \cdot \Vector{n}_i \right) \varphi - \left( \Gradient \varphi \cdot \Vector{n}_i \right) u \right) ~ d \sigma(\Vector{x}),
\end{gather}
for which the second part cancels with the corresponding terms in $l_h$ as $u = u_D$ on $\partial \Omega^-$, and by the following terms:
\begin{gather}
    \int_{\partial K_i^+ \cap \partial \Omega} u_N \varphi ~ d \sigma(\Vector{x}),
\end{gather}
which appear in $l_h$ due to \eqref{equation:cdr_neumann}.

\newpage
\subsubsection{Convection Term Discretization}

The bilinear form $b_h$ represents the discretization of the convection term. Applying Green's theorem yields, for $i \in \SpaceIndices$:
\begin{align}
    \int_{K_i} \left( \Convection \cdot \Gradient u \right) \varphi ~ d \omega(\Vector{x}) &= \int_{\partial K_i} \left( \Convection \cdot \Vector{n}_i \right) u \varphi ~ d \sigma(\Vector{x}) - \int_{K_i} u \Divergence \left( \Convection \varphi \right) ~ d \omega(\Vector{x}) \notag \\
    &= \int_{\partial K_i^-} \left( \Convection \cdot \Vector{n}_i \right) u \varphi ~ d \sigma(\Vector{x}) + \int_{\partial K_i^+} \left( \Convection \cdot \Vector{n}_i \right) u \varphi ~ d \sigma(\Vector{x}) \label{equation:cdr_convection_discretization_1} \\
    & - \int_{K_i} u \Divergence \left( \Convection \varphi \right) ~ d \omega(\Vector{x}). \notag
\end{align}
Consider, on the inflow part of $\partial K_i$, the ``information coming from outside'' the element $K_i$, that is, write $u^-$ instead of $u$, where $u^-$ is a simplified notation for $u_{\mid \Gamma_{ji}}$ for all $j \in \Neighbours{i}$. Additionally, consider $u^- = u_D$ on $\partial K_i^- \cap \partial \Omega$. Starting from \eqref{equation:cdr_convection_discretization_1}, further rearrange the resulting terms:
\begin{align}
    \int_{K_i} \left( \Convection \cdot \Gradient u \right) \varphi ~ d \omega(\Vector{x}) &= \int_{\partial K_i^-} \left( \Convection \cdot \Vector{n}_i \right) u^- \varphi ~ d \sigma(\Vector{x}) + \int_{\partial K_i^+} \left( \Convection \cdot \Vector{n}_i \right) u \varphi ~ d \sigma(\Vector{x}) \notag \\
    &- \int_{K_i} u \Divergence \left( \Convection \varphi \right) ~ d \omega(\Vector{x}) \notag \\
    &= - \int_{K_i} u \Divergence \left( \Convection \varphi \right) ~ d \omega(\Vector{x}) \pm \int_{\partial K_i} \left( \Convection \cdot \Vector{n}_i \right) u \varphi ~ d \sigma(\Vector{x}) \notag \\
    &+ \int_{\partial K_i^- \setminus \partial \Omega} \left( \Convection \cdot \Vector{n}_i \right) u^- \varphi ~ d \sigma(\Vector{x}) + \int_{\partial K_i^- \cap \partial \Omega} \left( \Convection \cdot \Vector{n}_i \right) u^- \varphi ~ d \sigma(\Vector{x}) \notag \\
    &+ \int_{\partial K_i^+ \setminus \partial \Omega} \left( \Convection \cdot \Vector{n}_i \right) u \varphi ~ d \sigma(\Vector{x}) + \int_{\partial K_i^+ \cap \partial \Omega} \left( \Convection \cdot \Vector{n}_i \right) u \varphi ~ d \sigma(\Vector{x}) \notag \\
    &= \int_{K_i} \left( \Convection \cdot \Gradient u \right) \varphi ~ d \omega(\Vector{x}) + \int_{\partial K_i^- \setminus \partial \Omega} \left( \Convection \cdot \Vector{n}_i \right) \left( u^- - u \right) \varphi ~ d \sigma(\Vector{x}) \notag \\
    &- \int_{\partial K_i^- \cap \partial \Omega} \left( \Convection \cdot \Vector{n}_i \right) u \varphi ~ d \sigma(\Vector{x}) + \int_{\partial K_i^- \cap \partial \Omega} \left( \Convection \cdot \Vector{n}_i \right) u^- \varphi ~ d \sigma(\Vector{x}) \notag \\
    &= \int_{K_i} \left( \Convection \cdot \Gradient u \right) \varphi ~ d \omega(\Vector{x}) - \int_{\partial K_i^- \setminus \partial \Omega} \left( \Convection \cdot \Vector{n}_i \right) \left[ u \right] \varphi ~ d \sigma(\Vector{x}) \\
    &- \int_{\partial K_i^- \cap \partial \Omega} \left( \Convection \cdot \Vector{n}_i \right) u \varphi ~ d \sigma(\Vector{x}) + \int_{\partial K_i^- \cap \partial \Omega} \left( \Convection \cdot \Vector{n}_i \right) u_D \varphi ~ d \sigma(\Vector{x}). \notag
\end{align}
Note that the last term is transferred to $l_h$ while the other terms define $b_h$ as in \eqref{equation:cdr_bh}.

\newpage
\subsubsection{Interior and Boundary Penalty}

The bilinear form $J^{\kappa}_h$ represents the interior and boundary penalty that replaces the continuity of conforming finite elements. The first term vanishes due to \eqref{equation:assumption_regularity_1} and \eqref{equation:assumption_regularity_2}, while the second term cancels with the corresponding term in $l_h$.

\subsubsection{Approximate Solution}

\begin{definition}[$\SpaceSolution_h^p$]
    \begin{gather}
        \SpaceSolution_h^p = \left\{ u \in \SpaceLp{2}(\Omega) \text{ such that } u_{\mid K} \in \SpacePolynomials{p}(K) \text{ for all } K \in \Omega_h \right\}.
    \end{gather}
\end{definition}

The discontinuous Galërkin semi-discrete problem is to find an approximate solution $u_h \in C^1(I; \SpaceSolution_h^p)$ to \eqref{equation:cdr} with the initial-boundary conditions \eqref{equation:cdr_dirichlet}, \eqref{equation:cdr_neumann}, and \eqref{equation:cdr_initial}, such that:
\begin{align}
    \left( \partial_t u_h(t), \varphi_h \right)_{\SpaceLp{2}(\Omega)} + A_h(u_h(t), \varphi_h) &= l_h(\varphi_h) \quad \text{ for all } \varphi_h \in \SpaceSolution_h^p \text{ and for all } t \in I, \\
    \left( u_h(0), \varphi_h \right)_{\SpaceLp{2}(\Omega)} &= \left( u_0, \varphi_h \right)_{\SpaceLp{2}(\Omega)} \quad \text{ for all } \varphi_h \in \SpaceSolution_h^p.
\end{align}

\newpage
\subsection{Time Discretization Setting}

Following \cite{Feistauer2007}, for the discretization in time of this problem, consider $\left\{ I_{\tau} \right\}_{\tau \in \tIndices}$ to be a family of partitions of $\overline{I}$, each consisting of intervals $I_n$ indexed by $n \in \TimeIndices$. A partition $I_{\tau}$ is such that:
\begin{gather}
    I_{\tau} = \left\{ 0 = t_0 < t_1 < \dots < t_{N_\tau} = T \right\} \quad \text{ for all } \tau \in \tIndices.
\end{gather}
Moreover, denote by $I_n = \left( t_{n - 1}, t_n \right)$ and $\overline{I}_n = \left[ t_{n - 1}, t_n \right]$ so that:
\begin{gather}
    \overline{I} = \left[ 0, T \right] = \bigcup_{n = 1}^{N_{\tau}} \overline{I}_n \quad \text{ for all } \tau \in \tIndices,
\end{gather}
and:
\begin{gather}
    I_n \cap I_m = \emptyset \quad \text{ for all } n, m \in \TimeIndices \text{ such that } n \neq m \text{ and for all } \tau \in \tIndices.
\end{gather}

\newpage
\subsection{Discontinuous Galërkin Discretization in Time}


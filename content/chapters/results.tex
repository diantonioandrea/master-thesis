This chapter presents the numerical results obtained testing the algorithm based on \cref{chapter:cdr} and implemented as in \cref{chapter:implementation}.

\section{Testing Conditions}

The goal of the error analysis is to evaluate the behavior of the error $\Error$, as defined in \cref{subsection:error_estimates}, for a fixed exact solution $u$ while varying $p$ and $q$.

Following \cite{Feistauer2007}, fix $\Omega = \left( 0, 1 \right) \times \left( 0, 1 \right)$ and $I = \left( 0, 1 \right)$, and define the exact solution as follows:
\begin{align}
    u(t, \Vector{x}) &= \left( 1 - e^{-t} \right) \left( 2 x_1 + 2 x_2 - x_1 x_2 + 2 \left( 1 - e^{\frac{\ConvectionNoVector_1 \left( x_1 - 1 \right) }{\Boundary}} \right) \left( 1 - e^{\frac{\ConvectionNoVector_2 \left( x_2 - 1 \right) }{\Boundary}} \right) \right),
\end{align}
where $\Boundary$ controls the steepness of the boundary layer in the exact solution, set to $\Boundary = 0.05$.

Moreover, fix:
\begin{align}
    \ConvectionNoVector_1(t, \Vector{x}) &= 1, \\
    \ConvectionNoVector_2(t, \Vector{x}) &= 1, \\
    \Reaction(t, \Vector{x}) &= 0.5,
\end{align}
and consider two choices for $\Diffusion$:
\begin{align}
    \Diffusion &= 0.005, \\
    \Diffusion &= 0,
\end{align}
which correspond to the parabolic and hyperbolic cases, respectively.

\newpage
\subsection{Error Analysis}

Two notable quantities to consider are $\Norm{\Error}_{\SpaceLp{2}(I; \SpaceLp{2}(\Omega))}$ and $\Norm{\Error}_{\sqrt{\Diffusion} \SpaceLp{2}(I; \SpaceHk{1}(\Omega))}$, for which \cref{theorem:estimates_ht} gives an estimate. 

This work wants to extend the error analysis by additionally considering $\Norm{\Error}_{\SpaceLp{2}(\left\{ T \right\} \times \Omega)}$ and $\Norm{\Error}_{\SpaceLp{\infty}(I; \SpaceLp{2}(\Omega))}$, where:
\begin{align}
    \Norm{\Error}_{\SpaceLp{2}(\left\{ T \right\} \times \Omega)} &= \left( \int_{\Omega} \Error^2(T, \Vector{x}) ~ d \omega (\Vector{x}) \right)^{\frac12}, \\
    \Norm{\Error}_{\SpaceLp{\infty}(I; \SpaceLp{2}(\Omega))} &= \sup_{t \in \overline{I}} \left( \int_{\Omega} \Error^2(t, \Vector{x}) ~ d \omega (\Vector{x}) \right)^{\frac12}.
\end{align}

\newpage
\subsection{Parabolic Case}

\begin{figure}[!ht]
    \begin{subfigure}[t]{0.49\textwidth}
        \centering
        \begin{tikzpicture}[scale=.9]
\begin{loglogaxis}[
xlabel={$DoFs$},
% xtick={},
% xticklabels={},
% ylabel={ _LABERROR },
grid=both,
legend pos=north east
]
\addplot[color=\documentcolor, dashed, semithick] coordinates {(5250, 0.008158029807153699) (15000, 0.003778963652924082) (42000, 0.0021149591552920614) (120000, 0.0010673187064462785) (336000, 0.000643075)};
\addplot[color=\documentcolor, dotted, semithick] coordinates {(5250, 0.010289171190271694) (15000, 0.005041703966636022) (42000, 0.0025723) (120000, 0.0012604259916590055) (336000, 0.000643075)};
\addplot[color=\documentcolor, solid, thick] coordinates {(5250, 0.009223600498712696) (15000, 0.004410333809780053) (42000, 0.0023436295776460307) (120000, 0.0011638723490526419) (336000, 0.000643075)};
\addplot[color=\accentcolor, mark=triangle, very thick] coordinates {(5250, 0.00843558) (15000, 0.00461507) (42000, 0.00236626) (120000, 0.00113055) (336000, 0.000643075)};
\legend{$h^2$,$\tau^2$,$h^2 + \tau^2$,$\Norm{\Error}_{\SpaceLp{2}(I; \SpaceLp{2}(\Omega))}$}
\end{loglogaxis}
\end{tikzpicture}
    \end{subfigure}
    \hfill
    \begin{subfigure}[t]{0.49\textwidth}
        \centering
        \begin{tikzpicture}[scale=.9]
\begin{loglogaxis}[
xlabel={$DoFs$},
% xtick={},
% xticklabels={},
% ylabel={ _LABERROR },
grid=both,
legend pos=north east
]
\addplot[color=\documentcolor, dashed, semithick] coordinates {(5250, 0.036223220032647725) (15000, 0.02465362396030036) (42000, 0.01844358758028077) (120000, 0.013102116919882468) (336000, 0.0101701)};
\addplot[color=\documentcolor, dotted, semithick] coordinates {(5250, 0.040680343047462784) (15000, 0.02847626860949256) (42000, 0.0203402) (120000, 0.01423813430474628) (336000, 0.0101701)};
\addplot[color=\documentcolor, solid, thick] coordinates {(5250, 0.038451781540055255) (15000, 0.026564946284896463) (42000, 0.019391893790140384) (120000, 0.013670125612314375) (336000, 0.0101701)};
\addplot[color=\accentcolor, mark=triangle, very thick] coordinates {(5250, 0.0347433) (15000, 0.0264992) (42000, 0.019354) (120000, 0.0135571) (336000, 0.0101701)};
\legend{$\BigO{h}$,$\BigO{\tau}$,$\BigO{h + \tau}$,$\Norm{\Error}_{\sqrt{\Diffusion} \SpaceLp{2}(I; \SpaceHk{1}(\Omega))}$}
\end{loglogaxis}
\end{tikzpicture}
    \end{subfigure}
\end{figure}

\begin{figure}[!ht]
    \begin{subfigure}[t]{0.49\textwidth}
        \centering
        \begin{tikzpicture}[scale=.9]
\begin{loglogaxis}[
xlabel={$DoFs$},
% xtick={},
% xticklabels={},
% ylabel={ _LABERROR },
grid=both,
legend pos=north east
]
\addplot[color=\documentcolor, dashed, semithick] coordinates {(5250, 0.012528130622855953) (15000, 0.005803282334337986) (42000, 0.003247902396270913) (120000, 0.0016390609603866948) (336000, 0.000987558)};
\addplot[color=\documentcolor, dotted, semithick] coordinates {(5250, 0.01580088375745027) (15000, 0.007742448526039943) (42000, 0.003950232) (120000, 0.0019356121315099858) (336000, 0.000987558)};
\addplot[color=\documentcolor, solid, thick] coordinates {(5250, 0.014164507190153112) (15000, 0.006772865430188965) (42000, 0.0035990671981354565) (120000, 0.0017873365459483402) (336000, 0.000987558)};
\addplot[color=\accentcolor, mark=triangle, very thick] coordinates {(5250, 0.0129324) (15000, 0.00707916) (42000, 0.00362616) (120000, 0.00173093) (336000, 0.000987558)};
\legend{$\BigO{h^2}$,$\BigO{\tau^2}$,$\BigO{h^2 + \tau^2}$}
\end{loglogaxis}
\end{tikzpicture}
    \end{subfigure}
    \hfill
    \begin{subfigure}[t]{0.49\textwidth}
        \centering
        \begin{tikzpicture}[scale=.9]
\begin{loglogaxis}[
xlabel={$DoFs$},
% xtick={},
% xticklabels={},
% ylabel={ _LABERROR },
grid=both,
legend pos=north east
]
\addplot[color=\documentcolor, dashed, semithick] coordinates {(5250, 0.012766398501156247) (15000, 0.005913652812632463) (42000, 0.0033096730495457195) (120000, 0.0016702336539985972) (336000, 0.00100634)};
\addplot[color=\documentcolor, dotted, semithick] coordinates {(5250, 0.0161013949160176) (15000, 0.007889699288239311) (42000, 0.00402536) (120000, 0.0019724248220598277) (336000, 0.00100634)};
\addplot[color=\documentcolor, solid, thick] coordinates {(5250, 0.014433896708586925) (15000, 0.006901676050435888) (42000, 0.0036675165247728597) (120000, 0.0018213292380292124) (336000, 0.00100634)};
\addplot[color=\accentcolor, mark=triangle, very thick] coordinates {(5250, 0.0129324) (15000, 0.00707916) (42000, 0.00362616) (120000, 0.00176517) (336000, 0.00100634)};
\legend{$\BigO{h^2}$,$\BigO{\tau^2}$,$\BigO{h^2 + \tau^2}$}
\end{loglogaxis}
\end{tikzpicture}
    \end{subfigure}
\end{figure}

\newpage

\begin{figure}[!ht]
    \begin{subfigure}[t]{0.49\textwidth}
        \centering
        \begin{tikzpicture}[scale=.9]
\begin{loglogaxis}[
xlabel={$DoFs$},
% xtick={},
% xticklabels={},
% ylabel={ _LABERROR },
grid=both,
legend pos=north east
]
\addplot[color=\documentcolor, dashed, semithick] coordinates {(15750, 0.001746168162036185) (45000, 0.0005505125271136028) (126000, 0.00023049478861934987) (360000, 8.26322e-05)};
\addplot[color=\documentcolor, dotted, semithick] coordinates {(15750, 0.0019272758508139458) (45000, 0.0006610576) (126000, 0.00024091049317399307) (360000, 8.26322e-05)};
\addplot[color=\documentcolor, solid, thick] coordinates {(15750, 0.0018367220064250652) (45000, 0.0006057850635568014) (126000, 0.00023570264089667147) (360000, 8.26322e-05)};
\addplot[color=\accentcolor, mark=triangle, very thick] coordinates {(15750, 0.0014825) (45000, 0.000617285) (126000, 0.000246978) (360000, 8.26322e-05)};
\legend{$\BigO{h^3}$,$\BigO{\tau^3}$,$\BigO{h^3 + \tau^3}$}
\end{loglogaxis}
\end{tikzpicture}
    \end{subfigure}
    \hfill
    \begin{subfigure}[t]{0.49\textwidth}
        \centering
        \input{content/tikz/2_2_par_l2h1.tex}
    \end{subfigure}
\end{figure}

\begin{figure}[!ht]
    \begin{subfigure}[t]{0.49\textwidth}
        \centering
        \input{content/tikz/2_2_par_l2T.tex}
    \end{subfigure}
    \hfill
    \begin{subfigure}[t]{0.49\textwidth}
        \centering
        \begin{tikzpicture}[scale=.9]
\begin{loglogaxis}[
xlabel={$DoFs$},
% xtick={},
% xticklabels={},
% ylabel={ _LABERROR },
grid=both,
legend pos=north east
]
\addplot[color=\documentcolor, dashed, semithick] coordinates {(15750, 0.0027077670663352865) (45000, 0.0008536747622204869) (126000, 0.00035742616957212364) (360000, 0.000128137)};
\addplot[color=\documentcolor, dotted, semithick] coordinates {(15750, 0.002988609109956489) (45000, 0.001025096) (126000, 0.0003735777077681092) (360000, 0.000128137)};
\addplot[color=\documentcolor, solid, thick] coordinates {(15750, 0.0028481880881458875) (45000, 0.0009393853811102435) (126000, 0.0003655019386701164) (360000, 0.000128137)};
\addplot[color=\accentcolor, mark=triangle, very thick] coordinates {(15750, 0.00229241) (45000, 0.000954386) (126000, 0.00038189) (360000, 0.000128137)};
\legend{$\BigO{h^3}$,$\BigO{\tau^3}$,$\BigO{h^3 + \tau^3}$,$\Norm{\Error}_{\SpaceLp{\infty}(I; \SpaceLp{2}(\Omega))}$}
\end{loglogaxis}
\end{tikzpicture}
    \end{subfigure}
\end{figure}

\newpage

\begin{figure}[!ht]
    \begin{subfigure}[t]{0.49\textwidth}
        \centering
        \begin{tikzpicture}[scale=.9]
\begin{loglogaxis}[
xlabel={$DoFs$},
% xtick={},
% xticklabels={},
% ylabel={ _LABERROR },
grid=both,
legend pos=north east
]
\addplot[color=\documentcolor, dashed, semithick] coordinates {(10500, 0.004792495159929444) (30000, 0.001510924708761123) (84000, 0.0006326109837892596) (240000, 0.00022679053894358858) (672000, 0.000106066)};
\addplot[color=\documentcolor, dotted, semithick] coordinates {(10500, 0.0016970512482484274) (30000, 0.0008315567747544475) (84000, 0.000424264) (240000, 0.00020788919368861186) (672000, 0.000106066)};
\addplot[color=\documentcolor, solid, thick] coordinates {(10500, 0.0032447732040889357) (30000, 0.0011712407417577852) (84000, 0.0005284374918946298) (240000, 0.0002173398663161002) (672000, 0.000106066)};
\addplot[color=\accentcolor, mark=triangle, very thick] coordinates {(10500, 0.00210328) (30000, 0.000965358) (84000, 0.000455247) (240000, 0.000206094) (672000, 0.000106066)};
\legend{$h^3$,$\tau^2$,$h^3 + \tau^2$,$\Norm{\Error}_{\SpaceLp{2}(I; \SpaceLp{2}(\Omega))}$}
\end{loglogaxis}
\end{tikzpicture}
    \end{subfigure}
    \hfill
    \begin{subfigure}[t]{0.49\textwidth}
        \centering
        \begin{tikzpicture}[scale=.9]
\begin{loglogaxis}[
xlabel={$DoFs$},
% xtick={},
% xticklabels={},
% ylabel={ _LABERROR },
grid=both,
legend pos=north east
]
\addplot[color=\documentcolor, dashed, semithick] coordinates {(10500, 0.010996883365782237) (30000, 0.005093977776140967) (84000, 0.0028509284354104026) (240000, 0.0014387271934965318) (672000, 0.000866854)};
\addplot[color=\documentcolor, dotted, semithick] coordinates {(10500, 0.0034674111456195426) (30000, 0.0024271902291239087) (84000, 0.001733708) (240000, 0.0012135951145619544) (672000, 0.000866854)};
\addplot[color=\documentcolor, solid, thick] coordinates {(10500, 0.0072321472557008906) (30000, 0.0037605840026324376) (84000, 0.0022923182177052013) (240000, 0.001326161154029243) (672000, 0.000866854)};
\addplot[color=\accentcolor, mark=triangle, very thick] coordinates {(10500, 0.00961887) (30000, 0.00543628) (84000, 0.00296588) (240000, 0.00139258) (672000, 0.000866854)};
\legend{$h^2$,$\tau$,$h^2 + \tau$,$\Norm{\Error}_{\sqrt{\Diffusion} \SpaceLp{2}(I; \SpaceHk{1}(\Omega))}$}
\end{loglogaxis}
\end{tikzpicture}
    \end{subfigure}
\end{figure}

\begin{figure}[!ht]
    \begin{subfigure}[t]{0.49\textwidth}
        \centering
        \input{content/tikz/2_1_par_l2T.tex}
    \end{subfigure}
    \hfill
    \begin{subfigure}[t]{0.49\textwidth}
        \centering
        \begin{tikzpicture}[scale=.9]
\begin{loglogaxis}[
xlabel={$DoFs$},
% xtick={},
% xticklabels={},
% ylabel={ _LABERROR },
grid=both,
legend pos=north east
]
\addplot[color=\documentcolor, dashed, semithick] coordinates {(10500, 0.017782512941530295) (30000, 0.0056062734109507415) (84000, 0.0023472977292171822) (240000, 0.0008415043853357522) (672000, 0.000393557)};
\addplot[color=\documentcolor, dotted, semithick] coordinates {(10500, 0.006296894368665797) (30000, 0.0030854844116119783) (84000, 0.001574228) (240000, 0.0007713711029029946) (672000, 0.000393557)};
\addplot[color=\documentcolor, solid, thick] coordinates {(10500, 0.012039703655098046) (30000, 0.00434587891128136) (84000, 0.0019607628646085913) (240000, 0.0008064377441193733) (672000, 0.000393557)};
\addplot[color=\accentcolor, mark=triangle, very thick] coordinates {(10500, 0.00568751) (30000, 0.00289989) (84000, 0.00152017) (240000, 0.000760563) (672000, 0.000393557)};
\legend{$\BigO{h^3}$,$\BigO{\tau^2}$,$\BigO{h^3 + \tau^2}$,$\Norm{\Error}_{\SpaceLp{\infty}(I; \SpaceLp{2}(\Omega))}$}
\end{loglogaxis}
\end{tikzpicture}
    \end{subfigure}
\end{figure}

\newpage
\subsection{Hyperbolic Case}

\begin{figure}[!ht]
    \centering
    \input{content/tikz/1_1_hyp_l2l2.tex}
\end{figure}

\begin{figure}[!ht]
    \begin{subfigure}[t]{0.49\textwidth}
        \centering
        \begin{tikzpicture}[scale=.9]
\begin{loglogaxis}[
xlabel={$DoFs$},
% xtick={},
% xticklabels={},
% ylabel={ _LABERROR },
grid=both,
legend pos=north east
]
\addplot[color=\documentcolor, dashed, semithick] coordinates {(5250, 0.014275648287195318) (15000, 0.006612767699369434) (42000, 0.003700944193199409) (120000, 0.00186768948186613) (336000, 0.00112531)};
\addplot[color=\documentcolor, dotted, semithick] coordinates {(5250, 0.01800490958616746) (15000, 0.008822423342059918) (42000, 0.00450124) (120000, 0.0022056058355149795) (336000, 0.00112531)};
\addplot[color=\documentcolor, solid, thick] coordinates {(5250, 0.01614027893668139) (15000, 0.007717595520714677) (42000, 0.0041010920965997046) (120000, 0.0020366476586905547) (336000, 0.00112531)};
\addplot[color=\accentcolor, mark=triangle, very thick] coordinates {(5250, 0.0138078) (15000, 0.00772809) (42000, 0.00405183) (120000, 0.0019929) (336000, 0.00112531)};
\legend{$\BigO{h^2}$,$\BigO{\tau^2}$,$\BigO{h^2 + \tau^2}$,$\Norm{\Error}_{\SpaceLp{2}(\left\{ T \right\} \times \Omega)}$}
\end{loglogaxis}
\end{tikzpicture}
    \end{subfigure}
    \hfill
    \begin{subfigure}[t]{0.49\textwidth}
        \centering
        \input{content/tikz/1_1_hyp_linfl2.tex}
    \end{subfigure}
\end{figure}

\newpage

\begin{figure}[!ht]
    \centering
    \input{content/tikz/2_2_hyp_l2l2.tex}
\end{figure}

\begin{figure}[!ht]
    \begin{subfigure}[t]{0.49\textwidth}
        \centering
        \begin{tikzpicture}[scale=.9]
\begin{loglogaxis}[
xlabel={$DoFs$},
% xtick={},
% xticklabels={},
% ylabel={ _LABERROR },
grid=both,
legend pos=north east
]
\addplot[color=\documentcolor, dashed, semithick] coordinates {(15750, 0.003061217763257518) (45000, 0.0009651067769617394) (126000, 0.0004040817812398179) (360000, 0.000144863)};
\addplot[color=\documentcolor, dotted, semithick] coordinates {(15750, 0.0033787187267973096) (45000, 0.001158904) (126000, 0.0004223416146812521) (360000, 0.000144863)};
\addplot[color=\documentcolor, solid, thick] coordinates {(15750, 0.0032199682450274134) (45000, 0.0010620053884808696) (126000, 0.000413211697960535) (360000, 0.000144863)};
\addplot[color=\accentcolor, mark=triangle, very thick] coordinates {(15750, 0.00249395) (45000, 0.00105991) (126000, 0.000430618) (360000, 0.000144863)};
\legend{$h^3$,$\tau^3$,$h^3 + \tau^3$,$\Norm{\Error}_{\SpaceLp{2}(\left\{ T \right\} \times \Omega)}$}
\end{loglogaxis}
\end{tikzpicture}
    \end{subfigure}
    \hfill
    \begin{subfigure}[t]{0.49\textwidth}
        \centering
        \input{content/tikz/2_2_hyp_linfl2.tex}
    \end{subfigure}
\end{figure}

\newpage

\begin{figure}[!ht]
    \centering
    \begin{tikzpicture}[scale=.9]
\begin{loglogaxis}[
xlabel={$DoFs$},
% xtick={},
% xticklabels={},
% ylabel={ _LABERROR },
grid=both,
legend pos=north east
]
\addplot[color=\documentcolor, dashed, semithick] coordinates {(10500, 0.004803068235820148) (30000, 0.0015142580708361528) (84000, 0.0006340066333867431) (240000, 0.0002272908782239687) (672000, 0.0001063)};
\addplot[color=\documentcolor, dotted, semithick] coordinates {(10500, 0.0017007952377652389) (30000, 0.0008333913332868003) (84000, 0.0004252) (240000, 0.00020834783332170009) (672000, 0.0001063)};
\addplot[color=\documentcolor, solid, thick] coordinates {(10500, 0.0032519317367926937) (30000, 0.0011738247020614766) (84000, 0.0005296033166933715) (240000, 0.00021781935577283438) (672000, 0.0001063)};
\addplot[color=\accentcolor, mark=triangle, very thick] coordinates {(10500, 0.00219274) (30000, 0.00100901) (84000, 0.000472772) (240000, 0.000210667) (672000, 0.0001063)};
\legend{$\BigO{h^3}$,$\BigO{\tau^2}$,$\BigO{h^3 + \tau^2}$}
\end{loglogaxis}
\end{tikzpicture}
\end{figure}

\begin{figure}[!ht]
    \begin{subfigure}[t]{0.49\textwidth}
        \centering
        \input{content/tikz/2_1_hyp_l2T.tex}
    \end{subfigure}
    \hfill
    \begin{subfigure}[t]{0.49\textwidth}
        \centering
        \begin{tikzpicture}[scale=.9]
\begin{loglogaxis}[
xlabel={$DoFs$},
% xtick={},
% xticklabels={},
% ylabel={ _LABERROR },
grid=both,
legend pos=north east
]
\addplot[color=\documentcolor, dashed, semithick] coordinates {(10500, 0.017798417739451355) (30000, 0.005611287699200359) (84000, 0.002349397167927927) (240000, 0.0008422570324583753) (672000, 0.000393909)};
\addplot[color=\documentcolor, dotted, semithick] coordinates {(10500, 0.006302526352896214) (30000, 0.0030882440894042357) (84000, 0.001575636) (240000, 0.0007720610223510589) (672000, 0.000393909)};
\addplot[color=\documentcolor, solid, thick] coordinates {(10500, 0.012050472046173785) (30000, 0.004349765894302297) (84000, 0.001962516583963963) (240000, 0.0008071590274047171) (672000, 0.000393909)};
\addplot[color=\accentcolor, mark=triangle, very thick] coordinates {(10500, 0.00568812) (30000, 0.00290104) (84000, 0.00152116) (240000, 0.00076117) (672000, 0.000393909)};
\legend{$h^3$,$\tau^2$,$h^3 + \tau^2$,$\Norm{\Error}_{\SpaceLp{\infty}(I; \SpaceLp{2}(\Omega))}$}
\end{loglogaxis}
\end{tikzpicture}
    \end{subfigure}
\end{figure}
\section{Background and Motivation}

\acrfull{pdes} are fundamental in modeling a wide range of physical, biological, and engineering phenomena, including fluid dynamics, electromagnetism, and structural mechanics. Their study dates back to the works of Euler, Laplace, and Fourier, who formulated many of the classical equations still used today. Over time, the development of solution techniques has played a crucial role in advancing physics, engineering, and applied mathematics. However, solving \acrshort{pdes} analytically is often infeasible, except for specific cases with simple geometries and boundary conditions. This limitation necessitates numerical methods that approximate solutions efficiently and accurately.

Among numerical techniques, Galërkin methods have significantly influenced computational mathematics. First introduced by Boris Galërkin in the early 20th century, these methods provide a systematic framework for approximating \acrshort{pdes} solutions by projecting them onto a finite-dimensional function space. They have since evolved into various formulations, including the \acrfull{fem} and the \acrfull{dgm}, both of which offer significant advantages. By transforming the original problem into a system of algebraic equations, these methods enable numerical solutions to complex \acrshort{pdes} that arise in real-world applications. Their flexibility in handling irregular geometries, adaptive refinement, and stability properties makes them particularly effective for problems where classical approaches, such as finite difference methods, struggle.

\newpage
\section{Thesis Outline}

This thesis focuses on the discontinuous Galërkin method, first introduced in \citetitle{Reed1973} as an extension of Galërkin methods that allows discontinuities between elements, offering enhanced stability properties and adaptability for high-order approximations. The study is structured to provide a coherent progression from theoretical foundations to numerical implementation and results.

\Cref{chapter:preliminaries} provides a brief review of the necessary mathematical preliminaries, introducing key function spaces that are fundamental for formulating weak solutions of \acrshort{pdes}. This chapter establishes notation and ensures completeness, laying the groundwork for the subsequent theoretical developments.

\Cref{chapter:dg} investigates the \acrshort{dgm} for a class of parabolic problems. After a brief introduction to Bochner spaces, it presents the model parabolic problem and its weak formulation. The chapter then develops a discretization that is continuous in space and discontinuous in time, analyzing its stability and accuracy.

\Cref{chapter:cdr} specialises the \acrshort{dgm} to the case of \acrfull{cdr} problems, detailing a discontinuous discretization of both space and time and providing a comprehensive error analysis of the method on which numerical tests are based.

\Cref{chapter:implementation} details the practical aspects of implementing the algorithm based on the \acrshort{dgm} for \acrshort{cdr} problems. It discusses the various components of the computational framework, including the handling of mesh generation, numerical integration, and the solution of the resulting algebraic systems. This chapter bridges the theoretical aspects developed in earlier chapters with their computational realization.

\Cref{chapter:results} presents the numerical results obtained by testing the implemented algorithm on a known exact solution of a \acrshort{cdr} problem. The chapter describes the testing conditions, evaluates the observed numerical behavior against theoretical predictions, and analyzes the effectiveness of the method. These results validate the algorithm and provide insights into its performance, concluding the thesis.
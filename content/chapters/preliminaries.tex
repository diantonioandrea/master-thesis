This chapter develops the mathematical basics behind the theory of \acrfull{pdes} and of the \acrfull{dgm}.

\section{Function Spaces}

The function spaces introduced in this section provide an ideal setting for studying and characterizing solutions of \acrshort{pdes}.

\subsection{Lebesgue Spaces} \label{subsection:lebesgue}

Following \cite[p.~89]{Brezis2010}, let $(\Omega, \SigmaAlgebraA, \omega)$ denote a measure space, where $\Omega \subseteq \RealNumbersTo{d}$, $\SigmaAlgebraA$ is a $\sigma$-algebra on $\Omega$, and $\omega$ is a measure. Assume that $\Omega$ is $\sigma$-finite, that is there exist $\left\{ \Omega_n \right\}_{n \in \NaturalNumbers} \subset \SigmaAlgebraA$ such that:
\begin{align}
    \Omega & = \bigcup_{n \in \NaturalNumbers} \Omega_n, \\
    \omega(\Omega_n) & < +\infty &\text{ for all } n \in \NaturalNumbers.
\end{align}

\begin{definition}[$\SpaceLp{1}(\Omega, \SigmaAlgebraA, \omega)$]
    \begin{align}
        \SpaceLp{1}(\Omega, \SigmaAlgebraA, \omega) = \left\{ f \colon \Omega \rightarrow \RealNumbers \text{ measurable such that } \int_{\Omega} \lvert f(\Vector{x}) \rvert ~ d \omega(\Vector{x}) < \infty \right\},
    \end{align}
    with:
    \begin{align}
    \lVert f \rVert_{\SpaceLp{1}(\Omega)} = \lVert f \rVert_{1} = \int_{\Omega} \lvert f(\Vector{x}) \rvert ~ d \omega(\Vector{x}).
    \end{align}
\end{definition}

Without losing generality, let $\SpaceLp{1}(\Omega) = \SpaceLp{1}(\Omega, \SigmaAlgebraA, \omega)$ in what follows.

\subsubsection{Known Results}

\begin{theorem}[Beppo Levi, or monotone convergence]
    Let $\left\{ f_n \right\}_{n \in \NaturalNumbers} \subset \SpaceLp{1}(\Omega)$ such that:
    \begin{align}
        & 0 \leq f_1(\Vector{x}) \leq \dots \leq f_n(\Vector{x}) \leq \dots < +\infty &\text{ for } \omega \text{-a.e. } \Vector{x} \in \Omega, \\
        & \sup_{n \in \NaturalNumbers} \int_\Omega f_n(\Vector{x}) ~ d \omega(\Vector{x}) < +\infty,
    \end{align}
    then there exists $f \in \SpaceLp{1}(\Omega)$ such that for $\omega$-a.e. $\Vector{x} \in \Omega$:
    \begin{align}
        \lim_{n \rightarrow \infty} f_n(\Vector{x}) = f(\Vector{x}),
    \end{align}
    and
    \begin{align}
        \lim_{n \rightarrow \infty} \lVert f_n - f \rVert_1 = 0.
    \end{align}
\end{theorem}

\begin{theorem}[Lebesgue, or dominated convergence]
    Let $\left\{ f_n \right\}_{n \in \NaturalNumbers} \subset \SpaceLp{1}(\Omega)$ for which there exists $g \in \SpaceLp{1}(\Omega)$ such that for $\omega$-a.e. $\Vector{x} \in \Omega$:
    \begin{align}
        \lim_{n \rightarrow \infty} f_n(\Vector{x}) & = f(\Vector{x}), \\
        \lvert f_n(\Vector{x}) \rvert & < g(\Vector{x}),
    \end{align}
    then $f \in \SpaceLp{1}(\Omega)$ and:
    \begin{align}
        \lim_{n \rightarrow \infty} \lVert f_n - f \rVert_1 = 0.
    \end{align}
\end{theorem}

\begin{lemma}[Fatou]
    Let $\left\{ f_n \right\}_{n \in \NaturalNumbers} \subset \SpaceLp{1}(\Omega)$ such that:
    \begin{align}
        f_n(x) & \geq 0 &\text{ for } \omega \text{-a.e. } \Vector{x} \in \Omega, \\
        \sup_{n \in \NaturalNumbers} \int_{\Omega} f_n(\Vector{x}) ~ d \omega(\Vector{x}) & < +\infty,
    \end{align}
    then:
    \begin{align}
        \int_{\Omega} \liminf_{n \rightarrow \infty} f_n(\Vector{x}) ~ d \omega(\Vector{x}) \leq \liminf_{n \rightarrow \infty} \int_{\Omega} f_n(\Vector{x}) ~ d \omega(\Vector{x}).
    \end{align}
\end{lemma}

\subsubsection{Definition and Elementary Properties}

\begin{definition}[$\SpaceLp{p}(\Omega)$]
    Let $1 < p < +\infty$,
    \begin{align}
        \SpaceLp{p}(\Omega) = \left\{ f \colon \Omega \rightarrow \RealNumbers \text{ such that } \left| f \right|^p \in \SpaceLp{1}(\Omega) \right\},
    \end{align}
    with:
    \begin{align}
    \lVert f \rVert_{\SpaceLp{p}(\Omega)} = \lVert f \rVert_{p} = \left( \int_{\Omega} \lvert f(\Vector{x}) \rvert^p ~ d \omega(\Vector{x}) \right)^{1/p}.
    \end{align}
\end{definition}

\begin{definition}[$\SpaceLp{\infty}(\Omega)$]
    Let $C \geq 0$,
    \begin{align}
        \SpaceLp{\infty}(\Omega) = \left\{ f \colon \Omega \rightarrow \RealNumbers \text{ measurable such that } \lvert f(\Vector{x}) \rvert < C \text{ for } \omega \text{-a.e. } \Vector{x} \in \Omega \right\},
    \end{align}
    with:
    \begin{align}
    \lVert f \rVert_{\SpaceLp{\infty}(\Omega)} = \lVert f \rVert_{\infty} = \inf_{C \geq 0} \left\{ \lvert f(\Vector{x}) \rvert < C \text{ for } \omega \text{-a.e. } \Vector{x} \in \Omega \right\}.
    \end{align}
\end{definition}

\begin{theorem}
    Let $1 \leq p \leq +\infty$, then $\SpaceLp{p}(\Omega)$ is a vector space and $\lVert \cdot \rVert_p$ is a norm.
\end{theorem}

\begin{theorem}[Fischer-Riesz]
    Let $1 \leq p \leq +\infty$, then $(\SpaceLp{p}(\Omega), \lVert \cdot \rVert_p)$ is a Banach space.
\end{theorem}

\begin{definition}[Conjugate exponent]
    Let $1 \leq p \leq +\infty$, denote by $p^{\prime}$ its conjugate exponent defined by:
    \begin{align}
        \frac{1}{p} + \frac{1}{p^{\prime}} = 1.
    \end{align}
\end{definition}

\begin{theorem}[Hölder's inequality]
    Let $1 \leq p \leq +\infty$, $f \in \SpaceLp{p}(\Omega)$, and $g \in \SpaceLp{p^{\prime}}(\Omega)$, then $fg \in \SpaceLp{1}(\Omega)$ and:
    \begin{align}
        \lVert fg \rVert_1 \leq \lVert f \rVert_p \lVert g \rVert_{p^{\prime}}.
    \end{align}
\end{theorem}

\subsubsection{Characterizations}

\begin{theorem}
    Let $1 < p < +\infty$, then $\SpaceLp{p}(\Omega)$ is reflexive.
\end{theorem}

\begin{theorem}[Riesz representation theorem for $1 < p < +\infty$] \label{theorem:riesz_p}
    Let $1 < p < +\infty$ and $\varphi \in (\SpaceLp{p}(\Omega))^*$, then for all $f \in \SpaceLp{p}(\Omega)$ there exists a unique $u \in \SpaceLp{p^{\prime}}(\Omega)$ such that:
    \begin{align}
        \langle f, \varphi \rangle = \int_{\Omega} f(\Vector{x}) u(\Vector{x}) ~ d \omega(\Vector{x}),
    \end{align}
    and:
    \begin{align}
        \lVert u \rVert_{p^{\prime}} = \lVert \varphi \rVert_{p^*},
    \end{align}
    where:
    \begin{align}
        \lVert \varphi \rVert_{p^*} = \lVert \varphi \rVert_{(\SpaceLp{p})^*}.
    \end{align}
\end{theorem}

As a consequence of \cref{theorem:riesz_p}, the following identification result can be stated:

\begin{theorem}
    Let $1 < p < +\infty$, then:
    \begin{align}
        (\SpaceLp{p}(\Omega))^* = \SpaceLp{p^{\prime}}(\Omega).
    \end{align}
\end{theorem}

\begin{theorem}[Riesz representation theorem for $p = 1$] \label{theorem:riesz_1}
    Let $\varphi \in (\SpaceLp{1}(\Omega))^*$, then for all $f \in \SpaceLp{1}(\Omega)$ there exists a unique $u \in \SpaceLp{\infty}(\Omega)$ such that:
    \begin{align}
        \langle f, \varphi \rangle = \int_{\Omega} f(\Vector{x}) u(\Vector{x}) ~ d \omega(\Vector{x}),
    \end{align}
    and:
    \begin{align}
        \lVert u \rVert_{\infty} = \lVert \varphi \rVert_{1^*}.
    \end{align}
\end{theorem}

As a consequence of \cref{theorem:riesz_1}, the following identification result can be stated:

\begin{theorem}
    \begin{align}
        (\SpaceLp{1}(\Omega))^* = \SpaceLp{\infty}(\Omega).
    \end{align}
\end{theorem}

\begin{theorem}
    Let $1 \leq p < +\infty$, then $C_c(\RealNumbersTo{d})$ is dense in $\SpaceLp{p}(\RealNumbersTo{d})$.
\end{theorem}

\begin{theorem}
    let $1 \leq p < +\infty$ and assume that $(\Omega, \SigmaAlgebraA, \omega)$ is a separable measure space, that is there exist $\left\{ E_n \right\}_{n \in \NaturalNumbers} \subset \SigmaAlgebraA$ such that $\sigma \left( \left\{ E_n \right\}_{n \in \NaturalNumbers} \right) = \SigmaAlgebraA$, then $\SpaceLp{p}(\Omega)$ is separable.
\end{theorem}

\newpage
\subsection{Hilbert Spaces}

Following \cite[p.~131]{Brezis2010}, let $H$ be a vector space.

\begin{definition}[Scalar product]
    A scalar product $\left( \cdot, \cdot \right)_H$ is a bilinear form in $H \times H$ with values in $\RealNumbers$ such that:
    \begin{align}
        (u, v)_H & = (v, u)_H &\text{ for all } u, v \in H, \\
        (u, u)_H & \geq 0 &\text{ for all } u \in H, \\
        (u, u)_H & \neq 0 &\text{ for all } u \in H \setminus \left\{ 0_H \right\}.
    \end{align}
    Moreover, $\lvert \cdot \rvert_H$, defined for $u \in H$ as:
    \begin{align} % Notation to be changed.
        \lvert u \rvert_H = (u, u)_H^{\nicefrac12},
    \end{align}
    is a norm.
\end{definition}

\subsubsection{Definition}

\begin{definition}[Hilbert space] % [!] Better norm notation.
    A Hilbert space is a vector space $(H, (\cdot, \cdot)_H)$ such that $H$ is complete for the norm $\lvert \cdot \rvert_H$.
\end{definition}

\begin{remark}
    The space $\SpaceLp{2}(\Omega)$ equipped with $(\cdot, \cdot)_{\SpaceLp{2}(\Omega)}$ such that, for all $u, v \in \SpaceLp{2}(\Omega)$:
    \begin{align}
        (u, v)_{\SpaceLp{2}(\Omega)} = \int_{\Omega} u(\Vector{x}) v(\Vector{x}) ~ d \omega(\Vector{x}),
    \end{align}
    is a Hilbert space. Moreover, for all $u \in \SpaceLp{2}(\Omega)$:
    \begin{align}
        (u, u)_{\SpaceLp{2}(\Omega)} = \lVert u \rVert_2^2.
    \end{align}
\end{remark}

\subsubsection{Characterizations}

\begin{theorem}
    $H$ is uniformly convex, and thus it is reflexive.
\end{theorem}

\begin{theorem}[Riesz-Fréchet representation theorem for Hilbert spaces]
    Let $\varphi \in H^*$, then for all $u \in H$ there exists a unique $f \in H$ such that:
    \begin{align}
        \langle u, \varphi \rangle = (u, f)_H,
    \end{align}
    and:
    \begin{align}
        \lvert f \rvert_H = \lVert \varphi \rVert_{H^*}.
    \end{align}
\end{theorem}

\subsubsection{Notable Results}

\begin{definition}[Continuous bilinear form]
    A bilinear form $a \colon H \times H \rightarrow \RealNumbers$ is said to be continuous if there exists $\ContC{a} \geq 0$ such that for all $u, v \in H$:
    \begin{align}
        \lvert a(u, v) \rvert \leq \ContC{a} \lvert u \rvert_H \lvert v \rvert_H.
    \end{align}
\end{definition}

\begin{definition}[Coercive bilinear form]
    A bilinear form $a \colon H \times H \rightarrow \RealNumbers$ is said to be coercive if there exists $\CoerC{a} > 0$ such that for all $u \in H$:
    \begin{align}
        \lvert a(u, u) \rvert \geq \CoerC{a} \lvert u \rvert_H^2.
    \end{align}
\end{definition}

\begin{definition}[Symmetric bilinear form]
    A bilinear form $a \colon H \times H \rightarrow \RealNumbers$ is said to be symmetric if for all $u, v \in H$:
    \begin{align}
        a(u, v) = a(v, u).
    \end{align}
\end{definition}

The following results concerning bilinear forms defined on Hilbert spaces are central to the development of the methods that will be introduced and applied in \cref{chapter:dg} and \cref{chapter:cdr}, particularly \cref{theorem:lm}, which provides existence and uniqueness for the problems to be studied.

\begin{theorem}[Stampacchia]
    Let $a \colon H \times H \rightarrow \RealNumbers$ be a continuous a coercive bilinear form on $H$ and let $K \subset H$ be a nonempty closed and convex subset, then, for all $\varphi \in H^*$, there exists a unique $u \in K$ such that, for all $v \in H$:
    \begin{align}
        a(u, v - u) \geq \langle u - v, \varphi \rangle.
    \end{align}
    Morever, if $a(\cdot, \cdot)$ is symmetric, then:
    \begin{align}
        \frac{1}{2} a(u, u) - \langle u, \varphi \rangle = \min_{v \in K} \left\{ a(v, v) - \langle v, \varphi \rangle \right\}.
    \end{align}
\end{theorem}

\begin{corollary}[Lax-Milgram] \label{theorem:lm}
    Let $a \colon H \times H \rightarrow \RealNumbers$ be a continuous a coercive bilinear form on $H$, for all $\varphi \in H^*$, there exists a unique $u \in H$ such that, for all $v \in H$:
    \begin{align}
        a(u, v) = \langle v, \varphi \rangle.
    \end{align}
    Morever, if $a(\cdot, \cdot)$ is symmetric, then:
    \begin{align}
        \frac{1}{2} a(u, u) - \langle u, \varphi \rangle = \min_{v \in K} \left\{ a(v, v) - \langle v, \varphi \rangle \right\}.
    \end{align}
\end{corollary}

\newpage
\subsection{Sobolev Spaces}

Following \cite[p.~267]{Brezis2010}, let $(\Omega, \SigmaAlgebraA, \omega)$ denote the same measure space introduced with Lebesgue spaces in \cref{subsection:lebesgue}.

\subsubsection{Definition and Elementary Properties}

\begin{definition}[$\SpaceWkp{1}{p}(\Omega)$]
    Let $1 \leq p \leq +\infty$, then:
    \begin{align} % [!] Fix align.
        \SpaceWkp{1}{p}(\Omega) = \left\{ \vphantom{\int_{\Omega}} u \in \SpaceLp{p}(\Omega) \text{ for which there exist } \left\{ g_j \right\}_{j = 1}^{d} \subset \SpaceLp{p}(\Omega) \text{ such that } \right. \notag \\ 
        \left. \int_{\Omega} u \varphi_{x_j} = - \int_{\Omega} g_j \varphi \right. \notag \\
        \left. \text{ for all } \varphi \in C_c^{\infty}(\Omega) \text{ and for all } j = 1, \dots, d \vphantom{\int_{\Omega}} \right\}.
    \end{align}
    Define, for all $j = 1, \dots, d$:
    \begin{align}
        u_{x_j} = g_j,
    \end{align}
    Therefore:
    \begin{align}
        \Gradient u = \left( u_{x_1}, \dots, u_{x_d} \right).
    \end{align}
    Moreover, $\SpaceWkp{1}{p}(\Omega)$ is equipped with the norm $\lVert \cdot \rVert_{\SpaceWkp{1}{p}}$, such that for all $u \in \SpaceWkp{1}{p}(\Omega)$:
    \begin{align}
        \lVert u \rVert_{\SpaceWkp{1}{p}(\Omega)} = \lVert u \rVert_{1, p} = \lVert u \rVert_p + \sum_{j = 1}^d \lVert u_{x_j} \rVert_p.
    \end{align}
\end{definition}

\begin{definition}[$\SpaceHk{1}(\Omega)$]
    \begin{align}
        \SpaceHk{1}(\Omega) = \SpaceWkp{1}{2}(\Omega).
    \end{align}
    $\SpaceHk{1}(\Omega)$ is equipped with the scalar product $(\cdot, \cdot)_{\SpaceHk{1}(\Omega)}$, such that for all $u, v \in \SpaceHk{1}(\Omega)$:
    \begin{align}
        (u, v)_{\SpaceHk{1}(\Omega)} = (u, v)_1 = (u, v)_{\SpaceLp{2}(\Omega)} + \sum_{j = 1}^d (u_{x_j}, v_{x_j})_{\SpaceLp{2}(\Omega)}.
    \end{align}
\end{definition}

\begin{theorem}
    Let $1 \leq p < +\infty$ and $u \in \SpaceWkp{1}{p}(\Omega)$, then there exist $\left\{ u_n \right\}_{n \in \NaturalNumbers} \subset C_c^{\infty}(\Omega)$ such that:
    \begin{align}
        & \lim_{n \rightarrow \infty} u_{n \mid \Omega} = u \Restriction_{\Omega} \text { in } \SpaceLp{p}(\Omega)\\
        & \lim_{n \rightarrow \infty} \Gradient u_{n \mid K} = \Gradient u \Restriction_{K} \text { in } \SpaceLp{p}(K) &\text{ for all } K \subset \subset \Omega.
    \end{align}
\end{theorem}

\subsubsection{Generalisations}

\begin{definition}[$\SpaceWkp{k}{p}(\Omega)$]
    Let $k \geq 2$ integer and let $1 \leq p \leq +\infty$, then:
    \begin{align}
        \SpaceWkp{k}{p}(\Omega) = \left\{ u \in \SpaceWkp{k - 1}{p}(\Omega) \text{ for which } u_{x_j} \in \SpaceWkp{k - 1}{p}(\Omega) \text{ for all } j = 1, \dots, d \right\}.
    \end{align}
    Alternatively:
    \begin{align} % [!] Fix align.
        \SpaceWkp{k}{p}(\Omega) = \left\{ \vphantom{\int_{\Omega}} u \in \SpaceLp{p}(\Omega) \text{ such that there exists } g_{\alpha} \in \SpaceLp{p}(\Omega) \text{ such that } \right. \notag \\ 
        \left. \int_{\Omega} u \partial^{\alpha} \varphi = (-1)^{\lvert \alpha \rvert} \int_{\Omega} g_{\alpha} \varphi \right. \notag \\
        \left. \text{ for all } \alpha \text{ with } \lvert \alpha \rvert \leq k \text{ and for all }\varphi \in C_c^{\infty}(\Omega) \vphantom{\int_{\Omega}} \right\}.
    \end{align}
    Define, for all $\alpha \text{ with } \lvert \alpha \rvert \leq k$:
    \begin{align}
        \partial^{\alpha} u = g_{\alpha}.
    \end{align}
    Moreover, $\SpaceWkp{k}{p}(\Omega)$ is equipped with the norm $\lVert \cdot \rVert_{\SpaceWkp{k}{p}}$, such that for all $u \in \SpaceWkp{k}{p}(\Omega)$:
    \begin{align}
        \lVert u \rVert_{\SpaceWkp{k}{p}(\Omega)} = \lVert u \rVert_{k, p} = \sum_{\lvert \alpha \rvert \leq k} \lVert \partial^{\alpha} u \rVert_p.
    \end{align}
\end{definition}

\begin{definition}[$\SpaceHk{k}(\Omega)$]
    Let $k \geq 2$ integer, then:
    \begin{align}
        \SpaceHk{k}(\Omega) = \SpaceWkp{k}{2}(\Omega).
    \end{align}
    $\SpaceHk{k}(\Omega)$ is equipped with the scalar product $(\cdot, \cdot)_{\SpaceHk{k}(\Omega)}$, such that for all $u, v \in \SpaceHk{k}(\Omega)$:
    \begin{align}
        (u, v)_{\SpaceHk{k}(\Omega)} = (u, v)_k = \sum_{\lvert \alpha \rvert \leq k} (\partial^{\alpha} u, \partial^{\alpha} v)_{\SpaceLp{2}(\Omega)}.
    \end{align}
\end{definition}

Note that $(u, v)_{\SpaceLp{2}(\Omega)} = (u, v)_0$ as, formally, $\SpaceLp{2}(\Omega)$ and $\SpaceHk{0}(\Omega)$ coincide. % [!] Maybe definition.

\begin{remark}
    Let $k \geq 1$ integer and let $1 \leq p \leq +\infty$, $(\SpaceWkp{k}{p}(\Omega), \lVert \cdot \rVert_{k, p})$ is a Banach space.
\end{remark}

\begin{remark}
    Let $k \geq 1$ integer, $(\SpaceHk{k}(\Omega), (\cdot, \cdot)_k)$ is a Hilbert space.
\end{remark}

\subsubsection{Notable Results}

Consider the case in which $\Omega = \RealNumbersTo{d}$.

\begin{definition}[Critical Sobolev exponent]
    Let $d \geq 2$ and $1 \leq p < d$, denote by $p^{\star}$ its associated critical Sobolev exponent defined by:
    \begin{align}
        \frac{1}{p^{\star}} = \frac{1}{p} - \frac{1}{d}.
    \end{align}
\end{definition}

\begin{theorem}[Sobolev-Gagliardo-Nirenberg's inequality]
    Let $1 \leq p < d$, then:
    \begin{align}
        \SpaceWkp{1}{p}(\RealNumbersTo{d}) \subset \SpaceLp{p^{\star}}(\RealNumbersTo{d}),
    \end{align}
    and there exists $C \geq 0$ such that for all $u \in \SpaceWkp{1}{p}(\RealNumbersTo{d})$:
    \begin{align}
        \lVert u \rVert_{p^{\star}} \leq C \lVert \Gradient u \rVert_p.
    \end{align}
\end{theorem}

\begin{corollary}
    Let $1 \leq p < d$, then\footnote{$\hookrightarrow$ denotes a continuous embedding.}:
    \begin{align}
        \SpaceWkp{1}{p}(\RealNumbersTo{d}) \hookrightarrow \SpaceLp{q}(\RealNumbersTo{d}) &\text{ for all } p \leq q \leq p^{\star}.
    \end{align}
\end{corollary}

\begin{corollary}
    \begin{align}
        \SpaceWkp{1}{d}(\RealNumbersTo{d}) \hookrightarrow \SpaceLp{q}(\RealNumbersTo{d}) &\text{ for all } N \leq q < +\infty.
    \end{align}
\end{corollary}

\begin{theorem}[Morrey's inequality]
    Let $p > d$, then:
    \begin{align}
        \SpaceWkp{1}{p}(\RealNumbersTo{d}) \hookrightarrow \SpaceLp{\infty}(\RealNumbersTo{d}).
    \end{align}

    Furthermore, there exists $C \geq 0$ such that for all $u \in \SpaceWkp{1}{p}(\RealNumbersTo{d})$:
    \begin{align}
        \lvert u(\Vector{x}) - u(\Vector{y}) \rvert &\leq C \lVert \Vector{x} - \Vector{y} \rVert^{\alpha} \lVert \Gradient u \rVert_p &\text{ for } \omega \text{-a.e. } \Vector{x}, \Vector{y} \in \RealNumbersTo{d},
    \end{align}
    where:
    \begin{align}
        \alpha = 1 - \frac{d}{p}.
    \end{align}

    In particular\footnote{Up to the choice of a continuous representative.}:
    \begin{align}
        \SpaceWkp{1}{p}(\RealNumbersTo{d}) \subset C(\RealNumbersTo{d}).
    \end{align}
\end{theorem}

\begin{corollary} \label{corollary:embedding}
    Let $k \geq 1$ integer and $1 \leq p < +\infty$, then:
    \begin{align}
        \SpaceWkp{k}{p}(\RealNumbersTo{d}) & \hookrightarrow \SpaceLp{q}(\RealNumbersTo{d}) &\text{ where } \frac{1}{q} = \frac{1}{p} - \frac{k}{d} \text{ if } \frac{1}{p} - \frac{k}{d} > 0, \\
        \SpaceWkp{k}{p}(\RealNumbersTo{d}) & \hookrightarrow \SpaceLp{q}(\RealNumbersTo{d}) &\text{ for all } p \leq q < +\infty \text{ if } \frac{1}{p} - \frac{k}{d} = 0, \\
        \SpaceWkp{k}{p}(\RealNumbersTo{d}) & \hookrightarrow \SpaceLp{\infty}(\RealNumbersTo{d}) & \text{ if } \frac{1}{p} - \frac{k}{d} < 0.
    \end{align}

    Moreover, set:
    \begin{align}
        & m = \left[ k - \frac{d}{p} \right], \\
        & \theta = k - \frac{d}{p} - m,
    \end{align}
    then there exists $C \geq 0$ such that for all $u \in \SpaceWkp{k}{p}(\RealNumbersTo{d})$ for all $\alpha$ with $\lvert \alpha \rvert \leq m$:
    \begin{align}
        \lVert \partial^{\alpha} u \rVert_{\infty} \leq \lVert u \rVert_{k, p},
    \end{align}
    and
    \begin{align}
        \lvert \partial^{\alpha} u(\Vector{x}) - \partial^{\alpha} u(\Vector{y}) \rvert &\leq C \lVert \Vector{x} - \Vector{y} \rVert^{\theta} \lVert u \rVert_{k, p} &\text{ for } \omega \text{-a.e. } \Vector{x}, \Vector{y} \in \RealNumbersTo{d}.
    \end{align}

    In particular:
    \begin{align}
        \SpaceWkp{k}{p}(\RealNumbersTo{d}) \subset C^m(\RealNumbersTo{d}).
    \end{align}
\end{corollary}

Consider now the case in which $\Omega \subset \RealNumbersTo{d}$.

\begin{corollary}
    Let $1 \leq p \leq +\infty$, then:
    \begin{align}
        \SpaceWkp{1}{p}(\Omega) & \hookrightarrow \SpaceLp{p^{\star}}(\Omega) & \text{ if } p < d, \\
        \SpaceWkp{1}{p}(\Omega) & \hookrightarrow \SpaceLp{q}(\Omega) &\text{ for all } p \leq q < +\infty \text{ if } p = d, \\
        \SpaceWkp{1}{p}(\Omega) & \hookrightarrow \SpaceLp{\infty}(\Omega) & \text{ if } p > d.
    \end{align}

    Furthermore, if $p > d$ there exists $C \geq 0$ such that for all $u \in \SpaceWkp{1}{p}(\Omega)$:
    \begin{align}
        \lvert u(\Vector{x}) - u(\Vector{y}) \rvert &\leq C \lVert \Vector{x} - \Vector{y} \rVert^{\alpha} \lVert u \rVert_{1, p} &\text{ for } \omega \text{-a.e. } \Vector{x}, \Vector{y} \in \RealNumbersTo{d},
    \end{align}
    where:
    \begin{align}
        \alpha = 1 - \frac{d}{p}.
    \end{align}

    In particular:
    \begin{align}
        \SpaceWkp{1}{p}(\Omega) \subset C(\overline{\Omega}).
    \end{align}
\end{corollary}

\begin{corollary}
    Results stated in \cref{corollary:embedding} remain true for $\Omega \subset \RealNumbersTo{d}$.
\end{corollary}

\begin{theorem}[Rellich-Kondrachov]
    Let $1 \leq p \leq +\infty$ and assume that $\Omega$ is bounded and of class $C^1$, then\footnote{$\hookrightarrow \hookrightarrow$ denotes a compact embedding.}:
    \begin{align}
        \SpaceWkp{1}{p}(\Omega) & \hookrightarrow \hookrightarrow \SpaceLp{q}(\Omega) &\text{ for all } p \leq q < p^{\star} \text{ if } p < d, \\
        \SpaceWkp{1}{p}(\Omega) & \hookrightarrow \hookrightarrow \SpaceLp{q}(\Omega) &\text{ for all } p \leq q < +\infty \text{ if } p = d, \\
        \SpaceWkp{1}{p}(\Omega) & \hookrightarrow \hookrightarrow C(\overline{\Omega}) & \text{ if } p > d.
    \end{align}

    In particular:
    \begin{align}
        \SpaceWkp{1}{p}(\Omega) \hookrightarrow \SpaceLp{p}(\Omega).
    \end{align}
\end{theorem}

\begin{definition}[$\SpaceWkp{1}{p}_0(\Omega)$]
    Let $1 \leq p < +\infty$, $\SpaceWkp{1}{p}_0(\Omega)$ denotes the closure of $C_c^{\infty}(\Omega)$ in $\SpaceWkp{1}{p}(\Omega)$.

    Moreover, set $\SpaceHk{1}_0(\Omega) = \SpaceWkp{1}{2}_0(\Omega)$.
\end{definition}

\begin{theorem}
    Let $1 \leq p < +\infty$ and assume that $\Omega$ is of class $C^1$. Then, for $u \in \SpaceWkp{1}{p}(\Omega) \cap C(\overline{\Omega})$, $u = 0$ on $\partial \Omega$ if and only if $u \in \SpaceWkp{1}{p}_0(\Omega)$.
\end{theorem}

\begin{theorem}[Poincaré's inequality]
    Let $1 \leq p < +\infty$ and assume that $\Omega$ is a bounded open set. Then there exists $C \geq 0$ such that for all $u \in \SpaceWkp{1}{p}_0(\Omega)$:
    \begin{align}
        \lVert u \rVert_p \leq C \lVert \Gradient u \rVert_p.
    \end{align}
\end{theorem}

\begin{definition}[$\SpaceWkp{-1}{p^{\prime}}(\Omega)$]
    Let $1 \leq p < +\infty$, $\SpaceWkp{-1}{p^{\prime}}(\Omega)$ denotes the dual space of $\SpaceWkp{1}{p}_0(\Omega)$ and $\SpaceHk{-1}(\Omega)$ the dual space of $\SpaceHk{1}_0(\Omega)$.

    Moreover:
    \begin{align}
        \SpaceHk{1}_0(\Omega) \subset \SpaceLp{2}(\Omega) \subset \SpaceHk{-1}(\Omega).
    \end{align}
\end{definition}
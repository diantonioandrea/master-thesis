This chapter develops the mathematical basics behind the theory of \acrlong{pdes} and of the \acrlong{dgm}.

\section{Function Spaces}

The function spaces introduced in this section turn out to be the ideal setting to study and characterize the solutions of \acrshort{pdes}.

\subsection{Lebesgue Spaces}

Following \cite[p. 89]{Brezis2010}, let $(\Omega, \MA, \mu)$ denote a measure space, where $\Omega \subseteq \Rd$, $\MA$ is a $\sigma$-algebra on $\Omega$, and $\mu$ is a measure. Assume that $\Omega$ is $\sigma$-finite, that is there exist $\left\{ \Omega_n \right\}_{n \in \N} \subset \MA$ such that:
\begin{gather}
    \Omega = \bigcup_{n \in \N} \Omega_n, \\
    \mu(\Omega_n) < +\infty \text{ for all } n \in \N.
\end{gather}

\begin{definition}[$\Lp{1}(\Omega, \MA, \mu)$]
    \begin{gather}
        \Lp{1}(\Omega, \MA, \mu) = \left\{ f \colon \Omega \rightarrow \R \text{ measurable such that } \int_{\Omega} \lvert f(\Vector{x}) \rvert ~ d \mu < \infty \right\},
    \end{gather}
    with:
    \begin{gather}
    \lVert f \rVert_{\Lp{1}(\Omega)} = \lVert f \rVert_{1} = \int_{\Omega} \lvert f(\Vector{x}) \rvert d \mu.
    \end{gather}
\end{definition}

Without losing generality, let $\Lp{1}(\Omega) = \Lp{1}(\Omega, \MA, \mu)$ in what follows.

\subsubsection{Known Results}

\begin{theorem}[Beppo Levi, or Monotone Convergence]
    Let $\left\{ f_n \right\}_{n \in \N} \subset \Lp{1}(\Omega)$ such that:
    \begin{gather}
        0 \leq f_1(\Vector{x}) \leq \dots \leq f_n(\Vector{x}) \leq \dots < +\infty \text{ for } \mu \text{-a.e. } \Vector{x} \in \Omega, \\
        \sup_{n \in \N} \int_\Omega f_n(\Vector{x}) ~ d \mu < +\infty,
    \end{gather}
    then there exists $f \in \Lp{1}(\Omega)$ such that for $\mu$-a.e. $\Vector{x} \in \Omega$:
    \begin{gather}
        \lim_{n \rightarrow \infty} f_n(\Vector{x}) = f(\Vector{x}),
    \end{gather}
    and
    \begin{gather}
        \lim_{n \rightarrow \infty} \lVert f_n - f \rVert_1 = 0.
    \end{gather}
\end{theorem}

\begin{theorem}[Lebesgue, or Dominated Convergence]
    Let $\left\{ f_n \right\}_{n \in \N} \subset \Lp{1}(\Omega)$ for which there exists $g \in \Lp{1}(\Omega)$ such that for $\mu$-a.e. $\Vector{x} \in \Omega$:
    \begin{gather}
        \lim_{n \rightarrow \infty} f_n(\Vector{x}) = f(\Vector{x}), \\
        \lvert f_n(\Vector{x}) \rvert < g(\Vector{x}),
    \end{gather}
    then $f \in \Lp{1}(\Omega)$ and:
    \begin{gather}
        \lim_{n \rightarrow \infty} \lVert f_n - f \rVert_1 = 0.
    \end{gather}
\end{theorem}

\begin{lemma}[Fatou]
    Let $\left\{ f_n \right\}_{n \in \N} \subset \Lp{1}(\Omega)$ such that:
    \begin{gather}
        f_n(x) \geq 0 \text{ for } \mu \text{-a.e. } \Vector{x} \in \Omega, \\
        \sup_{n \in \N} \int_{\Omega} f_n(\Vector{x}) ~ d \mu < +\infty,
    \end{gather}
    then:
    \begin{gather}
        \int_{\Omega} \liminf_{n \rightarrow \infty} f_n(\Vector{x}) \leq \liminf_{n \rightarrow \infty} \int_{\Omega} f_n(\Vector{x}).
    \end{gather}
\end{lemma}

\subsubsection{Definition and elementary properties}

\begin{definition}[$\Lp{p}(\Omega)$]
    Let $1 < p < +\infty$,
    \begin{gather}
        \Lp{p}(\Omega) = \left\{ f \colon \Omega \rightarrow \R \text{ such that } \left| f \right|^p \in \Lp{1}(\Omega) \right\},
    \end{gather}
    with:
    \begin{gather}
    \lVert f \rVert_{\Lp{p}(\Omega)} = \lVert f \rVert_{p} = \left( \int_{\Omega} \lvert f(\Vector{x}) \rvert^p ~ d \mu \right)^{1/p}.
    \end{gather}
\end{definition}

\begin{definition}[$\Lp{\infty}(\Omega)$]
    Let $C \geq 0$,
    \begin{gather}
        \Lp{\infty}(\Omega) = \left\{ f \colon \Omega \rightarrow \R \text{ measurable such that } \lvert f(\Vector{x}) \rvert < C \text{ for } \mu \text{-a.e. } \Vector{x} \in \Omega \right\},
    \end{gather}
    with:
    \begin{gather}
    \lVert f \rVert_{\Lp{\infty}(\Omega)} = \lVert f \rVert_{\infty} = \inf_{C \geq 0} \left\{ \lvert f(\Vector{x}) \rvert < C \text{ for } \mu \text{-a.e. } \Vector{x} \in \Omega \right\}.
    \end{gather}
\end{definition}

\begin{theorem}
    Let $1 \leq p \leq +\infty$, then $\Lp{p}(\Omega)$ is a vector space and $\lVert \cdot \rVert_p$ is a norm.
\end{theorem}

\begin{theorem}[Fischer-Riesz]
    Let $1 \leq p \leq +\infty$, then $(\Lp{p}(\Omega), \lVert \cdot \rVert_p)$ is a Banach space.
\end{theorem}

\begin{definition}[Conjugate Exponent]
    Let $1 \leq p \leq +\infty$, denote by $1 \leq p^{\prime} \leq +\infty$ its conjugate exponent such that:
    \begin{gather}
        \frac{1}{p} + \frac{1}{p^{\prime}} = 1,
    \end{gather}
\end{definition}

\begin{theorem}[Hölder's inequality]
    Let $1 \leq p \leq +\infty$, $f \in \Lp{p}(\Omega)$, and $g \in \Lp{p^{\prime}}(\Omega)$, then $fg \in \Lp{1}(\Omega)$ and:
    \begin{gather}
        \lVert fg \rVert_1 \leq \lVert f \rVert_p \lVert g \rVert_{p^{\prime}}.
    \end{gather}
\end{theorem}

\subsubsection{Characterizations}

% [!]

\newpage
\subsection{Sobolev Spaces}

% [!]

\newpage
\subsection{Bochner Spaces}

% [!]
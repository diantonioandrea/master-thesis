\section{Implementation Overview}

This chapter outlines the implementation of the Space-Time Discontinuous Galërkin Method as presented in \cref{chapter:cdr}.

The algorithm has been implemented as part of a comprehensive library, \href{https://github.com/diantonioandrea/ivo}{ivo}\footnote{Named after \href{https://en.wikipedia.org/wiki/Ivo_Babuška}{Ivo Babuška}.}, which was written entirely in \lstinline{C++23} from scratch for this thesis.

The library has been designed to include all the necessary components for the implementation, use, and testing of the algorithm. These components comprise geometric methods and objects in $\RealNumbersTo{2}$ and $\RealNumbersTo{3}$ for mesh generation, dense and sparse matrices for problem formulation, and linear solvers for its solution, as well as additional linear algebra routines and utilities for the definition and manipulation of the principal equation driving the algorithm.

Moreover, the implementation of the algorithm differs slightly from the version introduced and analyzed in \cref{chapter:cdr}.

\newpage
\section{Prismatic Meshes Over Polygonal Domains}

Although it is not necessary to explicitly construct the full space-time mesh for this problem, doing so simplifies the process by providing a straightforward method for accessing individual elements based on their spatial index and time level. 

\subsection{Construction of the Polygonal Meshes}

The construction of the polygonal meshes over the problem's polygonal domain is based on the Voronoi mesh construction process and Lloyd's relaxation method.

Let $\Omega \subset \RealNumbersTo{2}$ be a polygonal (and hence bounded) domain, and let $\nSpaceIndices \in \NaturalNumbers$ denote the desired number of elements for $\Omega_h$, i.e., the mesh to be constructed. As before, it is customary to let $h$ denote both the mesh index and its characteristic parameter.

The function used to generate polygonal meshes is \lstinline{mesher2}, defined as follows:

\begin{lstlisting}[style=cpp]
namespace ivo {

    std::vector<Polygon21> mesher2(
        const Polygon21 &, 
        const Natural &);

}
\end{lstlisting}

This method first randomly generates $\nSpaceIndices$ points in $\Omega$ and computes their Voronoi cells by iteratively subdividing $\Omega$ using the bisectors corresponding to each point.

The initial diagram, consisting of $\nSpaceIndices$ cells, is then iteratively relaxed using Lloyd's process. At each iteration, a new diagram is generated by considering the centroids of the current cells as the generating points for the Voronoi diagram. This process continues until the residual, defined as the sum of the distances between each point and its previous position, falls below a fixed tolerance.

Finally, postprocessing is performed by collapsing edges that are too short, as such edges may introduce errors in the solution of the problem.

\subsection{Construction of the Prismatic Meshes}



\newpage
\section{Convection-Diffusion-Reaction Problems}


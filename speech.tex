\documentclass[12pt]{article}

% Language.
\usepackage[italian]{babel}

% Colors.
% Colors.
\newcommand{\documentcolor}{solarized-base02} % geometry.tex
\newcommand{\accentcolor}{solarized-orange} % localization.tex
\newcommand{\urlcolor}{solarized-orange} % localization.tex

% Includes.
% Speech geometry.

% Geometry.
\usepackage{geometry}
\geometry{
    a4paper,
    left=30mm,
    right=30mm,
    top=30mm,
    bottom=30mm,
    includehead,
    includefoot
}

\setlength{\headheight}{15pt}

% Afterpage.
\usepackage{afterpage} % Geometry and graphics.
% Document's main variables.

% Titles and data.
\newcommand{\documenttitle}{The Space-Time Discontinuous Galërkin Method}
\newcommand{\brokendocumenttitle}{The Space-Time \\ Discontinuous Galërkin Method}

\newcommand{\documentauthor}{Andrea Di Antonio}
\newcommand{\documentdate}{Sessione di Marzo 2025 \\ Anno Accademico 2023-2024}

% Metadata.
\author{Relatore: Lourenço Beirão Da Veiga \\ Laureando: \documentauthor, 858798}
\title{Riassunto della Tesi Magistrale \\ \documenttitle}
\date{\documentdate} % Variables.
\input{./include/sp_style.tex} % Style.
% PDF information and localization.

\usepackage[T1]{fontenc}
\usepackage[utf8]{inputenc}

\usepackage{bookmark}
\usepackage{hyperref}
\hypersetup{
	linktocpage=true,
	colorlinks=true,
	linkcolor=\accentcolor,
	urlcolor=\urlcolor,
	pdftitle={\documenttitle},
	pdfpagemode=FullScreen,
	pdfauthor={\documentauthor}
} % Localization.
\input{./include/p_math.tex} % Math.
% Codes.

\usepackage{courier}
\usepackage{listings}

\lstdefinestyle{default}{
	basicstyle=\small\ttfamily\color{solarized-base01},
	frame=single,
	rulecolor=\color{solarized-base01},
	xleftmargin=15pt,
	xrightmargin=15pt,
	framesep=10pt,
	breakatwhitespace=false,
	breaklines=true,
	showstringspaces=false,
	showtabs=false,
	tabsize=4,
	captionpos=b,
	keepspaces=true
}

\lstdefinestyle{cpp}{ % C++.
	language=c++,
	% backgroundcolor=\color{solarized-base3},
	basicstyle=\small\ttfamily\color{\documentcolor},
	keywordstyle=\bfseries\color{\accentcolor},
	commentstyle=\itshape\color{solarized-base00},
	stringstyle=\color{solarized-red},
	frame=single,
	rulecolor=\color{\documentcolor},
	xleftmargin=15pt,
	xrightmargin=15pt,
	framesep=10pt,
	numbers=left,
	numbersep=20pt,
	numberstyle=\color{solarized-base00},
	breakatwhitespace=false,
	breaklines=true,
	showstringspaces=false,
	showtabs=false,
	tabsize=4,
	captionpos=b,
	keepspaces=true
}

\lstset{style=default} % Code.
% Acronyms.
\usepackage[acronym, toc, nomain]{glossaries}
\glsdisablehyper

% Partial differential equations.
\newacronym{pde}{PDE}{Partial Differential Equation}
\newacronym{pdes}{PDEs}{Partial Differential Equations}

% Discontinuous Galërkin methods.
\newacronym{dg}{$dG$}{Discontinuous Galërkin}
\newacronym{dgm}{$dG$M}{Discontinuous Galërkin Method}

% Gauss-Radau.
\newacronym{gr}{GR}{Gauss-Radau}

% Gauss-Legendre.
\newacronym{gl}{GL}{Gauss-Legendre}

% Convection-Diffusion-Reaction.
\newacronym{cdr}{CDR}{Convection-Diffusion-Reaction} % Code.

% Speech title.
\title{Discorso di Presentazione della Tesi Magistrale \\ \documenttitle}

\begin{document}

    %% Title and ToC.

    \pagenumbering{roman}
    \maketitle
    \tableofcontents

    %% Speech.

    \newpage
    \pagenumbering{arabic}
    
    \newpage
    \section{Introduzione}

    \subsection{Titolo, Slide 1}

    Buongiorno a tutti, sono Andrea Di Antonio e oggi presenterò la mia tesi dal titolo \textit{The Space-Time Discontinuous Galërkin Method}, in cui viene esposto il metodo di Galërkin discontinuo e la sua applicazione ai problemi di Convezione-Diffusione-Reazione.

    \subsection{Tavola dei Contenuti, Slide 2}

    Inizierò con un'introduzione, il cui scopo è fornire un contesto generale al lavoro tramite una breve discussione sull'importanza delle equazioni alle derivate parziali nei vari campi delle scienze, evidenziando le difficoltà incontrate nella loro risoluzione analitica e mostrando come tali difficoltà possano essere superate attraverso l'uso di metodi numerici.

    \subsection{Equazioni alle Derivate Parziali, Slide 3-5}

    Le equazioni alle derivate parziali rappresentano uno strumento matematico fondamentale in numerose discipline scientifiche, poiché permettono di modellare e prevedere fenomeni e sistemi di natura fisica, biologica o ingegneristica. Progressi nella teoria di queste equazioni motivano sviluppi applicativi e viceversa.

    In questa slide sono riportati alcuni esempi di equazioni differenziali provenienti da diversi ambiti della fisica, dalla meccanica quantistica alla dinamica dei fluidi.

    Dal punto di vista analitico, la risoluzione di alcune equazioni differenziali risulta spesso ostacolata da elementi quali domini e condizioni al contorno complessi, presenza di termini non lineari o interazioni tra più equazioni simultanee. Ciò rende necessario lo sviluppo di metodi numerici specifici.

    Due esempi significativi di metodi numerici per equazioni differenziali alle derivate parziali sono il \textit{Metodo alle Differenze Finite} e il \textit{Metodo agli Elementi Finiti}. Il primo, basato sull'approssimazione degli operatori differenziali tramite differenze finite definite su griglie di punti, permette implementazioni relativamente semplici ma risulta limitato nella gestione di geometrie e condizioni complesse. Il secondo, invece, discretizzando il dominio in componenti geometriche più semplici, sulle quali vengono definite basi di funzioni generalmente polinomiali, consente di superare le limitazioni del primo metodo, pur richiedendo una maggiore complessità implementativa e un costo computazionale superiore.

    \newpage
    \section{Metodo di Galërkin Discontinuo}

    \subsection{Tavola dei Contenuti, Slide 6}

    Il metodo di Galërkin discontinuo, oggetto centrale di questa tesi, appartiene alla famiglia dei metodi agli elementi finiti.

    In questa parte della presentazione introdurrò inizialmente questo metodo nella sua forma generale, tramite lo studio di un problema modello, per poi considerare nello specifico la sua applicazione ai problemi di Convezione-Diffusione-Reazione.

    In entrambi i casi, il procedimento adottato è il medesimo: partendo da un'introduzione del problema e delle sue caratteristiche principali, si procede dapprima con una discretizzazione spaziale e poi con la completa discretizzazione spazio-temporale, concludendo infine con una stima dell'errore del metodo.

    \subsection{Caratteristiche del Metodo, Slide 7}

    Il metodo di Galërkin discontinuo, introdotto nel 1973 nell'ambito delle equazioni relative al trasporto di neutroni, si distingue per l'utilizzo di funzioni continue a tratti, ammettendo quindi discontinuità tra gli elementi. Questa caratteristica consente di adottare approssimazioni polinomiali di ordine elevato e offre una grande flessibilità nella scelta delle geometrie e degli elementi utilizzati.

    \subsection{Contesto del Problema Modello Parabolico, Slide 8-10}

    L'introduzione del problema modello parabolico, necessaria alla presentazione del metodo $dG$, parte dalla sua contestualizzazione, ovvero dalla scelta di un dominio spaziale $\Omega$ e di un intervallo temporale $I$. Vengono quindi introdotte le cosiddette triple di Gelfand, caratterizzate dalle proprietà qui elencate, e gli spazi $\SpaceX$, che descrivono le funzioni e le loro derivate, fondamentali per la formulazione e l'analisi dei risultati principali relativi al problema modello.

    È possibile quindi enunciare il problema parabolico scegliendo una funzione $f$, una condizione iniziale $u_0$, e cercando una funzione $u \in \SpaceX(I; V; V^*)$ tale per cui siano soddisfatte le due relazioni indicate.

    Affinché questa definizione sia ben posta, è necessario introdurre l'operatore $A$, noto anche come operatore parabolico, avente le seguenti proprietà. In particolare, è richiesto che tale operatore associ ad ogni tempo $t \in I$ un operatore lineare da $V$ al suo duale che sia continuo e coercivo, condizioni indispensabili per la corretta definizione delle formulazioni deboli in problemi di questa natura.

    \subsection{Formulazione Debole del Problema Modello Parabolico, Slide 11-12}

    Per sviluppare una formulazione $dG$ del problema modello parabolico, occorre ora introdurre una sua formulazione debole.

    Si definiscono pertanto gli spazi $\SpaceTrial$ e $\SpaceTest$, noti rispettivamente come spazi \textit{Trial} e \textit{Test}, e gli operatori $b$ e $l$, ottenuti moltiplicando le equazioni precedenti per una funzione di test e integrando successivamente lungo il dominio del problema.

    Ciò permette di considerare la formulazione debole del problema parabolico in funzione degli operatori appena introdotti e di citare un risultato che riassume la buona positura del problema, garantendo esistenza, unicità e stabilità della soluzione.

    \subsection{Semi-Discretizzazione del Problema Modello Parabolico, Slide 13-14}

    La prima discretizzazione introdotta riguarda la componente spaziale e avviene considerando una famiglia di mesh $\Omega_h$, indicizzata dal parametro $h$, e una sequenza di sottospazi $V_h$, ciascuno dotato di una propria base. Da qui deriva la definizione degli spazi semi-discreti $\SpaceTrial_h$ e $\SpaceTest_h$.

    Sfruttando versioni analoghe degli operatori introdotti nella formulazione debole, e riscrivendo l'operatore $A$ come operatore bilineare a tempo fissato, si ottiene la formulazione semi-discreta del problema modello, che consiste nel cercare una funzione in $\SpaceTrial_h$ tale che sia soddisfatta la relazione indicata per ogni funzione test appartenente a $\SpaceTest_h$.

    \subsection{Discretizzazione del Problema Modello Parabolico, Slide 15-16}

    Analogamente, si considera una famiglia di partizioni $I_{\tau}$ dell'intervallo temporale, con indice $\tau$, e si definiscono spazi polinomiali a valori in $V_h$. Ciò porta a una nuova definizione dello spazio $\SpaceTrial$, ora indicato come $\SpaceTrial_{h \tau}$, costituito da funzioni polinomiali a tratti, definizione condivisa anche dagli spazi delle funzioni test.

    Questo consente di ridefinire gli operatori $b$ e $l$ tenendo conto delle eventuali discontinuità introdotte e di formulare la versione completamente discreta del problema modello, riportata nella parte inferiore della slide.

    \subsection{Analisi dell'Errore del Problema Modello Parabolico, Slide 17}

    Ottenuta la formulazione discreta, vengono formulati diversi risultati volti alla sua caratterizzazione. In particolare, il risultato principale riguarda l'analisi dell'errore del metodo $dG$ applicato al problema modello che, sotto ipotesi adeguate sulla regolarità della soluzione esatta, fornisce una stima dell'errore stesso.

    È importante notare come questa stima risulti in un certo senso incompleta, poiché dipende dall'operatore di interpolazione spaziale. Infatti, nella formulazione appena illustrata non sono stati approfonditi dettagliatamente gli aspetti della discretizzazione spaziale, approfondimento invece effettuato nella trattazione dei problemi di Convezione-Diffusione-Reazione.

    \subsection{Contesto del Problema di CDR, Slide 18-19}

    Il problema di Convezione-Diffusione-Reazione, rilevante ad esempio nell'ambito della dinamica dei fluidi, è definito tramite la seguente equazione e le rispettive condizioni al bordo, distinte in funzione del coefficiente di convezione sul bordo del dominio $\Omega$.

    Tale problema è accompagnato da specifiche assunzioni: da un lato, ipotesi standard di regolarità sui dati iniziali, sulle condizioni al bordo e sui coefficienti dell'equazione, necessarie per garantire la buona positura del problema; dall'altro, stime sui coefficienti che assicurano l'applicabilità dei risultati di esistenza e unicità alla formulazione debole considerata nella tesi.

    \subsection{Formulazione Debole del Problema di CDR, Slide 20}

    Per questo problema è possibile fornire una caratterizzazione più completa della soluzione attraverso la formulazione debole, ottenendo risultati sulla regolarità e una relazione che costituisce il fondamento del successivo processo di discretizzazione.

    \subsection{Semi-Discretizzazione del Problema di CDR, Slide 21-22}

    Analogamente al problema modello già discusso, anche per il problema di CDR il punto di partenza della discretizzazione è la formulazione debole appena introdotta.

    Per ottenere la relazione presentata, si procede moltiplicando l'equazione CDR per una funzione di test e integrando successivamente lungo ciascun elemento della mesh spaziale, aggiungendo opportuni termini volti a comporre implicitamente le forme bilineari qui riportate. Queste ultime rappresentano rispettivamente la discretizzazione dei termini di diffusione, convezione e reazione, oltre a un ulteriore termine di stabilizzazione.

    L'introduzione del seguente spazio discreto permette, quindi, di enunciare formalmente la formulazione semi-discreta del problema.

    \subsection{Discretizzazione del Problema di CDR, Slide 23-24}

    La formulazione completamente discreta segue un procedimento simile a quello illustrato in precedenza. Definendo ora il seguente spazio discreto, costituito da funzioni polinomiali a tratti sia nello spazio che nel tempo, si ottiene la seguente formulazione discreta.

    Gli operatori $B$ e $L$ derivano direttamente dalla discretizzazione precedente, considerando esplicitamente eventuali discontinuità. È opportuno notare che la forma bilineare $A$ viene qui riscritta con l'aggiunta di un indice temporale $n$, che permette di includere nella definizione la possibilità di adottare mesh differenti per ogni livello temporale.

    \subsection{Analisi dell'Errore del Problema di CDR, Slide 25}

    A differenza del problema modello, prima di procedere all'analisi dell'errore è necessario dimostrare alcuni risultati preliminari, non riportati in queste slide, che stabiliscono la buona positura della formulazione discreta e analizzano le proprietà di approssimazione delle basi utilizzate, nello specifico quella dei polinomi di Legendre.

    Basandosi su tali risultati, si enuncia dunque il seguente teorema che fornisce la principale stima dell'errore del metodo applicato al problema CDR. Questa stima costituisce poi il punto di partenza per i test numerici effettuati nella tesi.

    \newpage
    \section{Implementazione e Risultati}

    \subsection{Tavola dei Contenuti, Slide 26}

    Seguono ora i dettagli relativi all'implementazione del metodo e ai risultati ottenuti tramite i test numerici svolti. Verranno descritti più approfonditamente gli aspetti più significativi e interessanti dell'implementazione stessa.

    \subsection{Caratteristiche dell'Implementazione, Slide 27}

    Ho implementato l'algoritmo utilizzando \lstinline{C++23}, sviluppando interamente da zero tutti i componenti necessari, tra cui il generatore della mesh e le funzioni dedicate alla costruzione e alla risoluzione del problema. Il risultato è un codice strutturato come una libreria, il cui obiettivo principale è definire e risolvere problemi di CDR.

    \subsection{Mesh Prismatiche e Poligonali, Slide 28-34}

    Una delle parti fondamentali dell'implementazione riguarda la discretizzazione del dominio spaziale, realizzata tramite mesh poligonali spaziali e semplici partizioni temporali intervallari, la cui combinazione genera le cosiddette mesh prismatiche. Un esempio grafico di tale mesh è riportato in figura.

    Il processo di costruzione della mesh prevede inizialmente la generazione casuale di un insieme di punti nel dominio spaziale, in numero pari agli elementi desiderati, seguito dalla determinazione del primo diagramma di Voronoi generato da tali punti tramite le bisettrici.

    Questo diagramma è successivamente perfezionato applicando iterativamente l'algoritmo di \textit{Lloyd}, che consiste nel ricalcolare il baricentro di ogni cella del diagramma di Voronoi e generare nuovi punti a partire da questi baricentri. Tale procedura prosegue fino al raggiungimento di una tolleranza prefissata sullo spostamento dei punti.

    Infine, il diagramma subisce una fase di post-elaborazione, in cui vengono rimossi i lati troppo piccoli, che potrebbero introdurre errori numerici nei calcoli.

    \subsection{Funzioni Polinomiali e Quadrature, Slide 35}

    Per quanto riguarda le basi polinomiali, la scelta adottata, coerentemente con la teoria sviluppata, è quella dei polinomi di Legendre, facilmente generabili tramite le relazioni qui illustrate. Per l'algoritmo di quadratura, la scelta è caduta sulla quadratura di Gauss-Legendre, in virtù della necessità di uno schema di quadratura con ordine sufficientemente alto da evitare errori dovuti ad approssimazioni numeriche insufficienti.

    La quadratura sugli elementi viene effettuata su una sotto-triangolazione del dominio per semplificare la gestione delle mappe di riferimento.

    \subsection{Costruzione e Soluzione del Problema, Slide 36-41}

    Nei metodi agli elementi finiti, il problema discretizzato viene risolto valutando gli operatori precedentemente definiti, $B$ e $L$, sulle funzioni di base scelte. Questa valutazione conduce alla costruzione di una matrice, detta matrice di stiffness, e di un vettore, che compongono il sistema lineare associato al problema. La risoluzione di tale sistema consente di determinare i coefficienti relativi alle funzioni di base della soluzione del problema discreto.

    Le quadrature presentate in precedenza, distinte in quadrature di volume e di interfaccia, spaziali e temporali, definiscono gli elementi della matrice indicati in questa slide.

    La costruzione avviene effettuando le quadrature necessarie e inserendo i risultati negli elementi corrispondenti della matrice. Analogamente si procede per la costruzione del vettore del sistema lineare.

    Il sistema lineare viene poi risolto a blocchi, ciascuno corrispondente a un intervallo temporale. Durante questa fase vengono completate le quadrature che legano ciascun intervallo temporale al precedente o al dato iniziale, risolvendo infine i sottosistemi lineari risultanti tramite \lstinline{GMRES}.

    \subsection{Caratteristiche dei Test, Slide 42-43}

    Tra i vari test effettuati nella tesi, ho scelto di illustrare in dettaglio il test di convergenza dell'algoritmo, in quanto particolarmente significativo.

    I test sono strutturati scegliendo una soluzione esatta particolare, in questo caso una soluzione regolare ma caratterizzata dalla presenza di boundary layers, insieme ai coefficienti dell'equazione qui riportati. I dati iniziali e al bordo vengono direttamente dedotti dalla soluzione scelta, mentre due diversi valori per il coefficiente di diffusione consentono di distinguere due regimi: il caso parabolico e quello iperbolico.

    Il test di convergenza consiste nel valutare l'errore dell'algoritmo utilizzando le norme elencate, applicandole a una successione di mesh aventi parametri spaziali e temporali confrontabili.

    Sono stati inclusi sia risultati ottenuti con norme classiche, già utilizzate nella letteratura considerata, sia risultati relativi a due norme aggiuntive, che possono offrire una diversa prospettiva sull'analisi dell'errore.

    \subsection{Risultati in Norme Classiche, Slide 44-46}

    I grafici riportati evidenziano il corretto andamento dell'errore nelle due scelte del grado polinomiale, sia nella norma $l2l2$ che nella norma $l2h1$, confermando la validità dei risultati ottenuti anche nel regime iperbolico.

    \subsection{Risultati in Norme Nuove, Slide 47-48}

    Per quanto riguarda le ulteriori norme introdotte, ho considerato un caso con grado polinomiale non omogeneo, utile a evidenziare la differente natura di queste due norme: la norma $l2T$, da un lato, fornisce un indicatore locale dell'errore, sottolineando il potere approssimativo spaziale dell'algoritmo; dall'altro, la norma $linfl2$ costituisce un indicatore globale dell'errore, descrivendo correttamente le caratteristiche dell'approssimazione di ordine inferiore dell'algoritmo.

    Gli stessi risultati vengono riportati anche per il caso iperbolico.

    \newpage
    \section{Conclusioni}

    \subsection{Tavola dei Contenuti, Slide 49}

    Per concludere, seguono alcune considerazioni finali e possibili sviluppi futuri.

    \subsection{Considerazioni Finali, Slide 50}

    Come evidenziato dai grafici mostrati e da ulteriori risultati non inclusi in questa presentazione, i test effettuati confermano il corretto funzionamento del metodo, mostrando risultati coerenti con le attese. In particolare, è da sottolineare la consistenza dei risultati ottenuti nel passaggio dal regime parabolico a quello iperbolico.

    Tra i possibili sviluppi futuri di questo lavoro, oltre a migliorie riguardanti l'implementazione che al momento rimane confinata all'ambito accademico, vi è certamente la direzione dell'adattività, sia rispetto agli ordini di approssimazione che a un raffinamento adattivo delle mesh, sia spaziali che temporali. Un ulteriore sviluppo importante è costituito dall'introduzione di tecniche specifiche di precondizionamento e stabilizzazione, necessarie per garantire l'affidabilità dei risultati quando si adottano approssimazioni polinomiali ad alto ordine.

    \subsection{Ringraziamenti, Slide 51}

    Grazie.

\end{document}
